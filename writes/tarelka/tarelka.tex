\documentclass{article}
\usepackage{a4wide}
\usepackage{amsmath}

\usepackage{ucs}
\usepackage[utf8x]{inputenc}
\usepackage[T2A]{fontenc} % Output encoding
\usepackage[russian]{babel} % Russian words for Chapter, etc


\usepackage{igor_macros}

\def\vec#1{\mbox{\boldmath$#1$}}

\begin{document}

Идеальная несжимаемая жидкость при условии стационарного потока описывается уравнениями:
\eqn{
(\vec{v}\cdot\vec{\nabla})\vec{v} = \vec{\nabla}\frac{p}{\rho} + \vec{g}
}
и уравнением непрерывности:
\eqn{
\vec{\nabla} \vec{v} = 0.
}

\p
В цилиндрических координатах при условии цилиндрической симметрии (я предполагая, что тарелка круглая!) вдоль оси z данное векторное уравнение записывается в виде (смотри Ландау-Лифшиц, Гидродинамика, гл.2, параграф 15):

\p

\eqn{
\splitDo{
v_r \frac{\partial v_r}{\partial r} 
      + v_z \frac{\partial v_r}{\partial z}  
     & = -\frac{1}{\rho} \frac{\partial p}{\partial r} \\
v_r \frac{\partial v_z}{\partial r} 
      + v_z \frac{\partial v_z}{\partial z}  
     & = -\frac{1}{\rho} \frac{\partial p}{\partial z} - g\\
\frac{1}{r} \frac{\partial (rv_r)}{\partial r}
+ \frac{\partial v_z}{\partial z} & = 0,
}}

где во втором уравнении стоит $-g$, поскольку ось $z$ направлена вверх. Здесь важно заметить, что в статье про расчёт тарелки ось $z$ направлвна по горизонтали, но я не стал менять оси под обозначения статьи, поскольку в книжках по физике принято как раз направлять $z$ вверх и радиальное направление по горизонтали и я к такому соглашению слишком привык.

\p
Обозначая $v \equiv v_z$ (вертикальная скорость), $u \equiv v_r$ (радиальная скорость) как принято в статье, я получаю:

\eqnl{all-eqn}{
\splitDo{
u \frac{\partial u}{\partial r} 
      + v \frac{\partial u}{\partial z}  
     & = -\frac{1}{\rho} \frac{\partial p}{\partial r} \\
u \frac{\partial v}{\partial r} 
      + v \frac{\partial v}{\partial z}  
     & = -\frac{1}{\rho} \frac{\partial p}{\partial z} - g\\
\frac{1}{r} \frac{\partial (ru)}{\partial r}
+ \frac{\partial v}{\partial z} & = 0.
}}

\p

Последняя система - это система 3 частных уравнений относительно 3 неизвестных, $u$, $v$, $p$ и при заданных граничных условиях определяемых формой тарелки и внешним давлением такая система должна иметь решение. Проблема, разумеется, как его найти.
\p 
Предположим, следуя статье, что горизонтальная скорость $u$ не зависит от высоты (или глубины), т.е. $u = u(r)$. Тогда:
\eqn{
\frac{\partial u}{\partial z} = 0.
}
\p
Дифференцируя по $z$ третие уравнение в \rf{all-eqn}, получаем:
\eqn{
\splitDo{
0 &= \frac{\partial}{\partial z}
     \left(\frac{1}{r} \frac{\partial (ru)}{\partial r}\right)
     + \frac{\partial^2 v}{\partial z^2} \\
  & = \frac{1}{r} \frac{\partial}{\partial r}
     \left(r\frac{\partial u}{\partial z}\right)
     + \frac{\partial^2 v}{\partial z^2}
}}
или
\eqn{
  0 = \frac{\partial^2 v}{\partial z^2}.
}
\p
Следовательно, $v$ явлеется линейной функцией от $z$:
\eqnl{v-form}{
v(z, r) = D(r)z + V(r),
}  
где $D(r)$ и $E(r)$ - неизвестные фунлции только от $r$. Подставляя эту форму в третие уравнение \rf{all-eqn}, находим:
\eqn{
\frac{1}{r} \frac{d(ru)}{dr} + D(r) = 0
}
или
\eqnl{D-form}{
D(r) = -\frac{1}{r} \frac{d(ru)}{dr} = -u' - \frac{u}{r}, 
}
где штрихом обозначено диференцирование по $r$ и
\eqnl{v-explicit-form}{
v(z, r) = z\left(-u' - \frac{u}{r}\right) + V(r).
}

\p
Продифиринцируем теперь первое уравнение \rf{all-eqn} по $z$:
\eqn{
\frac{\partial}{\partial z}
\left(u\frac{\partial u}{\partial r} 
      + v \frac{\partial u}{\partial z} \right)
    = -\frac{1}{\rho} \frac{\partial^2 p}{\partial r \partial z}
}
или
\eqn{
0 = \frac{\partial^2 p}{\partial r \partial z}.
}
\p
Наиболее общая форма $p(z, r)$, удовлетворящая последнему равнению, даётся формулой:
\eqnl{p-form}{
p(z, r) = P(z) + R(r),
}
где $P(z)$ и $R(r)$ - неизвестные функции.
\p
Подстановка \rf{p-form} и \rf{v-form} во второе уравнение \rf{all-eqn} позволяет найти $P(z)$:
\eqm{
u \frac{\partial v}{\partial r} 
      + v \frac{\partial v}{\partial z}  
     = 
u \frac{\partial}{\partial r}\left(D(r)z + V(r)\right)
      + (D(r)z + V(r)) D(r)  
      \\ =
u (zD' + V')+ (zD + V)D = z(uD' +D^2) + (uV' +VD) 
= -\frac{1}{\rho} \frac{dP}{dz} - g     
}
или
\eqnl{eqn-for-P}{
z(uD' +D^2) + (uV' +VD)  + \frac{1}{\rho} \frac{dP}{dz} + g = 0.
}
\p
Если продиффиринцировать дважды последние уравнение по $z$, то получается:
\eqn{
\frac{d^3P}{dz^3} = 0.
}
с общим решением
\eqnl{P-kvadra}{
P(z) = P_1 + Sz +Wz^2,
}
где $P_1$, $S$, $W$ - произвольные константы. Таким образом, из предположениа, что $u$ зависит только от $r$ следует, что $v$ - линейная функциа $z$, а $p$ - квадратичная функция $z$.

\p
\rf{eqn-for-P} даёт:
\eqn{
z(uD' +D^2) + (uV' +VD)  + \frac{1}{\rho}\left(2Wz + S\right) + g 
= z\left(uD' +D^2 +\frac{2W}{\rho}\right) + \left(uV' + VD + \frac{S}{\rho}+ g\right) = 0.
}
Последние равенство жыполняется при произвольных $z$, поэтому получается 2 уравнения:
\eqnl{VD-conditions}{
\splitDo{
0 = & uD' +D^2 +\frac{2W}{\rho} \\
0 = & uV' + VD + \frac{S}{\rho} +g.
}}
\p
Используя \rf{D-form}, из первого равенства получается:
\eqn{
\splitDo{
0 = & u\left(-u'' - \frac{u'}{r} + \frac{u}{r^2}\right) + \left(u' + \frac{u}{r}\right)^2 + \frac{2W}{\rho} \\
= &  -uu'' - \frac{uu'}{r} + \frac{u^2}{r^2}
  + (u')^2 + \frac{2u'u}{r} + \frac{u^2}{r^2} + \frac{2W}{\rho} \\
= &  -uu'' + \frac{uu'}{r} + (u')^2  + \frac{2u^2}{r^2} + \frac{2W}{\rho}. 
}} 
или:
\eqn{
 uu'' - (u')^2  - \frac{uu'}{r}  - \frac{2u^2}{r^2} =  \frac{2W}{\rho}. 
}

\p
Последние равенство можно представить в виде:
\eqnl{ur-W-eqn}{
ru^2\left(\frac{(ur)'}{ur^2}\right)' = \frac{2W}{\rho}.
}
В самом деле:
\eqm{
 ru^2\left(\frac{(ur)'}{ur^2}\right)' 
 = 
 ru^2\left(\frac{u'r + u}{ur^2}\right)' 
 = 
 ru^2\left(\frac{(u'r + u)'ur^2 - (u'r + u)(ur^2)'}{u^2r^4}\right)
 \\
 =  \frac{(u'r + u)'ur^2 - (u'r + u)(ur^2)'}{r^3} 
 = \frac{(u''r + u' + u')ur^2 - (u'r + u)(u'r^2 + 2ru)}{r^3}
\\
 = \frac{uu''r^3 + 2uu'r^2 - (u')^2r^3 - 2uu'r^2 - uu'r^2 - 2ru^2}{r^3}
 = \frac{uu''r^3 - (u')^2r^3 - uu'r^2 - 2ru^2}{r^3}
\\
= uu'' - (u')^2 - \frac{uu'}{r} - \frac{2u^2}{r^3}.
}

\p 

Перейдём теперь к задаче о нахождении $R(r)$ в \rf{p-form}. Подставим \rf{p-form} в первое уравнение \rf{all-eqn}, получим:
\eqn{
u \frac{\partial u}{\partial r} = -\frac{1}{\rho} \frac{\partial P(z) + R(r)}{\partial r}
}
или
\eqn{
uu' = -\frac{R'}{\rho}. 
}
Интегрируя по $r$, получаем:
\eqn{
-\frac{1}{2}\rho u^2 + C = R,
}
где $C$ - константа интегрирования. Тогда \rf{p-form}, \rf{P-kvadra} и последние выражение дают:
\eqnl{p-via-z-and-r}{
p(z, r) = P_1 + Sz +Wz^2 - \frac{1}{2}\rho u^2 + C.
}
или 
\eqnl{p-via-u}{
p(z, r) = P_2 + Sz +Wz^2 - \frac{1}{2}\rho u^2.
}
где $P_2 = P_1 +C$ - новая константа, которая как и должна определяться начальными условиями.

\p
Рассмотрим теперь столбик воды на расстоянии $r$ от центра тарелка. Етот столбик имеет определенную высотy $h$, которая зависит только от $r$, $h = h(r)$. Поскольку давление на тарелкой постоянно и равно $P_0$, то и давление в верхней точке столбика тоже $P_0$. Считая, что плоскость $z = 0$ соответствует дну тарелки, и, следовательно в верхней точке столбика $z = h$, откуда получается:
\eqnl{P-null}{
P_0 = p(h, r) = P_2 + Sh +Wh^2 - \frac{1}{2} \rho \left(u(r)\right)^2.
}
\p
Последняя формула позволяет зная $u(r)$ найти зажисимость $h(r)$. Воспользуемся теперь граничным усложоем, что через тареку проходит $Q$ единиц массы жидкости в секунду (соответсвенно, $Q/\rho$ единиц обьёма). Тогда эта же масса жидкости должна проходить через каждый цилиндр с произвольным радиусом $r$. Плошадь поверхности такого цилиндра $2\pi r h$ и в предполовении, что $u$ не зависит от $z$ поток $Q$ определяется через:
\eqn{
Q = 2\pi r h \rho u,
} 
откуда:
\eqnl{u-via-Q}{
u = \frac{Q}{2\pi r h \rho}.
} 
\p
Подставляя последний результат в \rf{P-null}, получим:
\eqn{
P_0 = P_2 + Sh +Wh^2 - \frac{1}{2} \rho \frac{Q^2}{(2\pi r h \rho)^2},
}
\eqnl{p-z-rel-one}{
Sh +Wh^2 = P_0 - P_2 + \frac{Q^2}{8(\pi r h)^2 \rho}.
}

Последнее уравнение опеределяет $h$ через $r$ как решение уравнения четвертой степени и через \rf{u-via-Q} также определяет зависимость $u(r)$. Эта зависимисть должна быть согласована с уравнением \rf{ur-W-eqn}. Если соответствия не будет, то степень его нарушения определит возможность применения приближения $u$ - функция только от $r$. 

\p
Посмотрим, насколько хорошо ето выполняется, если использовать предполовение статьи о линейной зависимости $p$ от $z$. В силу \rf{p-via-z-and-r} это даёт:
\eqn{
W = 0
}   
и \rf{p-z-rel-one}, \rf{ur-W-eqn} переходят в кубическое уравнение
\eqn{
Sh = P_0 - P_2 + \frac{Q^2}{8(\pi r h)^2 \rho}.
}
и простое дифференциальное уравнение
\eqn{
\left(\frac{(ur)'}{ur^2}\right)' = 0.
} 

\p
Его решением будет:
\eqn{
u = \frac{r_0u_0}{r}\exp({-r^2/r_0^2}),
}
где $u_0$, $r_0$ - константы интегрирования. Тогда:
\eqn{
h(r) = \frac{Q}{2\pi r u \rho} = \frac{Q}{2\pi r_0 u_0 \rho} \exp({r^2/r_0^2})
}
и согласование будет выполнено, если:
\eqn{
Sh = \frac{SQ}{2\pi r_0 u_0 \rho} \exp({r^2/r_0^2}).
}
будет равно
\eqn{
P_0 - P_2 + \frac{Q^2}{8(\pi r h)^2 \rho} = P_0 - P_2 +\frac{\rho u^2}{2},
}
т.е. если будет выполняться:
\eqn{
\frac{SQ}{2\pi r_0 u_0 \rho} \exp({r^2/r_0^2}) 
=
P_0 - P_2 +\frac{\rho r_0^2 u_0^2}{2r^2} \exp({-2r^2/r_0^2}).
}

\p
Последние не может выплняться при произвольном $r$, что и выражает мою проблему...

\end{document}

\documentclass{letter}
\begin{document}

Dear Pr. ,

I am a Master student at Bergen University and Pr. Osland recommend me to
ask you about the questions arrised for me during my study of the general 
relativity.

The questions are related to the properties of gravitational horesont
and can be formulated as follows. 

Although for an free-falling particle that moves towards the black hole hole
it takes the finite proper time interval not only to cross the horisont but 
also reach the singularity, for any observer (at least as I understand)
whose world line does not cross the horizont it will take infinite amount
of proper time. From this I can deduce that the possible way to describe the 
no-way-back property of the horisont is to say that because the horisont
crossing occurs  at the future infinity from the outside world point 
of view the particle should move back in time to return from black hole.
Is this right description?

And the second related questions is about a heavy star collapse.
How long does it take to collapse to a black hole for 
a observer that stays outside of the star gravitational radius?
As I understand the general answer is infinity again, and for the simplest 
models such as the collapse of dust matter with zero pressure
it can be shown explicitly. This also means that new black holes can former
only in infinite time from the external observer point of view. Or probably
I do not understand something here, because there are a lot of literature about
thermodynamics of such objects. How thermodynamics can be 
applied to something that steal does not exist? The same problem is with the 
understanding of the black hole evaporation. 

So could you help me with these problems and/or give me some references where
I can find the answer?

\end{document}

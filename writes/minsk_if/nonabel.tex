\documentclass[a4paper]{article}
\usepackage[utf8]{inputenc} % Input encoding
\usepackage[T2A]{fontenc} % Output encoding
\usepackage[russian]{babel} % Russian words for Chapter, etc

% Original header: \documentstyle[14pt,russian]{artr}

\def\baselinestretch{1.4}
\textheight 210mm
%\textwidth 150mm
\topmargin 20.0mm
%\hoffset 10.0mm


\def\MAP#1{\ifmmode{#1}\else{$ #1 $}\fi}
\def\abs#1{\MAP{\mid\!\!#1\!\!\mid}}
\def\ket#1{\MAP{\mid\!\!#1\!\!>}}
\def\bra#1{\MAP{<\!\!#1\!\!\mid}}
\def\dirProd#1#2{\MAP{<\!\! #1 \!\! \mid \!\! #2 \!\!>}}
\def\vct#1{\MAP{\vec #1}}
\def\pDer#1#2{\MAP{\frac{\partial #1}{\partial #2}}}
\def\oDer#1#2{\MAP{\frac{d#1}{d#2}}}

\def\be{\begin{equation}}
\def\ee{\end{equation}}
\def\bel#1{\begin{equation}\label{#1}}
\def\rf#1{(\ref{#1})}

\def\H{\MAP{\bf H}}
\def\X{\MAP{\vct x}}
\def\E{\MAP{E\,}}


%\pagestyle{empty}

\begin{document}

Рассмотрим квантовую систему с гамильтонианом $\H(\X)$ и заданной зависимостью
внешних параметров \X = $\{x_1, \ldots, x_d\}$ от времени:
\be
\X = \X(t),
\ee
и пусть в начальный
момент времени система находится в состоянии \ket{\psi(t_0)},
соответствующем уровню с энергией \E и
степенью вырождения $N$, так что:
\be
\H(\X(t_0))\ket{\psi(t_0)} = \E(\X(t_0))\ket{\psi(t_0)}.
\ee

Тогда в случае применимости адиабатического приближения можно считать,
что в последующие моменты времени система будет оставаться на данном
уровне, при этом конкретная зависимость \ket{\psi(t)} дается выражением:
\be
\ket{\psi(t)} =
\exp\left(-i\int_{t_0}^{t} E(t_1) d\,t_1\right)c^i(\X(t))\ket{n_i(\X(t))},
\ee
где по повторяющимся латинским индексам подразумевается суммирование от
1 до $N$, \ket{n_i} образуют базис в пространстве состояний с энергией
\E:
\be
\dirProd{n^i}{n_j} = \delta^i_j, \mbox{ и} \quad
\H(\X)\ket{n_i} = \E(\X)\ket{n_i}, \quad i, j = 1\ldots N,
\ee
а коэфиценты $c^i$ определяются из решения системы уравнений:
\bel{cEqn}
\oDer{x^\mu}{t}\left(\pDer{c^i}{x^\mu} + A^i_{\mu j} c^j\right) = 0
\ee
с начальными условиями, следующими из разложения \ket{\psi} по базису
\ket{n_i} в момент времени $t_0$:
\be
\ket{\psi(t_0)} = c^i(\X(t_0))\ket{n_i(\X(t_0))}.
\ee
В \rf{cEqn} по повторяющимся греческим индексам идет суммирование от
1 до $d$ - размерности пространства параметров и по определению:
\bel{ADef}
A^i_{\mu j} \equiv \dirProd{n^i}{\oDer{n_j}{x^\mu}}.
\ee

\end{document}

% This file can processed by TEX.
\documentstyle{article}

\begin{document}

Dear Evgenij Arkadievich,
After some calculations I have got a result for the topological
charge that I can interpret in the next way. In the case of the
standard instanton on $S^4$ to calculate the charge one must
perform the integration only on sphere $S^3(r \rightarrow \inf)$
and this will give 1 but in the case of our connection we have 2
boundaries either we consider $CP^2$ with additional boundary
$S^3(r = 1)$ or $S^4$ where the connection has a divergence at $r
= 0$ so it should be surrounded by sphere $S^3(r \rightarrow 0)$.

Integration over $S^3(r = 1)$ is just 0 so the whole charge is 1
and only behaviour only on the infinity is important but
integration over $S^3(r \rightarrow 0)$ gives -1 so the total
charge is 0 and from $S^4$ point of view our connection is
trivial.   

Here is calculation main points:

The definition of topological charge:
$$
k = \frac{1}{2\pi^2}\int det(F)
$$
$$
det(F) = F_{11} \wedge F_{22} - F_{12} \wedge F_{21}
= \frac 1 2 (Tr(F) \wedge Tr(F) - Tr(F \wedge F)) = -d\theta
$$
where $\theta$ is:
$$
\frac 1 2 (Tr(A \wedge dA) - Tr(A) \wedge dTr(A)) + 
\frac 1 3 Tr(A \wedge A \wedge A).
$$ 
Hence:
$$
k = -\frac{1}{2\pi^2}\int d\theta = 
-\frac{1}{2\pi^2}\int_{boundary} \theta
$$

Now if $A$ has the form:
$$
A_{11} = f_1\omega_0, \quad A_{12} = f_3\omega_1, \quad
A_{21} = -f_3\bar\omega_1, \quad A_{22} = f_2\omega_0
$$
where:
$$
\omega_0 = \zeta_1d\bar\zeta_1 + \zeta_2d\bar\zeta_2
- d\zeta_1\bar\zeta_1 - d\zeta_2\bar\zeta_2,
$$
$$
\omega_1 = \zeta_2d\zeta_1 - \zeta_1d\zeta_2 \quad
\bar\omega_1 = \bar\zeta_2d\bar\zeta_1 - \bar\zeta_1d\bar\zeta_2
$$
$$
f_i = f_i(r^2) \quad r^2 = 
\zeta_1\bar\zeta_1 + \zeta_2\bar\zeta_2
$$
then for $k$ I have:
$$
k = K(\inf) - K(r_0), \quad
K(r) = -2r^4(2f_1f_2 + f_3^2((f_1 - f_2)r^2 - 1)) 
$$
with $r_0 = 1$ for $CP^2$ and $r_0 = 0$ for $S^4$.

In the case of the standard instanton:
$$
f_1 = \frac{1}{2}f_3, \quad 
f_2 = -\frac{1}{2}f_3, \quad
f_3 = \frac{1}{1 + r^2},
$$ 
and:
$$
K(r) = \frac{r^4}{(1 + r^2)^2}(1 + \frac{2}{1 + r^2}),
$$
so:
$$
k(S^4) = K(\inf) - K(0) = 1 - 0 = 1.
$$

In the case of our potential $A$:
$$
f_1 = \frac{1}{2r^2}, \quad 
f_2 = -\frac{1}{2r^2} \left(\frac{r^2 - 1}{r^2 + 1}\right)^2,
\quad
f_3 =  \frac{1}{r^2} \frac{r^2 - 1}{r^2 + 1},
$$
$$ 
K(r) = \left(\frac{r^2 - 1}{r^2 + 1}\right)^2
(2 - \left(\frac{r^2 - 1}{r^2 + 1}\right)^2),
$$ 
so:
$$
 k(CP^2) = K(\inf) - K(1) = 1 - 0 = 1,
$$
$$
 k(S^4) = 1 - K(0) = 1 - 1 = 0.
$$
\end{document}

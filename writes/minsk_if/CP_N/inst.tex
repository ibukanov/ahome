%\documentstyle[russian]{artr}
%\documentstyle[14pt,russian]{artr}

\documentstyle{article}

%\pagestyle{empty}

%This macro will enable to use others in non-math mode
\def\MAP#1{\ifmmode{#1}\else{$#1$}\fi}

%standard abs
\def\abs#1{\MAP{\mid\!\!#1\!\!\mid}} 

%definition for dirac vectors and product
\def\ket#1{\MAP{\mid\!\!#1\!\!>}}
\def\bra#1{\MAP{<\!\!#1\!\!\mid}}
\def\dirProd#1#2{\MAP{<\!\!#1\!\mid\!#2\!\!>}}

%derivatives: partial and ordinary in different forms.
\def\pDer#1#2{\MAP{\frac{\partial #1}{\partial #2}}}
\def\pDerF#1#2{\MAP{\partial_#2 #1}}
\def\pDerB#1#2{\MAP{\left(#1\right)^\prime_#2}}
\def\oDer#1#2{\MAP{\frac{d#1}{d#2}}}
\def\oDerB#1#2{\MAP{left(#1\right)^\prime}}

%I prefer to use next command to write math formula in red line.
\def\be{\[}
\def\ee{\]}
\def\bel#1{\begin{equation}\label{#1}} % to label formula
\def\eel{\end{equation}} % to label formula

%standard representation for refferences to formulas
\def\rf#1{(\ref{#1})}

\begin{document}

\title{Connection and topological charges for some systems}

%\maketitle

\section{About this document (Introduction).}

The following calculation is a modification of one that I did last year
with usage of non-orthogonal basis. In this version I avoided to use it 
so now I do not need to speak about correctness of the usage.

Unfortunately I have to use a lot of non-standard notations just to keep
formulas more or less compact. I only hope that comments in this very 
draft representation make reading not do difficult.     
 

\section{General words.}


Consider a $M$-level quantum system with hamiltonian $H$ which has 
$N$ degenerated level with energy $E$ and let $\ket{n^i}$, $i = 1..N$ 
form a orthogonal basis in the level state space, So:
\bel{BasisDef}
\dirProd{n_i}{n^j} = \delta_i^j, \quad H\ket{n^i} = E\ket{n^i},
\quad i,j=1..N.   
\eel

Then corresponding nonabelian connection which appears during evolution
of the system has next form:
\be
(A^i_j)_\mu = \dirProd{n^i}{\pDer{n_j}{x^\mu}}.
\ee

To make following calculations more compact I introduce now 
indexless notation. Let $n$ denotes $M\times N$ matrix defined by:
\be
(n)^i_a = (\ket{n^i})_a, 
\quad i=1..N, \quad a=1..M 
\ee
where $(\ket{n})_a$ is $a$ component of a vector \ket{n}. Then \rf{BasisDef}
can be rewritten as:
\bel{BasisRedef}
n^+n=I_N, \quad H^\prime n = 0, \quad H^\prime \equiv H - EI_M,
\eel
and by introducing matrix valued differential form $A$:
\be
(A)^i_j = (A^i_j)_\mu \wedge dx^\mu
\ee
I have:
\bel{ADef}
A = n^+dn.
\eel

It is easy to see that under $U(N)$ transformation 
\be
n \rightarrow \tilde n =  n\omega
\ee
given by unitary $N\times N$ matrix $\omega$ identities \rf{BasisRedef} 
will be hold:
\be
\tilde n^+ \tilde n = \omega^+n^+n\omega = \omega^+I_N\omega = 
\omega^+\omega = I_N,
\ee
\be
H^\prime \tilde n = H^\prime n \omega = 0 \omega = 0.
\ee
and for $\tilde A$ I have:
\be 
\tilde A = \tilde n^+ d\tilde n = \omega^+ n^+ d(n\omega)
=  \omega^+ n^+ dn \omega + \omega^+ n^+n d\omega = 
\omega^{-1} A \omega + \omega^{-1} d\omega,
\ee
i.e. standard connection transformation law.



\section{Basis representation.}

As it follows from the previous section results 
to construct the particular connection one should select a
parametrization of $n$ and then use \rf{ADef} to calculate $A$. One possible
way to choose $n$ is based on next approach. 

The requirement the energy level should be $N$-degenerated means that 
$rank\, H^\prime = M-N$ or in other words that there is 
$(M-N)\times (M-N)$ submatrix in $H^\prime$ with non-zero determinant.
By permutation of rows and columns it is always possible to move the submatrix 
to the right bottom conner and write for $H^\prime$: 
\be
H^\prime = \pmatrix{h_{11} & h_{12} \cr h_{21} & h_{22} \cr },
\quad det h_{22} \ne 0, 
\quad n =  \pmatrix{\rho \cr \rho_\prime \cr}
\ee
where $h_{11}$,$h_{12}$,$h_{21}$,$h_{22}$,$\rho$,$\rho_\prime$ are 
$N\times N$, $N\times (M-N)$,$(M-N)\times N$,$(M-N)\times (M-N)$,
$N\times N$, $(M-N)\times N$ matrixes.


So $H^\prime n = 0$ is equivalent to: 
\bel{eqn_for_n}
h_{21}\rho + h_{22}\rho_\prime = 0 \mbox{ or } 
\rho_\prime = - (h_{22})^{-1}h_{21}\rho
\eel
because other equations:
\bel{first_eqn_for_H} 
h_{11}\rho + h_{12}\rho_\prime = 0
\eel
 due to $det h_{22} \ne 0$
and $rank H^\prime = M-N$ is just linear combination of \rf{eqn_for_n}. 
Moreover it can be used to express $h_{11}$ in terms of others $h_{ij}$:
\be
h_{11} = -h_{12}\rho_\prime \rho^{-1} = h_{12}(h_{22})^{-1}h_{21},
\ee
the last equation is the same as $H^\prime n = 0$ and from it follows 
that elements of $h_{21}$, $h_{22}$ can be treated as independent variables 
and origin $H$ can be expressed in terms of them and energy value $E$
by following procedure:

1. Define $h_{12} = h_{21}^+$.

2. Define $h_{11} = h_{12}(h_{22})^{-1}h_{21}$. This fully determines 
$H^\prime$.

3. Define $H = H^\prime + E$.

The requirement of basis orthogonality $n^+n = 1$ leads to: 
\be
 1_N = \rho^+\rho + \rho_\prime^+\rho_\prime = 
 \rho^+(1_N + h_{21}^+(h_{22})^{-2}h_{21})\rho.
\ee 
So $det \rho \ne 0$ which gives:
\bel{RhoCondition}
 (\rho^+)^{-1}(\rho)^{-1} = (\rho\rho^+)^{-1} = 
 1_N + h_{21}^+(h_{22})^{-2}h_{21}.
\eel 
Last equation is only one condition for $\rho$ and for any such $\rho$ I have
for $n$:
\bel{n_is}
n = \pmatrix{\rho \cr - (h_{22})^{-1}h_{21}\rho \cr} 
= \pmatrix{1_N \cr x^+ \cr} \rho,
\eel
where by definition for $N\times (M-N)$: $x^+ = - (h_{22})^{-1}h_{21}$.

From the fact that $h_{21}$ is independent from $h_{22}$ and its elements 
can be any complex values and from $det h_{22} \ne 0$ 
it follow that elements of $x$
can be any complex values too and possible values of $x$ cover
whole :
\bel{space_is}
C^{N\times (M-N)} \equiv R^{2N\times (M-N)}.
\eel

Important note: 
The last statement does not refer to the whole possible topology of 
hamiltonian parameter space due to for example the restriction 
$det h_{22} \ne 0$, it just refer to domain for $x$.  
I do not like this prove that the domain is $C^{N\times (M-N)}$ -
but currently I do not know how to express it in more strict way.
And in any case it is important to write some consideration of possible
topology of quantum system states. 
 
\section{Connection and field tensor.}

Calculations of connection $A$ from \rf{n_is} straightforward:
\be
A = n^+dn = \rho^+\pmatrix{1_N^+ & x \cr } 
d(\pmatrix{1_N \cr x^+ \cr } \rho) 
\ee
\be
= \rho^+xdx^+\rho + \rho^+(1+xx^+)d\rho
= \rho^{-1}\rho\rho^+(xdx^+\rho + (1+xx^+)d\rho).
\ee
But 
\be
(\rho\rho^+)^{-1} = 1_N + h_{21}^+(h_{22})^{-2}h_{21} = 1 + xx^+,
\ee
this leads to:
\be
A = \rho^{-1}(1 + xx^+)^{-1}(xdx^+\rho + (1+xx^+)d\rho)
 = \rho^{-1}(1 + xx^+)^{-1}xdx^+\rho + \rho^{-1}d\rho
\ee

Last formula looks like transformed by matrix $\rho$ connection:
\be
\tilde A = (1 + xx^+)^{-1}xdx^+, 
\ee
but because in general $\rho\rho^+ = (1 + xx^+)^{-1} \ne 1$ this is not a 
$U(N)$ transformation. Nevertheless this representation is useful in calculation
of $F$ because it will look as transformed by $\rho$ $\tilde F$. Proof is
the same as for ordinary $U(N)$ transformation :
\be
F = dA + A \wedge A = d(\rho^{-1}\tilde A \rho + \rho^{-1}d\rho) +
\rho^{-1}(\tilde A \rho + d\rho) \wedge \rho^{-1}(\tilde A \rho + d\rho)
\ee
\be
= -\rho^{-1}d\rho\rho^{-1}\wedge\tilde A \rho + \rho^{-1}d\tilde A\rho
- \rho^{-1}\tilde A \wedge d\rho - \rho^{-1}d\rho\wedge\rho^{-1}d\rho
\ee
\be
+ \rho^{-1}\tilde A\wedge\tilde A\rho + \rho^{-1}\tilde A d\rho
+ \rho^{-1}d\rho\wedge\rho^{-1}\tilde A \rho 
+ \rho^{-1}d\rho\wedge\rho^{-1}d\rho
\ee
\be
= \rho^{-1}(d\tilde A + \tilde A\wedge\tilde A)\rho = \rho^{-1}\tilde F\rho,
\ee
where next identity is used: $d\rho^{-1} = -\rho^{-1}d\rho\rho^{-1}$ and
$d(\omega_P\wedge\omega_Q) = d\omega_P \wedge \omega_Q 
+(-1)^P \omega_P\wedge\omega_Q$ for any differential forms of rank $P$ and $Q$.

For $\tilde F$ I have:
\be
\tilde F = d((1 + xx^+)^{-1}xdx^+) + 
(1 + xx^+)^{-1}xdx^+\wedge (1 + xx^+)^{-1}xdx^+
\ee
\be
= -(1 + xx^+)^{-1}d(1 + xx^+)(1 + xx^+)^{-1}\wedge xdx^+
\ee
\be
+ (1 + xx^+)^{-1}dx\wedge dx^+ 
+(1 + xx^+)^{-1}xdx^+\wedge (1 + xx^+)^{-1}xdx^+
\ee
\be
= (1 + xx^+)^{-1}\left(
(-d(xx^+) + xdx^+)(1 + xx^+)^{-1}x + dx\right)\wedge dx^+
\ee
\be
= (1 + xx^+)^{-1}dx\left(-x^+(1 + xx^+)^{-1}x + 1\right)\wedge dx^+,
\ee
But $(1 + xx^+)^{-1}x = x(1 + x^+x)^{-1}$  due to:
\be
(1 + xx^+)^{-1}x - x(1 + x^+x)^{-1} 
\ee
\be
=(1 + xx^+)^{-1}\left(x(1 + x^+x) - (1 + xx^+)x\right)(1 + x^+x)^{-1} = 0.
\ee
So:
\be
\tilde F = (1 + xx^+)^{-1}dx\left(-x^+x(1 + x^+x)^{-1} + 1\right)\wedge dx^+
\ee
\be
= (1 + xx^+)^{-1}dx\left(-x^+x + 1 + x^+x\right)(1 + x^+x)^{-1}\wedge dx^+,
\ee
or:
\bel{t_F_is}
\tilde F = (1 + xx^+)^{-1}dx \wedge (1 + x^+x)^{-1}dx^+,
\eel
\bel{F_is1}
F = \rho^{-1}(1 + xx^+)^{-1}dx \wedge (1 + x^+x)^{-1}dx^+ \rho
\eel

To avoid a calculation of inverse matrix in \rf{t_F_is} twice one can use 
next identities:
\bel{invert_prop}
(1 + xx^+)^{-1} = 1 - x(1 + x^+x)^{-1}x^+, 
\quad (1 + x^+x)^{-1} = 1 - x^+(1 + xx^+)^{-1}x.
\eel   
Here is straightforward prove for the first one:
\be 
(1 + xx^+)(1 - x(1 + x^+x)^{-1}x^+) 
= 1 + x(1 - (1 + x^+x)^{-1} - x^+x(1 + x^+x)^{-1})x^+
\ee
\be
= 1 + x (1 - (1 + x^+x)(1 + x^+x)^{-1})x^+ = 1 + 0 = 1.
\ee

\section{Connection and field tensor in terms of hamiltonian matrix elements.}

To get an expression for $A$ and $F$ via matrix elements of $H$ 
$\rho$ should be fixed in some way. One possibility to do it 
is to require 
$\rho = \rho^+$. This gives from \rf{RhoCondition}:
\be
\rho^{-2} = 1 + h_{21}^+(h_{22})^{-2}h_{21}
\ee
or:
\be
\rho = (1 + h_{21}^+(h_{22})^{-2}h_{21})^{-\frac{1}{2}} 
= (1 + xx^+)^{-1/2}
\ee

and:

\bel{A_is} 
A = (1 + xx^+)^{-1/2}xdx^+(1 + xx^+)^{-1/2} 
+ (1 + xx^+)^{1/2} d((1 + xx^+)^{-1/2}),
\eel

\bel{F_is}
F = (1 + xx^+)^{-1/2} dx \wedge (1 + x^+x)^{-1} dx^+ (1 + xx^+)^{-1/2}.
\eel

By using \rf{A_is} and \rf{F_is} calculation can be done in principle
for any $M$ and $N$. Of cause due to matrix structure of $x$ the final 
expressions via hamiltonian matrix elements will be very complicated.
But in two important cases when
$N=1$ and $N=M-1$it 
is possible to get this result in general form. 
In the first one which corresponds to non-degenerated case
$x$ is $1\times (M-1)$ and $xx^+$ are just numbers which are commute. 
So defining $r^2 = xx^+$ = $h_{21}^+h_{22}^{-2}h_{21}$ I have:
\be
A = (1 + r^2)^{-1/2}xdx^+(1 + r^2)^{-1/2} 
+ (1 + r^2)^{1/2} d((1 + r^2)^{-1/2})
\ee
\be
= \frac{1}{1 + r^2}xdx^+ - \frac{1}{2(1 + r^2)}d(r^2)
= \frac{1}{2(1 + r^2)}(xdx^+ - dxx^+),
\ee
\be
F = \frac{1}{1 + r^2} dx \wedge (1_N + x^+x)^{-1} dx^+
\ee 
\be
= \frac{1}{1 + r^2} dx \wedge (1N - \frac{1}{1 + r^2}x^+x) dx^+ 
\mbox{ from \rf{invert_prop}}
\ee
\be
 = \frac{1}{1 + r^2} dx \wedge dx^+
- \frac{1}{(1 + r^2)^2} dxx^+ \wedge xdx^+.
\ee

In the second case $x$ becames $N\times 1$ matrix so $xx^+$ with $A,F$ are
$N\times N$ ones and $x^+x$ with $h_{22}$ are just number. 
So define now $r^2 = x^+x$ = $\frac{1}{h_{22}^2}h_{21}h_{21}^+$ 
I have from \rf{invert_prop}:
\be
(1 + xx^+)^{-1} = 1_N - \frac{1}{1 + r^2}xx^+,
\ee
and by writing $(1 + xx^+)^{-1/2}$ = $1 + xf(r^2)x^+$
I get an quadratic equation for f:
\bel{rec_two}
(1 + xfx^+)^2 = 1 + x(2f + f^2r^2)x^+ = 1 - x\frac{1}{1 + r^2}x^+
\eel
\be
\mbox{or: } 2f + f^2r^2 = - \frac{1}{1 + r^2},
\ee
\be
f^2 + frac{2}{r^2}f + \frac{1}{r^2(1 + r^2)} = 0
\ee
with solution:
\be
f(r^2) = -frac{1}{r^2}\left( 1 \pm \frac{1}{\sqrt{1 + r^2}}\right).
\ee
The same technic gives:
\be
(1 + xx^+)^{1/2} = 1 - \frac{f}{1 + fr^2}xx^+.
\ee

(Here I will write final expression for A and H in more or less form,
but I do not select which up to now)

\section{Topological charges.}

From Postnikov book, "Differential geometry" it follows that to calculate
different Chern number for particular connection on complex manifold 
one should find all 
$N$ invariant polynomials of $F$ such as $Tr(F)$, $det(F)$ and so on. 
(The total number should be $N$ just because $F$ is $N\times N$ matrix).
And then from these polynomials that will be differential forms of even order
from 2 to $2N$ because $F$ itself is 2-oreder by making external product
construct all possible linear independent forms $\omega_A, A = 1..$ 
of order $D$ where $D$ is parameter space dimension, which is even 
due to complex nature of manifold.
Then corresponding Chern number is given by:
\be
   c_A = \frac{1}{(2\pi i)^{D/2}} \int \omega_A.
\ee    

In the case of $F$ represented by \rf{F_is} $D = 2N\times(M-N)$ as it follows
from \rf{space_is}. 

And again explicit calculation of $c_A$ can be done in two mentioned cases.

\subsection{Topological charge for $N=1$ system.}

Here the parameter space dimension is $2N$ and 
$F$ is just ordinary 2-form and there is only one invariant polynomial 
which is $F$ itself so the connection characterized by only one Chern number:
\be
c_1 = \frac{1}{(2\pi i)^N} \int (F)^N.
\ee

By representing F in the form: 
\bel{F_for_N_one}
F = a_2 - a_1 \wedge a_1^+
\eel
with 2-form $a_2 = \frac{1}{1 + r^2} dx \wedge dx^+$
and 1 form $a_1 = \frac{1}{1 + r^2} dxx^+$ $(F)^N$ becomes:
\be
(F)^N = (a_2 - a_1 \wedge a_1^+)^N
= a_2^N - Na_1 \wedge a_1^+ \wedge a_2^{N-1}
\ee
from $(a_1^+ \wedge a_2)^k = 0, k > 1$ for any 1-forms $a_1, a_2$. So:
\be
(F)^N = \frac{1}{(1 + r^2)^{N + 1}}
\left((1 + r^2)(dx \wedge dx^+)^N 
- N(dxx^+ \wedge xdx^+)(dx \wedge dx^+)^{N-1}\right).
\ee

In components of $x$: $(x)_{i,1} = z_i, i = 1..N$ with complex numbers $z_i$
I have:
\be
(dx \wedge dx^+) = \sum_{i = 1}^Ndz_i \wedge dz_i^*,
\quad (dxx^+) = \sum_{i = 1}^N dz_i z_i^*,
\ee
\be
\mbox{and: } (dx \wedge dx^+)^k = 
\sum_{i_1, i_2, \cdots,i_k }
dz_{i_1} \wedge dz_{i_1}^* \wedge dz_{i_2} \wedge dz_{i_2}^*
\wedge \cdots \wedge dz_{i_k} \wedge dz_{i_k}^*
\ee
\be
= \sum_{i_1 \ne i_2 \ne \cdots \ne i_k \ne i_1}
dz_{i_1} \wedge dz_{i_1}^* \wedge dz_{i_2} \wedge dz_{i_2}^*
\wedge \cdots \wedge dz_{i_k} \wedge dz_{i_k}^*
\ee
\be
= k!\sum_{1 \le i_1 < i_2 < \cdots < i_k \le N}
dz_{i_1} \wedge dz_{i_1}^* \wedge dz_{i_2} \wedge dz_{i_2}^*
\wedge \cdots \wedge dz_{i_k} \wedge dz_{i_k}^,
\ee
\be
\mbox{so: }
(dx \wedge dx^+)^N = 
N!dz_1 \wedge dz_1^*\wedge \cdots \wedge dz_N \wedge dz_N^*
= N!\Omega_N \mbox{  by definition,}
\ee
and $(dxx^+ \wedge xdx^+)(dx \wedge dx^+)^{N-1}$ =
\be
(N-1)!
\left(\sum_{i,j} dz_i z_i^* \wedge z_j dz_j^* \right) \wedge
\sum_{1 \le i_1 < i_2 < \cdots < i_{N-1} \le N}
dz_{i_1} \wedge dz_{i_1}^* 
\wedge \cdots \wedge dz_{i_{N-1}} \wedge dz_{i_{N-1}}^*
\ee 
\be
=
(N-1)!
(\sum_{i} \abs{z_i}^2 dz_i \wedge dz_i^* ) \wedge
\sum_{1 \le i_1 < i_2 < \cdots < i_{N-1} \le N}
dz_{i_1} \wedge dz_{i_1}^* 
\wedge \cdots \wedge dz_{i_{N-1}} \wedge dz_{i_{N-1}}^*
\ee 
\be
=
(N-1)!
\left(\sum_{i} \abs{z_i}^2\right)
\sum_{1 \le i_1 < i_2 < \cdots < i_N \le N}
dz_{i_1} \wedge dz_{i_1}^* 
\wedge \cdots \wedge dz_{i_N} \wedge dz_{i_N}^*
\ee
\be
= (N-1)! r^2 dz_1 \wedge dz_1^*\wedge \cdots \wedge dz_N \wedge dz_N^*
= (N-1)! r^2 \Omega_N.
\ee

Finally I get:
\be
(F)^N = \frac{1}{(1 + r^2)^{N + 1}}
\left((1 + r^2)N! - N(N-1)!r^2\right) \Omega_N = 
\frac{N!}{(1 + r^2)^{N + 1}}\Omega_N,
\ee
\be
c_1 = \frac{N!}{(2\pi i)^N} \int \frac{1}{(1 + r^2)^{N + 1}}\Omega_N.
\ee

To calculate previous integral I set $z_i = x_{2i} +ix_{2i - 1}$, $i = 1..N$
so:
\be
   dz_i \wedge dz_i^* = 
   (dx_{2i} + idx_{2i - 1}) \wedge (dx_{2i} - idx_{2i - 1}) 
\ee
\be
   = i(-dx_{2i} \wedge dx_{2i - 1} + dx_{2i - 1} \wedge dx_{2i})
   = 2idx_{2i - 1} \wedge dx_{2i},
\ee
\be
   \Omega_N = (2i)^N dx_1 \wedge \cdots \wedge dx_{2N}, 
\ee
\be
\mbox{and: } c_1 = \frac{N!}{\pi^N}\int \frac{1}{(1 + r^2)^{N + 1}}
dx_1 \wedge \cdots \wedge dx_{2N} \equiv
\frac{N!}{\pi^N}\int \frac{1}{(1 + r^2)^{N + 1}}d^{2N}x,
\ee
\be
\mbox{with } r^2 = \sum_{i = 1}^N x_i^2.
\ee
By introducing spherical coordinates and intergrating over unit sphere 
$S^{2N-1}$ which gives $\frac{2\pi^N}{(N-1)!}$ I have:
\be
c_1 = \frac{2N!\pi^N}{(N-1)!\pi^N} 
\int_0^{\infty}\frac{1}{(1 + r^2)^{N + 1}} r^{2N-1}dr
\ee
\be
= N\int_1^{\infty}\frac{(s + 1)^{N-1}}{s^{N + 1}} ds \mbox{ with } s = r^2 + 1.
\ee
Appling Newton binomial formula:
\be
c_1 = N\sum_{j=0}^{N-1} \frac{(N-1)!}{j!(N-1-j)!}
\int_1^{\infty}s^{j-N-1}(-1)^{N-1-j}
\ee
\be
= \sum_{j=0}^{N-1} \frac{N!}{j!(N-1-j)!}
\frac{1}{j-N}s^{j-N}\bigl|_1^{\infty}(-1)^{N-1-j}
\ee
\be
= \sum_{j=0}^{N-1}\frac{N!}{j!(N-j)!}(-1)^{N-1-j}
= - \sum_{j=0}{N-1}\frac{N!}{j!(N-j)!}(-1)^{N-j}
\ee
\be
= -\left(\sum_{j=0}^{N}\frac{N!}{j!(N-j)!}1^j(-1)^{N-j} - 1\right)
\ee
\be
\mbox{ and finally: } c_1 = -((1-1)^N - 1) = 1.
\ee

This mean that for any $M$ level non-degenerated system corresponding 
abelian connection has always unit topological charge.

 
 
\subsection{Topological charge for $N=M-1$ system.}

In this case the parameter space dimension is $2N$ again and 
field tensor has the form from \rf{F_is1}:
\be
F = a \wedge a^+,
\ee
with $N\times 1$ matrix $a = (1 + xx^+)^{-1/2} dx (1 + x^+x)^{-1/2}$.

Now to calculate $N$ different invariant polynomials of $F$
I will use standard formula:
\be
det(X) = \exp (Tr\log X)
\ee   

(Of cause I should say here a lot of words how can function of differential
matrix-valued forms be defined, but because even forms are commute
and any polynomial of F is even form, it is possible to define
any function of $F$ just by corresponding Tailor sum and because 
$F^k = 0 for k > N$ by properties of differential forms such expansion
will contain only finite number of terms and any matrix property that 
can be proven only by using the expansion such as previous one will
be hold for even order differentials forms too.) 

From Tailor expansion of $\log$:
\be
Tr\log (\lambda - F) = Tr(\log\lambda 1_N) + Tr\log (1 - \frac{1}{\lambda}F)
= N\log\lambda - Tr\left(\sum_{k=1}^{\infty} \frac{1}{k}
\frac{1}{\lambda^k}F^k\right)
\ee
\be
= N\log\lambda - \left(\sum_{k=1}^{N} \frac{1}{k} 
\frac{1}{\lambda^k}Tr(F^k)\right), 
\ee
But:
\be
Tr(F^k) = Tr((a \wedge a^+)^k) = Tr(a (a^+ \wedge a)^{k - 1} a^+)
= - Tr(a^+(a^+ \wedge a)^{k - 1}a) 
\ee
\be
= - Tr((a^+ \wedge a)^k) \equiv -(a^+ \wedge a)^k
\ee
because $a^+ \wedge a$ is just an ordinary 2-form. This leads to:
\be
Tr\log (\lambda - F) 
= N\log\lambda + \left(\sum_{k=1}^{N} \frac{1}{k} 
\frac{1}{\lambda^k}(a^+ \wedge a)^k\right) 
\ee
\be
= N\log\lambda - \log\left(1 - \frac{1}{\lambda}(a^+ \wedge a)\right)
= \log\left(\lambda^N(1 - \frac{1}{\lambda}(a^+ \wedge a))^{-1}\right),
\ee
So:
\be
det(\lambda - F) = \exp(Tr\log (\lambda - F))
= \lambda^N(1 - \frac{1}{\lambda}(a^+ \wedge a))^{-1}.
\ee
Again appling Tailor expansion:
\be
(1 - \frac{1}{\lambda}(a^+ \wedge a))^{-1}
= \sum_{k=0}^{\infty}\frac{1}{\lambda^k}(a^+ \wedge a)^k
= \sum_{k=0}^{N}\frac{1}{\lambda^k}(a^+ \wedge a)^k
\ee
I have:
\be
det(\lambda - F) = \sum_{k=0}^{N}\lambda^{N-k}(a^+ \wedge a)^k,
\ee
which immediately gives that all invariant polynomials of $F$ has the form
\be
(-1)^k(a^+ \wedge a)^k = (-a^+ \wedge a)^k
\ee
and product of them which
gives 2N-form is always $(-a^+ \wedge a)^N$. This means that all Chern
number in the case $N=M-1$ are the same and given by formula:
\be
c = \frac{1}{(2\pi i)^N} \int (-a^+ \wedge a)^N.
\ee
\be
a^+ \wedge a = 
(1 + x^+x)^{-1/2}dx^+ (1 + xx^+)^{-1/2} \wedge
(1 + xx^+)^{-1/2} dx (1 + x^+x)^{-1/2}
\ee
\be
= (1 + x^+x)^{-1}dx^+\wedge (1 + xx^+)^{-1}dx
\ee
because $x^+x \equiv r^2$ is just number and using \rf{rec_two}:
\be
a^+ \wedge a = 
\frac{1}{1 + r^2} dx^+ \wedge dx 
- \frac{1}{1 + r^2} dx^+x \wedge \frac{1}{1 + r^2} x^+dx
\ee
\be
 = a^\prime_2 - a^\prime_1 \wedge a^{\prime +}_1, 
 \quad a^\prime_2 = \frac{1}{1 + r^2} dx^+ \wedge dx,
 \quad a^\prime_1 = \frac{1}{1 + r^2} dx^+x.
\ee

If now I let $x^\prime = x^+$ I have:
\be
x^\prime x^{\prime +} = x^+x = r^2,
 \quad a^\prime_2 = \frac{1}{1 + r^2} dx^\prime \wedge dx^{\prime +},
 \quad a^\prime_1 = \frac{1}{1 + r^2} dx^\prime x{\prime +}.
\ee

But this the same as expression \rf{F_for_N_one} for $F$ in $N=1$ case
if one forget about ', which immediately gives:
\be
(-a^+ \wedge a)^N = (-1)^N\frac{N!}{(1 + r^2)^{N + 1}}\Omega^\prime_N,
\ee
with 
\be
\Omega^\prime_N =
dz^\prime_1 \wedge dz^{\prime *}_1\wedge 
\cdots \wedge dz^\prime_N \wedge dz^{\prime *}_N
\ee
and taking in account:
\be
z^\prime_j = (x^\prime)_{1j} = ((x)_{j1})^* = z_j^*:
\ee
\be
\Omega^\prime_N = 
dz_1^* \wedge dz_1\wedge 
\cdots \wedge dz_N^* \wedge dz_N
= (-1)^N 
dz_1 \wedge dz_1^*\wedge 
\cdots \wedge dz_N \wedge dz_N^* = (-1)^N\Omega_N,
\ee
I have:
\be
(-a^+ \wedge a)^N = \frac{N!}{(1 + r^2)^{N + 1}}\Omega_N
\ee
or the same expression as in $N=1$ case. And using result of previous section:
\be
c_A = \frac{1}{(2\pi i)^N} \int \frac{N!}{(1 + r^2)^{N + 1}}\Omega_N = 1.
\ee
So in the $N=M-1$ case all possible Chern numbers are equal to one.




%\begin{thebibliography}{10}
%\bibitem{FirstBerryWork}
% Berry~M.~V. // Proc. Roy. Soc. 1987. V. A392. P.~45.
%\end{thebibliography}

\end{document}

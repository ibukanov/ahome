


Consider a $M$-level quantum system with hamiltonian $H$ wich has 
$N$ degenerated level with energy $E$ and let $\ket{n^i}$, $i = 1..N$ 
form a ortigonal basis in the level state space, So:
\bel{BasisDef}
\dirProd{n_i}{n^j} = \delta_i^j, \quad H\ket{n^i} = E\ket{n^i},
\quad i,j=1..N.   
\ee

Then corresponding nonabelian connection which appears during evolution
of the system has next form:
\be
(A^i_j)_\mu = \dirProd{n^i}{\pDer{n_j}{x^\mu}}.
\ee

To make following calculations more compact I introduse now 
indexless notation. Let $n$ denotes $M\times N$ matrix defined by:
\be
(n)^i_a = (\ket{n^i})_a, 
\quad i=1..N, \quad a=1..M 
\ee
where $(\ket{n})_a$ is $a$ component of a vector \ket{n}. Then \rf{BasisDef}
can be rewritten as:
\bel{BasisRedef}
n^+n=I_N, \quad H^\prime n = 0, \quad H^\prime \equiviv H - EI_M,
\ee
and by introdusing matrix valued differencial form $A$:
\be
(A)^i_j = (A^i_j)_\mu \wedge dx^\mu
\ee
I have:
\bel{ADef}
A = n^+dn.
\ee

It is easy to see that under $U(N)% transformation 
\be
n \rightarrow \tilde n =  n\omega
\ee
given by unitary $N\times N$ matrix $\omega$ identities \rf{BasisRedef} 
will be hold:
\be
\tilde n^+ \tilde n = \omega^+n^+n\omega = \omega^+I_N\omega = 
\omega^+\omega = I_N,
\ee
\be
H^\prime \tilde n = H^\prime n \omega = 0 \omega = 0.
\ee
and for $\tilde A$ I have:
\be 
\tilde A = \tilde n^+ d\tilde n = \omega^+ n^+ d(n\omega)
=  \omega^+ n^+ dn \omega + \omega^+ n^+n d\omega = 
\omega^{-1} A \omega + \omega^{-1} d\omega,
\ee
i.e. standard conection transformation law.

So to construct the particular connection one should select a particular
parametrization of $n$ and then use \rf{ADef} to calculate $A$. One possible
way to choose $n$ is based on next approach. 



The requirement the energy level should be $N$-degenerated means that 
$rank\, H^\prime = M-N$ or in other words that there is 

\be
H^prime = pmatrix{h_{11} & h_{12} \cr h_{21} & h_{22} \cr },
\quad det h_{22} \ne 0, 
\quad n =  pmatrix{\rho \cr \rho_\prime \cr}
\ee
$h_{11}$,$h_{12}$,$h_{21}$,$h_{22}$,$\rho$,$\rho_\prime$ are 
$N\times N$, $N\times (M-N)$,$(M-N)\times N$,$(M-N)\times (M-N)$,
$N\times N$, $(M-N)\times N$ matrixes.


So $H^\prime n = 0$ is equivalent to: 
\bel{eqn_for_n}
h_{21}\rho + h_{22}\rho_\prime = 0 \mbox{ or } 
\rho_\prime = - (h_{22})^{-1}h_{21}\rho
\ee
because other quations $h_{11}\rho + h_{12}\rho_\prime = 0$ due to 
$det h_{22} \ne 0$
and $rank H^\prime = M-N$ is just linear combination of \rf{eqn_for_n}. 
Now $n^+n = 1$ leads to:
\be
 1_N = \rho^+\rho + \rho_\prime^+\rho_\prime = 
 \rho^+(1_N + h_{21}^+(h_{22})^{-2}h_{21})\rho.
\ee 
So $det \rho \ne 0$ which gives:
\bel{RhoCondition}
 (\rho^+)^{-1}(\rho)^{-1} = (\rho\rho^+)^{-1} = 
 1_N + h_{21}^+(h_{22})^{-2}h_{21}.
\ee 
Last eqution is only one condition for $\rho$ and for any such $\rho$ I have
for $n$:
\be
n = \pmatrix{\rho \cr - (h_{22})^{-1}h_{21}\rho \cr} 
= \pmatrix{1_N \cr x^+ \cr} \rho,
\ee
where by definition for $N\times (M-N)$: $x^+ = - (h_{22})^{-1}h_{21}$.


Now calculations of connection $A$ strightforward:
\be
A = n^+dn = \rho^+pmatrix{1_N^+ & x \cr } 
d(pmatrix{1_N \cr x^+ \cr } \rho) 
\ee
\be
= \rho^+xdx^+\rho + \rho^+(1+xx^+)d\rho
= \rho^{-1}\rho\rho^+(xdx^+\rho + (1+xx^+)d\rho).
\ee
But 
\be
(\rho\rho^+)^{-1} = 1_N + h_{21}^+(h_{22})^{-2}h_{21} = 1 + xx^+,
\ee
this leads to:
\be
A = \rho^{-1}(1 + xx^+)^{-1}(xdx^+\rho + (1+xx^+)d\rho)
 = \rho^{-1}(1 + xx^+)^{-1}xdx^+\rho + \rho^{-1}d\rho
\ee

Last formula looks like transformed by matrix $\rho$ connectin:
\be
\tilde A = (1 + xx^+)^{-1}xdx^+, 
\ee
but because in general $\rho\rho^+ = (1 + xx^+)^{-1} \ne 1$ this is not a 
$U(N)$ transformation. Neverthless this representation is useful in calculation
of $F$ because it will look as transformed by $\rho$ $\tilda F$. Proof is
the same as for ordinary $U(N)$ transformation :
\be
F = dA + A \wedge A = d(\rho^{-1}\tilda A \rho + \rho^{-1}d\rho) +
\rho^{-1}(\tilda A \rho + d\rho) \wedge \rho^{-1}(\tilda A \rho + d\rho)
\ee
\be
= -\rho^{-1}d\rho\rho^{-1}\wedge\tilda A \rho + \rho^{-1}d\tilda A\rho
- \rho^{-1}\tilda A \wedge d\rho - \rho^{-1}d\rho\wedge\rho^{-1}d\rho
\ee
\be
+ \rho^{-1}\tilda A\wedge\tilda A\rho + \rho^{-1}\tilda A d\rho
+ \rho^{-1}d\rho\wedge\rho^{-1}\tilda A \rho 
+ \rho^{-1}d\rho\wedge\rho^{-1}d\rho
\ee
= \rho^{-1}(d\tilda A + \tilda A\wedge\tilda A)\rho = \rho^{-1}\tilda F\rho,
\ee
where next identity is used: $d\rho^{-1} = -\rho^{-1}d\rho\rho^{-1}$ and
$d(\omega_P\wedge\omega_Q) = d\omega_P \wedge \omega_Q 
+(-1)^P \omega_P\wedge\omega_Q$ for any differencial forms of brank $P$ and $Q$.

For $\tilda F$ I have:
\be
\tilda F = d((1 + xx^+)^{-1}xdx^+) + 
(1 + xx^+)^{-1}xdx^+\wedge (1 + xx^+)^{-1}xdx^+
\ee
\be
= -(1 + xx^+)^{-1}d(1 + xx^+)(1 + xx^+)^{-1}\wedge xdx^+
+ (1 + xx^+)^{-1}dx\wedge dx^+ 
+(1 + xx^+)^{-1}xdx^+\wedge (1 + xx^+)^{-1}xdx^+
\ee
\be
= (1 + xx^+)^{-1}\left(
(-d(xx^+) + xdx^+)(1 + xx^+)^{-1}x + dx\right)\wedge dx^+
\ee
\be
= (1 + xx^+)^{-1}dx\left(-x^+(1 + xx^+)^{-1}x + 1\right)\wedge dx^+,
\ee
But $(1 + xx^+)^{-1}x = x(1 + x^+x)^{-1}$  due to:
\be
(1 + xx^+)^{-1}x - x(1 + x^+x)^{-1} = 
(1 + xx^+)^{-1}\left(x(1 + x^+x) - (1 + xx^+)x\right)(1 + x^+x)^{-1} = 0.
\ee
So:
\be
\tilda F = (1 + xx^+)^{-1}dx\left(-x^+x(1 + x^+x)^{-1} + 1\right)\wedge dx^+
\ee
\be
= (1 + xx^+)^{-1}dx\left(-x^+x + 1 + x^+x\right)(1 + x^+x)^{-1}\wedge dx^+,
\ee
or:
\be
\tilda F = (1 + xx^+)^{-1}dx \wedge (1 + x^+x)^{-1}dx^+.
\ee


One possible way to fix $\rho$ is to require $\rho$ to be hermition matrix or
$\rho = \rho^+$. This gives from \rf{RhoCondition}:
\be
\rho^{-2} = 1 + h_{21}^+(h_{22})^{-2}h_{21}
\ee
or;
\be
\rho = (1 + h_{21}^+(h_{22})^{-2}h_{21})^{-\frac{1}{2}} 
= (1 + xx^+)^{-\frac{1}{2}}
\ee



Topological charges.

Direct calculation of 



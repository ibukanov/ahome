\documentclass[a4paper]{article}
\usepackage[utf8]{inputenc} % Input encoding
\usepackage[T2A]{fontenc} % Output encoding
\usepackage[russian]{babel} % Russian words for Chapter, etc

% Original was: \documentstyle[14pt,russian]{artr}

\def\baselinestretch{1.4}
\textheight 210mm
%\textwidth 150mm
\topmargin 20.0mm
%\hoffset 10.0mm

\def\Hhat{ \ifmmode {\hat H} \else {$\hat H$} \fi }
\def\Hhatf#1{ \ifmmode {\hat H(#1)} \else {$\hat H(#1)$} \fi}

\def\HZhat{ \ifmmode {\hat H_0} \else {$\hat H_0$} \fi}
\def\HZtilde{ \ifmmode {\tilde H_0} \else {$\tilde H_0$} \fi}

\def\Vhat{ \ifmmode {\hat V} \else {$\hat V$} \fi}
\def\Vform{ \ifmmode {\Vhat^{-1}d\Vhat} \else {$\Vhat^{-1}d\Vhat$} \fi}

\def\Jg#1#2{ \ifmmode {J^{(#1)}_{#2}} \else {$J^{(#1)}_{#2}$} \fi}

\def\der#1{\partial_#1}

\def\lam#1{ \ifmmode {\lambda_#1} \else {$\lambda_#1$} \fi}

\def\cosT#1{\cos \theta_{#1}}

\def\KetVect#1{|#1\!\!>}
\def\BraVect#1{<\!\!#1|}
\def\DirProd#1#2{<\!\!#1|#2\!\!>}

\def\mod#1{\mid\!\!#1\!\!\mid}

%\pagestyle{empty}

\begin{document}

\begin{center}

{\large\bf Фаза Берри для невырожденной трехуровневой квантовой системы. }

Буканов~И.В., Толкачев~Е.А, Трегубович~А.Я.

{\it Институт физики им. Б.И.~Степанова АН Беларуси}

\end{center}

\begin{abstract}
   Построены замкнутые выражения, позволяющие вычислить фазу Берри,
возникающую при динамической эволюции параметров невырожденной
трехуровневой квантовой системы общего вида.
\end{abstract}

Как известно, в последние годы уделяется значительное внимание исследованию
геометро - топологических зффектов в квантовых системах. К их числу
относится, в частности, комплекс наблюдаемых и предсказываемых явлений
под общим названием фаза Берри~\cite{FirstBerryWork,AllBerryPhase}.
Универсальная геометрическая природа этой фазы позволяет с единых позиций
интерпретировать такие, казалось бы, разнородные явления, как нейтринные
осциляции~\cite{NaumovWork} и некоторые закономерности в молекулярных
спектрах~\cite{AboutPhaseInSpectra}. Экспериментальное наблюдение фазы Берри
в нейтронных и оптических экспериментах породило определенные надежды на
возможность ее практического использования.

Однако вскоре выяснилось, что конкретные вычисления данного эффекта можно
провести лишь для ряда простых квантовых систем, описываемых квадратичным
гамильтонианом~\cite{About2Hamiltonian} или обладающих 3-х параметрической
группой динамической симметрии~\cite{FirstBerryWork,AboutExactSolution}.
Попытки в рамках стандартного подхода
получить результат, например, для трехуровневой системы, удались лишь
ценой существенных ограничений на вид соответствующего
гамильтониана ~\cite{CeulemansWork,KorenblitWork}.

В данной работе на примере трехуровневой системы общего вида развит новый
метод вычисления геометрической фазы для квантовых систем, описываемых
гамильтонианами, которые реализуют некоторое представление алгебры Ли
группы динамической симметрии.

Рассмотрим квантовую систему с гамильтонианом \Hhatf{x}, $x\in \Omega$,
где $\Omega $ --- $N$ мерное пространство параметров, и пусть параметры
изменяются во времени, описывая в $\Omega$ гладкий замкнутый контур $C$,
так что $x=x(t)$ и для некоторых $t_0$, $t_1$: $x(t_0)=x(t_1)$.

  Тогда эволюция состояния $\KetVect{\psi}$ системы будет описываться 
временным
уравнением Шредингера:
$$
   i\hbar \der{t}\KetVect{\psi} = \Hhatf{x(t)}\KetVect{\psi}
$$

Если теперь состояние $\KetVect{\psi}$ системы в момент $t_0$
удовлетворяет условию:
$$
\KetVect{\psi} = \KetVect{n(x(t_0))}, \quad \mbox{ где }
  \Hhatf{x}\KetVect{n(x)}=E_n\KetVect{n(x)},
$$
причем для всех $x \in C$ зависимость $E_n$, $\KetVect{n}$ от $x$ достаточно
гладкая и уровень энергии с номером $n$ невырожден, то в случае применимости
адиабатического приближения можно считать~\cite{AboutAdiabat}:
\begin{equation}
  \KetVect{\psi(t_1)} \approx \exp(i\alpha_n(t_1)) \KetVect{n(x(t_1))},
\end{equation}
причем фазовый множитель $\alpha_n(t)$
дается выражением \cite{FirstBerryWork}:
$$
\alpha_n(t_1) = - \hbar^{-1} \int_{t_0}^{t_1}E_n(x(t)) dt + \gamma_n(t_1),
$$
где $\gamma_n(t_1)$ -- геометрическая фаза:
\begin{equation}
\gamma_n(t_1) = \int_{t_0}^{t_1} \dot \gamma_n(t) dt
   = i \oint_{C}\DirProd{n}{\der{\mu}n} dx^{\mu}
  \label{GammaPhase}
\end{equation}

Перепишем теперь ~ (\ref{GammaPhase}) в терминах унитарного преобразования,
диагонализирующего \Hhat. В силу эрмитовости \Hhat такое
преобразование всегда существует и по определению:
\begin{equation}
\Hhatf{t} = \Vhat(t) \HZhat(t) \Vhat^{-1}(t),
\quad \Vhat^{-1} = \Vhat^{+},
\quad \HZhat = diag\{E_1(t),\ldots,E_n(t),\ldots\} \label{HamiltDecomp}
\end{equation}

Тогда:
\begin{equation}
\KetVect{n} = \Vhat\KetVect{n_0},
\quad \HZhat\KetVect{n_0} = E_n\KetVect{n_0}
  \label{SelfVector}
\end{equation}
Откуда для $\dot \gamma(t)$ получаем:
\begin{equation}
\dot \gamma(t)dt
  = i\BraVect{n_0}\Vhat^{+} \der{\mu}\Vhat\KetVect{n_0} dx^{\mu}
  = i(\Vform)_{nn},
  \label{FullGammaPhase}
\end{equation}
где введена матричнозначная дифференциальная форма
$d\Vhat = \der{\mu}\Vhat dx^{\mu}$

Применим теперь ~ (\ref{FullGammaPhase}) к вычислению геометрической фазы в
трехуровневой системе.
Прежде всего заметим, что данная система обладает группой
динамической симметрии $SU(3)$, т.к. ее гамильтониан является
произвольной эрмитовой $3 \times 3$ матрицей c нулевым следом и,
следовательно, $\Hhat \in su(3)$, где $su(3)$ --- алгебра группы $SU(3)$.
Тогда $\Hhat$ можно представить в виде:
\begin{equation}
\Hhat = \Vhat \HZtilde \Vhat^{-1},
\label{HamiltRepresent}
\end{equation}
где $\Vhat \in SU(3)$,
\begin{equation}
  \Vhat = V_1 V_2 V_3,
  \label{VRepresent}
\end{equation}
$V_i = \exp(\alpha_i\Jg{i}{+} - \alpha_i^{\ast}\Jg{i}{-})$,
$\alpha_i$  ---  3 комплексных параметра,
$$
\HZtilde = \sum_{i=1}^{3} E_{(i)}\Jg{i}{3},
$$
$\Jg{i = 1, 2, 3}{3, \pm}$ --- генераторы
$su(3)$ в базисе Картана-Вейля, причем $\Jg{1}{3}+\Jg{2}{3}+\Jg{2}{3} = 0$
и выполняются коммутационные соотношения:
$$
[\Jg{i}{\pm}, \Jg{j}{\mp} ] = [ \Jg{i}{3},\Jg{j}{3} ] = 0,
  i \not= j; \quad
[ \Jg{i}{+}, \Jg{i}{-} ] = \Jg{i}{3}; \quad
[ \Jg{i}{3}, \Jg{i}{\pm} ] = \pm 2\Jg{i}{\pm}; \quad
$$
$$
[ \Jg{i}{3}, \Jg{j}{\pm} ] = \mp \Jg{j}{\pm}, i \not= j; \quad
[ \Jg{i}{\pm}, \Jg{j}{\pm} ] = \mp \varepsilon_{ijk}\Jg{k}{\pm};  \quad
$$
%     , где индексы 1,2,3 можно циклически    переставлять;

  В случае фундаментального представления, которое и
реализуется в случае трехуровневой системы, $\Jg{i}{3, \pm}$
можно выразить через матрицы Гелл - Манна:
$$
\Jg{1}{\pm} = (\lam{1} \pm i\lam{2}) / 2; \quad
\Jg{2}{\pm} = (\lam{4} \mp i\lam{5}) / 2; \quad
\Jg{3}{\pm} = (\lam{6} \pm i\lam{7}) / 2; \quad
$$
$$
\Jg{1}{3} = \lam{3}; \quad
\Jg{2}{3} = (-\lam{3} - \sqrt{3}\lam{8}) / 2; \quad
\Jg{3}{3} = (-\lam{3} + \sqrt{3}\lam{8}) / 2; \quad
$$
  Поэтому в рассматриваемом представлении:
$$
(\Jg{i}{\pm})_{nn} = 0; \quad (\Jg{i}{3})_{nm} = 0, \quad n \not= m
$$
Откуда: $ \HZtilde = diag\{E_1,E_2,E_3\}$, где $E_k$ --- уровни
энергии системы.
Тем сасмым показано, что разложение ~ (\ref{HamiltRepresent})
имеет вид ~ (\ref{HamiltDecomp}).

Вычислим теперь $\dot \gamma_1 dt$ в явном виде
в рассматриваемой параметризации $\Hhat$. Для этого запишем:
$$
V_i = \exp(\alpha_i\Jg{i}{+} - \alpha_i^{\ast}\Jg{i}{-}) =
\exp(\theta_i(e^{i\phi_i}\Jg{i}{+} - e^{-i\phi_i}\Jg{i}{-})), \quad
\alpha_i = \theta_i e^{i\phi_i}
$$
   Тогда, перемножая $V_i$, получим для первого столбца
$\Vhat$ $=$ $V_1V_2V_3$ в фундаментальном представлении:
\begin{equation}
    V_{11} = \cosT{1}\cosT{2}, \quad
    V_{21} = -\cosT{2} \sin \theta_1 e^{-i\phi_1}, \quad
    V_{31} = \sin \theta_2 e^{i\phi_2}
  \label{VMatrixExpression}
\end{equation}

Откуда:
\begin{equation}
  \dot \gamma_1 dt
   = i (\Vform)_{11}
   = - Im(\sum_{i=1}^{3} V^\ast_{i1} dV_{i1})
   = \sin^2\theta_1 \cos^2\theta_2 d\phi_1 -
                     \sin^2\theta_2 d\phi_2
  \label{FullFormExpression}
\end{equation}

В принципе, ~ (\ref{FullFormExpression}) и дает выражение для
геометрической фазы одного из уровней, необходимо лишь по заданной
зависимости \Vhat от времени проинтегрировать форму $\dot \gamma_1 dt$.
Однако в реальных
задачах обычно бывают известны только матричные элементы
гамильтониана и для того, чтобы применять ~ (\ref{FullFormExpression}),
необходимо найти зависимость $\theta_{1,2}$ , $\phi_{1,2}$ от $H_{ij}$.

Для этого представим~(\ref{HamiltDecomp}) в виде
$HV = VH_0$ или:
$$
  H_{i1}V_{11} + H_{i2}V_{21} + H_{i3}V_{31} = V_{ij}(H_0)_{11} = 
V_{i1}E_{1},
    \quad i = 1..3
$$
Откуда после простых преобразований получим:
$$
{V_{21} \over V_{11}} =
    {H_{23}H_{31} - H_{21} \Delta_3 \over \Delta_2 \Delta_3 - 
\mod{H_{23}}^2},
  \qquad
{V_{31} \over V_{11}} =
    {H_{32}H_{21} - H_{31} \Delta_2 \over \Delta_2 \Delta_3 - 
\mod{H_{23}}^2},
$$
где по определению: $\Delta_i = H_{ii} - E_1, \quad i = 2..3 $

Но из явного вида $V_{i1}$ ~ (\ref{VMatrixExpression}) следует:
$$
V_{21} / V_{11} = - tg \theta_1 e^{-i\phi_1}, \quad
V_{31} / V_{11} \cosT{1} = tg \theta_2 e^{i\phi_2}, \quad
$$
или:
\begin{equation}
- tg \theta_1 e^{-i\phi_1} =
    {H_{23}H_{31} - H_{21} \Delta_3 \over \Delta_2 \Delta_3 - 
\mod{H_{23}}^2},
\quad
tg \theta_2 e^{i\phi_2} =
    {H_{32}H_{21} - H_{31} \Delta_2 \over \Delta_2 \Delta_3 - 
\mod{H_{23}}^2}
\label{ThetaPhiFromH}
\end{equation}

Это позволяет выразить $\theta_{1,2}$ , $\phi_{1,2}$
через $H_{ij}$~, что и дает искомую
связь параметров $V_{i1}$ с матричными элементами гамильтониана,
при этом входящее в ~ (\ref{ThetaPhiFromH}) значение $E_1$
может быть найдено путем решения следующего кубического уравнения на
собственные значения для $H$~, например, с помощью формул Кардано:
\begin{equation}
    E^3 + Sp{\overline H}E - DetH = 0,
\label{SelfEquation}
\end{equation}
где ${\overline H}$ -- взаимная матрица~\cite{AboutMatrix},
$$
    Sp{\overline H} = - (H_{22})^2 - (H_{33})^2 - H_{22}H_{33}
     - \mod{H_{23}}^2 - \mod{H_{31}}^2 - \mod{H_{12}}^2,
$$
$$
    DetH = (H_{22} + H_{33})(\mod{H_{23}}^2 - H_{22}H_{33})
$$
$$
    - H_{22}\mod{H_{31}}^2 - H_{33}\mod{H_{12}}^2 + 2Re(H_{12}H_{23}H_{31})
$$

Здесь следует сделать два замечания. Во первых, в качестве $E_1$
может быть выбрано любое из решений (\ref{SelfEquation}),
т. к. стоящие на диагонале в $H_0$ значения
$E_i$ могут быть упорядочены произвольным образом, и, следовательно,
полученные формулы для геометрической фазы применимы для любого из уровней
3 уровневой системы. Во вторых, (\ref{FullFormExpression}),
(\ref{ThetaPhiFromH}) не применимы при условии:
\begin{equation}
  \Delta_2 \Delta_3 = \mod{H_{23}}^2
%\label{BadCondition}
\end{equation}
Однако это не ограничивает общность полученных результатов,
т. к. если гамильтониан системы не вырожден, то обращение в ноль
знаменателя в (\ref{ThetaPhiFromH}) является чисто координатной особенностью
и связано с конкретным выбором параметризации \Vhat~(\ref{VRepresent}).

В качестве примера использования полученных результатов рассмотрим систему,
в которой изменяются только диагональные элементы
гамильтониана~\cite{NaumovWork}. Тогда, беря в качестве независимых
параметров $H_{22}$, $H_{33}$,
подставляя (\ref{ThetaPhiFromH}) в (\ref{FullFormExpression}) и учитывая
$$
   dH_{12} = dH_{23} = dH_{31} = 0, \qquad H_{11} = - H_{22} - H_{33},
$$
получаем:
$$
  \dot \gamma_1 dt = Im(H_{12}H_{23}H_{31}) {d(H_{22} - H_{33}) \over L},
$$
   где
$$
    L = (\Delta_2 \Delta_3 - \mod{H_{23}}^2)^2
     + \mod{H_{23}H_{31} - H_{12}^{\ast} \Delta_3}^2
     + \mod{H_{12}^{\ast}H_{23}^{\ast} - H_{31}\Delta_2}^2,
$$
$$
    \Delta_{i} = \Delta_{i}(t) = H_{ii}(t) - E(H_{22}(t), H_{33}(t)), 
\qquad i = 2, 3
$$
   Откуда, считая, что $\dot \gamma_1dt$ является формой потенциала $A$
  некоторого векторного поля на плоскости ($H_{22}$, $H_{33}$), для тензора
  напряженности $F=dA$ имеем:
\begin{equation}
   F =
    6 Im(H_{12}H_{23}H_{31}) { \Delta_2 \Delta_3 - \mod{H_{23}} \over 
L^2 } E
  (1-{ \partial E \over \partial H_{22} }-{ \partial E \over \partial 
H_{33} })
   dH_{33} \wedge dH_{22}
  \label{FOnPlate}
\end{equation}
   Из (\ref{FOnPlate}) следует, что в общем случае произвольной зависимости
$H_{22}$, $H_{33}$ от времени $F \ne 0$, и в данной
системе может наблюдаться топологическая фаза, в отличии от утверждения,
содержащегося в работе~\cite{NaumovWork}.
%Это можно явно показать в случае

%Заключение.

Примененный в данной работе метод, в принципе, позволяет найти 
геометрическую
фазу в случае
произвольной $SU(N)$ системы, однако, т.к. в окончательные формулы,
как и в $SU(3)$ случае, будут входить собственные значения $E$
гамильтониана, то в силу ограничений, накладываемых теоремой Абеля, явные
аналитические выражения в общем случае можно будет получить
только при $N \le 4$.

\begin{thebibliography}{10}

\bibitem{FirstBerryWork}
  Berry~M.~V. // Proc. Roy. Soc. 1987. V. A392. P.~45.
\bibitem{AllBerryPhase}
  Shapere~A., Wilczek~F. Geometric Phases in Physics. World Scientific,
  Singapore, 1989.
\bibitem{NaumovWork}
  Наумов~В.~А. // ЖЭТФ. 1992. Т. 101. С.~3.
\bibitem{AboutPhaseInSpectra}
  Виницкий~С.И., Дербов~В.Л., Дубовик~В.М., Марковски~Б.Л., 
Степановский~Ю.П.
  // УФН. 1990. Т. 160. N6. С.~1.
\bibitem{About2Hamiltonian}
  Dodonov~V.V., Man'ko~V.I. in Topological Phases in: Quantum Theory.
  Eds. Markovski~B., Vinitsky~S.I. World Scientific, Singapore, 1989. P.~74.
\bibitem{AboutExactSolution}
  Chaturvedi~S., Sriram~M., Shrinivasan~V. // J.Phys. 1987. V. A20. P.~L1071
\bibitem{CeulemansWork}
  Ceulemans~A., Szopa~M. // J.Phys. 1991. V. A24. P.~4495.
\bibitem{KorenblitWork}
  Korenblit~S.E., Kuznetsov~V.E., Naumov~V.A. in: Proc. Int. Conf.
  Quantum systems : New trends and methods. Minsk, 1994.
  Eds. Barut~A.O. et al. World Scientific, Singapure, 1995. P.~208.
\bibitem{AboutAdiabat}
  Messiah~A. Quantum Mechanics. North-Holland, Amsterdam, 1970. V.~2.
\bibitem{AboutMatrix}
  Гантмахер Ф. Р. Теория матриц. Москва, 1988.
\end{thebibliography}

\pagebreak

\begin{center}

{\large\bf Berry phase for a nondegenerated three-level quantum system. }

Bukanov~I.V., Tolkachev~E.A, Tregubovich~A.Ya.

{\it Stepanov~B.I. Institute of Physics, Academy of Sciences of Belarus.}

\end{center}

\begin{center}
{\bf Abstract. }
\end{center}

\begin{center}
  Closed expressions have been built enabling to calculate Berry phase
  appearing during dynamic evolution of parameters
  of a general-type nondegenerated three-level quantum system.

%  Построены замкнутые выражения, позволяющие вычислить фазу Берри,
%возникающую при динамической эволюции параметров невырожденной
%трехуровневой квантовой сиситемы общего вида.
\end{center}

\end{document}



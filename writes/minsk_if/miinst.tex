\documentclass[a4paper]{article}
\usepackage[utf8]{inputenc} % Input encoding
\usepackage[T2A]{fontenc} % Output encoding
\usepackage[russian]{babel} % Russian words for Chapter, etc

% Original was: \documentstyle[14pt,russian]{artr}

\def\baselinestretch{1.4}
\textheight 210mm
%\textwidth 150mm
\topmargin 20.0mm
%\hoffset 10.0mm


\def\MAP#1{\ifmmode{#1}\else{$#1$}\fi}
\def\abs#1{\MAP{\mid\!\!#1\!\!\mid}}
\def\ket#1{\MAP{\mid\!\!#1\!\!>}}
\def\bra#1{\MAP{<\!\!#1\!\!\mid}}
\def\dirProd#1#2{\MAP{<\!\! #1 \! \mid \! #2 \!\!>}}
\def\vct#1{\MAP{\vec #1}}
\def\pDer#1#2{\MAP{\frac{\partial #1}{\partial #2}}}
\def\oDer#1#2{\MAP{\frac{d#1}{d#2}}}

\def\be{\begin{equation} \nonumber}
\def\ee{\end{equation}}
\def\bel#1{\begin{equation}\label{#1}}
\def\rf#1{(\ref{#1})}

\def\H{\MAP{\hat H}}
\def\X{\MAP{\vct x}}
\def\E{\MAP{E\,}}


%\pagestyle{empty}

\begin{document}

\begin{center}

{\large\bf Фаза Берри и параметрический инстантон в динамике вырожденной
трехуровневой системы. }

Буканов~И.В., Толкачев~Е.А., Трегубович~А.Я.

{\it Институт физики им. Б.И.~Степанова АН Беларуси}

\end{center}

\begin{abstract}
  Построены замкнутые выражения, позволяющие вычислить фазу Берри,
возникающую при динамической эволюции параметров вырожденной
трехуровневой квантовой системы и показано, что соответствующий
неабелев потенциал в пространстве параметров несет единичный
топологический заряд и имеет ассимтоматику обычного инстантонна.
\end{abstract}


Как известно, при циклической адиабатической эволюции параметров
гамильтониана квазистационарной квантовой системы соответствующие
собственные векторы приобретают за период наряду с динамической
дополнительную фазy~\cite{FirstBerryWork,AllBerryPhase},
имеющую прозрачный геометрический смысл~\cite{SimonWork}
Этот эффект, носящий название фаза Берри, полностью
определяется топологией пространства параметров системы и имеет надежное
экспериментальное подтверждение~\cite{BerryPhaseExperimentalTest}.
Конкретные расчеты, однако, проделаны
лишь для ряда простых систем~\cite{WorkAboutSimpleSystem},
когда пространство параметров является
двумерной сферой. При этом естественно фаза Берри выражается через
циркуляцию по замкнутому контуру потенциалов-связностей монопольного типа:
абелева для невырожденных уровней~\cite{FirstBerryWork}
и неабелева в случае
вырождения~\cite{WilczekZeeWork}.

В тоже время большой интерес вызывают более сложные, в частности
многоуровневые, квантовые системы, имеющие широкие физические приложения.
Здесь пока получены отдельные
результаты~\cite{CeulemansWork,KorenblitWork}.
Недавно~\cite{PreviousWork} нами был
предложен метод расчета геометрической фазы для невырожденной трехуровневой
системы, уточняющий и обобщающий известные резуьтаты.

В настоящей работе расчитана фаза Берри для вырожденной трехуровневой
системы и установлена ее связь с SU(2) инстантонным потенциалом. Таким
образом многоуровневые системы с изменяющимися параметрами являются удобным
инструментом для наблюдения эффектов инстантонных связностей.


Рассмотрим квантовую систему с гамильтонианом $\H(\X)$ и заданной зависимостью
внешних параметров \X = $\{x_1, \ldots, x_d\}$ от времени, описывающей в
пространстве параметров кривую $C$:
\be
\X = \X(t),
\ee
и пусть в начальный
момент времени система находится в состоянии \ket{\psi(t_0)},
соответствующем уровню с энергией \E и
степенью вырождения $N$, так что:
\be
\H(\X(t_0))\ket{\psi(t_0)} = \E(\X(t_0))\ket{\psi(t_0)}.
\ee

Тогда в случае применимости адиабатического приближения можно считать,
что в последующие моменты времени система будет оставаться на данном
уровне, при этом конкретная зависимость \ket{\psi(t)} дается выражением:
\be
\ket{\psi(t)} =
\exp\left(-i\int_{t_0}^{t} E(t_1) d\,t_1\right) \ket{\phi(t)}.
\ee
Первый сомножитель в данном выражении представляет собой стандартную
фазу адиабатического приближения~\cite{AboutAdiabat},
отражающую динамические
свойства системы, а в обшем случае нетривиальная зависимость \ket{\phi}
от времени, связаная с геометрией системы и носящая имя геометрической
фазы или фазы Берри, может быть представлена в виде:
\be
\ket{\phi(t)} = b^i(t)\ket{n_i(\X(t))},
\ee
где по повторяющимся латинским индексам подразумевается суммирование от
1 до $N$, \ket{n_i} образуют базис в пространстве состояний с энергией
\E:
\bel{defBasis}
\dirProd{n^i}{n_j} = \delta^i_j, \quad \mbox{и} \quad
\H(\X)\ket{n_i} = \E(\X)\ket{n_i}, \quad i, j = 1\ldots N,
\ee
а коэфиценты $b^i$ определяются из решения системы уравнений:
\bel{cEqn}
\oDer{b^i}{t} + A^i_{\mu j} \oDer{x^\mu}{t}b^j = 0
\ee
с начальными условиями, следующими из разложения \ket{\psi} по базису
\ket{n_i} в момент времени $t_0$:
\be
\ket{\psi(t_0)} = \ket{\phi(t_0)} = b^i(\X(t_0))\ket{n_i(\X(t_0))}.
\ee
В \rf{cEqn} по повторяющимся греческим индексам идет суммирование от
1 до $d$ - размерности пространства параметров и по определению:
\be
A^i_{\mu j} = \dirProd{n^i}{\pDer{n_j}{x^\mu}}.
\ee
или вводя матричнозначную дифференциальную форму
$(A)^i_j = A^i_{\mu j}dx^\mu$:
\bel{ADef}
(A)^i_j = \dirProd{n^i}{dn_j}.
\ee

Решение уравнений \rf{cEqn} зависит только от формы кривой $C$
и с геометрической точки зрения решение задает
паралельный перенос в расслоенном пространстве~\cite{Postnikov}, базой которого
является пространство параметров системы, слоем -- состояния с энергией
\E, структурной группой -- $U(N)$ -- группа преобразований
базиса, сохраняющая условия~\rf{defBasis},
а коэфиценты $A$ задают связность или
калибровочный потенциал
данного расслоения. При этом если соответствующий тензор наряженности
$F$:
\be
F = dA + A \wedge A
\ee
не равен нулю, то никаким выбором базиса \rf{defBasis}
нельзя обратить $A$ в ноль тождественно.

Решение~\rf{cEqn} может быть представлено в виде:
\bel{formalSolution}
b_i(t) = \left({\cal P}\exp(\int_C A_\mu d\!x^\mu)\right)^j_i b_j(t_0),
\ee
где ${\cal P}$ обозначает $P$-упорядоченную экспоненту и интегрирование
ведется вдоль $C$ между точками $\X(0)$ и $\X(t)$. Однако это чисто
формальный результат и в практических расчетах проще непосредственно
решать~\rf{cEqn}. Тем не менее в случае отсутствия вырождения $N = 1$
в~\rf{formalSolution} $P$-упорядоченние снимается, что дает стандартную
формулу для абелевой фазы:
\be
b_1 = \exp(\int_C A_\mu d\!x^\mu)b_1(t_0),
\ee

Таким образом, для вычисления $A$ необходимо выбрать параметризацию
базиса \rf{defBasis}. В случае трехуровневой системы с вырождением
это можно сделать следующим образом. Легко видеть, что гамильтониан
любой такой системы может быть представлен в виде:
\bel{hIs}
\H = EI_3 + (E_3 - E)\ket{n_3}\bra{n_3},
\ee
где $I_3$ - единичная 3x3 матрица и \ket{n_3} - комплексный трехмерный вектор,
нормированный на единицу:
\[
\dirProd{n_3}{n_3} = 1.
\]

Учитывая произвол в фазе \ket{n_3}, получаем, что гамильтониан \rf{hIs}
зависит от четырех рараметров. По построению вектор является собственным
вектором \H с собственным значением $E_3$. Другие два вектора, отвечающие
двукратно вырожденному уровню с энергией $E$, находятся из уравнений:
\bel{nsEqn}
\dirProd{n_3}{n_j}  = 0, \quad \dirProd{n_i}{n_j} = \delta_{ij},
\quad i, j  = 1\cdots 2,
\ee
которые инвариантны относительно преобразований из группы $U(2)$:
\be
\ket{n_i} = U_{ij}\ket{n_j}, \quad UU^+ = 1
\ee

Таким образом, вектор \ket{n_3} и соответственно гамильтониан~\rf{hIs}
можно параметризовать тремя комплексными параметрами:
\be
\ket{n_3} = (\xi_1, \xi_2, \xi_3)^T,
\ee
где $(\dots)^T$ -- операция транспонирования,
подчиняющиеся двум условиям:
\bel{xiCondition}
\abs{\xi_1}^2 + \abs{\xi_2}^2 + \abs{\xi_2}^2 = 1, \quad Arg(\xi_3) = 0.
\ee

Следовательно, в случае вырожденной трехуровневой системы
пространство параметров гамильтониана есть $CP(2)$~\cite{AboutCpTwoSpace}.

Естественно ожидать, что в расслоении с базой $CP(2)$ и структурной группой
$U(2) \approx SU(2) \otimes U(1)$ в качестве связностей, определяющих
геометрическую фазу, возникнут выражения инстантонного типа.

Чтобы убедиться в этом, введем с помощью стереографической проекции
в $CP(2)$ новые координаты $\zeta(\zeta_1, \zeta_2)$:
\bel{stereoProject}
 \xi_i = \frac{2\zeta_i}{1 + r^2}, i = 1,2,
\quad \xi_3 = \frac{r^2 - 1}{\abs{r^2 + 1}},
\ee
причем в силу \rf{xiCondition}:
\bel{rCondition}
r^2 \equiv \abs{\zeta_1}^2 + \abs{\zeta_2}^2 \ge 1.
\ee

Здесь следует заметить, что наличие условий \rf{rCondition} отражает тот факт,
что в данном случае строится стереографическая проекция не сферы $S^4$,
а топологически не эквивалентного ей пространства $CP(2)$.

В данной параметризации уравнения \rf{nsEqn} легко решаются
и полная система собственных векторов \H имеет вид:
\bel{nsIs}
\ket{n_1} = \frac{1}{r}(\bar \zeta_2, - \bar \zeta_1, 0)^T,
\ee
\be
\ket{n_2} = \frac{r^2 - 1}{r(r^2 +1)}(\zeta_1, \zeta_2, \frac{2r^2}{1 - r^2})^T,
\ee
\be
\ket{n_3} = \frac{2}{r^2 +1}(\zeta_1, \zeta_2, \frac{r^2 - 1}{2})^T.
\ee


Подставляя \rf{nsIs} в \rf{ADef}, получаем, что связность,
определяющая перенос вектора \ket{n_3} имеет вид:
\be
A_{ab} = \dirProd{n_3}{dn_3} = -\frac{4}{(1 + r^2)^2}\omega_0,
\ee
\be
\omega_0 = d\bar\zeta_1 \zeta_1 + d\bar\zeta_2 \zeta_2
- \bar\zeta_1 d\zeta_1 - \bar\zeta_2 d\zeta_2
\ee
а неабелева связность, соответствующая переносу пары
$(\ket{n_1}, \ket{n_2})$, может быть представлена в виде $2 \time 2$
матрицы:
\be
 A = \left( \begin{array}{cc}
 A_{11} & A_{12} \\
 -A_{12}^* & A_{22} \end{array} \right),
\ee
\bel{AIs}
A_{11} = \frac{1}{2r^2}\omega_0, \quad
A_{12} = \frac{r^2 - 1}{r^2(r^2 + 1)}\omega_1, \quad
A_{22} = - \frac{1}{2r^2}
\left( \frac{1 - r^2}{1 + r^2} \right)^2\omega_0,
\ee
\be
\omega_1 = \zeta_2 d\zeta_1 - \zeta_1 d\zeta_2.
\ee
 При этом $Tr(A) = -A_{ab}$ в силу:
\be
0 = \dirProd{n_1}{dn_1} + \dirProd{n_2}{dn_2} + \dirProd{n_3}{dn_3} \equiv
Tr(A) + A_{ab}.
\ee

В выражении для $A$ легко выделяется инстантонная часть
$A^\prime$~\cite{InstantonExpression}:
\be
A = A^\prime + A^{\prime\!\prime},
\ee
где
\be
A^\prime = \frac{1}{1 + r^2}\left(\begin{array}{cc}
 \omega_0/2 & \omega_1 \\
 -\omega_1^* & -\omega_0/2 \end{array} \right),
\ee
а добавочные член $A^{\prime\!\prime}$
убывает на бесконечности быстрее, чем $r^{-2}$ и поэтому при
$r \rightarrow \infty$ $A$ имеет ту же ассимтотику,
что и обычный инстантон. Инстантонный характер данной связности
подтверждается и прямым расчетом топологического заряда, который
в силу $Tr(A) \ne 0$ необходимо расчитывать по
формуле~\cite{Postnikov}:
\be
k = \frac{1}{4\pi} \int_{CP(2)} det(F) \equiv
\frac{1}{4\pi} \int_{CP(2)} (F^1_1 \wedge F^2_2 - F^1_2 \wedge F^2_1)
\ee
\be
= \frac{1}{8\pi} \int_{r^2 \ge 1} d\left(
Tr(A \wedge dA) - Tr(A) \wedge Tr(dA) + \frac{2}{3} Tr(A \wedge A \wedge A)
\right),
\ee
что в данном случае дает:
\be
k = 1.
%\frac{1}{8\pi} \int_{r^2 \ge 0} f
%d\zeta_1d\bar\zeta_1d\zeta_2d\bar\zeta_2 = 1,
\ee
%\be
% f = .
%\ee
 Тем не менее так как пространство параметров рассматриваемой системы
не является сферой $S^4$, связность $A$ не может быть сведена к
$SU(2)$ инстантону.

Полученные формулы полностью решают задачу о вычислении
фазы Берри для трехуровневой системы с вырождением.
Напомним, что невырожденная система была рассмотрена
ранее~\cite{PreviousWork}.

В заключении отметим, что предлагаемый подход
автоматически расспространяется на $(N + 1)$ уровневые системы с
$N$-кратным вырождением. При этом будут возникать
аналогичные инстантонные связности над $CP(N)$.

Несомненно представляет интерес постановка экспериментов
по обнаружению эффектов параметрических инстантонов.

\begin{thebibliography}{10}

\bibitem{FirstBerryWork}
 Berry~M.~V. // Proc. Roy. Soc. 1987. V. A392. P.~45.
\bibitem{AllBerryPhase}
 Shapere~A., Wilczek~F. Geometric Phases in Physics. World Scientific,
 Singapore, 1989.
\bibitem{SimonWork}
 Simon~R., Kimble~H.J., Sudarshan~E.C.G.
 // Phys. Rev. Lett. 1988. V.~61. P.~19
\bibitem{BerryPhaseExperimentalTest}
 Здесь должна стоять ссылка на работу про экспериментальное
 подтверждение фазы Беррии.
\bibitem{WorkAboutSimpleSystem}
 Work about simple system phase calculation
\bibitem{WilczekZeeWork}
 Wilczek~F., Zee~A. // Phys. Rev. Lett. 1984. V.~52. P.~2111
\bibitem{CeulemansWork}
 Ceulemans~A., Szopa~M. // J.Phys. 1991. V. A24. P.~4495.
\bibitem{KorenblitWork}
 Korenblit~S.E., Kuznetsov~V.E., Naumov~V.A. in: Proc. Int. Conf.
 Quantum systems : New trends and methods. Minsk, 1994.
 Eds. Barut~A.O. et al. World Scientific, Singapure, 1995. P.~208.
\bibitem{PreviousWork}
 Буканов~И.В., Толкачев~Е.А., Трегубович~А.Я.
 // ЯФ. 1996. Т.~59. N~4. С.~108-110.
\bibitem{AboutAdiabat}
 Messiah~A. Quantum Mechanics. North-Holland, Amsterdam, 1970. V.~2.
\bibitem{AboutCpTwoSpace}
 Дубровин Б.А., Новиков~С.П., Фоменко~А.Т.
 Современная геометрия. Москва, 1986.
\bibitem{Postnikov}
 Постников~М.М. Дифференциальная геометрия. Москва, 1989.
\bibitem{InstantonExpression}
 Здесь должна стоять ссылка на работу про инстантоны, например
работу Белавина, Полякова, Шварца, Тюпкина.
\end{thebibliography}

\pagebreak

\begin{center}

{\large\bf Berry phase and the parametric instanton in the dinamics of
a degenerated three-level quantum system. }

Bukanov~I.V., Tolkachev~E.A., Tregubovich~A.Ya.

{\it Stepanov~B.I. Institute of Physics, Academy of Sciences of Belarus.}

\end{center}

\begin{center}
{\bf Abstract. }
\end{center}

\begin{center}
 We have built closed expressions enabling to calculate Berry phase
 appearing during a dynamic evolution of parameters
 of a degenerated three-level quantum system and have shown that
 the corresponding nonabelian potential in the parameter's space
 has unit topological charge and the assimptomatic behavior
 of an ordinary instanton.
\end{center}

\end{document}

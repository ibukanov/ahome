\chapter{Conventions and useful relations}
\label{Conventions}

\begin{itemize}

\item 
$A^\dag$, $A^T$, $A^*$ denote respectively
the hermitian conjugation, matrix transposition and complex conjugation of $A$.
For a scalar quantity $a$, $a^T = a$ and $a^\dag = a^*$.

\item 
Lower-case Greek letters are used for Lorentz indices
and two-component spinor indices. In most cases the letters
$\mu$, $\nu$, $\rho$, $\sigma$ are Lorentz indices, whereas 
$\alpha$ and $\beta$ are two-component spinor indices. Lower-case Roman letters
are used to index four-component spinors. The index $r$ is often used 
to denote "+" or "-" polarization states. 

\item 
The Cartesian metric tensor $\eta_{\mu\nu}$ is given by
\bel{convCartesianMetric}
\eta_{\mu\nu} = \left\{ \matrix{1  & \mbox{if} \quad \mu = \nu = 0 \cr
                                -1 & \mbox{if} \quad \mu = \nu \ne 0 \cr
                                0  & \mbox{otherwise} } \right. .
\ee

\item
$g_{\mu\nu}$ stands for an arbitrary symmetric matrix with signature -2, i.e.
the difference between the number of positive and negative eigenvalues of 
$g_{\mu\nu}$ should be -2.

\P
These requirements mean
that one can find a non-singular real matrix $L$ such that 

\bel{gamma-metric-transformation}
\eta_{\mu\nu} = L_{\mu}{}^\rho L_{\nu}{}^\sigma g_{\rho\sigma}
\quad \mbox{or} \quad 
g_{\rho\sigma} = (L^{-1})_\rho{}^\mu (L^{-1})_\sigma{}^\nu \eta^{\mu\nu}
.
\ee  

Another useful form of~\rf{gamma-metric-transformation} is
\bel{gamma-metric-transformation2}
\eta^{\mu\nu} = (L^{-1})_\rho{}^\mu (L^{-1})_\sigma{}^\nu g^{\rho\sigma}
\quad \mbox{or} \quad 
g^{\rho\sigma}= L_{\mu}{}^\rho L_{\nu}{}^\sigma\eta^{\mu\nu}
.
\ee

In most cases $g$ is used to denote the determinant of $g_{\mu\nu}$, 
\be
g = \det g < 0.
\ee

\item 
I use $p$ mostly for particle 4-momenta such that 
\be
p^2  \equiv p_\mu p^\mu = g_{\mu\nu}p^\mu p^\mu = m^2 > 0
\ee
with some
mass $m$. I also use $q$ to denote the normalized momenta, 
\bel{convQPRelation}
q \equiv p/m, \quad q^2 = 1.  
\ee  

\item 
$(a \cdot b)$ and $\slsh{a}$ represent 
the index-less notations for scalar products, 
\be
(a \cdot b) \equiv a_\mu b^\mu, \quad \slsh{a} = a_\mu \gu\mu.
\ee

\item 
$A_{(\mu_1 \dots \mu_k)}$ and $A_{[\mu_1 \dots \mu_k]}$ 
denote the symmetric and antisymmetric parts of $A_{\mu_1 \dots \mu_k}$,

\beml{convention-symmetry-anti-symmetry}
A_{(\mu_1 \dots \mu_k)} & = &
\frac{1}{k!}\sum_{\sigma}
A_{\mu_{\sigma(1)} \dots \mu_{\sigma(k)}},

\nel
A_{[\mu_1 \dots \mu_k]} & = &
\frac{1}{k!}\sum_{\sigma}
\sign(\sigma) A_{\mu_{\sigma(1)} \dots \mu_{\sigma(k)}},
\ee

where the sum is taken over the permutation group of the set 
$\{ 1, \dots, k \}$, $\sigma(n)$ gives the result of
the permutation $\sigma$ for the number $n$ and 
$\sign(\sigma)$ is $1$
if $\sigma$ forms an even permutation of $\{ 1, \dots, k \}$
 and $-1$ otherwise. (A permutation 
is even if it is obtained by an even number of simple permutations 
and the simple permutation is the exchange
of two neighboring indices.)

\P 
It is easy to check that
\beml{contractionWithSymAnti}
&
A_{(\mu_1 \dots \mu_k)}B^{(\mu_1 \dots \mu_k)} = 
A_{\mu_1 \dots \mu_k}B^{(\mu_1 \dots \mu_k)} = 
A_{(\mu_1 \dots \mu_k)}B^{\mu_1 \dots \mu_k} ,

\nel &
A_{[\mu_1 \dots \mu_k]}B^{[\mu_1 \dots \mu_k]} = 
A_{\mu_1 \dots \mu_k}B^{[\mu_1 \dots \mu_k]} = 
A_{[\mu_1 \dots \mu_k]}B^{\mu_1 \dots \mu_k} ,

\nel &
A_{[\mu_1 \dots \mu_k]}B^{(\mu_1 \dots \mu_k)} = 0 .

\ee

\item

$\varepsilon^{\mu\nu\rho\sigma}$ is the fully
antisymmetric 4-dimensional tensor defined by

\bel{convEpsilonUpIs}
\varepsilon^{\mu\nu\rho\sigma} = \varepsilon^{[\mu\nu\rho\sigma]} = 
   \frac{1}{\sqrt{-g}} e^{\mu\nu\rho\sigma},
\quad
e^{\mu\nu\rho\sigma} = e^{[\mu\nu\rho\sigma]}, 
\quad 
e^{0123} = 1,
\ee

The pseudo-tensor $e^{\mu\nu\rho\sigma}$ satisfies for any $A^\mu{}_\nu$
\bel{convDetIs}
A^\mu{}_{\mu'}A^\nu{}_{\nu'}A^\rho{}_{\rho'}A^\sigma{}_{\sigma'}
                        e^{\mu'\nu'\rho'\sigma'} 
= \det(A) \, e^{\mu\nu\rho\sigma}
\ee

By the definition $e_{\mu\nu\rho\sigma} = e^{\mu\nu\rho\sigma}$ and 
for the covariant version of $\varepsilon^{\mu\nu\rho\sigma}$ 
one can find

\beml{convEpsilonDownIs}
\varepsilon_{\mu\nu\rho\sigma} 
& = & g_{\mu\mu'}g_{\nu\nu'}g_{\sigma\sigma'}g_{\rho\rho'}
        \varepsilon^{\mu'\nu'\sigma'\rho'} 

\nel 
&  = & \frac{1}{\sqrt{-g}} g_{\mu\mu'}g_{\nu\nu'}g_{\sigma\sigma'}g_{\rho\rho'}
                        e^{\mu'\nu'\sigma'\rho'}   
= \frac{1}{\sqrt{-g}} g e_{\mu\nu\rho\sigma},
\nel
\varepsilon_{\mu\nu\rho\sigma} & = & -\sqrt{-g} \, e_{\mu\nu\rho\sigma}
\ee

\P
In the case when $g_{\mu\nu}$ equals the Cartesian 
metric~\rf{convCartesianMetric}
, $g = -1$ and $\varepsilon^{\mu\nu\rho\sigma}$ is given by
\be
\varepsilon^{\mu\nu\rho\sigma} = e^{\mu\nu\rho\sigma},
\quad
\varepsilon^{0123} = 1,
\quad
\varepsilon_{0123} = -e_{0123} = -e^{0123} = -1.
\ee


\end{itemize}

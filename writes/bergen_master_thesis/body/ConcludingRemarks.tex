\chapter{Concluding remarks}

An important feature of any realistic supersymmetric generalization of
the SM Lagrangian is the presence of fermion fields that couple with other 
particles not only by usual terms like $\bar\psi_1 X \psi_2$ 
but also via products where instead of $\bar\psi$ one have $\psi^T C$ 
term (see~\rf{gammaScalarProducts}). In general, such terms violate the fermion number 
conservation law. An example of such particles in the MSSM is charginos.
Also SUSY models predict the existence of massive Majorana 
fields that satisfy~\rf{gammaMajoranaRestriction}, $\bar\psi_M = \psi_M^T C$.
If such fermion fields are found and their parameters lie inside 
the predicted bounds it would be a very strong argument 
for supersymmetry, even in the absence of other data.

\P
In this study I considered different properties of fermion fields.
In particular I showed without using the CPT theorem that the electric 
and magnetic dipole moments of Majorana particle are zero and 
that the particles
cannot couple with vector fields.

\P
I calculated in the framework of the MSSM model 
the cross section for the lightest chargino production process,
$e^+e^-\rightarrow \chi^+\chi^-$, in the case of CP violation
in the chargino sector. The results show that these effect do not 
lead to any significant contribution to the cross section  and cannot be 
used to loosen the experimental bounds that can be set on the theory
parameters if particles are not found in the experiment.


\P
Charginos are expected to be easy to discover since their 
production cross section may be as high as several pb 
and they would lead to events with
characteristic decays similar to $W$-boson pair production plus 
missing energy. The typical decay processes would be 
(see for example~\cite{CharginoSignatures}):
\bem
\chi^+ & \rightarrow & l^+ + \nu + \xi,
\nel
\chi^+ & \rightarrow & q + \bar{q}' + \xi,
\nel
\chi^+ & \rightarrow & q + \bar{q}' + \tilde{g}, \quad 
\tilde{g}\; \rightarrow\; q + \bar{q}' + \xi,
\ee

where $\xi$ is the lightest neutralino which is stable in the framework 
of the MSSM, and $\tilde{g}$ is the gluino, the gluon superpartner.


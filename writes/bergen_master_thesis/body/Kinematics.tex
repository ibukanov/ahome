\chapter{Kinematics}\label{CrossSectionApp}

Consider the process 
\bel{csProcess}
f_1(\pa, \ra), f_2(\pb, \rb) \rightarrow f'_1(\ppa, \rpa), f'_2(\ppb, \rpb),
\ee
where
$f_{1, 2}$, $f'_{1, 2}$ are some fermions,
$\pa$, $\ra$, $\pb$, $\rb$ denote the momenta and
spin polarization states of the incoming particles
and the "prime" versions are used for the outgoing ones. The particle momenta
satisfy the on-shell conditions, 
\be
\pa^2 = \ma^2, \quad  \pb^2 = \mb^2,
\quad
\ppa^2 = \mpa^2, \quad  \ppb^2 = \mpb^2,
\ee

\P
The amplitude is given by 
the $S$-matrix element that has the following general structure, 
\beml{csSis}
S(f_1, f_2 \rightarrow f'_1, f'_2)
& = &
-i (2\pi)^{4}
\left(\frac{\ma\mb\mpa\mpb}{\Ea\Eb\Epa\Epb}\right)^{\frac{1}{2}}
\delta(\ppa + \ppb - \pa - \pb)

\nel &&{}\times 
 M(\pa, \ra, \pb, \rb, \ppa, \rpa, \ppb, \rpb).
\ee

Here $M$ is some smooth Lorentz-invariant function of momenta and spins
(Feynman amplitude) and $E$ is defined through 
\be
    E^2 = \absp^2 + m^2.
\ee

The Dirac delta-function in \rf{csSis} reflects the 4-momentum conservation 
in the process \rf{csProcess},
\bel{cs-conservation-law}
\pa + \pb \; = \; \ppa +\ppb,
\ee


\P
The center-of-mass unpolarized differential cross section is according
to~\cite{SMTextBook} given by
\bel{cs-cross1}
\left({d\sigma \over d\Omega}\right)_{\rm c.m.}
\; = \;{1 \over 4 \pi^2 (\Ea + \Eb)^2}{\abspp\over \absp}
\ma\mb\mpa\mpb \, X(\pa, \pb, \ppa, \ppb),
\ee

with
\bel{cs1}
X(\pa, \pb, \ppa, \ppb) \; = \; {1 \over 4}\sum_\ra\sum_\rb\sum_\rpa\sum_\rpb
   M^\dagger(\ra, \rb, \rpa, \rpb) M(\ra, \rb, \rpa, \rpb),
\ee

where $E$, $E'$, $\vecp$ and $\vecpp$ are 
the energies and 3-momenta of the 
incoming and outgoing particles in the center-of-mass frame,

\be
\pa|_{\rm c.m} = (\Ea, \vecp), 
\quad 
\pb|_{\rm c.m}  = (\Eb, -\vecp),
\quad 
\ppa|_{\rm c.m} = (\Epa, \vecpp),
\quad 
\ppb|_{\rm c.m} = (\Epb, -\vecpp).
\ee

To rewrite the cross section in an explicitly Lorentz-invariant form 
I express it through the total momentum squared $s$ and two other 
kinematic invariants $t$ and $u$,
\be
s = (\pa + \pb)^2 =  (\ppa + \ppb)^2,
\ee
\be
t = (\ppa - \pa)^2 =  (\pb - \ppb)^2,
\ee
\be
u = (\ppb - \pa)^2 =  (\pb - \ppa)^2.
\ee

These Lorentz scalars are related by the identity
\bem
s + t + u & = & (\pa + \pb)^2 + (\ppa - \pa)^2 + (\ppb - \pa)^2
\nel
& = & \ma^2 + \mb^2 + \mpa^2 + \ma^2 + \mpb^2 + \ma^2 
\nel
&&{}  + 2(\pa \cdot \pb) - 2(\pa \cdot \ppa) - 2(\pa \cdot \ppb)
\nel
& = & 3\ma^2 + \mb^2 + \mpa^2 + \mpb^2 
  + 2\pa \cdot(\pb - \ppa -  \ppb)
\nel
& = & 3\ma^2 + \mb^2 + \mpa^2 + \mpb^2 
  - 2(\pa \cdot \pa)
\ee
or
\bel{sum-s-t-u}
s + t + u = \ma^2 + \mb^2 + \mpa^2 + \mpb^2 .
\ee


In the c.m. frame $s$ may be written as
\be
s = (\Ea + \Eb)^2 - (\vecp - \vecp)^2 = (\Ea + \Eb)^2
.
\ee

To get an expression for the c.m. momentum $\absp$ I proceed
as following,
\be
(\pa \cdot \pb) = (\pa \cdot \pb)|_{\rm c.m} = \Ea\Eb + \absp^2 
= \sqrt{\ma^2 + \absp^2}\sqrt{\mb^2 + \absp^2}
    + \absp^2
    .
\ee

Squaring, I find
\bem
(\ma^2 + \abs{\vec{p}}^2)(\mb^2 + \absp^2) 
 & = &
[(\pa \cdot \pb) - \absp^2]^2, 
\nel
\absp^2(\ma^2 + \mb^2) + \ma^2\mb^2 
& = & 
(\pa \cdot \pb)^2 - 2(\pa \cdot \pb)\absp^2,
\nel
(\pa \cdot \pb)^2 - \ma^2\mb^2 
&=& \absp^2[\ma^2 + \mb^2 + 2 (\pa \cdot \pb)] 
,
\ee
or
\be
(\pa \cdot \pb)^2 - \ma^2\mb^2 \;=\; \absp^2(\pa + \pb)^2 \;=\; s\absp^2 
.
\ee

Thus
\be
\absp^2 
={1 \over s}[(\pa \cdot \pb)^2 - \ma^2\mb^2]
={1 \over 4s}\{[2(\pa \cdot \pb)]^2 - [2\ma\mb]^2\}.
\ee

Furthermore,

\bem
\lefteqn{
[2(\pa \cdot \pb)]^2 - [2\ma\mb]^2 \; = \; (s - \ma^2 - \mb^2)^2 - [2\ma\mb]^2
}
\nel
&=& (s - \ma^2 - \mb^2 - 2\ma\mb)(s - \ma^2 - \mb^2 + 2\ma\mb)
\nel
&=& [s - (\ma + \mb)^2][s - (\ma - \mb)^2]
.
\ee

So
\bel{cs-abs-p-squared}
\absp^2   =
{1 \over 4s} [s - (\ma + \mb)^2][s - (\ma - \mb)^2]
.
\ee

In the same way I find 
\bel{cs-abs-pp-squared}
\abspp^2  = 
{1 \over 4s} [s - (\mpa + \mpb)^2][s - (\mpa - \mpb)^2],
\ee
and thus
\bel{cs-pp-over-p}
{\abspp\over \absp} 
= \sqrt{[s - (\mpa + \mpb)^2][s - (\mpa - \mpb)^2] \over 
        [s - (\ma + \mb)^2][s - (\ma - \mb)^2]}
.
\ee

\P

To be Lorentz-invariant, the quantity $X$ can depend on the particle momenta 
only through scalar products of different $p$. With
\beml{cs-t-explicit}
t    & = & (\ppa - \pa)^2 = \mpa^2 + \ma^2 - 2(\ppa \cdot \pa)
\nel & = & (\pb - \ppb)^2 = \mb^2 + \mpb^2 - 2(\pb \cdot \ppb)
,
\ee

\beml{cs-u-explicit}
u    & = & (\ppb - \pa)^2 = \mpb^2 + \ma^2 - 2(\ppb \cdot \pa)
\nel & = & (\pb - \ppa)^2 = \mb^2 + \mpa^2 - 2(\pb \cdot \ppa)
,
\ee

all such products are given by
\begin{eqnarray}\label{cs-scalar-products-1}
(\pa \cdot \pb) & = &{1 \over 2} (s - \ma^2 - \mb^2),
\\
(\ppa \cdot \ppb) & = & {1 \over 2} (s - \mpa^2 - \mpb^2),
\\
(\ppa \cdot \pa) & = & {1 \over 2} (\mpa^2 + \ma^2 - t),
\\
(\ppb \cdot \pa) & =& {1 \over 2} (\mpb^2 + \ma^2 - u)
\nel &=& 
{1 \over 2} [\mpb^2 + \ma^2 - (\mpa^2 + \ma^2 + \mpb^2 + \mb^2 - s - t)]
\nel &=& 
{1 \over 2} (s + t - \mpa^2 - \mb^2),
\\
(\ppa \cdot \pb) &=& {1 \over 2} (\mpa^2 + \mb^2 - u)
= {1 \over 2} (s + t - \mpb^2 - \ma^2) ,
\\
(\ppb \cdot \pb) 
&=& {1 \over 2} (\mpb^2 + \mb^2 - t)
= {1 \over 2} (s + u - \mpa^2 - \ma^2)
.
\label{cs-scalar-products-2}
\end{eqnarray}

Thus $X(\pa,\pb,\ppa,\ppb)$ may be viewed as a function of $s$ and $t$,
\bel{cs-X-of-s-t}
X = X(s, t).
\ee

\P
The variable $t$ can be expressed via the center-of-mass frame 
scattering angle $\theta$ defined by 
$\vecpa\cdot\vecppa = \vecp\cdot\vecpp = \absp\abspp\cos\theta$,

\be
(\ppa \cdot \pa) = (\ppa \cdot \pa)|_{\rm c.m} = \Epa\Ea -  \vecpa\cdot\vecppa
= \Epa\Ea -  \absp\abspp\cos\theta
.
\ee
Thus
\be
{1 \over 2} (\ma^2 + \mpa^2 - t) 
  = \Epa\Ea -  \absp\abspp\cos\theta,
\ee
and
\bel{cs-t-of-theta}
t =   2\absp\abspp\cos\theta - t_0,
\ee
where I defined
\bel{cs-tmp-t0-is}
t_0 = 2\Epa\Ea  - (\ma^2 + \mpa^2)
.
\ee

To calculate $2\Epa\Ea$ in~\rf{cs-tmp-t0-is} I use~\rf{cs-abs-p-squared} 
and~\rf{cs-abs-pp-squared}  to express
$\Ea$ and $\Epa$,
\bem
\Ea^2 & = & \absp^2 + \ma^2 = 
{1 \over 4s} [s - (\ma + \mb)^2][s - (\ma - \mb)^2]  + \ma^2,
\nel
4s\Ea^2 &=& s^2 + s [4\ma^2 - (\ma + \mb)^2 - (\ma - \mb)^2]
+ (\ma^2 - \mb^2)^2 
\nel
& = & s^2 + 2 s (\ma^2 -\mb^2) + (\ma^2 - \mb^2)^2 
= (s + \ma^2 -\mb^2)^2
.
\ee
So
\be
\Ea^2 = {1 \over 4s}(s + \ma^2 -\mb^2)^2, 
\ee
and similarly 
\be
\Epa^2 = {1 \over 4s}(s + \mpa^2 -\mpb^2)^2.
\ee

Taking the square roots I find
\be
2\Epa\Ea = {1 \over 2s}(s + \mpa^2 -\mpb^2)(s + \ma^2 -\mb^2).
\ee

It gives 
\beml{cs-t0-is}
t_0 & = &  {1 \over 2s}(s + \mpa^2 -\mpb^2)(s + \ma^2 -\mb^2) - \ma^2 - \mpa^2
\nel
&=& {1 \over 2s}
\{s^2 + s[-2(\ma^2 + \mpa^2) + (\ma^2 - \mb^2) + (\mpa^2 -\mpb^2)]
\nel && \qquad{}
   + (\mpa^2 -\mpb^2)(\ma^2 -\mb^2)\}
\nel &=&
{1 \over 2s}[s^2 - s(\ma^2 + \mpa^2 + \mb^2 + \mpb^2) 
   + (\mpa^2 -\mpb^2)(\ma^2 -\mb^2)].
\nel
\ee

Now I may write for the c.m. differential cross-section~\rf{cs-cross1}
\bel{cs-cross2}
\left({d\sigma \over d\Omega}\right)_{\rm c.m.}
\; = \; {d\sigma(s, t(\theta)) \over d\Omega}
\; = \;{1 \over 4 \pi^2 s}{\abspp\over \absp}
\ma\mb\mpa\mpb \, X(s, t).
\ee

Then for the total cross-section for the process \rf{csProcess}, defined by
\be
\sigma_T(s) = 
\int {d\sigma \over d\Omega} d\Omega
= \int {d\sigma \over d\Omega} \sin\theta d\theta d\phi
,
\ee

I have 
\bel{cs-cross3}
\sigma_T(s) = 
\int {d\sigma(s, t(\theta)) \over d\Omega} \sin\theta d\theta d\phi
 = 2\pi \int_{-1}^1 {d\sigma(s, t(\theta)) \over d\Omega} d(\cos\theta) .
\ee

From~\rf{cs-t-of-theta} I find
\be
d(\cos\theta) = {dt \over 2\absp\abspp}
.
\ee

Thus
\bem
\sigma_T(s)  
& = & {\pi \over \absp\abspp} \int_{t_-}^{t_+} 
\left({d\sigma \over d\Omega}\right)_{\rm c.m.} dt
\nel
& = &
{\ma\mb\mpa\mpb \over 4 \pi^2 s}{\abspp\over \absp}
{\pi \over \absp\abspp} \int_{t_-}^{t_+} X(s, t) dt ,
\nel
& = &{\ma\mb\mpa\mpb \over 4 \pi s \absp^2}
 \int_{t_-}^{t_+} X(s, t) \, dt
 ,
\ee
where $t_{\pm}$ are given by
\be
t_{\pm} = t\left|_{\cos\theta = \pm 1}\right. = \pm 2\absp\abspp - t_0
.
\ee

\P
In the case when 
\bel{cs-ultra-approximation}
\mpa = \mpb = m' \gg \ma = \mb = m, \quad s \ge 4m'^2 \gg m^2
\ee

the kinematic relations of this appendix can be simplified. 
To write the formulas
in a compact form I introduce the dimension-less quantity $\rho$, 
\bel{csRhoIs}
\rho = {m'^2 \over s},
\quad \rho \le {1 \over 4}.
\ee

With this notation I have according 
to~\rf{cs-abs-p-squared}--\rf{cs-pp-over-p} 
\be
\absp^2 = 
{s \over 4} - m^2 \approx {s \over 4}, 
\qquad
\abspp^2 = {s \over 4}(1 - 4\rho),
\ee
\be
{\abspp \over \absp} 
\approx \sqrt{1 - 4\rho}, 
\qquad
\abspp \absp = {\abspp \over \absp} \absp^2 = 
{s \over 4}\sqrt{1 - 4\rho}
.
\ee

So under this approximation $t_\pm$ becomes
\bel{cs-tmp-t-pm-approximation}
t_{\pm} \approx \pm {s \over 2}\sqrt{1 - 4\rho} - t_0
\approx \pm{s \over 2}\sqrt{1 - 4\rho} - {1 \over 2}(s - 2m'^2) 
= {s \over 2} (2\rho - 1 \pm \sqrt{1 - 4\rho}).
\ee

The scalar products in~\rf{cs-scalar-products-1}--\rf{cs-scalar-products-2}
now read
\be
(\pa \cdot \pb) \approx {s \over 2},
\qquad 
(\ppa \cdot \ppb) = {s \over 2}(1 - 2\rho),
\ee

\beml{csKinApproximation}
(\ppa \cdot \pa) = (\ppb \cdot \pb) & \approx & {m'^2 - t \over 2},
\nel
(\ppb \cdot \pa) = (\ppa \cdot \pb) & \approx & {m'^2 - u \over 2} 
\approx {s + t - m'^2 \over 2},
\ee

and the cross sections~\rf{cs-cross2}and~\rf{cs-cross3} are given by

\bel{cs-tmp-dif-approximation}
\left({d\sigma \over d\Omega}\right)_{\mbox{c.m.}}
\approx 
{m^2 m'^2  \over 4\pi^2 s} \sqrt{1 - 4\rho}\, X(s, t)
,
\ee

\bel{cs-tmp-total-approximation}
\sigma_T(s)  
\approx {m^2 m'^2 \over \pi s^2} \int_{t_-}^{t_+} X(s, t) \, dt 
.
\ee

\P
I next present several useful results and find typical integrals 
that occur during calculations 
of the total cross section under the approximation~\rf{cs-ultra-approximation}:

\bem
\lefteqn{
(\qpb \cdot \qa) (\qpa \cdot \qb) + (\qpb \cdot \qb) (\qpa \cdot \qa)
}
\nel
& \equiv &
{1 \over \ma\mb\mpa\mpb}
[(\ppb \cdot \pa) (\ppa \cdot \pb) + (\ppb \cdot \pb) (\ppa \cdot \pa)]
\nel
& = & {1 \over m^2m'^2}
[(\ppb \cdot \pa)^2 + (\ppa \cdot \pa)^2]
\approx {1 \over 4m^2m'^2}[(s + t - m'^2)^2 + (m'^2 - t)^2]
\nel & = &
{1 \over 4m^2m'^2}[(m'^2 - t)^2 -2s(m'^2 - t) + s^2 + (m'^2 - t)^2],
\ee
or
\bel{cs-4q-sum}
(\qpb \cdot \qa) (\qpa \cdot \qb) + (\qpb \cdot \qb) (\qpa \cdot \qa)
\approx {2(m'^2 - t)^2 + s(s + 2t - 2m'^2) \over 4m^2m'^2}
.
\ee

Through similar steps I have
\beml{cs-4q-sub}
\lefteqn{
(\qpb \cdot \qa) (\qpa \cdot \qb) - (\qpb \cdot \qb) (\qpa \cdot \qa)
}
\nel
& \equiv &
{1 \over \ma\mb\mpa\mpb}
[(\ppb \cdot \pa) (\ppa \cdot \pb) - (\ppb \cdot \pb) (\ppa \cdot \pa)]
\nel
& \approx & {1 \over 4m^2m'^2}[(s + t - m'^2)^2 - (m'^2 - t)^2],
\nel
& \approx & {s \over 4m^2m'^2}(s + 2t - 2m'^2)
.
\ee

The direct consequences of~\rf{cs-tmp-t-pm-approximation} are
\beml{cs-t-pm-pm}
t_+ - t_- & \approx & s\sqrt{1 - 4\rho}, 
\nel
t_+ + t_- & \approx & s(2\rho - 1),
\nel
t_+ t_- & \approx &
{s^2 \over 4} (2\rho - 1 + \sqrt{1 - 4\rho})(2\rho - 1 - \sqrt{1 - 4\rho})
\nel 
&=& {s^2 \over 4} [(2\rho - 1)^2 - (1 - 4\rho)]
= s^2\rho^2
.
\ee

Thus from the definition of $\rho$~\rf{csRhoIs}
\bel{cs-t-pm-mul}
t_+ t_- \approx  m'^4
.
\ee

\P
In the following integrals I assume that 
$x$ and $z$ are complex numbers and $r$ is real.

\beml{cs-int-1}
\int_{t_-}^{t_+} (2m'^2 - 2t - s) dt
& = & (t_+ - t_-)(2m'^2 - s) - (t_+^2 - t_-^2)
\nel 
& = &
(t_+ - t_-)(2m'^2 - s -t_+ - t_-)
\nel
& = & (t_+ - t_-)[s(2\rho - 1) -t_+ - t_-]
\nel
& \approx & 0,
\ee
where~\rf{cs-t-pm-pm} was used. Next

\bem
\int_{t_-}^{t_+} (m'^2 - t)^2 dt 
&=&\int_{t_-}^{t_+} (m'^4 - 2 m'^2t + t^2) dt 
\nel 
& = & 
m'^4(t_+ - t_-) - m'^2(t_+^2 - t_-^2) + {1 \over 3}(t_+^3 - t_-^3)
\nel 
& = & 
(t_+ - t_-)\{m'^4 - m'^2(t_+ + t_-) + {1 \over 3}[(t_+ + t_-)^2 - t_+t_-]\}.
\ee

At this point, according to~\rf{cs-t-pm-pm}--\rf{cs-t-pm-mul} I can
write the approximation
\beml{cs-int-2}
\int_{t_-}^{t_+} (m'^2 - t)^2 dt
& \approx & 
s\sqrt{1 - 4\rho}\{m'^4 - m'^2(2m'^2 - s) 
   + {1 \over 3}[(2m'^2 - s)^2 - m'^4]\}
\nel 
& = & 
s\sqrt{1 - 4\rho}\,[m'^4(1 - 2 + 1) + m'^2s(1 -4/3) + s^2/3]
\nel 
& = & 
s\sqrt{1 - 4\rho}[s^2/3 - m'^2s/3]
\nel 
& = & {s^3 \over 3} (1 - \rho)\sqrt{1 - 4\rho}\,
.
\ee

I shall also need the integral 
\be
\int_{t_-}^{t_+} \Re\left({z \over t - x}\right) dt
\ee
for value of $x$ such that no pole is encountered,

\bel{csIntTmp3}
\int_{t_-}^{t_+} \Re\left({z \over t - x}\right) dt 
\;=\; \int_{t_- - x}^{t_+ - x} \Re\left({z \over t}\right) dt 
\;=\;  \Re\left(z\ln{t_+ - x \over t_- - x}\right)
.
\ee

To simplify the last expression I use,
\begin{eqnarray}
\Re(z_1 z_2) &=& \Re(z_1)\Re(z_2) - \Im(z_1)\Im(z_2), 
\\
\label{cs-Im-z1-z2}
\Im(z_1 z_2) &=& \Re(z_1)\Im(z_2) + \Im(z_1)\Re(z_2), 
\\
\Re(\ln z_3) &=& \ln\abs{z_3}, 
\\
\Im(\ln z_3) &=& \arctg{\Im(z_3) \over \Re(z_3)} 
%+ \left\{\matrix{0, & \Re(z) \ge 0 \cr \pi, & \Re(z) < 0}\right.
,
\end{eqnarray}
that are valid for any complex $z_1$,$z_2$ and $z_3 \ne 0$. Thus

\bem
\Re\left[z \ln {t_+ - x \over t_- - x}\right] 
&=&
\Re(z)\Re\left[ \ln {t_+ - x \over t_- - x}\right] 
- \Im(z)\Im\left[ \ln {t_+ - x \over t_- - x}\right] 
\nel
&=&
\Re(z)\ln\abs{t_+ - x \over t_- - x}
- \Im(z)\Im\left[ \ln {t_+ - x \over t_- - x}\right],
\ee

where
\bem
\lefteqn{
\Im\left[ \ln {t_+ - x \over t_- - x}\right]
= \Im\left[ \ln {(t_+ - x)(t_- - x^*) \over \abs{t_- - x}^2}\right]
= \Im\left[ \ln (t_+t_- + \abs{x}^2 - x t_- - x^*t_+)\right]
}
&& \fillEqn
\nel
&=& \arctg{\Im(m'^4 + \abs{x}^2 - x t_- - x^*t_+) 
         \over \Re(m'^4 + \abs{x}^2 - x t_- - x^*t_+)}
= \arctg{\Im(x)(t_+ - t_-) 
         \over m'^4 + \abs{x}^2 - \Re(x)(t_- + t_+)}
,         
\ee

or
\bel{cs-tmp-im-log}
\phi(x) \equiv \Im\left[ \ln {t_+ - x \over t_- - x}\right]
= 
\arctg{s\sqrt{1 - 4\rho}\,\Im(x)
         \over m'^4 + \abs{x}^2 - s(2\rho - 1)\Re(x)}
.
\ee

It gives for~\rf{csIntTmp3}, 
\bel{cs-int-3}
\int_{t_-}^{t_+} \Re\left({z \over t - x}\right) dt 
\; \approx \;
\Re(z)\ln\abs{t_+ - x \over t_- - x}
- \Im(z) \phi(x).
\ee

Next I consider the integral (where again no pole is encountered)

\bem
\lefteqn{
\int_{t_-}^{t_+} {(t - r)^2 \over t - x} dt 
\; = \; 
\int_{t_- - x}^{t_+ - x} {(t + x - r)^2 \over t} dt 
= 
\int_{t_- - x}^{t_+ - x} 
\left[{(x - r)^2 \over t} + 2(x - r) + t \right] dt 
}
&& \fillEqn
\nel
& = &
(x - r)^2 \ln{t_+ - x \over t_- - x}
+ 2(x - r)(t_+ - t_-) + {1 \over 2}[(t_+ - x)^2 - (t_- - x)^2]
.
\ee

But
\be
(t_+ - x)^2 - (t_- - x)^2 = (t_+ - x + t_- - x)(t_+ - t_-)
= [s(2\rho - 1) - 2x](t_+ - t_-),
\ee

so
\beml{cs-tmp-int1}
\int_{t_-}^{t_+} {(t - r)^2 \over t - x} dt 
& = &
(x - r)^2 \ln{t_+ - x \over t_- - x}
+ (t_+ - t_-)[2(x - r) + {s \over 2}(2\rho - 1) - x]
\nel
& = &
(x - r)^2 \ln{t_+ - x \over t_- - x}
+ s\sqrt{1 - 4\rho}(x - 2r + s\rho - s/2)
.
\ee

I use~\rf{cs-tmp-int1} to calculate the following two integrals.
First I consider

\bem
\lefteqn{
\int_{t_-}^{t_+} \Re\left(z{(t - r)^2 \over t - x}\right) dt 
}
\nel
& \approx &
\Re\left[z(x - r)^2 \ln{t_+ - x \over t_- - x}\right]
+ s\sqrt{1 - 4\rho}\Re[z(x  - 2r + s\rho - s / 2)]
\nel
& = &
\Re\left[z(x - r)^2 \ln{t_+ - x \over t_- - x}\right]
+ s\sqrt{1 - 4\rho}\Re[zx  - z(2r - s\rho + s / 2)]
\nel
& = &
\Re[z(x - r)^2] \Re\left[\ln{t_+ - x \over t_- - x}\right]
-\Im[z(x - r)^2] \Im\left[\ln{t_+ - x \over t_- - x}\right]
\nel
&& {}
+ s\sqrt{1 - 4\rho}\{\Re(zx) - \Re(z)[2r + s(1 - 2\rho)/2]\}
,
\ee

or
\beml{cs-int-4}
\int_{t_-}^{t_+} \Re\left(z{(t - r)^2 \over t - x}\right) dt 
& \approx &
\Re[z(x - r)^2] \ln\abs{t_+ - x \over t_- - x}
-\Im[z(x - r)^2] \phi(x)
\nel
&&{}
+ s\sqrt{1 - 4\rho}\{\Re(zx) - \Re(z)[2r + s(1 - 2\rho)/2]\}
.
\nel
\ee

\P
In the following calculations I use 
\be
{1 \over \abs{t - x}^2} 
= {1 \over x - x^*} \left({1 \over t - x} - {1 \over t - x^*} \right)
= {1 \over 2i\Im(x)} \left({1 \over t - x} - {1 \over t - x^*} \right)
,
\ee
where it is supposed that $\Im(x) \ne 0$. 
Thus ($r$ is real)

\bem
\int_{t_-}^{t_+} {(t - r)^2 \over \abs{t - x}^2} dt 
& \equiv &
\Re\left\{
\int_{t_-}^{t_+} {(t - r)^2 \over \abs{t - x}^2} dt 
\right\}
\nel
& = &
\Re\left\{
{1 \over 2i\Im(x)}
\int_{t_-}^{t_+} \left[{(t - r)^2 \over t - x}
                       - {(t - r)^2 \over t - x^*}\right] dt 
\right\}
\nel
& = &
{1 \over 2\Im(x)}
\Im\left\{
\int_{t_-}^{t_+} \left[{(t - r)^2 \over t - x}
                       - {(t - r)^2 \over t - x^*}\right] dt 
\right\}
,
\ee

where I have used the fact that $r$ is real. So

\beml{cs-int-6}
\lefteqn{
\int_{t_-}^{t_+} {(t - r)^2 \over \abs{t - x}^2} dt 
}
\nel
& = &
{1 \over 2\Im(x)}
\Im\left[(x - r)^2 \ln {t_+ - x \over t_- - x}
+ s\sqrt{1 - 4\rho}(x - 2r + m'^2 - s / 2)
\right]
\nel&& {}
-{1 \over 2\Im(x)}
\Im\left[
(x^* - r)^2 \ln {t_+ - x^* \over t_- - x^*}
+ s\sqrt{1 - 4\rho}(x^* - 2r + m'^2 - s / 2)
\right]
\nel
& = &
{1 \over 2\Im(x)}
\Im\left[
(x - r)^2 \ln {t_+ - x \over t_- - x} 
- (x^* - r)^2 \ln {t_+ - x^* \over t_- - x^*}
+ s\sqrt{1 - 4\rho}(x - x^*)
\right]
\nel
\ee

I simplify the last expression with the help of the identity
\be
\Im(z - z^*) = \Im[2i\Im(z)] = 2\Im(z).
\ee

According to~\rf{cs-Im-z1-z2} I find for the integral~\rf{cs-int-6}
\bem
\int_{t_-}^{t_+} {(t - r)^2 \over \abs{t - x}^2} dt 
& = &
{1 \over \Im(x)}
\Im\left[(x - r)^2 \ln {t_+ - x \over t_- - x} + sx\sqrt{1 - 4\rho} \right]
\nel
& = &
{1 \over \Im(x)}
\Re[(x - r)^2] \Im\left[\ln{t_+ - x \over t_- - x}\right] 
\nel&&{}
+{1 \over \Im(x)}
\Im[(x - r)^2] \Re\left[\ln {t_+ - x \over t_- - x}\right] 
+ s\sqrt{1 - 4\rho}
\,
.
\ee

By taking into account that
\be
\Re[(x - r)^2] = [\Re(x - r)]^2 - [\Im(x - r)]^2 = [\Re(x) - r]^2 - \Im(x)^2,
\ee
\bel{csReImForSub}
\Im[(x - r)^2] = 2\Im(x - r)\Re(x - r) = 2\Im(x)[\Re(x) - r]
,
\ee

I finally find
\beml{cs-int-65}
\int_{t_-}^{t_+} {(t - r)^2 \over \abs{t - x}^2} dt
&\approx&
{[\Re(x) - r]^2 - \Im(x)^2 \over \Im(x)} \phi(x)
+ 2[\Re(x) - r]\ln\abs{t_+ - x \over t_- - x}
\nel
&&{}
+ s\sqrt{1 - 4\rho}\, .
\ee

When $\Im(x) = 0$, i.e when $x$ is real, I may use,
\be
\lim_{\varepsilon \rightarrow 0} 
\left({1 \over \varepsilon} \arctg a\varepsilon\right)
= a
\ee
to calculate the first term in~\rf{cs-int-65},
\bem
{ \phi(x)\over \Im(x)} &=& 
{ 1 \over \Im(x)} 
\arctg{s\sqrt{1 - 4\rho}\Im(x) \over m'^4 + \abs{x}^2 - s(2\rho - 1)\Re(x)}
\nel&\rightarrow &
{s\sqrt{1 - 4\rho}\over m'^4 + \abs{x}^2 - s(2\rho - 1)\Re(x)}
.
\ee

Thus for real $x$ equation~\rf{cs-int-65} becomes
\beml{cs-int-7}
\int_{t_-}^{t_+} {(t - r)^2 \over (t - x)^2} dt
&\approx&
{s\sqrt{1 - 4\rho}[x - r]^2 \over m_\chi^4 + x^2 - sx(2\rho - 1)}
+ 2(x - r)\ln\abs{t_+ - x \over t_- - x}
+ s\sqrt{1 - 4\rho}\,
.
\nel
\ee

\P
For reference purposes I collect here several key formulas.
For the process~\rf{csProcess} when the particle masses satisfy
\bel{csUltraApproximation}
\mpa = \mpb = m' \gg \ma = \mb = m, \quad s \ge 4m'^2 \gg m^2,
\ee

the c.m. differential  and total cross sections are given by
\bel{csDifApproximation}
\left({d\sigma \over d\Omega}\right)_{\rm c.m.}
\approx 
{m^2 m'^2  \over 4\pi^2 s} \sqrt{1 - 4\rho}\, X(s, t)
,
\ee

\bel{csTotalApproximation}
\sigma_T(s)  
\approx {m^2 m'^2 \over \pi s^2} \int_{t_-}^{t_+} X(s, t) \, dt 
,
\ee

where


\begin{eqnarray}
\label{cs-s}
s & = & (\pa + \pb)^2 = (\ppa + \ppb)^2 ,
\\
\label{cs-t-is}
t &=& (\ppa - \pa)^2 = (\pb - \ppb)^2 ,
\\
\label{cs-t-pm-approximation}
t_\pm & \approx & {s \over 2} (2\rho - 1 \pm \sqrt{1 - 4\rho})
,
\\
\label{cs-rho}
\rho &=& {m'^2 \over s} \; \le \; {1 \over 4},
\end{eqnarray}

and $X$ is the spin sum over initial and final particle spin states,
\bel{cs-X-is}
X(s, t) \equiv 
X(\pa, \pb, \ppa, \ppb) 
=  {1 \over 4}\sum_{\ra, \rb, \rpa, \rpb = +, -}
   M^\dagger(\ra, \rb, \rpa, \rpb) M(\ra, \rb, \rpa, \rpb)
.   
\ee

\P
Under the approximation~\rf{csUltraApproximation} the following relations hold,
\beml{cs-q-scalar-products}
(\qpb \cdot \qa) (\qpa \cdot \qb) + (\qpb \cdot \qb) (\qpa \cdot \qa)
& \approx &
{2(m'^2 - t)^2 + s(s + 2t - 2m'^2) \over 4m^2m'^2}
,
\nel
(\qpb \cdot \qa) (\qpa \cdot \qb) - (\qpb \cdot \qb) (\qpa \cdot \qa)
& \approx & {s \over 4m^2m'^2}(s + 2t - 2m'^2)
,
\ee

\beml{cs-t-expressions}
t_+ - t_- & \approx & s\sqrt{1 - 4\rho}, 
\nel
t_+ + t_- & \approx & s(2\rho - 1),
\nel
t_+ t_- & \approx & m'^4
.
\ee

\P
One may also find the integrals,

\bel{csInt1}
\int_{t_-}^{t_+} (2m'^2 - 2t - s) dt 
\; \approx \; 
0
,
\ee

\bel{csInt2}
\int_{t_-}^{t_+} (m'^2 - t)^2 dt 
\; \approx \;
{s^3 \over 3} (1 - \rho)\sqrt{1 - 4\rho} \,
,
\ee

\bel{csInt3}
\int_{t_-}^{t_+} \Re\left({z \over t - x}\right) dt
\; \approx \;
\Re(z)\ln\abs{t_+ - x \over t_- - x}
- \Im(z) \phi(x),
\ee

where $z$ is an arbitrary complex number, 
$x$ is a complex number such that no pole is encountered in~\rf{csInt3}
and
\bel{csImLog}
\phi(x) \; = \;
\arctg{s\sqrt{1 - 4\rho}\,\Im(x)
         \over m'^4 + \abs{x}^2 - s(2\rho - 1)\Re(x)}
.
\ee

Furthermore
\beml{csInt4}
\int_{t_-}^{t_+} \Re\left(z{(t - r)^2 \over t - x}\right) dt 
& \approx &
\Re[z(x - r)^2] \ln\abs{t_+ - x \over t_- - x}
-\Im[z(x - r)^2] \phi(x)
\nel
&&{}
+ s\sqrt{1 - 4\rho}\{\Re(zx) - \Re(z)[2r + s(1 - 2\rho)/2]\}
,
\nel
\ee

\beml{csInt65}
\int_{t_-}^{t_+} {(t - r)^2 \over \abs{t - x}^2} dt
&\approx&
{[\Re(x) - r]^2 - \Im(x)^2 \over \Im(x)} \phi(x)
+ 2[\Re(x) - r]\ln\abs{t_+ - x \over t_- - x}
\nel
&&{}
+ s\sqrt{1 - 4\rho}\, 
.
\ee

For a real $x$ the integral~\rf{csInt65} reads
\beml{csInt7}
\int_{t_-}^{t_+} {(t - r)^2 \over (t - x)^2} dt
&\approx&
{s\sqrt{1 - 4\rho}[x - r]^2 \over m_\chi^4 + x^2 - sx(2\rho - 1)}
+ 2(x - r)\ln\abs{t_+ - x \over t_- - x}
+ s\sqrt{1 - 4\rho}\,
.
\nel
\ee


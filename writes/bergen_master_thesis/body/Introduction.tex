\chapter{Introduction}

Although at the present time the Standard Model (SM) shows good agreement
with all available experimental data, there are several reasons 
to expect the existence of new physics beyond the SM. 

\P
First, to calculate measurable quantities the theory should be 
renormalized. Such procedure inevitably introduces some energy scale
at which the theory can not be applied any more. Although this scale can be 
set to an arbitrary high value, it means that the SM can be viewed 
only as a low energy approximation of some other theory.

\P 
The second problem comes from the fact that the SM does not describe 
gravitation. At the energy scale of $10^{19}$ GeV the gravitational 
interaction between particles becomes as strong as any other interaction. 
It is expected that at this energy the effects of quantum gravity will play 
an important role. 

\P 
During the last years several approaches were developed to overcome 
these problems. Most of them contain a new symmetry between fermions and 
bosons called {\it supersymmetry}~\cite{supersymmetry-appearance}. 
Moreover, it seems that only supersymmetric 
theories can be used to unify gravity with other fundamental 
forces. Still such theories require a renormalization to be carried out.

\P
Among other features the supersymmetric models predict the existence
of new particles that are the superpartners of the ordinary particles 
of the SM. It is expected that such particles can be detected at future
colliders. For this reason it is very important to study the properties 
of the new particles and processes that can be used to detect them.

\P 
For example, the simplest supersymmetric generalization of the SM
called the {\it Minimal Supersymmetric Standard Model} 
(MSSM)~\cite{howCharginosCome} contains new 
fermions called charginos and neutralinos 
that are Dirac and Majorana particles. They are of particular interest due 
to their coupling with leptons and gauge bosons that makes it 
possible to detect them in $e^+e^-$ collisions.

\P
In this thesis I consider some properties of Majorana fields and calculate 
the production cross-section for the lightest charginos.

\P
In chapter 2, I consider various properties of the Dirac algebra
and massive fermion fields in a quite general way. 
I prove that electric and magnetic dipole moments
of Majorana particles are zero. The proof uses only basic properties of 
the Dirac matrices and does not refer for example to the CPT theorem.  

\P Chapter 3 deals mostly with the quadratic and cubic terms 
in the MSSM Lagrangian that give masses to the charginos and scalar neutrinos 
and describes the coupling between these particles and the electromagnetic and 
$Z^0$ fields.  

\P The calculation of the lightest chargino production cross section is the 
topic of Chapter 4. I perform the calculation in the general case of
CP violation in the chargino sector.

\P
In chapter 5 I present numerical results for the chargino production 
cross section.








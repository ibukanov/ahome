\chapter{Charginos and Sleptons in MSSM}

\section{Charginos}

By definition, charginos $\chi_i$ are the two mass eigenstates formed
by the mixing of the fermionic partners of the charged gauge bosons (winos)
and those of the charged Higgs bosons~\cite{howCharginosCome}.
Consequently the charginos are described by the following quadratic terms 
in the MSSM Lagrangian that in terms of two component Weyl spinors 
read (see~\cite{mainBook}, page 47.)
\beml{lagran-free-chargino-1}
L & = & 
   -i\psi_1^-\sigma^a\partial_a\bar{\psi}_1^-  
   -i\psi_2^+\sigma^a\partial_a\bar{\psi}_2^+  
   -i\Lambda^-\sigma^a\partial_a\bar{\Lambda}^-  
   -i\Lambda^+\sigma^a\partial_a\bar{\Lambda}^+  
\nel &&
- \pba{cc}-i\Lambda^+ & \psi_2^+ \pea B
  \pba{c}-i\Lambda^- \\ \psi_1^- \pea
- \pba{cc}-i\bar{\Lambda}^+ & \bar{\psi}_2^+ \pea \bar{B}
  \pba{c}-i\bar{\Lambda}^- \\ \bar{\psi}_1^- \pea.
\ee


%(ALERT I should write the same by the terms of Dirac matrixes 
%- it would be better)

Here the fields $\Lambda^+$ and $\Lambda^-$ are charged SU(2) gaugino fields. 
They are defined through
\be 
\Lambda^+ \equiv \frac{1}{\sqrt{2}}(\Lambda^1 - i\Lambda^2)
\quad \mbox{and} \quad 
\Lambda^- \equiv \frac{1}{\sqrt{2}}(\Lambda^1 + i\Lambda^2).
\ee

The fields $\psi_1^-$ and $\psi_2^+$ are charged Higgsino fields 
and $\Lambda^1$, $\Lambda^2$ are $SU(2)$ gaugino fields. 
The mass matrix $B$ at the tree level is completely determined 
by three parameters: $\mu$, the mass coupling strength between the two Higgs
superfields; $\tan\beta$, the ratio of the vacuum expectation values
of the two Higgs doublets; and $m_{\tilde\Lambda}$, the $SU(2)$ gaugino
mass. The later quantity comes from the soft breaking terms in 
MSSM Lagrangian. These terms reflect the fact that supersymmetry cannot be 
an exact symmetry of nature, else the supersymmetric partners 
would have the same masses as the normal particles. In the absence 
of a fundamental understanding of the origin of supersymmetry breaking
one consider all possible terms that violate the SUSY and do not introduce 
quadratic divergences, the most trouble part the Standard Model. 

\P
The matrix $B$ has the following structure:
\bel{lagran-B-is}
B =  \pba{cc} 
   m_{\tilde\Lambda} & \sqrt{2}m_W \cos\beta \\
   \sqrt{2}m_W \sin\beta  & \mu 
\pea.
\ee

If the CP symmetry is violated in the chargino sector then $\mu \ne \mu^*$
and $B$ would not be real matrix.  

\P
To construct from the quadratic charged Higgsino and gaugino terms two free
Dirac fields I introduce the mass eigenstates as 
(\cite{mainBook}, page 48)
\bel{lagran-chargino-construction}
\pba{c} \chi_1^+ \\ \chi_2^+ \pea = 
   X\pba{c}-i\Lambda^+ \\ \psi_2^+ \pea
\quad \mbox{and} \quad
\pba{c} \chi_1^- \\ \chi_2^- \pea = 
   Y\pba{c}-i\Lambda^- \\ \psi_1^- \pea
\ee

The $2\times 2$ matrixes $X$ and $Y$ must be unitary to have standard  kinetic
terms for $\chi$ and should fulfill
\be
X^* BY^\hc = \pba{cc} -m_{\chi_1} & 0 \\ 0 & -m_{\chi_2} \pea 
.
\ee

According to appendix \rf{appMatrix} I have for the chargino masses 
from~\rf{lambdaForCompleMatrix} 
\beml{charginoMasses}
\lefteqn{
m_{\chi_1}^2 \; = \; S + \sqrt{T}, 
\qquad
m_{\chi_2}^2 \; = \; S - \sqrt{T}, 
}
\nel
S  &=& \frac{m_{\tilde\Lambda}^2 + \abs{\mu}^2}{2} + m_W^2
\nel
T &=& \frac{(m_{\tilde\Lambda}^2 + 2m_W^2\cos^2\beta - 
     2m_W^2\sin^2\beta - \abs{\mu}^2)^2}{4} 
\nel && {}
   + \abs{m_{\tilde\Lambda}\sqrt{2}m_W\sin\beta 
      + \sqrt{2}m_W\cos\beta\bar{\mu}}^2
\nel & = &
\frac{(m_{\tilde\Lambda}^2 - \abs{\mu}^2)^2}{4}
+ m_W^4(\cos^2\beta - \sin^2\beta)^2 
+ m_W^2(\cos^2\beta - \sin^2\beta)(m_{\tilde\Lambda}^2 - \abs{\mu}^2)
\nel && {}
+ 2m_W^2(m_{\tilde\Lambda}\sin\beta + \bar{\mu}\cos\beta)
    (m_{\tilde\Lambda}\sin\beta + \mu\cos\beta)
\nel & = &
\frac{(m_{\tilde\Lambda}^2 - \abs{\mu}^2)^2}{4}
+ m_W^4\cos^2(2\beta) 
\nel && {}
+ m_W^2\big[
   m_{\tilde\Lambda}^2(\cos^2\beta - \sin^2\beta)
  -\abs{\mu}^2(\cos^2\beta - \sin^2\beta)
\nel && \qquad {}
  + 2m_{\tilde\Lambda}^2\sin^2\beta
  + 2\abs{\mu}^2\cos^2\beta
  +2m_{\tilde\Lambda}\cos\beta\sin\beta(\mu + \bar{\mu})
\big] 
\nel & = &
\frac{(m_{\tilde\Lambda}^2 - \abs{\mu}^2)^2}{4}
+ m_W^4\cos^2(2\beta) 
+ m_W^2\big[
   m_{\tilde\Lambda}^2 + \abs{\mu}^2
  +2m_{\tilde\Lambda}\sin(2\beta)\Re(\mu)
\big]. 
\nel
\ee

The matrices $X$ and $Y$ are given by~\rf{matrixXandY}
\bel{lagran-X-andY-is}
X^* 
\; = \;
\pba{cc} 
   \cos\theta_X & \sin\theta_X e^{i\phi_X} \\
   -\sin\theta_X e^{-i\phi_X} & \cos\theta_X
\pea,
\ee
\bel{lagXandY}
Y 
\; = \;
\pba{cc} 
   \cos\theta_Y e^{i\psi_Y} & \sin\theta_Y e^{i\phi_Y} e^{i\psi_Y} \\
   -\sin\theta_Y e^{-i\phi_Y}e^{i\psi'_Y} & \cos\theta_Y e^{i\psi'_Y}
\pea,
\ee

with the factors given by~\rf{matrixXandYcoef} and~\rf{lagran-B-is} as

\be 
e^{i\phi_X} = \zOverAbs{m_{\tilde\Lambda}\sin\beta + \mu^* \cos\beta}, 
\quad
e^{i\phi_Y} = \zOverAbs{m_{\tilde\Lambda}\cos\beta + \mu \sin\beta}, 
\ee
\be
e^{i\psi_Y}
= 
  -\frac{m_{\tilde\Lambda}\cos\theta_X 
     + \sqrt{2}m_W \sin\beta\sin\theta_X e^{i\phi_X}}{m_{\chi_1}\cos\theta_Y}
        ,
\ee
\be
e^{i\psi'_Y} = 
\frac{-\sqrt{2} m_W \cos\beta \sin\theta_X e^{-i\phi_X} + \mu \cos\theta_X}
     {m_{\chi_2}\cos\theta_Y}
      .
\ee

Furthermore
\bel{lagrangian-theta-X-and-Y}
\tan\theta_X = \sqrt{1 + \eta_X^2} - \eta_X, 
\quad
\tan\theta_Y = \sqrt{1 + \eta_Y^2} - \eta_Y, 
\ee

and
\bel{lagrangian-eta-X-and-Y}
\eta_X = 
   \frac{m_{\tilde\Lambda}^2 - \abs{\mu}^2 + 2m_W^2\cos(2\beta)}
        {2\sqrt{2}m_W\abs{m_{\tilde\Lambda}\sin\beta + \mu^* \cos\beta}},
\quad 
\eta_Y = 
   \frac{m_{\tilde\Lambda}^2 - \abs{\mu}^2 - 2m_W^2\cos(2\beta)}
        {2\sqrt{2}m_W\abs{m_{\tilde\Lambda}\cos\beta + \mu^* \sin\beta}}
        .
\ee

\P
In terms of Dirac spinors the mass eigenstates are
\bel{lagran-chargino-as-Dirac-spinors}
\Psi_{\chi_1} = \pba{c} \chi_{1\alpha}^+ \\ \bar{\chi}_1^{-\dot\alpha} \pea,
\quad
\Psi_{\chi_2} = \pba{c} \chi_{2\alpha}^+ \\ \bar{\chi}_2^{-\dot\alpha} \pea
\ee

and the free chargino 
Lagrangian terms have the usual form of free Dirac fields,
\bel{lagran-free-charginos}
L = \bar{\Psi}_{\chi_1}(i\gu\mu\partial_\mu - m_{\chi_1}) {\Psi}_{\chi_1}
  + \bar{\Psi}_{\chi_2}(i\gu\mu\partial_\mu - m_{\chi_2}) {\Psi}_{\chi_2} 
  .
\ee

%\subsection{Neutralinos}
%
\section{Scalar neutrinos}

A possible channel for the chargino production process in $e^+$, $e^-$
collisions involves the exchange of a virtual scalar electron neutrino
$\tilde{\nu}_l$, the neutral scalar particle that 
is the supersymmetric partner of the electron neutrino.
There are two types of sneutrinos, i.e. right- and left-handed ones,
$\tilde{\nu}_l^R$ and $\tilde{\nu}_l^L$. 
In the framework of the MSSM the quadratic term in the Lagrangian 
that gives masses for sneutrinos reads

\be
\Lc \; = \;
 -\pmatrix{ \tilde{\nu}_l^{L\hc} & \tilde{\nu}_l^{R\hc}}
\Mc_{\tilde{\nu}} \pmatrix{{\tilde{\nu}_l^L} \cr {\tilde{\nu}_l^R}}
.
\ee 

The mass matrix has the following form (see~\cite{mainBook}, page 45)
\bel{sneutrinoMassMatrix}
\Mc_{\tilde{\nu}} = \pmatrix{\displaystyle
           \tilde{M}_U^2 + {m_Z^2 \over 2} \cos(2\beta) & 0 \cr
              0 & \tilde{m}_U^2 }.
\ee

As in the case of $m_{\tilde\Lambda}$ the mass parameters $\tilde{M}_U^2$
and $\tilde{m}_U^2$ are introduced by quadratic soft breaking terms in the
MSSM Lagrangian.

\P
The diagonal form of $\Mc_{\tilde{\nu}}$ comes from the assumption
that neutrinos are massless. Otherwise one would have non-diagonal 
$\Mc_{\tilde{\nu}}$ and mixing between $\tilde{\nu}^L$ and $\tilde{\nu}^L$.

\section{Interaction Lagrangian density}

The interactions of charginos with gauge bosons and 
with charged leptons and scalar neutrinos are given by
by~(see~\cite{mainBook}, page 67-69)
\beml{charginoCoupling}
L[\chi,\gamma] & = & 
  \sum_{i = 1..2} -\overline{\psi}_{\chi_i} e\gu{\rho} \psi_{\chi_i} A_\rho,

\nel

L[\chi, Z] & = &
    \frac{g}{4 \cos \theta_W} \sum_{i,j = 1..2}  
    
    \overline{\psi}_{\chi_j} \gu{\rho} \Bigl\lbrace 
          [2\delta_{ji} \cos (2\theta_W) 
            + Y_{i1}Y_{1j}^\hc  + X_{j1}X_{1i}^\hc 
        
\nel && \qquad {} 
          + \gu5 [Y_{i1}Y_{1j}^\hc  - X_{j1}X_{1i}^\hc ]\Bigr\rbrace

    \psi_{\chi_i} Z_\rho,
        
\ee

\bel{ChiLSNu}
L[l, \tilde{\nu}_l, \chi] = 
  \frac{g}{2} \sum_{l,i, p} 
       [- \psi^T_{\chi_i}C (\Vc_{lip} 
              + \Ac_{lip}\gd{5}) \psi_{l} (\tilde{\nu}_l^p)^\hc
       + \overline{\psi}_l (\Vc^\hc_{lip} 
                - \Ac^\hc_{lip}\gd{5})
                 C^{-1}\overline{\psi}_{\chi_i}^T \tilde{\nu}_l^p]
                 ,
\ee

where in~\rf{ChiLSNu} 
the sum is taken over two chargino generations $i$, $i = 1..2$, 
all lepton flavors $l$ and two sneutrino types $p$, 
$p = R,L$\footnote{
In~\rf{ChiLSNu} I have written the minus sign before the first term
as opposite to~\cite{mainBook}. With the plus sign
the Lagrangian term $L[l, \tilde{\nu}_l, \chi]$ would not be Hermitian.
}. I also use the convention $e = \abs{e}$, i.e.\ $e$ stands for the 
absolute value of the unit electric charge and $g = e / \sin\theta_W > 0$.

\P

Under the assumption that all neutrinos are massless, the coefficients 
in~\rf{ChiLSNu} read, 

\bem
\Vc_{liL} & = &
-X_{1i}^\hc  - Y_{i2}\frac{m_l}{\sqrt{2} m_W \cos\beta} 
,
\nel
\Vc_{liR} & = &
0,

\nel
\Ac_{liL} & = &

X_{1i}^\hc  - Y_{i2}\frac{m_l}{\sqrt{2} m_W \cos\beta}
,

\nel
\Ac_{liR} & = &

0.

\ee

\P
The zero values of $\Ac_{lip}$ and $\Vc_{lip}$ when $p=R$ 
means that cubic interaction between leptons, charginos and sneutrinos 
involves only one kind of sneutrinos, thus
\bel{lagranChiLSNu}
L[l, \tilde{\nu}_l, \chi] = 
  \frac{g}{2} \sum_{l,i} 
       [- \psi^T_{\chi_i}C (\Vc_{liL} 
              + \Ac_{liL}\gd{5}) \psi_{l} (\tilde{\nu}_l^R)^\hc
       + \overline{\psi}_l (\Vc^\hc_{liL} 
                - \Ac^\hc_{liL}\gd{5})
                 C^{-1}\overline{\psi}_{\chi_i}^T \tilde{\nu}_l^L]
.
\ee

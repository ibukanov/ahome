\chapter{Properties of Dirac matrices}\label{GammaApp}

The aim of this chapter is to show that all necessary properties
of the Dirac gamma matrices that are used in this work can be deduced from
the basic definition of the Dirac algebra in a representation-independent
way. I also will not suppose until section~\ref{gammaSectionDiracEquation}
that $g_{\mu\nu} = \eta_{\mu\nu}$, where $\eta_{\mu\nu}$ is the standard 
Minkowski metric. I present several simple applications
of the obtained results.

\section{Dirac algebra}

The $4\times 4$ matrices $\gu{\mu}$, $\mu=0..3$ are called Dirac matrices
if they satisfy the Dirac algebra,

\bel{gammaAlgebra}
\gu{\mu}\gu{\nu} + \gu{\nu}\gu{\mu} = 2 g^{\mu\nu}
\quad \mbox{or} \quad \{ \gu{\mu}, \gu{\nu}\} = 2 g^{\mu\nu},
\ee

where $g^{\mu\nu}$ is the metric tensor.
\P
The solution of \rf{gammaAlgebra} is not unique but for any two sets
$\gamma^{\prime\mu}$ and $\gamma^{\mu}$ there is a non-singular matrix $O$,
defined up to an arbitrary scalar factor, such that
\bel{gammaTransLaw}
\gamma^{\prime\mu} = O\gamma^{\mu}O^{-1}, \quad \mu = 0..3.
\ee

For the proof, see, for example, \cite{gammaTransLawProof}.

\P
By definitions, $\gd\mu = g_{\mu\nu}\gu\mu$. They satisfy
\bel{gammaDownAlgebra}
\{ \gd\mu , \gd\nu \} = 2 g_{\mu\nu},
\quad
\{ \gd\mu , \gu\nu \} = 2 \delta_\mu^\nu.
\ee

\P
To prove many identities that involve the Dirac matrices it is sometimes
helpful to use the following two-step approach.
\begin{itemize}
\item
Prove that the identity does not depend on a particular choice of $\gu\mu$.
\item
Show that the identity is valid in a certain representation.
\end{itemize}

To construct such reference representation $\gamma_r^\mu$ that can be used 
at the second step for an arbitrary $g^{\mu\nu}$ 
I consider the canonical Dirac representation $\gamma^\mu_D$ 
of the algebra~\rf{gammaAlgebra} with the usual Minkowski space-time
metric tensor $\eta^{\mu\nu}$. By definition, $\gamma^\mu_D$ 
satisfy
\bel{gamma-canonical}
\{ \gamma_D^{\mu}, \gamma_D^{\nu}\} = 2 \eta^{\mu\nu}
,
\ee

and are given by
\bel{gamma-standard-representation}
\gamma_D^0 = \pmatrix{\sigma^0 & 0 \cr 0 & -\sigma^0},
\quad
\gamma_D^{i} = \pmatrix{0 & \sigma^i \cr - \sigma^{i} & 0},
\quad i = 1,2,3,
\ee

where $\sigma^0$ and $\sigma^i$ are the usual Pauli 
matrices,
\bel{gamma-Pauli-matrices}
\sigma^0 = \pmatrix{ 1 & 0 \cr 0 & 1 },
\quad
\sigma^1 = \pmatrix{ 0 & 1 \cr 1 & 0 },
\quad
\sigma^2 = \pmatrix{ 0 & -i \cr i & 0 },
\quad
\sigma^3 = \pmatrix{ 1 & 0 \cr 0 & -1 }.
\ee

Then the matrices 
\bel{gamma-reference-representation}
\gamma_r^\mu = L_\rho{}^\mu \gamma_D^\rho ,
\ee

where $L_\rho{}^\mu$ is defined 
by~\rf{gamma-metric-transformation} 
and~\rf{gamma-metric-transformation2}, satisfy
\bem
\{ \gamma_r^\mu, \gamma_r^\nu \} & = &
\{ L_\rho{}^\mu \gamma_D^\rho, L_\sigma{}^\nu \gamma_D^\sigma \}
=
L_\rho{}^\mu L_\sigma{}^\nu \{\gamma_D^\rho, \gamma_D^\sigma \}
\nel
& = &
L_\rho{}^\mu L_\sigma{}^\nu  2\eta^{\rho\sigma}
 =  2g^{\mu\nu}
,
\ee
i.e. $\gamma_r^\mu$ 
form a representation of~\rf{gammaAlgebra}.

\P
According to~\rf{gammaTransLaw} one can find $O_L$ such that
\be
\gamma^{\mu} = O_L \gamma_r^\mu O_L^{-1}
.
\ee

Thus $\gamma^{\mu}$ can always be written as 
\bel{gamma-reference-representation2}
\gamma^{\mu} = L_\rho{}^\mu O_L \gamma_D^\rho O_L^{-1}
.
\ee


\P

An example application of~\rf{gamma-reference-representation2} 
is the proof that the matrix $\gd5\equiv \gu5$, defined by
\beml{gammaFiveDef} 
\gd5 & = & -i \frac{1}{4!}
       \varepsilon_{\mu\nu\rho\sigma} \gu\mu \gu\nu \gu\rho \gu\sigma
     = i (-g)^{\frac{1}{2}} \gu{[0} \gu1 \gu2 \gu{3]}
\nel &= & -i \frac{1}{4!}
       \varepsilon^{\mu\nu\rho\sigma} \gd\mu \gd\nu \gd\rho \gd\sigma 
     = -i (-g)^{-\frac{1}{2}} \gd{[0} \gd1 \gd2 \gd{3]},
\ee
where $\varepsilon^{\mu\nu\rho\sigma}$ is defined by~\rf{convEpsilonUpIs},
anti-commutes with $\gu\mu$.

\P

Indeed, I may write $\gd5$ as 
\bem
\gd5 & = & -i \frac{1}{4!}
       \varepsilon_{\mu\nu\rho\sigma} 
       L_{\mu'}{}^\mu  L_{\nu'}{}^\nu L_{\rho'}{}^\rho L_{\sigma'}{}^\sigma
\nel&& {} \times       
       (O_L \gamma_D^{\mu'} O_L^{-1})
       (O_L \gamma_D^{\nu'} O_L^{-1})
       (O_L \gamma_D^{\rho'} O_L^{-1})
       (O_L \gamma_D^{\sigma'} O_L^{-1})
\nel
& = &
 -i \frac{1}{4!}
       \varepsilon_{\mu\nu\rho\sigma} 
       L_{\mu'}{}^\mu  L_{\nu'}{}^\nu L_{\rho'}{}^\rho L_{\sigma'}{}^\sigma
       O_L \gamma_D^{\mu'} \gamma_D^{\nu'} \gamma_D^{\rho'} \gamma_D^{\sigma'} 
       O_L^{-1}.
\ee

According to~\rf{convEpsilonDownIs} and~\rf{convDetIs} I have
\bem
\varepsilon_{\mu\nu\rho\sigma} 
L_{\mu'}{}^\mu  L_{\nu'}{}^\nu L_{\rho'}{}^\rho L_{\sigma'}{}^\sigma
& = &
-\sqrt{-g} \, e_{\mu\nu\rho\sigma}
L_{\mu'}{}^\mu  L_{\nu'}{}^\nu L_{\rho'}{}^\rho L_{\sigma'}{}^\sigma
\nel
& = &
-\sqrt{-g} \, \det(L) e_{\mu'\nu'\rho'\sigma'}
= \det(L) \varepsilon_{\mu'\nu'\rho'\sigma'}.
\ee

So 
\be
\gd5 = -i \frac{1}{4!} \varepsilon_{\mu\nu\rho\sigma} 
     \det(L) O_L 
     \gamma_D^{\mu} \gamma_D^{\nu} \gamma_D^{\rho} \gamma_D^{\sigma} O_L^{-1}
     = i \sqrt{-g} \det(L) 
     O_L \gamma_D^{[0} \gamma_D^1 \gamma_D^2 \gamma_D^{3]} O_L^{-1}.
\ee

By taking the determinant of both sides in~\rf{gamma-metric-transformation}
I find
\be
\det(\eta) = \det(L)^2 \det(g) 
\quad \mbox{or} \quad 
\det(L) = \pm (-g)^{-{1\over 2}}.
\ee

For these reasons
\be
\gd5 =\pm iO_L \gamma_D^{[0} \gamma_D^1 \gamma_D^2 \gamma_D^{3]} O_L^{-1}
.
\ee

From~\rf{gamma-canonical} and the structure~\rf{convCartesianMetric} 
of $\eta^{\mu\nu}$ it follows that 
\be
\{\gamma_D^\mu , \gamma_D^\nu\} = 0, \quad \mu \ne \nu
,
\ee

i.e. $\gamma_D^\mu$ anticommutes
with $\gamma_D^\nu$, $\mu \ne \nu$. Thus
\be
\gamma_D^{[0} \gamma_D^1 \gamma_D^2 \gamma_D^{3]} 
= \gamma_D^{0} \gamma_D^1 \gamma_D^2 \gamma_D^{3}
.
\ee

It means that I can always represent $\gd5$ as
\bel{gamma-5-via-canonical}
\gd5 = \pm iO_L \gamma_D^{0} \gamma_D^1 \gamma_D^2 \gamma_D^{3} O_L^{-1}
\equiv \pm O_L \gamma_D^5 O_L^{-1},
\ee

where by definition
\bel{gamma-5-canonical}
\gamma_D^5 \equiv i\gamma_D^{0} \gamma_D^1 \gamma_D^2 \gamma_D^{3}.
\ee

The matrix $\gamma_D^5$ anticommutes with $\gamma_D^\mu$
because $\gamma_D^{0} \gamma_D^1 \gamma_D^2 \gamma_D^3$
contains $3$ matrices that anti-commute
with $\gamma_D^\mu$, $\mu = 0..3$ and one that commutes ($\gamma_D^\mu$). 
The net result is
\bel{gamma-5-anticommute}
\{ \gamma_D^5, \gamma_D^\mu \} = 0.
\ee

Now I have
\beml{gammaFiveAnti}
\{\gd5, \gu\mu\} 
&=& \pm \{ O_L \gamma_D^5 O_L^{-1}, L_\rho{}^\mu O_L \gamma_D^\rho O_L^{-1}\}
= \pm L_\rho{}^\mu O_L \{ \gamma_D^5, \gamma_D^\rho \} O_L^{-1}
\nel
& = & 0.
\ee

\P
The result~\rf{gamma-5-via-canonical} also may be used to find
$(\gd5)^2$,
\be
(\gd5)^2 = (\pm O_L \gamma_D^5 O_L^{-1})^2 = O_L (\gamma_D^5)^2 O_L^{-1}.
\ee

The calculation of $(\gamma_D^5)^2$ is straightforward,
\be
(\gamma_D^5)^2 = 
- \gamma_D^{0} \gamma_D^1 \gamma_D^2 \gamma_D^{3}
 \gamma_D^{0} \gamma_D^1 \gamma_D^2 \gamma_D^{3}
= -(\gamma_D^{0})^2 (\gamma_D^1)^2 (\gamma_D^2)^2 (\gamma_D^3)^2
= 1.
\ee

Thus 
\bel{gammaFiveSquared}
(\gd5)^2 = O_L O_L^{-1} = 1\footnote{
This result depends on the definition~\rf{gammaFiveDef} of $\gu5$. If one
sets instead of~\rf{gammaFiveDef} 
\be
\gd5 = (-g)^{\frac{1}{2}} \gu{[0} \gu1 \gu2 \gu{3]}
\ee 
then $\gamma_D^5 = \gamma_D^{0} \gamma_D^1 \gamma_D^2 \gamma_D^{3}$
and $\gd5^2 = -1$.
}
.
\ee


\section{Traces}

From \rf{gammaFiveAnti} and \rf{gammaFiveSquared} it follows that the trace
of an odd number of $\gu\mu$ matrices is zero,
\bem
\Tr(\gu{\mu_1}..\gu{\mu_{2k+1}})
& = &
\Tr(\gu{\mu_1}..\gu{\mu_{2k+1}}\gd5^2)
    \quad \mbox{(from $\Tr(ab) = \Tr(ba)$)}
\nel & = &
\Tr(\gd5\gu{\mu_1}..\gu{\mu_{2k+1}}\gd5)
\nel & = &
-\Tr(\gu{\mu_1}..\gu{\mu_{2k+1}}\gd5\gd5)
= 0.

\ee
It can be extended to the trace of an odd number of $\gu\mu$
and any number of $\gd5$ because $\gd5$ is a sum of 
products of four Dirac matrices.

For other important cases I have,

\bem
\Tr(\gu\mu\gu\nu) & = & \frac{1}{2}[\Tr(\gu\mu\gu\nu) + \Tr(\gu\nu\gu\mu)]
  = \frac{1}{2}\Tr(\gu\mu\gu\nu + \gu\nu\gu\mu)
\nel &=& \frac{1}{2}\Tr(2g^{\mu\nu}) = 4 g^{\mu\nu},

\nel
\Tr(\gu{[\mu}\gu{\nu]}) & = & 4 g^{[\mu\nu]} = 0,

\nel
\Tr(\gu\mu\gu\nu\gu\rho\gu\sigma)  & = &

\Tr(\gu\nu\gu\rho\gu\sigma\gu\mu)
\nel &=&
  2g^{\sigma\mu}\Tr(\gu\nu\gu\rho) - \Tr(\gu\nu\gu\rho\gu\mu\gu\sigma)

\nel &=&
  8g^{\sigma\mu}g^{\nu\rho}
  - 4g^{\rho\mu}\Tr(\gu\nu\gu\sigma)
  + \Tr(\gu\nu\gu\mu\gu\rho\gu\sigma)

\nel &=&
  8(g^{\sigma\mu}g^{\nu\rho} - g^{\rho\mu}g^{\nu\sigma})
  + 2g^{\nu\mu}\Tr(\gu\rho\gu\sigma)
  -\Tr(\gu\mu\gu\nu\gu\rho\gu\sigma)

\nel &=&
  8(g^{\mu\nu}g^{\rho\sigma} - g^{\mu\rho}g^{\nu\sigma}
    + g^{\mu\sigma}g^{\nu\rho})
  -\Tr(\gu\mu\gu\nu\gu\rho\gu\sigma),

\nel
\Tr(\gu\mu\gu\nu\gu\rho\gu\sigma)  & = &
  4(g^{\mu\nu}g^{\rho\sigma} - g^{\mu\rho}g^{\nu\sigma}
    + g^{\mu\sigma}g^{\nu\rho}),

\nel
\Tr(\gd5)  & = &
- \frac{i}{4!} \varepsilon_{\mu\nu\rho\sigma}
  \Tr(\gu\mu \gu\nu \gu\rho \gu\sigma)

\nel
& = &
-i \frac{4}{4!} \varepsilon_{\mu\nu\rho\sigma}
  (g^{\mu\nu}g^{\rho\sigma} - g^{\mu\rho}g^{\nu\sigma}
    + g^{\mu\sigma}g^{\nu\rho})
\nel
 & = & 0 \quad
 \mbox{(from the antisymmetry of $\varepsilon_{\mu\nu\rho\sigma}$)},

\nel
\Tr(\gu{\mu}\gu{\nu}\gd5)  & = &
- \frac{i}{4!} \varepsilon_{\eta\theta\rho\sigma}
  \Tr(\gu\mu \gu\nu \gu\eta \gu\theta \gu\rho \gu\sigma)

\nel  & = &
- \frac{i}{4!} \varepsilon_{\eta\theta\rho\sigma}
  \Tr(\gu\nu \gu\eta \gu\theta \gu\rho \gu\sigma \gu\mu)

\nel  & = &
- \frac{i}{4!} \varepsilon_{\eta\theta\rho\sigma}
  [2g^{\sigma\mu}\Tr(\gu\nu \gu\eta \gu\theta \gu\rho)
  -\Tr(\gu\nu \gu\eta \gu\theta \gu\rho \gu\mu \gu\sigma)],

\nel
\varepsilon_{\eta\theta\rho\sigma}
\Tr(\gu\nu \gu\eta \gu\theta \gu\rho) & = &

4\varepsilon_{\eta\theta\rho\sigma}
   (g^{\nu\eta}g^{\theta\rho} - g^{\nu\theta}g^{\eta\rho}
    + g^{\nu\rho}g^{\eta\theta}) = 0,

\nel
\Tr(\gu{\mu}\gu{\nu}\gd5)  & = &
\frac{i}{4!} \varepsilon_{\eta\theta\rho\sigma}
  \Tr(\gu\nu \gu\eta \gu\theta \gu\rho \gu\mu \gu\sigma)

=-\frac{i}{4!} \varepsilon_{\eta\theta\rho\sigma}
  \Tr(\gu\nu \gu\eta \gu\theta \gu\mu \gu\rho \gu\sigma)

\nel & = & \frac{i}{4!} \varepsilon_{\eta\theta\rho\sigma}
  \Tr(\gu\mu \gu\nu \gu\eta \gu\theta \gu\rho \gu\sigma)

\nel & = & -\Tr(\gu{\mu}\gu{\nu}\gd5) = 0.

\ee

To calculate $\Tr(\gu\mu\gu\nu\gu\rho\gu\sigma\gd5)$ I use another approach.
It is easy to see that this quantity is fully antisymmetric. For example,
for the exchange of the first two indices,
\bem
\Tr(\gu\mu\gu\nu\gu\rho\gu\sigma\gd5) & = &
-\Tr(\gu\nu\gu\mu\gu\rho\gu\sigma\gd5)
+ 2 g^{\mu\nu} \Tr(\gu\rho\gu\sigma\gd5)
\nel & = & -\Tr(\gu\nu\gu\mu\gu\rho\gu\sigma\gd5),
\ee
and the same holds 
for any other simple permutation. So it should be proportional
to $\varepsilon^{\mu\nu\rho\sigma}$,
\be
\Tr(\gu\mu\gu\nu\gu\rho\gu\sigma\gd5) = c \varepsilon^{\mu\nu\rho\sigma}.
\ee

To find the constant $c$ I contract the last equation with
$\varepsilon^{\mu\nu\rho\sigma}$ and use~\rf{convEpsilonUpIs} 
and~\rf{convEpsilonDownIs},

\be
\varepsilon_{\mu\nu\rho\sigma}\Tr(\gu\mu\gu\nu\gu\rho\gu\sigma\gd5)
=  \frac{4!}{-i}\frac{-i}{4!}
\varepsilon_{\mu\nu\rho\sigma}\Tr(\gu\mu\gu\nu\gu\rho\gu\sigma\gd5)
= 4! i \Tr(\gd5\gd5) = 4!4i,
\ee
\be
4!4i = c\varepsilon_{\mu\nu\rho\sigma}\varepsilon^{\mu\nu\rho\sigma}
=  4!c\varepsilon_{0123}\varepsilon^{0123}
= 4!c (-\sqrt{-g}){1 \over \sqrt{-g}}
= - 4!c,
\ee
\be
c =  -4i.
\ee


In summary,

\beml{gammaTraces}
\Tr(\gu\mu\gu\nu) & = & 4 g^{\mu\nu},

\nel
\Tr(\gu\mu\gu\nu\gu\rho\gu\sigma)  & = &
  4(g^{\mu\nu}g^{\rho\sigma} - g^{\mu\rho}g^{\nu\sigma}
    + g^{\mu\sigma}g^{\nu\rho}),

\nel
\Tr(\gu\mu\gu\nu\gu\rho\gu\sigma\gd5)
& = & -4i \varepsilon^{\mu\nu\rho\sigma},
\ee

\bel{gammaZeroTraces}
\Tr(\mbox{Product of odd $\gamma$})
= \Tr(\gd5) = \Tr(\gu{\mu}\gu{\nu}\gd5) = \Tr(\gu{[\mu}\gu{\nu]}) = 0.
\ee

\Pni

An application of~\rf{gammaTraces} and~\rf{gammaZeroTraces} is the
proof that the 16 matrices
\bel{gammaMatrixSet}
1_{4\times 4}, \quad \gu\mu,  \quad \sigma^{\mu\nu} = \frac{i}{2}\gu{[\mu}\gu{\nu]}, \quad
\gu\mu\gd5,  \quad \gd5
\ee
are linearly independent. To show it I consider the equation

\bem
X & \equiv & A 1_{4\times 4} + B_\mu \gu\mu + C_{\mu\nu} \sigma^{\mu\nu}
+ D_\mu \gu\mu\gd5 + E \gd5 = 0
\ee
with $C_{\mu\nu} = C_{[\mu\nu]}$.

Multiplying it by one of the matrices \rf{gammaMatrixSet} one finds that
$A = B_\mu = C_{\mu\nu} = D_\mu = E = 0:$

\bem
\Tr(X 1_{4\times 4}) & = & \Tr(X) = A \Tr(1_{4\times 4}) = 0,

\nel
\Tr(X \gu\rho) & = & B_\mu \Tr(\gu\mu \gu\rho) = 4 B^\rho = 0,

\nel
\Tr(X \sigma^{\rho\sigma}) & = &
   C_{\mu\nu} \Tr(\sigma^{\mu\nu}\sigma^{\rho\sigma})
   = -\frac{1}{4}C_{\mu\nu} \Tr(\gu{[\mu}\gu{\nu]}\gu{[\rho}\gu{\sigma]})

\nel & = &
-C_{\mu\nu}(g^{[\mu\nu]}g^{[\rho\sigma]}
    - g^{\eta[\mu}g^{\nu][\sigma} \delta^{\rho]}_\eta
    + g^{\eta[\mu}g^{\nu][\rho} \delta^{\sigma]}_\eta)

\nel & = &
2C_{\mu\nu}g^{\eta[\mu}g^{\nu][\sigma} \delta^{\rho]}_\eta
= C_{\mu\nu}(g^{\eta\mu}g^{\nu[\sigma}\delta^{\rho]}_\eta
   - g^{\eta\nu}g^{\mu[\sigma}\delta^{\rho]}_\eta)

\nel & = &
C_{\mu\nu}(g^{\nu[\sigma} g^{\rho]\mu}
   - g^{\mu[\sigma}g^{\rho]\nu}) = 2  C_{\mu\nu}g^{\nu[\sigma} g^{\rho]\mu}

\nel & = &
2C_{[\rho\sigma]} = 2C_{\rho\sigma} = 0.

\nel
\Tr(X \gu\rho\gd5)
& = & D_\mu \Tr(\gu\mu\gd5 \gu\rho\gd5) + E\Tr(\gd5\gu\rho\gd5)
\nel & = & -D_\mu \Tr(\gu\mu\gu\rho)
= 4 D^\rho = 0,

\nel
\Tr(X \gd5) & = & E \Tr(\gd5\gd5) =  4E = 0.

\ee

\P The linear independence of the 16 matrices \rf{gammaMatrixSet} means
that any $4\times 4$ matrix can be represented as a linear combination
of them. Also any matrix that commutes just with $\gu\mu$ , $\mu=0..3$
commutes with any of \rf{gammaMatrixSet} by construction of that set and hence
commutes with any matrix and should be
proportional to~$1_{4\times 4}$.



\section{Hermitian conjugation and transposition}

I have for the Dirac algebra \rf{gammaAlgebra},

\be
\{ \gu{\mu}, \gu{\nu} \}^\hc 
= \{ \gu{\nu}^\hc, \gu{\mu}^\hc \} 
= \{ \gu{\mu}^\hc, \gu{\nu}^\hc \}
= 2 g^{\mu\nu} 1_{4\times 4}^\hc = 2 g^{\mu\nu} 
, 
\ee

\be
\{ \gu\mu, \gu\nu \}^T 
= \{ \gu\nu^T, \gu{\mu}^T \} 
= \{ \gu{\mu}^T, \gu\nu^T \}  
= 2 g^{\mu\nu} 1_{4\times 4}^T = 2 g^{\mu\nu}
.
\ee

It means that $\cc{\gu{\mu}}$ and ${\gu{\mu}}^T$
form the same algebra as ${\gu{\mu}}$ and from \rf{gammaTransLaw}
one can conclude that there are matrices $B$ and $B_t$ such that
\bel{gammaComplexConj}
\gu{\mu}^\hc = B \gu{\mu} B^{-1}
,
\ee
\be
{\gu{\mu}}^T = B_t \gu{\mu} B_t^{-1}.
\ee

By convention instead of $B_t$ the matrix $C = B_t{\gd5}^{-1}$ 
is used such that
\be
{\gu{\mu}}^T = B_t \gu{\mu} B_t^{-1}
             = C \gd5 \gu{\mu} {\gd5}^{-1} C^{-1}
             = -C \gu{\mu} \gd5{\gd5}^{-1} C^{-1},
\ee
\bel{gammaTransport}
{\gu{\mu}}^T = -C \gu{\mu} C^{-1}.
\ee

\P
In commonly used representations of the Dirac algebra
$\gu{0}$ can play the role of $B$ but it
does not hold in the general case.

\P
The result of applying hermitian conjugation and transposition
to $\gd5$ can be obtained directly from \rf{gammaFiveDef},
\beml{gammaFiveUnder}
\cc{\gd5}
& = &
 -i (-g)^{\frac{1}{2}} {\gu{[3}}^\hc {\gu2}^\hc {\gu1}^\hc {\gu{0}}^\hc{}^]

\; = \;
-i (-g)^{\frac{1}{2}}  B {\gu{[3}} {\gu2} {\gu1} {\gu{0]}} B^{-1}
\nel
& = &
-i (-g)^{\frac{1}{2}}  B {\gu{[0}} {\gu1} {\gu2} {\gu{3]}} B^{-1}

\; = \; -B \gd5 B^{-1},

\nel

{\gd5}^T
& = &
 i (-g)^{\frac{1}{2}} {\gu{[3}}^T {\gu2}^T {\gu1}^T {\gu{0}}^T{}^]

\; = \;
i (-g)^{\frac{1}{2}}  C {\gu{[3}} {\gu2} {\gu1} {\gu{0]}} C^{-1}
\nel
& = &
C \gd5 C^{-1}.
\ee

\P
To establish some constraints
on the matrices $B$ and $C$ and find a relation between them
I use $\gamma = (\gamma^\hc)^\hc = (\gamma^T)^T$ together with
the following properties that hold for any non-singular matrix $M$,
\be
(M^{-1})^\hc M^\hc = (M M^{-1})^\hc = 1^\hc = 1 \Leftrightarrow  
(M^{-1})^\hc = (M^\hc)^{-1}, 
\ee
\be
(M^{-1})^T M^T = (M M^{-1})^T = 1^T = 1 \Leftrightarrow  
(M^{-1})^T = (M^T)^{-1}.
\ee

I have
\bem
\gu{\mu} &=& ({\gu\mu}^\hc)^\hc = (B\gu{\mu}B^{-1})^\hc
          = (B^{-1})^\hc {\gu{\mu}}^\hc B^\hc
= (B^{-1})^\hc B \gu{\mu} B^{-1} B^\hc,
\nel 
\gu{\mu} &=& ({\gu\mu}^T)^T = -(C\gu{\mu}C^{-1})^T
           =  -(C^{-1})^T \gu{\mu}^T C^T
=  (C^{-1})^T C \gu{\mu} C^{-1} C^T
.
\ee

Thus 
\bem
\gu{\mu}(B^{-1})^\hc B &=& (B^{-1})^\hc B \gu{\mu}
,
\nel
\gu{\mu}(C^{-1})^T C &=& (C^{-1})^T C \gu{\mu}
,
\ee
or 
\bel{gammaTempBCProp}
[\gu{\mu}, (B^{-1})^\hc B] = [\gu{\mu}, (C^{-1})^T C] = 0.
\ee

It means that $(B^{-1})^\hc B = (B^\hc)^{-1} B$ and 
$(C^{-1})^T C = (C^T)^{-1} C$ should be proportional
to the unit matrix or in other words 
$B^\hc = k_B B$, $C^T = k_C C$ with some complex $k_{B,C}$,
\bem &
B = k_B^\hc B^\hc = k_B^\hc k_B B \; \Rightarrow \; \abs{k_B}^2 = 1,
&\nel &
C = k_C C^T = k_C k_C C \; \Rightarrow \; k_C^2 = 1.
& \ee

So 
\bel{gamma-tmp-B-dagger-and-C-T}
\cc{B} = e^{i\phi_B}B, \quad C^T = \pm C.
\ee

Now I explore the fact that
 ${\gu\mu}^* = ({\gu\mu}^\hc)^T = ({\gu\mu}^T)^\hc$,

\be
 ({\gu\mu}^\hc)^T =  (B\gu\mu B^{-1})^T
    = (B^T)^{-1} {\gu\mu}^T B^T = -(B^T)^{-1} C \gu\mu C^{-1} B^T,
\ee
\be
({\gu\mu}^T)^\hc = -(C\gu\mu C^{-1})^\hc
    = -(C^\hc)^{-1} {\gu\mu}^\hc C^\hc = -(C^\hc)^{-1} B\gu\mu B^{-1} \cc{C}
.
\ee

Thus
\be
(C^\hc)^{-1} B\gu\mu B^{-1} C^\hc = (B^T)^{-1} C\gu\mu C^{-1} B^T,
\ee
\be
\gu\mu B^{-1} C^\hc = B^{-1}C^\hc (B^T)^{-1} C\gu\mu C^{-1} B^T,
\ee
\be
\gu\mu B^{-1} C^\hc (B^T)^{-1}C = B^{-1}C^\hc (B^T)^{-1} C\gu\mu,
\ee
\be
[\gu\mu, B^{-1} C^\hc (B^T)^{-1}C] = 0.
\ee

or 
\bel{gammaBCRelTmp1}
B^{-1} C^\hc (B^T)^{-1}C = k1_{4\times 4}
\ee
with some $k$. I may also write~\rf{gammaBCRelTmp1} as
\beml{gamma-tmp-C-dagger}
kC^{-1} & = & B^{-1} C^\hc (B^T)^{-1},
\nel
C^\hc & = & kBC^{-1}B^T.
\ee

Now I apply hermitian conjugation to~\rf{gamma-tmp-C-dagger} and 
use~\rf{gamma-tmp-B-dagger-and-C-T},

\be
C = k^\hc (B^\hc)^T (C^\hc)^{-1} B^\hc = e^{2i\phi_B} k^\ast B^T (C^\hc)^{-1}B,
\ee
\be
k = k C^{-1}C = B^{-1} C^\hc (B^T)^{-1} e^{2i\phi_B} k^* B^T (C^\hc)^{-1}B
  = k^* e^{2i\phi_B} ,
\ee

that gives
\be
k^2 = e^{2i\phi_B} k^* k = e^{2i\phi_B} \abs{k}^2
,
\ee

or simply
\be
k = \pm \abs{k} e^{i\phi_B}.
\ee

In equations~\rf{gammaComplexConj}--\rf{gammaTransport}
the matrices $B$, $C$ are defined up to some scale transformation,
\be
B \rightarrow B_s = s_B B, \quad C \rightarrow C_s = s_C C
.
\ee

Here the coefficients $s_B$, $s_C$ can be any non-zero complex numbers.
If I choose $s_B = \sqrt{\abs{k}}e^{i\phi_B/2}$ then
\be
B_s^\hc = \sqrt{\abs{k}}e^{-i\phi_B/2} B^\hc
= \sqrt{\abs{k}}e^{-i\phi_B/2}e^{i\phi_B}B
= \sqrt{\abs{k}}e^{i\phi_B/2}B = B_s
,
\ee

and from~\rf{gammaBCRelTmp1}
\be
B_s^{-1} C^\hc (B_s^T)^{-1}C
= (\abs{k}e^{i\phi_B})^{-1}B^{-1} C^\hc (B^T)^{-1}C
= (\abs{k}e^{i\phi_B})^{-1}k 
\ee

or
\be
B_s^{-1} C^\hc (B_s^T)^{-1}C 
= \pm (\abs{k}e^{i\phi_B})^{-1} \abs{k} e^{i\phi_B} = \pm 1
.
\ee

i.e. by an appropriate scale transformation one can always choose $B$
and $C$ such that
\bel{gamma-tmp-B-C-relation}
B^\hc = B, \quad C^T = \pm C,
\quad B^{-1} C^\hc (B^T)^{-1}C = \pm 1
.
\ee

This still does not fix $B$ and $C$,~\rf{gamma-tmp-B-C-relation} 
is invariant under
\bel{gamma-B-C-can-be-rescale}
B \rightarrow B_s = \abs{s} B, \quad C \rightarrow C_s = sC,
\ee

with an arbitrary non-zero complex $s$ .


\P
To find the signs in \rf{gamma-tmp-B-C-relation} I consider
the behavior of $B$ and $C$ under the transformation \rf{gammaTransLaw}.
According to~\rf{gammaComplexConj}--\rf{gammaTransport} the 
representation $\gamma'^{\mu}$ of the Dirac algebra has the matrices
$B'$ and $C'$ such that
\be
(\gamma'^{\mu})^\hc = B' \gamma'^{\mu} B'^{-1},
\quad
(\gamma'^{\mu})^T = - C' \gamma'^{\mu} C'^{-1}.
\ee

Then the statement~\rf{gammaTransLaw} permits me to write
\be
(O\gamma^{\mu}O^{-1})^\hc 
= (O^{-1})^\hc(\gamma^{\mu})^\hc O^\hc 
= B' O\gamma^{\mu}O^{-1} B'^{-1},
\ee
\be
(O^{-1})^\hc B \gamma^{\mu} B^{-1} O^\hc = B' O\gamma^{\mu}O^{-1} B'^{-1},
\ee
\be
\gamma^{\mu}B^{-1} O^\hc B' O = B^{-1} O^\hc B' O\gamma^{\mu},
\ee
\bel{gamma-tmp-B-B-prime1}
[\gamma^{\mu}, B^{-1} \cc{O} B' O] = 0.
\ee

In a similar way
\be
(O\gamma^{\mu}O^{-1})^T 
= (O^{-1})^T{\gamma^{\mu}}^T O^T 
= - C' O\gamma^{\mu}O^{-1} C'^{-1},
\ee
\be
(O^{-1})^T C{\gamma^{\mu}} C^{-1} O^T = C' O\gamma^{\mu}O^{-1} C'^{-1},
\ee
\be
\gamma^{\mu} C^{-1} O^T C' O = C^{-1} O^T C' O \gamma^{\mu},
\ee
\bel{gamma-tmp-C-C-prime1}
[\gamma^{\mu}, C^{-1} O^T C' O] = 0
.
\ee

Other forms of~\rf{gamma-tmp-B-B-prime1}--\rf{gamma-tmp-C-C-prime1} are
\be
B^{-1} O^\hc B' O = k^O_B 1_{4\times 4}
\; \Leftrightarrow \;
B' = k^O_B (O^\hc)^{-1} B O^{-1},
\ee
and
\be
C^{-1} O^T C' O = k^O_C 1_{4\times 4}
\; \Leftrightarrow  \;
C' = k^O_C {O^T}^{-1} C O^{-1},
\ee

with complex $k^O_B$ and $k^O_C$.

If I require $B'$ and $C'$ to satisfy \rf{gamma-tmp-B-C-relation}
I can set,
\bel{gammaTmpCBTrans}
B' = (O^\hc)^{-1} B O^{-1}, 
\quad 
C' = (O^T)^{-1} C O^{-1}
.
\ee

Such a choice is consistent with the fact that $O$ in~\rf{gammaTransLaw} 
is defined up to a scale factor and the ability to rescale $B$ and $C$
by~\rf{gamma-B-C-can-be-rescale}. Indeed, if I substitute 
in~\rf{gammaTmpCBTrans} $O$ by the matrix $s_O O$ I would have
\bem
B_s' & = &(s_O^* O^\hc)^{-1} B (s_O O)^{-1} = \abs{s_O}^{-2} B', 
\nel
C_s' & = & (s_O O^T)^{-1} C (s_O O)^{-1} = s_O^{-2} C',
\ee

which coincides with~\rf{gamma-B-C-can-be-rescale} if one identifies
$s_O^{-2}$ with $s$.

\P
The transformation law~\rf{gammaTmpCBTrans} 
leads to the conclusion that the signs in~\rf{gamma-tmp-B-C-relation} do not
depend on the particular representation of the Dirac algebra~\rf{gammaAlgebra}.
From equations~\rf{gamma-tmp-B-C-relation}
\bem
{C'}^T & = & (O^T)^{-1} C^T O^{-1} = \pm (O^T)^{-1} C O^{-1} = \pm C',
\nel
B'^{-1} C'^\hc (B'^T)^{-1}C' 
& = &
O B^{-1} O^\hc (O^{-1})^\hc C^\hc [(O^T)^{-1}]^\hc 
\nel&& {}\times
\{(O^{-1})^T B^T [(O^\hc)^{-1}]^T\}^{-1} (O^T)^{-1} C O^{-1}
\nel
& = &
O B^{-1} (O^{-1}O)^\hc C^\hc [(O^{-1}O)^\hc]^T (B^T)^{-1} (O^{-1}O)^T C O^{-1}
\nel
& = &
O B^{-1} C^\hc (B^T)^{-1} C O^{-1} = O(\pm 1_{4\times 4})O^{-1}
\nel
& = &
\pm 1_{4\times 4}
.
\ee

The last statement means that to determine the 
signs in~\rf{gamma-tmp-B-C-relation}
I may use, for example, the representation~\rf{gamma-reference-representation}.
In this representation the matrices
\be
B_D = \gamma_D^0 = \pmatrix{1_{2 \times 2} & 0 \cr 0 & -1_{2 \times 2}}
\ee

and
\be
C_D = \gamma_D^2\gamma_D^0 
= \pmatrix{0 & \sigma^2 \cr - \sigma^{2} & 0}
  \pmatrix{1_{2 \times 2} & 0 \cr 0 & -1_{2 \times 2}}
= \pmatrix{0 & -\sigma^2 \cr -\sigma^{2} & 0}
\ee

play the role of $B$ and $C$. Indeed one can easily check that
\bem
(\gamma_D^{\mu})^\hc & = & B_D \gamma_D^{\mu} B_D^{-1} 
,
\nel
(\gamma_D^{\mu})^T & = & - C_D \gamma_D^{\mu} C_D^{-1} 
.
\ee

Thus from the fact that in~\rf{gamma-reference-representation}
the matrix $L_\rho{}^\mu$ is real it follows that
\bem
(\gamma_r^{\mu})^\hc 
&=& (L_\rho{}^\mu \gamma_D^{\rho})^\hc 
= L_\rho{}^\mu (\gamma_D^{\rho})^\hc 
= B_D L_\rho{}^\mu \gamma_D^{\rho} B_D^{-1}
\nel
&=& B_D \gamma_r^{\mu} B_D^{-1}
,
\nel
(\gamma_r^{\mu})^T 
&=& (L_\rho{}^\mu \gamma_D^{\rho})^T 
= L_\rho{}^\mu (\gamma_D^{\rho})^T 
= -C_D L_\rho{}^\mu \gamma_D^{\rho} C_D^{-1}
\nel
&=& -C_D \gamma_r^{\mu} C_D^{-1}
.
\ee

I have for $B_D$ and $C_D$
\be
B_D^T =(\gamma_D^0)^T = B_D, 
\quad
C^\hc = (\gamma_D^2 \gamma_D^0)^\hc = -\gamma_D^0 \gamma_D^2
\ee

and
\be
C_D^T = \pmatrix{0 & -(\sigma^2)^T \cr -(\sigma^2)^T & 0}
      = \pmatrix{0 & \sigma^2 \cr \sigma^2 & 0} 
      = -C_D,
\ee

\bem
B_D^{-1} C_D^\hc (B_D^T)^{-1} C_D 
&=& -B_D C_D^\hc B_D C_D
= -\gamma_D^0 \gamma_D^0 \gamma_D^2 \gamma_D^0 \gamma_D^2 \gamma_D^0 
\nel
& = & (\gamma_D^0)^2 (\gamma_D^2)^2 (\gamma_D^0)^2
= -1.
\ee

Thus the signs in~\rf{gamma-tmp-B-C-relation} in both cases are minuses.

\P
By combining all results of this section
I have the next statement:\\
In four dimensions for any representation of the algebra \rf{gammaAlgebra}
and any metric tensor with the signature -$2$ one can always find such
matrices $B$ and $C$ that,
\bel{gammaBProperties}
\cc{\gu\mu} = B \gu\mu B^{-1}, \quad \cc{\gd5} = -B \gd5 B^{-1},
\quad B^\hc = B,
\ee

\bel{gammaCProperties}
{\gu\mu}^T = -C \gu\mu C^{-1}, \quad {\gd5}^T = C \gd5 C^{-1},
\quad {C}^T = -C,
\ee

\bel{gammaBCRelation}
B^{-1} C^\hc (B^T)^{-1} C = - 1
\quad \mbox{or} \quad
B^{-1} C^\hc = -C^{-1} {B^T}
.
\ee

The matrices $B$ and $C$ are defined up to a scale transformation with 
an arbitrary non-zero complex factor $s$,
\bel{gamma-B-C-rescale}
B \rightarrow B_s = \abs{s} B, \quad C \rightarrow C_s = sC,
\ee

The transformation law for $B$ and $C$ under 
the representation transformation \rf{gammaTransLaw} is given by
\be
B \rightarrow B' = (O^\hc)^{-1} B O^{-1},
\ee
\bel{gammaBCTransLaw}
C \rightarrow C' = (O^T)^{-1} C O^{-1}.
\ee




\section{Transformation properties of Dirac spinors}

Consider a real matrix $\Lambda^{\mu}{}_\rho$, 
$(\Lambda^{\mu}{}_\rho)^* = \Lambda^{\mu}{}_\rho$ with the property
\bel{gammaMatrixTransform}
g^{\mu\nu} = \Lambda^{\mu}{}_\rho \Lambda^{\nu}{}_\sigma g^{\rho\sigma}
.
\ee
All such $\Lambda^{\mu}{}_\rho$ form a representation of the Lorentz 
group~$\Lambda$.
To construct the Dirac algebra representation of $\Lambda$ I 
introduce the matrices $\gamma_\Lambda^{\mu}$ defined by
\bel{gamma-transformation-property}
\gamma_\Lambda^{\mu} = \Lambda^{\mu}{}_\rho\gamma^{\rho}
.
\ee

They satisfy the same Dirac algebra~\rf{gammaAlgebra} as $\gu\mu$:
\be
\{\gamma_\Lambda^{\mu}, \gamma_\Lambda^{\nu}\}
= \{\Lambda^{\mu}{}_\rho \gamma^{\rho}, \Lambda^{\nu}{}_\sigma\gamma^{\sigma}\}
= \Lambda^{\mu}{}_\rho \Lambda^{\nu}{}_\sigma \{\gamma^{\rho}, \gamma^{\sigma}\}
= 2\Lambda^{\mu}{}_\rho \Lambda^{\nu}{}_\sigma g^{\rho\sigma} = 2g^{\mu\nu}
.
\ee

So according to~\rf{gammaTransLaw} there is a matrix $S$ with the property
\beml{gammaRepresentation}
\gamma_\Lambda^{\mu} 
%= \Lambda^{\mu}{}_\rho\gamma^{\rho} 
= S \gamma^{\mu} S^{-1}.
\ee

By definition all such $S$ form a spinor representation 
of the group $\Lambda$.

\P

To find relations between $S$ and $\gd5$ I 
rewrite~\rf{gammaRepresentation} as 
\be
\gamma^{\mu} = \Lambda^{\mu}{}_\rho S^{-1}\gamma^{\rho}S
\ee

and substitute this expression for $\gu\mu$ in the 
definitions \rf{gammaFiveDef},
\bem
\gd5 & = & -i \frac{1}{4!}
       \varepsilon_{\mu\nu\rho\sigma} \gu\mu \gu\nu \gu\rho \gu\sigma

\nel & = &
 -i \frac{1}{4!} \varepsilon_{\mu\nu\rho\sigma} 
 \Lambda^{\mu}{}_{\mu'} S^{-1}\gamma^{\mu'}S
 \Lambda^{\nu}{}_{\nu'} S^{-1}\gamma^{\nu'} S
 \Lambda^{\rho }{}_{\rho'} S^{-1}\gamma^{\rho'} S
 \Lambda^{\sigma}{}_{\sigma'}S^{-1}\gamma^{\sigma'}S

\nel & = &
 -i \frac{1}{4!} 
 (\varepsilon_{\mu\nu\rho\sigma} 
 \Lambda^{\mu}{}_{\mu'} \Lambda^{\nu}{}_{\nu'} 
 \Lambda^{\rho }{}_{\rho'} \Lambda^{\sigma}{}_{\sigma'})
 S^{-1} \gamma^{\mu'}\gamma^{\nu'}
        \gamma^{\rho'}\gamma^{\sigma'} S
\ee

According to~\rf{convEpsilonUpIs} and~\rf{convDetIs} I have,
\be
\varepsilon_{\mu\nu\rho\sigma} 
 \Lambda^{\mu}{}_{\mu'} \Lambda^{\nu}{}_{\nu'} 
 \Lambda^{\rho }{}_{\rho'} \Lambda^{\sigma}{}_{\sigma'}
= -\sqrt{-g} e_{\mu\nu\rho\sigma} 
 \Lambda^{\mu}{}_{\mu'} \Lambda^{\nu}{}_{\nu'} 
 \Lambda^{\rho }{}_{\rho'} \Lambda^{\sigma}{}_{\sigma'}
= -\sqrt{-g} \det(\Lambda) \, e_{\mu'\nu'\rho'\sigma'}
\ee
\be
= \det(\Lambda) \, \varepsilon_{\mu'\nu'\rho'\sigma'}
\ee

and
\be
\gd5 = -i \frac{1}{4!} \det(\Lambda) \, \varepsilon_{\mu'\nu'\rho'\sigma'}
 S^{-1} \gamma^{\mu'}\gamma^{\nu'}
        \gamma^{\rho'}\gamma^{\sigma'} S
 = 
\det(\Lambda) S^{-1}\gd5 S
.
\ee

Thus
\bel{gamma-5-and-S}
\det(\Lambda) \gd5 = S\gd5 S^{-1}
.
\ee

\P
Now I calculate $(\gamma_\Lambda^{\mu})^\hc$ and $(\gamma_\Lambda^{\mu})^T$.
From~\rf{gammaComplexConj} and~\rf{gammaTransport}
\be
(\gamma_\Lambda^{\mu})^\hc = (\Lambda^{\mu}{}_\rho\gamma^{\rho})^\hc
= \Lambda^{\mu}{}_\rho\gu{\rho}^\hc
= \Lambda^{\mu}{}_\rho B\gu{\rho}B^{-1}
= B\gamma_\Lambda^{\mu}B^{-1}
,
\ee

\bel{gammaLorentzPrime1}
(\gamma_\Lambda^{\mu})^T = (\Lambda^{\mu}{}_\rho\gamma^{\rho})^T
= \Lambda^{\mu}{}_\rho\gu{\rho}^T
= -\Lambda^{\mu}{}_\rho C\gu{\rho}C^{-1}
= -C\gamma_\Lambda^{\mu}C^{-1}
.
\ee

But if I view $\gamma_\Lambda^{\mu}$ just as another representation of 
the Dirac algebra, then from the definition~\rf{gammaRepresentation}
and according to~\rf{gammaBCTransLaw}, where $S$ should be substituted for $O$,
I have
\be
(\gamma_\Lambda^{\mu})^\hc = B_\Lambda \gamma_\Lambda^{\mu} B_\Lambda^{-1},
\quad B_\Lambda = (S^\hc)^{-1} B S^{-1},
\ee
\bel{gammaLorentzPrime2}
(\gamma_\Lambda^{\mu})^T = -C_\Lambda \gamma_\Lambda^{\mu}C_\Lambda^{-1},
\quad C_\Lambda = (S^T)^{-1} C S^{-1}
.
\ee

The equations~\rf{gammaLorentzPrime1} and~\rf{gammaLorentzPrime2} mean that 
$B_\Lambda = \abs{s_\Lambda} B$ and $C_\Lambda = s_\Lambda C$
with some non-zero $s_\Lambda$
where I have used the fact that $B$ and $C$ are fixed only up to the 
transformation~\rf{gamma-B-C-rescale}. Thus
\bel{gamma-tmp-S-and-C-B}
\abs{s_\Lambda} B = (S^\hc)^{-1} B S^{-1},
\quad 
s_\Lambda C = (S^T)^{-1} C S^{-1}
.
\ee

The factor $s_\Lambda$ in~\rf{gamma-tmp-S-and-C-B} 
may be eliminated if one takes into account that in~\rf{gammaRepresentation}
$S$ is defined up to an arbitrary scale transformation. So if one chooses
a particular rule to fix $B$ and $C$ in any representation of the Dirac algebra
then in~\rf{gamma-tmp-S-and-C-B} the factor $s_\Lambda$ may be set to one
thus also fixing $S$. This permits me to write 
\be
B = (S^\hc)^{-1} B S^{-1},
\quad S^\hc B = B S^{-1},
\ee
\bel{gammaTmpSAndCB}
C = (S^T)^{-1} C S^{-1},
\quad S^T C = C S^{-1}
.
\ee

Still it does not define $S$ completely. In~\rf{gammaTmpSAndCB} $-S$
can be used instead of $S$ and to resolve this sign ambiguity some additional 
rule should be used.

\P

The Dirac matrices can be viewed as operators that act on four dimensional
complex space. Elements of this space are called 4-spinors
or simply spinors. The existence of the matrix $S$ makes it possible 
to establish the action of the Lorentz group on spinors, i.e. I can define,
\bel{gammaStandardTransformationLaw}
\psi \rightarrow \psi' = S\psi
.
\ee

Then for any two spinors $\psi$ and $\phi$ the following products 
are Lorentz-invariant,
\bel{gammaScalarProducts}
\bar{\psi} \phi, \quad {\psi}^T C \phi ,
\ee
if I use 
\bel{gamma-bar-definition}
\bar{\psi} \equiv \psi^\hc B
.
\ee

This fact follows directly 
from~\rf{gammaTmpSAndCB}:
\be
\bar\psi \phi \rightarrow \bar{\psi'} \phi'
= (S\psi)^\hc B S \phi
= \psi^\hc S^\hc B S \phi
= \psi^\hc B S^{-1} S \phi
= \psi^\hc B \phi  = \bar\psi \phi,
\ee
\be
{\psi}^T C \phi \rightarrow {\psi'}^T C \phi'
= {(S\psi)}^T C S \phi
= {\psi}^T{S}^T C S \phi
= {\psi}^T C S^{-1} S \phi
= {\psi}^T C \phi 
.
\ee

\P
The main difference between these two kinds of scalar products comes from
the fact that B is a hermitian matrix,
\bel{gamma-unused5}
B = B^\hc,  \quad   B_{ab} = B_{ba}^* 
\ee

and  C is an antisymmetric one,
\bel{gamma-unused6}
C = -C^T,  \quad  C_{ab} = -C_{ba},
\ee
where I have written explicitly the spinor indices. 

Thus
\beml{gamma-B-scalar-product-hermitian}
\bar{\psi}\psi
& \equiv& \bar{\psi}_a\psi_a 
= \psi_b^\hc B_{ba} \psi_a = B_{ab} \psi_a^\hc \psi_b
= B_{ba}^* \psi_a^\hc \psi_b
= B_{ba}^* \psi_a^\hc (\psi_b^\hc)^\hc
= B_{ba}^* (\psi_b^\hc \psi_a)^\hc
\nel &=& (\psi_b^\hc B_{ba} \psi_a)^\hc
= (\bar{\psi}\psi)^\hc,
\ee

and
\beml{gammaPsiCpsi}
\psi^T C \psi & \equiv & \psi_b C_{ba} \psi_a = C_{ab} \psi_a \psi_b 
= - C_{ba} \psi_a \psi_b
= - C_{ba} (\psi_b \psi_a + [\psi_a, \psi_b])
\nel & = &
- \psi_b C_{ba} \psi_a - C_{ba} [\psi_a, \psi_b]
= - \psi^T C \psi - C_{ba} [\psi_a, \psi_b],
\nel
\psi^T C \psi & = &- {1 \over 2} C_{ab} [\psi_b, \psi_a] 
= {1 \over 2} C_{ab} [\psi_a, \psi_b]
\ee

So $\bar{\psi}\psi$ is hermitian even if the components 
$\psi_a$ of the spinor $\psi$ have an internal structure. 
The product $\psi^T C \psi$ depends on 
the value of $[\psi_a, \psi_b]$. 
It is zero if $\psi_a$ are ordinary numbers that commute and for 
a Grassmann spinor $\psi^G$ with the properties
\be
\{\psi^G_a, \psi^G_b\} = 0, 
\quad [\psi^G_a, \psi^G_b] = 2\psi^G_a \psi^G_b
\ee

the equation~\rf{gammaPsiCpsi} becomes 
\be
(\psi^G)^T C \psi^G 
= C_{ab} \psi^G_a \psi^G_b 
= (\psi^G)^T C \psi^G
\ee

or just the trivial identity. 
Thus the product $(\psi^G)^T C \psi$ does not have to be zero
for the Grassmann spinors.

\P
Another important property of the product ${\psi}^T C \phi$ is that
it is not invariant under the phase transformation
$\psi \rightarrow e^{i\phi} \psi$:
\be
{\psi}^T C \phi \rightarrow e^{2i\phi} {\psi}^T C \phi
.
\ee

\P
I next calculate the values of 
$\bar{\psi} \gd5 \phi$, $\bar{\psi} \gu\mu \phi$,
$\psi^T C \gd5 \phi$ and $\psi^T C \gu\mu \phi$ after 
the transformation~\rf{gammaStandardTransformationLaw}, 
using also equation~\rf{gamma-5-and-S}:
\beml{gammaVectorIs}
\bar{\psi} \gd5 \phi & \rightarrow & \bar{\psi'} \gd5 \phi'
= \psi^\hc S^\hc B \gd5 S\phi = \bar{\psi} S^{-1} \gd5 S \phi
= \det(\Lambda) \bar{\psi} \gd5 \phi,
\nel
{\psi}^T C \gd5 \phi & \rightarrow & {\psi'}^T C \gd5 \phi'
= {\psi}^T C S^{-1} \gd5 S\phi
= \det(\Lambda) {\psi}^T C \gd5 \phi,

\nel
\bar{\psi} \gu\mu \phi & \rightarrow & \bar{\psi'} \gu\mu \phi'
=  \bar{\psi} S^{-1} \gu\mu S\phi
= \Lambda^{\mu}{}_\rho \bar{\psi} \gu\rho \phi,

\nel
{\psi}^T C \gu\mu \phi & \rightarrow & {\psi'}^T C \gu\mu \phi'
= \Lambda^{\mu}{}_\rho {\psi}^T C \gu\rho \phi,
\ee

Thus,
$\bar{\psi} \gd5 \phi$ and ${\psi}^T C \gd5 \phi$ are pseudo-scalars,
$\bar{\psi} \gu\mu \phi$ and ${\psi}^T C \gu\mu \phi$ are vectors, etc.






\section{Projection operators} \label{gammaProj}

Consider 4-vectors $p^\mu$, $s^\mu$ that describe a particle of mass $m$
with given momentum and spin. 
Such vectors should satisfy
\bel{gammaKinematicRelations}
p^2 = m^2 > 0, \quad
p^\mu = m q^\mu, \quad
q^2 = 1, \quad
s^2 = s_\mu s^\mu = -1,
\quad (p \cdot s) = (q \cdot s) = 0
\ee

Then four operators $\Lambda_\pm$, $\Sigma_\pm$ defined through

\be
\Lambda_\pm = \frac{1 \pm \qc}{2}, \qquad
\Sigma_\pm = \frac{1 \pm \slsh{s}\gd5}{2}
\ee

are projection operators because

\bem
(\Lambda_\pm)^2  & = & \frac{(1 \pm \qc)^2}{4}
= \frac{2 \pm 2\qc}{4} = \frac{1 \pm \qc}{2} = \Lambda_\pm,

\nel
\Lambda_+ \Lambda_- & = &
\frac{(1 + \qc)(1 - \qc)}{4} = \frac{1 - 1}{4} = 0,

\nel
(\Sigma_\pm)^2  & = & \frac{(1 \pm \slsh{s}\gd5)^2}{4}
= \frac{1 + \slsh{s}\gd5 \slsh{s}\gd5 \pm 2\slsh{s}\gd5}{4}
= \frac{1 - \slsh{s}^2 \gd5^2 \pm 2\slsh{s}\gd5}{4}
\nel & = &
\frac{2 \pm 2\slsh{s}\gd5}{4}

= \frac{1 \pm \slsh{s}\gd5}{2} = \Sigma_\pm,

\nel
\Sigma_+ \Sigma_-
& = & \frac{(1 + \slsh{s}\gd5)(1 - \slsh{s}\gd5)}{4}
= \frac{1 - \slsh{s}\gd5\slsh{s}\gd5}{4}
= \frac{1 + s^2 \gd5^2}{4} = 0.
\ee


With the convention that in $A_\pm B_\pm$
the first and second $\pm$ are treated independently, I have
\be
[\Lambda_\pm, \Sigma_\pm]  = 
\frac{1}{4}[1 \pm \qc, 1 \pm \slsh{s}\gd5]
 = \frac{(\pm1)(\pm1)}{4}[\qc, \slsh{s}\gd5].
\ee

But
\be
[\qc, \slsh{s}\gd5] = \qc\slsh{s}\gd5 - \slsh{s}\gd5\qc 
= (\qc\slsh{s} + \slsh{s}\qc)\gd5 = 2(q \cdot s)\gd5 = 0,
\ee 

so
\bel{gamma-L-and-S-commute}
[\Lambda_\pm, \Sigma_\pm]  = 0,
\ee

and the four operator products 
$\Lambda_\pm\Sigma_\pm$ are also linearly independent
projection operators acting in the spinor space. 
Moreover,these operators have unit trace,
\bem
\Tr(\Lambda_\pm\Sigma_\pm) & = &
\frac{1}{4} \Tr[(1 \pm \qc)(1 \pm \slsh{s}\gd5)]
\nel &=&
\frac{1}{4} \Tr(1 \pm \slsh{s}\gd5 \pm \qc + (\pm1)(\pm1) \qc\slsh{s}\gd5)
= \frac{1}{4} \Tr(1_{4\times 4}) = 1.
\ee

It can be shown that any projection operator $P$ 
with unit trace can be represented as
the normalized direct product of two spinors

\bel{gamma-P-as-direct-product}
P = \frac{1}{(\phi^L)^\hc \phi^R} \phi^R (\phi^L)^\hc \quad \mbox{or}
\quad (P)_{ab}
= \frac{1}{(\phi^L)^\hc \phi^R}\phi^R_a (\phi^L)^*_b,
\ee
\bel{gamma-P-as-direct-product-not-unique}
P\phi^R = \phi^R, \quad (\phi^L)^\hc P = (\phi^L)^\hc,
\ee

i.e. $\phi^R$ is a Right eigenvector of $P$ and $(\phi^L)^\hc$ is a Left one.
In the general case one can have $\phi^L \ne \phi^R$ and 
$(\phi^L)^\hc \ne (\phi^R)^T$ or in other words $P \ne P^\hc$ and $P \ne P^T$.
The eigenvectors are defined
up to a scale transformation with complex factors $\kappa_1$ and $\kappa_2$,
\bel{gammaProjTrans}
\phi^R \rightarrow \tilde{\phi}^R = \kappa_1 \phi^R,
\quad \phi^L \rightarrow \tilde{\phi}^L = \kappa_2 \phi^L,
\ee

This means that
for example $\Lambda_+\Sigma_r$, $r = +, -$ can be written as

\be
\Lambda_+\Sigma_r =
\frac{1}{(u^L_r)^\hc u_r^R} u_r^R (u^L_r)^\hc ,
\ee
with some spinors 
$u_{r}^{R,L} = u_r^{R,L}(q, s)$\footnote{
It is important to note here that $u_r$ depends only on
$q$ and $s$, i.e. it does not depend on the mass $m^2 = p^2$.
}.

\P
The spinors $u_r^L$ and $u_r^R$ can be related. To show this,
I calculate
\bem
(\Lambda_+\Sigma_\pm)^\hc 
& = & 
(\Sigma_\pm\Lambda_+ +[\Lambda_+, \Sigma_\pm])^\hc
= \cc{\Lambda_+}\cc{\Sigma_\pm} + 0^\hc
= \frac{1}{4}(1 \pm \cc{\qc}) (1 \pm \cc{\gd5}\cc{\slsh{s}})
\nel
& = & \frac{1}{4}(1 \pm B\qc B^{-1}) (1 \mp B{\gd5}{\slsh{s}}B^{-1})
\quad \mbox{(from \rf{gammaBProperties})}
\nel
& = & B\frac{1}{4}(1 \pm \qc) (1 \mp \gd5{\slsh{s}})B^{-1}
= B\frac{1}{4}(1 \pm \qc) (1 \pm {\slsh{s}}\gd5)B^{-1}
\nel & = & B\Lambda_\pm\Sigma_\pm B^{-1}.
\ee

Hence (in the following I omit the index $r = +, -$ , thus $u_r \equiv u$)
\bem
\left(\frac{1}{(u^L)^\hc u^R} u^R (u^L)^\hc\right)^\hc  &= &
\frac{1}{[(u^L)^\hc u^R]^\hc} u^L (u^R)^\hc
=
\frac{1}{(u^R)^\hc u^L} u^L (u^R)^\hc
= (\Lambda_+\Sigma)^\hc 
\nel
& =& B\Lambda_+\Sigma B^{-1}
\ee

or
\beml{gamma-tmp23}
\frac{1}{(u^R)^\hc u^L} u^L (u^R)^\hc 
=\frac{1}{(u^L)^\hc u^R} B u^R (u^L)^\hc B^{-1}
 = \frac{1}{(B^{-1}u^L)^\hc Bu^R} B u^R \cc{(B^{-1} u^L)},
\ee

where I have used $B^\hc = B$, thus 
\be
(u^L)^\hc u^R = (u^L)^\hc B^{-1} B u^R = (B^{-1}u^L)^\hc Bu^R.
\ee

The equation~\rf{gamma-tmp23} together with~\rf{gammaProjTrans} leads
to the conclusion that
$u^L = \kappa_1 B u^R$, $u^R = \kappa_2 B^{-1} u^L$
or just
\be
u^R = \kappa B u^L, \quad \kappa \equiv \kappa_1 = \frac{1}{\kappa_2}.
\ee

I have now,
\bel{gammaTmpProjPosIs}
\Lambda_+\Sigma =
\frac{1}{\cc{(\kappa B u)} u} u \cc{(\kappa B u)}
=\frac{1}{\cc{(B u)} u} u \cc{(B u)}
=\frac{1}{\cc{u}B u} u \cc{u}B
= \frac{1}{\ub u} u \ub,
\ee

where $u \equiv u^R$ is defined up to a scale transformation
with an arbitrary nonzero complex factor.

In the same way I may represent $\Lambda_-\Sigma_r$ as
\bel{gammaTmpProjNegIs}
\Lambda_-\Sigma_r = \frac{1}{\vb v} v \vb,
\quad v_r = v_r(q, s).
\ee

\P
To establish a connection between $u$ and $v$ I use~\rf{gamma-L-and-S-commute}
and explore the properties
of the matrix $C$ (I again omit the index $r$) 
\bem 
(\Lambda_+\Sigma_\pm)^T 
& = & (\Sigma_\pm \Lambda_+)^T = \Lambda_+^T \Sigma_\pm^T
= \frac{1}{4}(1 + {\qc}^T) (1 \pm {\gd5}^T{\slsh{s}}^T)
\nel
& = & \frac{1}{4}(1 - C\qc C^{-1}) (1 \mp C{\gd5}C^{-1}C\slsh{s}C^{-1})
 =  \frac{1}{4}C(1 - \qc) (1 \mp {\gd5}{\slsh{s}})C^{-1}
\nel
&=& \frac{1}{4}C(1 - \qc) (1 \pm {\slsh{s}}{\gd5})C^{-1}
= C\Lambda_-\Sigma_\pm C^{-1}.
\ee

It gives according to the definitions~\rf{gammaTmpProjPosIs} 
and~\rf{gammaTmpProjNegIs}
\be
\left({1 \over \ub u} u \ub\right)^T = {1 \over \ub u} \ub^T u^T 
= \frac{1}{\vb v} C v\vb C^{-1},
\ee

thus from the fact that the 
representation~\rf{gamma-P-as-direct-product} for the projection operator 
is defined up to the transformation~\rf{gammaProjTrans}
I get
\be
\ub^T = \kappa_1 Cv,
\quad
u^T = \kappa_2 \vb C^{-1},
\ee
with some complex $\kappa_1$ and $\kappa_2$.


Now I use the properties~\rf{gammaBProperties}--\rf{gammaBCRelation},
$C^T = -C$, $\cc{B} = B$ and $B^{-1} \cc{C} = -C^{-1} {B^T}$,
\be
\ub = (\ub^T)^T = (\kappa_1 Cv)^T = -\kappa_1 v^TC
.
\ee

Hence according to~\rf{gammaBProperties}, $B^\hc = B$,
\bem
\cc{u} B & = & -\kappa_1 v^TC,
\nel 
\cc{u} & = & -\kappa_1 v^TCB^{-1},
\nel
u &=& -\kappa_1^* B^{-1}\cc{C} \cc{v^T}
.
\ee

Furthermore from~\rf{gammaBCRelation} I find
\be
u = (u^T)^T = (\kappa_2 \vb C^{-1})^T = -\kappa_2 C^{-1} {\vb}^T
=  -\kappa_2 C^{-1} B^T \cc{v^T}
=  \kappa_2 B^{-1} \cc{C} \cc{v^T},
\ee

so
\be
-\kappa_1^* B^{-1}\cc{C} \cc{v^T} = \kappa_2 B^{-1} \cc{C} \cc{v^T},
\quad
-\kappa_1^* = \kappa_2,
\ee
\be
\ub^T = \kappa_1 Cv,
\quad
u^T = -\kappa_1^* \vb C^{-1}.
\ee

The last result also shows the important relation between
the signs of $\ub u$ and $\vb v$,
\be
\ub u = (\ub u)^T = u^T \ub^T = -\kappa_1^* \kappa_1 \vb C^{-1}Cv
 = - \abs{\kappa_1}^2 \vb v,
\ee
i.e. $\ub u$ and $\vb v$ should have opposite signs.

\P
I use the scale freedom in the definition of $u$ to choose
\bel{gammaTmpUNorm}
\ub u = f, \quad f = \pm 1
\ee
and then define the normalization for $v$ by the requirement
\beml{gammaTmpUVConnection} & \displaystyle
\ub^T = Cv \quad \mbox{or} \quad u^T = -\vb C^{-1}
\quad \Rightarrow \quad  \vb v = -f,
\nel & \displaystyle
v^T C = - \ub, \quad C^{-1} \vb^T = u, \quad
u^T C = -\vb, \quad  C^{-1}\ub^T = v.
\ee

\P
A particular choice for the sign in \rf{gammaTmpUNorm}
depends on the representation. A simple way to show this is to note that
the set $\gamma'^{\mu} = -\gu\mu$ also forms a representation
of~\rf{gammaAlgebra}.
In this representation $B' = B$, $\gd5' = \gd5$ and
\be
\Lambda'_+\Sigma'_+
= \frac{1}{4}(1 + \qc') (1 + \slsh{b}\gd5')
= \frac{1}{4}(1 - \qc) (1 - \slsh{s}\gd5) = \Lambda_-\Sigma_-
\ee

So one may choose $u'_+(q,s) = v_-(q, s)$ but then
\be
\ub'_+(q,s)u'_+(q,s) = \vb_-(q,s)v_-(q,s) = - \ub_-(q,s)u_-(q,s).
\ee

For all\footnote{``all'' here means that I have not found in the literature
any examples of representations where $\ub u = -1$}
commonly used representations the sign choice is
\bel{gamma-u-norm-sign-choice}
f = 1 \quad \Leftrightarrow \quad \ub_r u_r = - \vb_r v_r = 1.
\ee

\P
The spinors $u_\pm$ and $v_\pm$ form a basis in the spinor space and
satisfy the following orthogonality condition,
\be
\ub_+ u_- = \vb_+ v_- = \ub_r v_s = \vb_r u_s = 0,
\quad r,s = +,-,
\ee
because they are eigenvectors of linearly independent projectors.

Hence any spinor can be represented as a linear combination of them.

\P In summary I have the following statement from the results of this
section. Four operators
$\Lambda_\pm\Sigma_\pm$ defined through
\bel{gammaProjDef}
\Lambda_\pm = \frac{1 \pm \qc}{2}, \quad
\Sigma\pm = \frac{1 \pm \slsh{s}\gd5}{2},
\quad q^2 \equiv p^2/m^2 = 1,
\quad s^2 = -1,
\quad (q \cdot s) = 0
\ee
form the full set of projection operators and can be represented
as direct products of the spinors $u_\pm(q, s)$, $v_\pm(q, s)$,
\beml{gammaProjIs} &
\Lambda_+\Sigma_+  =  f  u_+(q, s) \ub_+(q, s),
\quad
\Lambda_+\Sigma_-  =  f  u_-(q, s) \ub_-(q, s),
\nel &
\Lambda_-\Sigma_+  =  -f  v_+(q, s) \vb_+(q, s),
\quad
\Lambda_-\Sigma_-  =  -f  v_-(q, s) \vb_-(q, s),
\nel &
\Lambda_+\Sigma_+ u_+(q, s) = \Lambda_+ u_+(q, s) =
\Sigma_+ u_+(q, s) = u_+(q, s),
\nel &
\Lambda_+\Sigma_- u_-(q, s) = \Lambda_+ u_-(q, s) =
\Sigma_- u_-(q, s) = u_-(q, s),
\nel &
\Lambda_-\Sigma_+ v_+(q, s) = \Lambda_- v_+(q, s) =
\Sigma_+ v_+(q, s) = v_+(q, s),
\nel &
\Lambda_-\Sigma_- v_-(q, s) = \Lambda_- v_-(q, s) =
\Sigma_- v_-(q, s) = v_-(q, s),
\ee

where $f$ depends on the particular representation of the Dirac
algebra~\rf{gammaAlgebra} and can be either $1$ or $-1$. 
By multiplying the $\gamma$-matrices by $-1$ one can always choose a
representation with $f = 1$. I will always suppose it to hold in the following.

\P
The spinors $u_\pm(q, s)$, $v_\pm(q, s)$ form a basis in 
the spinor space and satisfy
the orthonormality conditions,
\beml{gammaOrthoNorm} &
\ub_r(q, s) u_t(q, s) = \delta_{rt},
\quad
\vb_r(q, s) v_t(q, s) = -\delta_{rt},
\nel &
\ub_r(q, s) v_t(q, s) = \vb_r(q, s) u_t(q, s) = 0,
\nel & \displaystyle
\delta_{r, t = r} = 1, \quad \delta_{r, t \ne r} = 0,
\quad r,t = +, -.
\ee

They also satisfy 
\bel{gammaPOnSpinors}
\qc u_r(q, s) = u_r(q, s), \qquad \qc v_r(q, s) = -v_r(q, s),
\ee

and are related through
\beml{gammaSpinorRelation} &
v_r^T(q, s) C = - \ub_r(q, s), \quad C^{-1} \vb_r^T(q, s) = u_r(q, s),
\nel &
u_r^T(q, s) C = -\vb_r(q, s), \quad  C^{-1}\ub_r^T(q, s) = v_r(q, s).
\ee

Now I can easily obtain the standard formulae for
spin polarization sums, i.e.
\beml{gammaSpinPolarization} & \displaystyle
\sum_{r = +, -}  u_r \ub_r  = \sum_{r = +, -} \Lambda_+\Sigma_r
=  \Lambda_+(\Sigma_+ + \Sigma_-) = \Lambda_+ = \frac{\qc + 1}{2},
\nel & \displaystyle
 \sum_{r = +, -} v_r \vb_r   = -\sum_{r = +, -} \Lambda_-\Sigma_r
=  -\Lambda_+(\Sigma_+ + \Sigma_-) = -\Lambda_- = \frac{\qc - 1}{2}
.
\ee



\section{Relations between Dirac spinors}

Although the result~\rf{gammaSpinorRelation} of the previous 
section gives the possibility to calculate the spinor $v_r$ from $u_r$, it
does not show the relation between $u_+$ and $u_-$. Also it does not 
permit to calculate expressions like $\gu\mu u_\pm$. 
For this reason I present an alternative approach to 
express $u_\pm$ and $v_\pm$ through each other.
I first note that
\bem
\lefteqn{
\qc\gd5 = - \qc\gd5, \quad \slsh{s}\gd5\gd5 = -\gd5\slsh{s}\gd5
}
\nel & \Rightarrow &
\Lambda_+ \gd5 = \frac{1}{2}(1 + \qc)\gd5 =  \gd5\frac{1}{2}(1 - \qc)
= \gd5\Lambda_-,
\nel &&
\Sigma_+\gd5 = \gd5\frac{1}{2}(1 - \slsh{s}\gd5) = \gd5\Sigma_-,
\ee

which means
\bel{gamma-unused7}
\Lambda_+\Sigma_+ \gd5 v_-
= \Lambda_+\gd5\Sigma_-  v_-
= \gd5\Lambda_-\Sigma_-  v_- = \gd5 v_- ,
\ee

i.e. $\gd5 v_-$ is proportional to $u_+$.

\P
Consider now a 4-vector $w$ such that
\be
w^2 = -1, \quad (w \cdot s) =  (w \cdot q) = 0.
\ee

It is defined up to a rotation in the plane that is orthogonal
to the one built upon $q$ and $s$, i.e. it can be represented as
\bel{gamma-w-can-be-rotated}
w^\mu = w_1^\mu \cos\alpha  + w_2^\mu \sin\alpha ,
\ee

where $\alpha$ is the rotation angle in the plane defined by 
the space-like vectors $w_1$ and $w_2$ that are chosen such that
\be
q, \quad s, \quad w_1, \quad w_2
\ee

form an orthonormal basis in the Minkowski space.

 Then
\bem
\slsh{q}\slsh{w} & = & 2 (q \cdot w) - \slsh{w}\slsh{q}
= - \slsh{w}\slsh{q},
\nel
\slsh{s}\gd5\slsh{w} & = & - \slsh{s}\slsh{w} \gd5 = \slsh{w} \slsh{s}\gd5
.
\ee

Thus
\bel{gamma-s-prime-commutations}
\Lambda_\pm \slsh{w} = \slsh{w} \Lambda_\mp,
\quad
\Sigma_\pm \slsh{w} = \slsh{w} \Sigma_\pm 
,
\ee

and
\be
\Lambda_+\Sigma_+ \slsh{w} v_+
= \Lambda_+\slsh{w}\Sigma_+  v_+
= \slsh{w}\Lambda_-\Sigma_+  v_+ = \slsh{w}v_+,
\ee
\be
\Lambda_+\Sigma_+ \slsh{w}\gd5 u_-
=\slsh{w}\Lambda_-\Sigma_+ \gd5 u_-
=\slsh{w}\gd5\Lambda_+\Sigma_- u_-
=\slsh{w}\gd5 u_-.
\ee

Hence $\slsh{w}v_+$ and $\slsh{w}\gd5 u_-$ are proportional to
$u_+$. By choosing appropriate scale factors I may define
$u_-$, $v_\pm$ by

\be
u_+ = \slsh{w}\gd5 u_- = \slsh{w} v_+ = \gd5 v_- 
.
\ee

This definition together with the properties 
$\gd5^2 = -\slsh{w}^2 = 1$ and $\slsh{w}\gd5 = - \gd5\slsh{w}$
leads to

\bel{gammaTmpUVAltRel}
\ba{rclrclrcl}
u_+  & = & \slsh{w}\gd5 u_- ,
& u_+ & = & \slsh{w} v_+,
& u_+ & = & \gd5 v_-  ,
\\
u_-  & = & \slsh{w} \gd5 u_+ ,
& u_- & = & \gd5 v_+,
& u_-  & = & \slsh{w} v_-,
\\
v_+  &= &- \slsh{w} u_+ ,
& v_+ & = & \gd5 u_-,
& v_+ & = & - \slsh{w} \gd5 v_- ,
\\
v_- &=& \gd5 u_+ ,
& v_- & = & - \slsh{w} u_- 
& v_- & = & - \slsh{w} \gd5 v_+,
.
\ea
\ee

The spinors $v$ given by the last expression may differ from the ones obtained
from~\rf{gammaTmpUVConnection} but the difference can only be present in the
phase factors because I have from \rf{gammaTmpUVAltRel},
\be
\ba{rclclclcl}
\ub_- & = & \cc{(\slsh{w}\gd5u_+)}  B
&=& u_+^\hc \gd5^\hc \slsh{w}^\hc B
&=& -u_+^\hc B\gd5 BB^{-1}\slsh{w}B^{-1}B
&=& -\ub_+ \gd5 \slsh{w},
\\
\vb_+ &=& \cc{(-\slsh{w}u_+)}B &=& -u_+^\hc \slsh{w}^\hc B
&=& -u_+^\hc B \slsh{w} B^{-1}B &=& - \ub_+ \slsh{w}
,
\\
\vb_- &=& \cc{(\gd5 u_+)}B &=& u_+^\hc \gd5^\hc B
&=& -u_+^\hc B \gd5 B^{-1}B &=& - \ub_+ \gd5
.
\ea
\ee

Thus
\be
\ba{rclclcl}
\ub_- u_- &=& -\ub_+ \gd5 \slsh{w} \slsh{w}\gd5u_+ &=& \ub_+ \gd5 \gd5u_+
= \ub_+ u_+,
\\
\vb_+ v_+ &=& \ub_+ \slsh{w} \slsh{w} u_+ &=& -\ub_+ u_+,
\\
\vb_- v_- &=& -\ub_+ \gd5\gd5 u_+ &=& -\ub_+ u_+,
\ea
\ee

and from $\ub_+ u_+ = 1$ I have $1 = \ub_- u_- = - \vb_+ v_+ = -\vb_- v_-$,
i.e. the same normalization as the one
in~\rf{gammaTmpUNorm} and~\rf{gammaTmpUVConnection}.

\P
These results can be used to calculate for example the value of $\gu\mu u_+$.
I~represent it as a linear combination of $u_\pm$ and $v_\pm$,
\bel{gamma-on-u-spinor-equation}
\gu\mu u_+ = a^\mu u_+ + b^\mu u_- + c^\mu v_+ + d^\mu v_-,
\ee

and then multiply it by $u_\pm$ and $v_\pm$ to find the coefficients.
So

\bem
a^\mu & = & \ub_+\gu\mu u_+ = \Tr[\gu\mu u_+ \ub_+]
= \Tr[\gu\mu\Lambda_+\Sigma_+]
= \frac{1}{4} \Tr[\gu\mu(1 + \qc)(1 + \slsh{s}\gd5)]
\nel
& = & \frac{1}{4} \Tr[\gu\mu\qc] = q^\mu,

\nel
 b^\mu & = & \ub_-\gu\mu u_+ = - \Tr[\gu\mu u_+ \ub_+ \gd5 \slsh{w}]
= - \frac{1}{4} \Tr[\gu\mu (1 + \qc)(1 + \slsh{s}\gd5) \gd5 \slsh{w}]
\nel
& = & - \frac{1}{4}
\Tr[\gu\mu \gd5 \slsh{w} + \gu\mu \qc\slsh{s}\gd5\gd5 \slsh{w}]
= - \frac{1}{4}
\Tr[\gu\mu \qc\slsh{s}\slsh{w}]
\nel
& = &  - \frac{1}{4}
[q^\mu (s \cdot w) - s^\mu (q \cdot w) + w^\mu (q \cdot s)] = 0,

\nel
- c^\mu & = & \vb_+\gu\mu u_+ = - \Tr[\gu\mu u_+ \ub_+ \slsh{w}]
= - \frac{1}{4} \Tr[\gu\mu (1 + \qc)(1 + \slsh{s}\gd5) \slsh{w}]
,
\nel
c^\mu & = &
\frac{1}{4} \Tr[\gu\mu\slsh{w} - \gu\mu \qc\slsh{s}\slsh{w}\gd5]
= w^\mu + i \varepsilon^{\mu\nu\rho\sigma} q_\nu s_\rho w_\sigma,
\nel

- d^\mu & = & \vb_-\gu\mu u_+ = - \Tr[\gu\mu u_+ \ub_+ \gd5]
= - \frac{1}{4} \Tr[\gu\mu (1 + \qc)(1 + \slsh{s}\gd5) \gd5],
\nel
d^\mu & = &
\frac{1}{4} \Tr[\gu\mu\slsh{s}\gd5 \gd5]
= \frac{1}{4} \Tr[\gu\mu\slsh{s}] = s^\mu,
\ee

where I used \rf{gammaTraces} to calculate the traces. The result is

\bel{gamma-on-u-plus-spinor}
\gu\mu u_+ = q^\mu u_+
+ c^\mu v_+ + s^\mu v_-,
\ee

with
\bel{gamma-c-vector-is}
 c^\mu = w^\mu
  + i \varepsilon^{\mu\nu\rho\sigma} q_\nu s_\rho w_\sigma
.
\ee

If I substitute the expression~\rf{gamma-w-can-be-rotated} for $w$
in~\rf{gamma-c-vector-is} I would have
\be
c^\mu = \cos\alpha w_1^\mu + \sin\alpha w_2^\mu
  + i \cos\alpha \varepsilon^{\mu\nu\rho\sigma} q_\nu s_\rho (w_1)_\sigma
  +  i \sin\alpha \varepsilon^{\mu\nu\rho\sigma} q_\nu s_\rho (w_2)_\sigma.
\ee

But one can prove that from the fact that $q$, $s$, $w_1$, $w_2$ form
an orthonormal basis it follows that
\be
\varepsilon^{\mu\nu\rho\sigma} q_\nu s_\rho (w_1)_\sigma = \pm w_2^\mu,
\quad 
\varepsilon^{\mu\nu\rho\sigma} q_\nu s_\rho (w_2)_\sigma = \mp w_1^\mu,
\ee

where the sign depends on the basis orientation.
Thus
\bem
c^\mu &=& 
\cos\alpha w_1^\mu + \sin\alpha w_2^\mu
  \pm i \cos\alpha w_2^\mu
  \mp  i \sin\alpha w_1^\mu
\nel
& = &
w_1^\mu (\cos\alpha \mp i \sin\alpha) + w_2^\mu(\sin\alpha \pm i \cos\alpha)  
\nel
& = &
w_1^\mu e^{\mp i\alpha} \pm i w_2^\mu(\mp i \sin\alpha + \cos\alpha)  
\nel
& = &
e^{\mp i\alpha}(w_1^\mu \pm i w_2^\mu) 
  .
\ee

So
\be
\gu\mu u_+ = q^\mu u_+
+ e^{\mp i\alpha} (w_1^\mu \pm i w_2^\mu) v_+ + s^\mu v_-
.
\ee

Here the factor $e^{\mp i\alpha}$ reflects the ambiguity that is 
present in~\rf{gammaTmpUVAltRel} for the definition of $v_+$.

\P
By multiplying~\rf{gamma-on-u-plus-spinor} by $\slsh{w}$, $\gd5$ I also find
\bem
-\gd5 \gu\mu u_+ & = & \gu\mu \gd5 u_+ = -q^\mu \gd5 u_+
- c^\mu \gd5 v_+ - s^\mu \gd5 v_-,
\nel
-\slsh{w}\gu\mu u_+ & = & \gu\mu \slsh{w} u_+ - 2w^\mu u_+
= -q^\mu \slsh{w} u_+
- c^\mu \slsh{w} v_+ - s^\mu \slsh{w} v_-,
\nel
\gu\mu \slsh{w} u_+ & = &
2w^\mu u_+ - q^\mu \slsh{w} u_+
- c^\mu \slsh{w} v_+ - s^\mu \slsh{w} v_-
,
\nel
\slsh{w}\gd5 \gu\mu u_+ & = & \gu\mu \slsh{w}\gd5 u_+ - 2w^\mu \gd5u_+
= q^\mu \slsh{w}\gd5 u_+
+ c^\mu \slsh{w}\gd5 v_+ + s^\mu \slsh{w}\gd5 v_-
\nel
\gu\mu \slsh{w}\gd5 u_+ & = & 2w^\mu \gd5u_+
+ q^\mu \slsh{w}\gd5 u_+
+ c^\mu \slsh{w}\gd5 v_+ + s^\mu \slsh{w}\gd5 v_-
.
\ee

Now the relations~\rf{gammaTmpUVAltRel} result in
\bem
\gu\mu v_- &=& -q^\mu v_- - c^\mu u_- - s^\mu u_+,
\nel
\gu\mu v_+ &=& -q^\mu v_+ + (c^\mu - 2w^\mu) u_+ + s^\mu u_-,
\nel
\gu\mu u_- &=& q^\mu u_- - (c^\mu - 2w^\mu) v_- - s^\mu v_+
.
\ee

These may also be represented in a more compact form as,
\beml{gamma-on-spinors}
\gu\mu u_\pm  & = & q^\mu u_\pm +
(w^\mu \pm i \varepsilon^{\mu\nu\rho\sigma} q_\nu s_\rho w_\sigma)v_\pm
\pm s^\mu v_\mp,

\nel
\gu\mu v_\pm  & = & -q^\mu v_\pm -
(w^\mu \mp i \varepsilon^{\mu\nu\rho\sigma} q_\nu s_\rho w_\sigma)u_\pm
\pm s^\mu u_\mp
.
\ee

With the help of~\rf{gammaTmpUVAltRel} one also finds
\beml{gamma-5-gamma-on-spinors}
\gd5\gu\mu u_\pm  & = & q^\mu v_\mp +
(w^\mu \pm i \varepsilon^{\mu\nu\rho\sigma} q_\nu s_\rho w_\sigma)u_\mp
\pm s^\mu u_\pm,

\nel
\gd5\gu\mu v_\pm  & = & -q^\mu u_\mp -
(w^\mu \mp i \varepsilon^{\mu\nu\rho\sigma} q_\nu s_\rho w_\sigma)v_\mp
\pm s^\mu v_\pm
.
\ee


The relations~\rf{gamma-on-spinors} and~\rf{gamma-5-gamma-on-spinors} 
also mean that
\bel{gamma-u-and-v-1}
\ba{rclrcl}
\ub_\pm \gu\mu u_\pm & = & q^\mu,
\quad
& \vb_\pm \gu\mu v_\pm  &= &- q^\mu,
\cr
\ub_\pm \gu\mu\gd5 u_\pm  &=& \mp  s^\mu,
& \vb_\pm \gu\mu\gd5 v_\pm & =& \mp  s^\mu
\ea
.
\ee


\section{Dirac Equation}
\label{gammaSectionDiracEquation}

In this section I suppose that $g^{\mu\nu} = \eta^{\mu\nu}$, i.e.
it is the standard Minkowski metric \rf{convCartesianMetric} and for any two
4-vectors $p$, $r$
\be
(p \cdot r) = p^0 r^0 - p^1 r^1 - p^2 r^2 - p^3 r^3
\equiv p^0 r^0 - \vec{p} \vec{r}
\ee

\P
The linear first-order matrix partial differential equation defined by
\bel{gammaDiracEqn}
( i\gu\mu \partial_\mu - m)\psi(x_0, x_1, x_2, x_3) = 0,
\ee
is called the Dirac equation.

\P
A direct consequence of~\rf{gammaDiracEqn} can be obtained by multiplying
both sides of the equation by $(i\gu\nu \partial_\nu + m)$,

\be
0 = ( i\gu\nu \partial_\nu + m)( i\gu\mu \partial_\mu - m)\psi
= (- \gu\nu \partial_\nu \gu\mu \partial_\mu - m^2)\psi
= -(\gu\nu\gu\mu \partial_\nu \partial_\mu + m^2)\psi,
\ee
\be
\gu\nu\gu\mu \partial_\nu \partial_\mu =
(\gu\nu\gu\mu \partial_\nu \partial_\mu
+ \gu\mu\gu\nu \partial_\mu \partial_\nu )/ 2
=(\gu\nu\gu\mu + \gu\mu\gu\nu) \partial_\nu \partial_\mu  / 2
= g^{\mu\nu}\partial_\nu \partial_\mu
.
\ee
Thus
\be
(\partial^\mu \partial_\mu + m^2)\psi = 0,
\ee

i.e. any solution of the Dirac equation also satisfies 
the Klein-Gordon equation and hence may be represented in 
the form~\cite{SMTextBook},

\bel{gammatmpDiracPsiIs}
\psi(x) = \posPart\psi(x) + \negPart\psi(x)
= \int d^3p [\posPart\psi(\vecp) e^{-ipx} + \negPart\psi(\vecp) e^{ipx}],
\ee
where $px \equiv p^0 x^0 - \vec{p} \vec{x}$ with 
$p^0 \equiv E_p = +\sqrt{m^2 + \vec{p}^2}$.

To find equations for $\posPart\psi$ and $\negPart\psi$, I
substitute~\rf{gammatmpDiracPsiIs} into \rf{gammaDiracEqn},
\be
0 = \int d^3p [(p_\mu \gu\mu - m)\posPart\psi(\vecp) e^{-ipx}
     - (p_\mu \gu\mu + m)\negPart\psi(\vecp) e^{ipx}],
\ee
\be
(\pc - m)\posPart\psi(\vecp) = 0,
\quad
(\pc + m)\negPart\psi(\vecp) = 0,
\ee
or
\bel{gammaDiracTmpPsi}
(\qc - 1)\posPart\psi(\vecp) = 0,
\quad
(\qc + 1)\negPart\psi(\vecp) = 0,
\quad q \equiv p/m
.
\ee

From the definition~\rf{gammaProjDef} of the projection operators 
$\Lambda_\pm$ I may write for~\rf{gammaDiracTmpPsi}
\bel{gamma-select-positive-and-negative}
0 = \Lambda_-\posPart\psi(\vecp) = (1 - \Lambda_+)\posPart\psi(\vecp),
\quad \Lambda_+\posPart\psi(\vecp) = \posPart\psi(\vecp),
\ee
\be
0 = \Lambda_+\negPart\psi(\vecp) = (1 - \Lambda_-)\negPart\psi(\vecp),
\quad \Lambda_-\negPart\psi(\vecp) = \negPart\psi(\vecp).
\ee

It means that $\posPart\psi(\vecp)$ is an eigenvector of $\Lambda_+$,
so it can be represented as a linear combination of  $u_\pm(q, s)$,
the eigenvectors of $\Lambda_+$ (see \rf{gammaProjIs}),
\bel{gammaTmpPsiPlus}
\posPart\psi(\vecp) = c_+(p, s) u_+(q, s) + c_-(p, s) u_-(q, s).
\ee

In the same way,
\bel{gammaTmpPsiMinus}
\negPart\psi(\vecp) = d_+(p, s) v_+(q, s) + d_-(p, s) v_-(q, s).
\ee

Here the coefficients $c_\pm$ and $d_\pm$ are arbitrary complex functions
of $p$ and $s$,
defined by initial or boundary conditions for \rf{gammaDiracEqn}
and $s$ is a space-like four-vector that according to~\rf{gammaProjDef}
should satisfy, 
\be
(q \cdot s) = q^0s^0 - \vec{q}\vec{s} = 0, 
\quad s^2 = (s^0)^2 - \abs{\vec{s}}^2 = -1.
\ee

The requirement for the four-vector $s$ to be orthogonal to $q$ means that 
in general $s$ should be some function of $q$ because the only four-vector
that is orthogonal to any time-like unit vector is the zero vector
and it contradicts to the requirement $s^2 = -1$. The following
representation shows a possible choice for $s = s(q)$,
\be
s = (a, \sqrt{a^2 + 1}\,\vec{n}), 
\ee
where $\vec{n}$ is a unit constant 3-vector, $\abs{\vec{n}} = 1$,
and
\be 
a = a(q) 
= {\vec{q} \cdot \vec{n} \over 
   \sqrt{q_0^2 - (\vec{q} \cdot \vec{n})^2}}, 
\ee
\be
\sqrt{a^2 + 1} 
=\sqrt{q_0^2 - (\vec{q} \cdot \vec{n})^2 + (\vec{q} \cdot \vec{n})^2 
      \over q_0^2 - (\vec{q} \cdot \vec{n})^2} 
={q_0 \over \sqrt{q_0^2 - (\vec{q} \cdot \vec{n})^2}} 
= a{q_0 \over \vec{q} \cdot \vec{n}}
.      
\ee

For such $s$ I have
\be
s^2 = a^2 - (a^2 + 1) = -1, 
\ee
\be
(s \cdot q) = aq_0 - \sqrt{a^2 + 1}\, \vec{q} \cdot \vec{n}
= aq_0 - a{q_0 \over \vec{q} \cdot \vec{n}} \vec{q} \cdot \vec{n} = 0.
\ee
 
It may seem that in~\rf{gammaTmpPsiPlus} and~\rf{gammaTmpPsiMinus} 
$c_\pm(p, s)$ and $d_\pm(p, s)$ should not depend on the mass $m$, i.e.
they should depend only on $q = p/m$, but this does not hold for the general
solution of~\rf{gammaDiracTmpPsi}. For example, one can multiply 
$\psi(\vec{p})$ by some non-trivial function of $m$ thus
introducing the dependence of $\psi$ on $m$.

\P 
Now I can write the solution of the Dirac equation as
\bel{gammaTmpSolution}
\psi(x) =
\sum_{r = +, -}\int d^3p
    [ u_r(q,s(q))c_r(p, s(q)) e^{-ipx} + v_r(q,s(q))d_r(p, s(q)) e^{ipx} ]
\ee

To rewrite it in a more standard form I redefine $c$ and $d$,

\be
c_r(p) \rightarrow
\left(\frac{m}{(2\pi)^3 E_p}\right)^{\frac{1}{2}}c_r(p),
\quad
d_r(p) \rightarrow
\left(\frac{m}{(2\pi)^3 E_p}\right)^{\frac{1}{2}}d_r^*(p),
\ee

\bel{gammaFermionFieldClassic}
\psi(x) =
\sum_{r = +, -}\int d^3p \left(\frac{m}{(2\pi)^3 E_p}\right)^{\frac{1}{2}}
    [ u_r(q)c_r(p) e^{-ipx} + v_r(q)d_r^*(p) e^{ipx} ] ,
\ee
where I omitted the dependence on $s(q)$ in all quantities.

\P
This is a representation of the classical solution.
Under the second quantization
$c$ and $d$ become fermion and antifermion annihilation operators
with usual anticommutation relations,
\be
 \{c_r(p), c_{r'}^\hc(p')\}
= \{d_r(p), d_{r'}^\hc(p')\}
 = \delta_{rr'}\delta^3(\vec{p'} - \vec{p}),
 \quad r, r' = +,-,
\ee
\bel{gamma-c-and-d-anticommute}
\{c_r(p), c_{r'}(p')\} = \{d_r(p), d_{r'}(p')\} = 
\{c_r(p), d_{r'}(p')\} =\{c_r(p), d_{r'}^\hc(p')\} = 0, 
\ee

and $\psi$ has the standard form of a free fermion field:

\beml{gammaFermionField}
\lefteqn{
\psi(x) \; = \; \posPart\psi(x) + \negPart\psi(x)
}
\nel & = &
\sum_{r = +, -}\int d^3p \left(\frac{m}{(2\pi)^3 E_p}\right)^{\frac{1}{2}}
    [ u_r(p/m)c_r(p) e^{-ipx} + v_r(p/m)d_r^\hc(p) e^{ipx} ] ,
\nel
\lefteqn{
\bar{\psi}(x) \; = \;  
\overline{\posPart\psi}(x) + \overline{\negPart\psi}(x)
}
\nel & = &
\sum_{r = +, -}\int d^3p \left(\frac{m}{(2\pi)^3 E_p}\right)^{\frac{1}{2}}
    [ \ub_r(p/m)c_r^\hc(p) e^{ipx} + \vb_r(p/m)d_r(p) e^{-ipx} ].
\nel&&    
\ee


\section{Charge conjugation and Majorana spinors}

The Dirac equation for a particle in an external field $A_\mu$
is given by~\cite{SMTextBook}
\bel{gammaTmpDirExt}
[\gu\mu (i\partial_\mu + eA_\mu) - m]\psi(x) = 0,
\ee
where e is a coupling constant between $\psi(x)$ and $A_\mu$.
The constant $e$ also may be viewed as the 
charge annihilated by the $\psi$ field.

\P
This equation is invariant under the charge conjugation transformation
\bel{gammaChargeConj}
\psi \rightarrow \psi' = \psi^c \equiv C^{-1}{\bar\psi}^T,
\ee
\bel{gamam-C-on-A-and-e}
A_\mu \rightarrow A_\mu' = A_\mu^\hc
\quad
e \rightarrow e' = -e,
\ee
where the matrix $C$ is defined in~\rf{gammaCProperties}.

To show it I apply matrix transposition for 
$[\gu\mu (i\partial_\mu + e'A_\mu') - m]\psi'$
and use the property $C^T = -C$ of the matrix $C$,
\bem
\lefteqn{
\{[\gu\mu (i\partial_\mu + e'A_\mu') - m]\psi'\}^T 
= 
\psi'^T[\gu\mu^T (i{\bVec\partial}_\mu + e'A_\mu') - m]
}
\nel & = &
\psi'^T[-C\gu\mu C^{-1} (i{\bVec\partial}_\mu + e'A_\mu') - m] 
=
-\psi'^TC[\gu\mu (i{\bVec\partial}_\mu + e'A_\mu') + m]C^{-1}
\ee
where $F(x){\bVec\partial}_\mu \equiv \partial_\mu F(x)$.

Furthermore
\be
\psi'^T = (C^{-1}{\bar\psi}^T)^T
= \bar\psi (C^{-1})^T  = \bar\psi (C^T)^{-1}
= - \bar\psi C^{-1} 
\ee

from $(C^{-1})^T = (C^T)^{-1} = -C^{-1}$. 
So
\be
\{[\gu\mu (i\partial_\mu + e'A_\mu') - m]\psi'\}^T 
= \bar\psi C^{-1}C [\gu\mu (i{\bVec\partial}_\mu + e'A'_\mu) + m]C^{-1}
\ee
\be
= \bar\psi [\gu\mu (i{\bVec\partial}_\mu - e A_\mu^\hc) + m]C^{-1}
.
\ee

No I apply hermitian conjugation to~\rf{gammaTmpDirExt} and use 
the properties~\rf{gammaBProperties},
\bem
0 &=& \{[\gu\mu (i\partial_\mu + eA_\mu) - m]\psi\}^\hc
  = \psi^\hc[(\gu\mu)^\hc (-i{\bVec\partial}_\mu + eA_\mu^\hc) - m]
\nel & = &
\psi^\hc BB^{-1}[B\gu\mu B^{-1} (-i{\bVec\partial}_\mu + eA_\mu^\hc) - m]
=\bar\psi [\gu\mu (-i{\bVec\partial}_\mu + eA_\mu^\hc) - m]B^{-1}
,
\ee
\be
0 = \bar\psi [\gu\mu (-i{\bVec\partial}_\mu + eA_\mu^\hc) - m]
\ee

or
\be
  0 = \bar\psi [\gu\mu (i{\bVec\partial}_\mu - eA_\mu^\hc) + m]
.
\ee

It gives
\be
\{[\gu\mu (i\partial_\mu + e'A_\mu') - m]\psi'\}^T = 0
\ee
or simply
\be
[\gu\mu (i\partial_\mu + e'A_\mu') - m]\psi' = 0.
\ee

\P
A field is called neutral with respect to the charge $e$
if it is invariant under the charge conjugation 
transformation~\rf{gammaChargeConj}. For the field $A$ it simply means
that it should be hermitian, $A = A^\hc$. In the case of a neutral fermion
field which is called by definition a Majorana field, the necessary
conditions are given by the following requirement on the Majorana 
spinor $\psi_M$
\bel{gammaMajoranaRestriction}
\psi_M = \psi_M^c = C^{-1}(\bar\psi_M)^T = C^{-1}{\bar\psi_M}^T
.
\ee

\P
Now I consider the ``Dirac equation'' for a free Majorana spinor, 
\be
(i\gu\mu \partial_\mu - m)\psi_M = 0.
\ee

Its solution is given by \rf{gammaFermionFieldClassic} together 
with the restriction~\rf{gammaMajoranaRestriction}.
I have then from~\rf{gammaSpinorRelation}
$C^{-1} \vb_r^T = u_r$, $C^{-1}\ub_r^T = v_r$ and
\be
\psi^c_M = C^{-1}{\bar\psi_M}^T =

\sum_{r = +, -}\int d^3p \left(\frac{m}{(2\pi)^3 E_p}\right)^{\frac{1}{2}}
    [ v_r(q)c_r^*(p) e^{ipx} + u_r(q)d_r(p) e^{-ipx} ] ,
\ee

It will be equal to
\be
\psi_M =
\sum_{r = +, -}\int d^3p \left(\frac{m}{(2\pi)^3 E_p}\right)^{\frac{1}{2}}
    [ u_r(q)c_r(p) e^{-ipx} + v_r(q)d_r^*(p) e^{ipx} ]
\ee

only if $d_r(p) = c_r(p)$. The Majorana solution
of the Dirac equation can thus be written 

\be
\psi_M =
\sum_{r = +, -}\int d^3p \left(\frac{m}{(2\pi)^3 E_p}\right)^{\frac{1}{2}}
    [ u_r(q)c_r(p) e^{-ipx} + v_r(q)c_r^*(p) e^{ipx} ]
\ee

After the second quantization $d_r(p) = c_r(p)$ means that
particle and antiparticle annihilation
operators coincide so there
is no difference between Majorana particles and antiparticles.
For such a field I have,

\beml{gammaMajoranaField}
\lefteqn{
\psi_M(x) \; = \; \posPart\psi_M(x) + \negPart\psi_M(x)
}
\nel & = &
\sum_{r = +, -}\int d^3p \left(\frac{m}{(2\pi)^3 E_p}\right)^{\frac{1}{2}}
    [ u_r(p/m)c_r(p) e^{-ipx} + v_r(p/m)c_r^\dag(p) e^{ipx} ] ,
\nel
\lefteqn{
\bar{\psi}_M(x) \; = \;  
\overline{\posPart{\psi_M}}(x) + \overline{\negPart{\psi_M}}(x)
}
\nel & = &
\sum_{r = +, -}\int d^3p \left(\frac{m}{(2\pi)^3 E_p}\right)^{\frac{1}{2}}
    [ \ub_r(p/m)c_r^\dag(p) e^{ipx} + \vb_r(p/m)c_r(p) e^{-ipx} ],
\ee

with the anticommutational relations
\bel{gamma-Majorana-c-anticommute}
 \{c_r(p), c_{r'}^\dag(p')\}
 = \delta_{rr'}\delta^3(\vec{p'} - \vec{p}),
\quad
\{c_r(p), c_{r'}(p')\} = 0,
 \quad r, r' = +,-.
\ee


\section{Properties of Majorana fields}


The representation~\rf{gammaMajoranaField} of the Majorana field $\psi_M(x)$
can be used to establish some constraints on a possible structure 
of the Hamiltonian. I consider a term in the Hamiltonian that 
has the following structure
\bel{gamma-Majorana-H-I-term}
H_I = \int d^3x \Norder[\bar\psi_M(x) M(x) \psi_M(x)]
\ee

where N stands for the normal ordered product and $M$ is a $4 \times 4$ matrix,
whose coefficients may depend on some other fields but do not depend on 
$\psi_M$. I substitude~\rf{gammaMajoranaField} 
into~\rf{gamma-Majorana-H-I-term}:

\bem
H_I & = & 
\sum_{r, r' = +, -}\int d^3\!x\, d^3\!p \, d^3\!p'
\left(\frac{m^2}{(2\pi)^6 E_pE_{p'}}\right)^{1 \over 2}
\nel && {} \times
\Norder\Bigl\{[\ub_r(q) c_r^\hc(p) e^{ipx} + \vb_r(q) c_r(p) e^{-ipx}]

\nel
&& {}\qquad \times
 M(x)[u_{r'}(q') c_{r'}(p') e^{-ip'x} + v_{r'}(q') c_{r'}^\hc(p') e^{ip'x}]
\Bigr\}

\nel
& = & 
\sum_{r, r' = +, -}\int d^3\!x\, d^3\!p \, d^3\!p'
\left(\frac{m^2}{(2\pi)^6 E_pE_{p'}}\right)^{1 \over 2}
\nel
&& {} \times
\Norder\Bigl\{\ub_r(q) M(x)e^{ix(p - p')} 
    u_{r'}(q') c_r^\hc(p) c_{r'}(p')
\nel && {} \quad 
+\vb_r(q) M(x)e^{-ix(p - p')} 
        v_{r'}(q') c_r(p) c_{r'}^\hc(p')
\nel && {} \quad 
+\ub_r(q) M(x)e^{ix(p + p')} 
        v_{r'}(q') c_r^\hc(p) c_{r'}^\hc(p')
\nel && {} \quad 
+\vb_r(q) M(x)e^{-ix(p + p')} 
        u_{r'}(q') c_r(p) c_{r'}(p')
\Bigr\}
.
\ee

If I use
\be
\ba{rclrcl}
\Norder[M c^\hc c] &=& (-1)^s c^\hc \Norder[M] c, 
& \Norder[M c c^\hc ] &=& - (-1)^s c^\hc \Norder[M] c, 
\\
\Norder[M c c ] &= &\Norder[M] c c, 
& \Norder[M c^\hc c^\hc ] & = & c^\hc c^\hc \Norder[M],
\ea
\ee

where $s = 0$ if $M c^\hc = c^\hc M$ and $s = 1$ if $M c^\hc = -c^\hc M$, 
then I find
 

\beml{gamma-Majorana-H-I-1}
H_I 
& = & 
\sum_{r, r' = +, -}\int d^3\!p \, d^3\!p'
\left(\frac{m^2}{E_pE_{p'}}\right)^{1 \over 2}
\nel
&& {} \times
[(-1)^s c_r^\hc(p) \ub_r(q) M(\vecp - \vecpp) u_{r'}(q') c_{r'}(p')
\nel && {} \quad 
-(-1)^s c_{r'}^\hc(p') \vb_r(q) M(\vecpp - \vecp) v_{r'}(q') c_r(p) 
\nel && {} \quad 
+c_r^\hc(p) c_{r'}^\hc(p') \ub_r(q) M(\vecp + \vecpp) v_{r'}(q') 
\nel && {} \quad 
+\vb_r(q) M(-\vecp - \vecpp) u_{r'}(q') c_r(p) c_{r'}(p')
],
\ee

where by definition
\be
M(\vecp) = \int d^3x \Norder[M(x)]e^{ixp} 
\equiv \int d^3x \Norder[M(t, \vec{x})] 
        e^{itE_p} e^{-i\scriptVec{x}\scriptVec{p}}
.
\ee

Now I take into account that for a scalar quantity $a$ one has $a = a^T$, thus
from~\rf{gammaSpinorRelation}
\bem
\ub_r(q) M u_{r'}(q') &=& [\ub_r(q) M u_{r'}(q')]^T
= u_{r'}^T(q') M^T \ub_r^T(q) = - \vb_{r'}(q') C^{-1} M^T C v_r(q)
\nel
& \equiv & - \vb_{r'}(q') \tilde{M} v_r(q),
\ee

with
\be
\tilde{M} = C^{-1} M^T C.
\ee

Similarly
\bem
\vb_r(q) M v_{r'}(q') & = & - \ub_{r'}(q') \tilde{M} u_r(q),
\nel
\vb_r(q) M u_{r'}(q') & = & - \vb_{r'}(q') \tilde{M} u_r(q),
\nel
\ub_r(q) M v_{r'}(q') & = & - \ub_{r'}(q') \tilde{M} v_r(q).
\ee

Thus I may write for $H_I$
\bem
H_I & = & \sum_{r, r' = +, -} \int d^3\!p \, d^3\!p'
\left(\frac{m^2}{E_pE_{p'}}\right)^{1 \over 2}
\nel
&& {} \times
[-(-1)^s c_r^\hc(p) \vb_{r'}(q') \tilde{M}(\vecp - \vecpp) v_r(q) c_{r'}(p')
\nel && {} \quad 
+(-1)^s c_{r'}^\hc(p') \ub_{r'}(q') \tilde{M}(\vecpp - \vecp) u_r(q) c_r(p) 
\nel && {} \quad 
- c_r^\hc(p) c_{r'}^\hc(p') \ub_{r'}(q') \tilde{M}(\vecp + \vecpp)v_r(q) 
\nel && {} \quad 
-\vb_{r'}(q') \tilde{M}(-\vecp - \vecpp)u_r(q)  c_r(p) c_{r'}(p')
],
\ee

or if I exchange $r \leftrightarrow r'$ and $p \leftrightarrow p'$ and
then change the order of summation and integration,

\beml{gamma-Majorana-H-I-2}
H_I & = & \sum_{r, r' = +, -}\int d^3\!p \, d^3\!p'
\left(\frac{m^2}{E_pE_{p'}}\right)^{1 \over 2}
\nel
&& {} \times
[- (-1)^s c_{r'}^\hc(p') \vb_{r}(q) \tilde{M}(\vecpp - \vecp) 
    v_{r'}(q') c_{r}(p)
\nel && {} \quad 
+ (-1)^s c_{r}^\hc(p) \ub_{r}(q) \tilde{M}(\vecp - \vecpp) u_{r'}(q') c_{r'}(p') 
\nel && {} \quad 
-  c_{r'}^\hc(p') c_{r}^\hc(p) \ub_{r}(q) \tilde{M}(\vecp + \vecpp)v_{r'}(q') 
\nel && {} \quad 
-\vb_{r}(q) \tilde{M}(-\vecp - \vecpp)u_{r'}(q')  c_{r'}(p') c_{r}(p)
]
\nel
& = & \sum_{r, r' = +, -}\int d^3\!p \, d^3\!p'
\left(\frac{m^2}{E_pE_{p'}}\right)^{1 \over 2}
\nel
&& {} \times
[(-1)^s c_{r}^\hc(p)\ub_{r}(q) \tilde{M}(\vecp - \vecpp) u_{r'}(q') c_{r'}(p')
\nel && {} \quad 
-(-1)^s c_{r'}^\hc(p') \vb_{r}(q) \tilde{M}(\vecpp - \vecp) v_{r'}(q') c_{r}(p)
\nel && \quad {}  
+c_{r}^\hc(p) c_{r'}^\hc(p')\ub_{r}(q) \tilde{M}(\vecp + \vecpp)v_{r'}(q')   
\nel && \quad  {} 
+\vb_{r}(q) \tilde{M}(-\vecp - \vecpp)u_{r'}(q')  c_{r}(p) c_{r'}(p') 
],
\ee

where at the last step I have used
\be
c_{r'}^\hc(p') c_{r}^\hc(p) = -c_{r}^\hc(p) c_{r'}^\hc(p'),
\quad
c_{r'}(p') c_{r}(p) = - c_{r}(p) c_{r'}(p'),
\ee

the direct consequence of~\rf{gamma-Majorana-c-anticommute}.

Now~\rf{gamma-Majorana-H-I-2} together with~\rf{gamma-Majorana-H-I-1}
gives

\beml{gamma-Majorana-H-I-3}
H_I & = & {1 \over 2} (H_I + H_I)
\nel
& = & {1 \over 2}\sum_{r, r' = +, -}\int d^3\!p \, d^3\!p'
\left(\frac{m^2}{E_pE_{p'}}\right)^{1 \over 2}
\nel
&& {} \times
\{(-1)^s c_{r}^\hc(p) \ub_{r}(q) [\tilde{M}(\vecp - \vecpp) + M(\vecp - \vecpp)]
        u_{r'}(q')  c_{r'}(p') 
\nel && {} \quad 
-(-1)^s c_{r'}^\hc(p') \vb_{r}(q) 
    [\tilde{M}(\vecpp - \vecp) + M(\vecpp - \vecp)]v_{r'}(q') c_{r}(p)
\nel && {} \quad 
+c_{r}^\hc(p) c_{r'}^\hc(p') 
    \ub_{r}(q) [\tilde{M}(\vecp + \vecpp) + M(\vecp + \vecpp)]v_{r'}(q')   
\nel && {} \quad 
+\vb_{r}(q) [\tilde{M}(-\vecp - \vecpp) + M](-\vecp - \vecpp)]
        u_{r'}(q')  c_{r}(p) c_{r'}(p') 
\} 
\nel
& = & \sum_{r, r' = +, -}\int d^3\!p \, d^3\!p'
\left(\frac{m^2}{E_pE_{p'}}\right)^{1 \over 2}
\nel
&& {} \times
\{(-1)^s c_{r}^\hc(p) \ub_{r}(q) M_S(\vecp - \vecpp)u_{r'}(q') c_{r'}(p') 
\nel && {} \quad 
-(-1)^s c_{r'}^\hc(p') \vb_{r}(q) M_S(\vecpp - \vecp)v_{r'}(q') c_{r}(p)
\nel && \quad {} 
+c_{r}^\hc(p) c_{r'}^\hc(p') \ub_{r}(q) M_S(\vecp + \vecpp)v_{r'}(q')   
\nel && \quad {} 
+\vb_{r}(q) M_S(-\vecp - \vecpp)u_{r'}(q')  c_{r}(p) c_{r'}(p') 
\}
,
\ee

where
\bel{gamma-M-S-p-is}
M_S(\vecp) = {1 \over 2} \int d^3\!x M(x)e^{ixp} \Norder[C^{-1}M^T(x)C + M(x)]
.
\ee

\P
Suppose now that $M(x)$ is given by $f_\mu(x) \gu\mu$ or 
$f_{\mu\nu}(x)\sigma^{\mu\nu}$ with
some tensors $f_\mu(x)$ and $f_{\mu\nu}(x)$.
Then \rf{gammaCProperties} gives
\bem
C^{-1}{\gu\mu}^T C  & = & - \gu\mu, 
\nel
C^{-1}{\sigma^{\mu\nu}}^T C  & = & 
\frac{i}{4}C^{-1}[\gu\mu, \gu\nu]^T C
\; = \; 
\frac{i}{4}[\gu\nu, \gu\mu] = \sigma^{\nu\mu} = - \sigma^{\mu\nu},
\ee

thus in both cases I have $C^{-1}M^TC + M = 0$ or $M_S(p) = 0$ and 
according~\rf{gamma-Majorana-H-I-2} $H_I = 0$.
So I have
\bel{gamma-Majorana-HI-may-be-zero}
\int d^3\,x \Norder\{\bar\psi_M(x) 
   [f_\mu(x) \gu\mu + f_{\mu\nu}(x)\sigma^{\mu\nu}] \psi_M(x)\} \equiv 0 .
\ee

for any Majorana field $\psi_M(x)$ and arbitrary operators 
$f_\mu(x)$ and $f_{\mu\nu}(x)$.

\P
The statement~\rf{gamma-Majorana-HI-may-be-zero} 
shows that the electric and magnetic dipole moment
of any Majorana particle should be zero because 
they can be described by a coupling term in the interaction Hamiltonian
proportional to $\bar\psi_M(x) \sigma^{\mu\nu} \psi_M(x) F_{\mu\nu}(x)$
where $F_{\mu\nu}(x)$ is the electromagnetic field tensor and thus should 
be zero. It also means that the coupling of 
the Majorana field current $\bar\psi_M(x) \gu\mu \psi_M(x)$ 
with an arbitrary vector field vanishes. 

\P
It is important to note that to prove~\rf{gamma-Majorana-HI-may-be-zero} 
I have used only 
the definition of Majorana particles and basic properties 
of the Dirac matrices. This approach is different from the one presented,
for example, in~\cite{MajoranaZeroMomentWork},~\cite{MajoranaZeroMomentWork2},
where the proof is based on the CPT theorem.

\P
These results can be generalized to the case of several Majorana fields
$\psi_M^a$, $a = 1..N$. One can show that 
for any hermitian $N\times N$ matrices $L$ and $M$ with operator coefficients,  
$L_{ab} = L_{ba}^\hc$, $M_{ab} = M_{ba}^\hc$ the following equations hold,

\beml{gammaMajoranaElectroZeroSums}
\int d^3\,x \sum_{a, b=1..N} 
\Norder[L_{ab}(x) \bar\psi_M^a(x) \gu\mu \psi_M^b(x)]
&=& 0,
\nel
\int d^3\,x \sum_{a, b=1..N} 
\Norder[M_{ab}(x) \bar\psi_M^a(x) \sigma^{\mu\nu} \psi_M^b(x)] &=& 0
.
\ee


\P
Another interesting property of Majorana particles is that the usual 
Lagrangian for the classical Dirac field cannot be used for Majorana fields.
If I simply write for the free classical Majorana field that satisfies
$[\psi_a(x), \psi_b(x')] = 0$,
\bel{gammaMajoranaLagrange}
L_M = \bar\psi (i\gu\mu \partial_\mu - m)\psi
\ee

together with $\psi = \psi^c = C^{-1}{\bar\psi}^T$ or
$\bar\psi = -\psi^T C$ I would have

\bel{gamma-classic-Majorana-mass}
\bar\psi m \psi  =  -\psi_M^T C m \psi
                 = -(\psi_M^T C m \psi)^T = \psi_M^T C m \psi.
\ee
                 
Thus, such a mass term would be zero. Furthermore
\bem
\bar\psi \gu\mu \partial_\mu \psi  & = & - \psi_M^T C \gu\mu \partial_\mu \psi
= -(\psi_M^T C \gu\mu \partial_\mu \psi)^T
= \psi_M^T  \gu\mu^T \bVec{\partial_\mu} C \psi
\nel & = &
- \psi_M^T  C\gu\mu \bVec{\partial_\mu} C^{-1}C \psi
= - \partial_\mu(\psi_M^T)  C\gu\mu \psi
\nel & = &
- \partial_\mu(\psi_M^T  C\gu\mu \psi) + \psi_M^T  C\gu\mu\partial_\mu \psi
= \partial_\mu(\bar\psi \gu\mu \psi) - \bar\psi \gu\mu \partial_\mu \psi,
\ee

So
\be
2 \bar\psi \gu\mu \partial_\mu \psi = \partial_\mu(\bar\psi \gu\mu \psi),
\ee

or
\be
L  = \frac{1}{2}\partial_\mu(\bar\psi \gu\mu \psi),
\ee

but this is useless because $\partial_\mu(\bar\psi \gu\mu \psi)$ 
is a total divergence.

\P
It may seem that a possible way to overcome the problem is to 
introduce explicitly a real Lagrange multiplier $\lambda$ 
to handle $\psi_M = \psi_M^c$ or $\psi_M^T C + \bar\psi = 0$,
\be
L_M
= \bar\psi (i\gu\mu \partial_\mu - m)\psi + (\psi^T C + \bar\psi) \lambda
= \bar\psi (i\gu\mu \partial_\mu - m)\psi 
   - \lambda^T C\psi  + \bar\psi \lambda
\ee

It gives the following equations of motion:
\be
(i\gu\mu \partial_\mu - m)\psi + \lambda = 0,
\quad
\bar\psi (-i\gu\mu \bVec{\partial_\mu} - m) - \lambda^T C = 0,
\quad
\psi^T C + \bar\psi = 0,
\ee

Then from the second and third I have,
\be
\lambda^T C = \bar\psi (-i\gu\mu \bVec{\partial_\mu} - m)
                 = \psi^T C(i\gu\mu \bVec{\partial_\mu} + m).
\ee

Thus
\bem
(\lambda^T C)^T & = & -C\lambda
= (\psi^T C (i\gu\mu \bVec{\partial_\mu} + m))^T
= (i\gu\mu^T \partial_\mu + m) C^T \psi
\nel
& = & - (- C i\gu\mu C^{-1}\partial_\mu + m) C \psi
= C (i\gu\mu \partial_\mu - m) \psi,
\ee

or
\be
(i\gu\mu \partial_\mu - m) \psi + \lambda = 0,
\ee

i.e the first equation.

Hence $\lambda$ cannot be fixed and this method does not work.
\P
The solution to the problem is to require even in the classical 
case $\psi_M$ to be a Grassmann spinor $\psi_M^G$, 
$\{\psi^G_a, \psi^G_b\} = 0$. 
For such spinors relations like $\psi_M^T C \psi = (\psi_M^T C \psi)^T$ 
do not hold (see~\rf{gammaPsiCpsi}) and the 
Lagrangian~\rf{gammaMajoranaLagrange} is not trivial in this case.



\section{Connection with 2-spinors}

\def\PL{\Pi_{\rm L}}
\def\PR{\Pi_{\rm R}}
\def\psiL{\psi_{\rm L}}
\def\psiR{\psi_{\rm R}}

The property~\rf{gammaFiveSquared} $\gd5^2 = 1$ 
permits to build two projection
operators $\PL$ and $\PR$,
\bel{gamma-helicity-projectors}
\PL = {1 - \gd5 \over 2},
\quad
\PR = {1 + \gd5 \over 2}.
\ee

They are called Left- and Right-handed projection operators and
satisfy the usual properties 
\be
\Pi_{\rm L,R}^2 = \Pi_{\rm L,R},
\quad 
\PL \PR = 0,
\quad 
\PL + \PR = 1.
\ee


Thus any spinor $\psi$ may be represented as a sum of its left and right-handed
part,
\bel{gamma-spinor=sum-of-left-and-right}
\psi = (\PL + \PR)\psi = \psi_{\rm L} + \psi_{\rm R},
\ee
\bel{gamma-left-aright-spinors}
\psi_{\rm L,R} \equiv \Pi_{\rm L,R} \psi
.
\ee

\P
From the fact that $\gd5$ anticommutes with $\gu\mu$ I find
\bem
\PL \gu\mu  &=& \gu\mu {1 + \gd5 \over 2} = \gu\mu \PR,
\nel
\PR \gu\mu  &=& \gu\mu \PL
.
\ee

In a similar way I have according to~\rf{gammaBProperties} 
and~\rf{gammaCProperties}
\bem
\PL^\hc  & = & {1 \over 2}(1 - \gd5^\hc) = {1 \over 2}(1 + B\gd5 B^{-1})
 = {1 \over 2} B (1 + \gd5)B^{-1} = B \PR B^{-1},
\nel
\PL^T & = & {1 \over 2}(1 - \gd5^T) = {1 \over 2}(1 - C\gd5 C^{-1})
 = {1 \over 2} C (1 -\gd5)C^{-1} = C \PL C^{-1}
 .
\ee

So
\bel{gamma-B-C-and-left}
B \PR = \PL^\hc B, \quad C \PL = \PL^T C
\ee
and
\bel{gamma-B-C-and-right}
B \PL = \PR^\hc B, \quad C \PR = \PR^T C
.
\ee

I use $1 = \PL + \PR = \PL^2 + \PR^2$ to rewrite the products
$\bar\psi \gu\mu \psi$ and $\bar\psi\psi$ in terms of $\psiL$ and $\psiR$,

\bem
\bar\psi \gu\mu \psi &=& \bar\psi \gu\mu (\PL^2 + \PR^2) \psi
= \bar\psi \gu\mu \PL \psiL + \bar\psi \gu\mu \PR \psiR
\nel
& = & \psi^\hc B \Pi_R \gu\mu \psiL + \psi^\hc B \Pi_L \gu\mu \psiR
= \psi^\hc \PL^\hc B \gu\mu \psiL + \psi^\hc \PR^\hc B \gu\mu \psiR
\nel
& = & \bar\psiL \gu\mu \psiL + \bar\psiR \gu\mu \psiR,
\ee

and
\bem
\bar\psi \psi & = & \bar\psi \PL^2 \psi + \bar\psi \PR^2 \psi
= \bar\psiR \psiL + \bar\psiL \psiR
.
\ee

So for the free Dirac field Lagrangian I have
\bel{gammma-free-Dirac-as-left-and-right}
L  = \bar\psi(i\gu\mu\partial_\mu - m)\psi 
= i\bar\psiL\gu\mu\partial_\mu\psiL
+ i\bar\psiR\gu\mu\partial_\mu\psiR
- m(\bar\psiL\psiR + \bar\psiR\psiL)
.
\ee

\P
I introduce now the spinors $\eta$ and $\zeta$ by
\bel{gamma-2-spinors}
\psiL = B^{-1}C^\hc \eta,
\quad
\psiR = \zeta^*,
\ee

These spinors satisfy 
\bem
\PR^*\zeta & = & \PR^* \psiR^* = (\PR \psiR)^* = \psiR^* = \zeta,
\nel
\PR^*\eta & = &\PR^* (C^\hc)^{-1} B \psiL = (\PR^T)^\hc (C^{-1})^\hc B \psiL
  = (C^{-1} \PR^T)^\hc B \psiL
  ,
\ee

or according to~\rf{gamma-B-C-and-right}
\bem
\PR^*\eta 
& = & (\PR C^{-1})^\hc B \psiL = (C^\hc)^{-1} \PR^\hc  B \psiL
= (C^\hc)^{-1} B \PL \psiL = (C^\hc)^{-1} B \psiL 
\nel
&=& \eta
.  
\ee

Thus both spinors are eigenvectors of the projection operator $\PR^*$.

Clearly all eigenvectors of the operator $\PR^*$ form 
a two dimensional-subspace in the spinor space. 
By definition, elements of this space are called 2-spinors, i.e.
the spinor $\phi$ is a 2-spinor if it satisfies 
\bel{gamma-2-spinor-is}
\PR^* \phi = \phi
.
\ee

Any 4-spinor $\psi$ may be represented as a sum of two 2-spinors
$\eta$ and $\zeta$,
\beml{gamma-4-as-2-and-2}
\psi &=& B^{-1}C^\hc \eta + \zeta^*
\nel
& = & \psiL + \psiR
.
\ee

I have for $\bar\psi$ (with $B^\hc = B$):
\beml{gamma-bar-4-as-2-and-2}
\bar\psi & = & \psi^\hc B = \eta^\hc C B^{-1} B + \zeta^T B 
\nel
& = & \eta^\hc C  + \zeta^T B 
\nel
& = & \bar\psiL + \bar\psiR.
\ee

\P In terms of 2-spinors $\eta$ and $\zeta$, the 
Lagrangian~\rf{gammma-free-Dirac-as-left-and-right} reads 
\bel{gamma-two-tmp1}
L =
i \eta^\hc C \gu\mu B^{-1}C^\hc \partial_\mu  \eta
+ i \zeta^T B\gu\mu\partial_\mu\zeta^*
- m\eta^\hc C \zeta^* - m\zeta^T B B^{-1}C^\hc \eta
.
\ee

To simplify the first term in~\rf{gamma-two-tmp1} I use~\rf{gammaBCRelation} 
and~\rf{gammaCProperties},
\bem
\eta^\hc C \gu\mu B^{-1}C^\hc \partial_\mu  \eta
&=& - \eta^\hc C \gu\mu C^{-1}{B^T} \partial_\mu  \eta
= \eta^\hc \gu\mu^T B^T \partial_\mu  \eta
= (\eta^\hc \gu\mu^T B^T \partial_\mu  \eta)^T
\nel
&=& \partial_\mu  \eta^T B \gu\mu \eta^*
= \partial_\mu (\eta^T B \gu\mu \eta^*) - \eta^T B \gu\mu \partial_\mu\eta^*
.
\ee

If I omit the total divergence in the last expression I may write
\bel{gammma-free-Dirac-as-two}
L = -i \eta^T B \gu\mu \partial_\mu\eta^* 
   + i \zeta^T B\gu\mu\partial_\mu\zeta^*
- 2m\Re(\zeta^T C^\hc \eta),
\ee

where 
\be
2\Re(\zeta^T C^\hc \eta) = \eta^\hc C \zeta^* + \zeta^T C^\hc \eta
.
\ee


\P
Suppose now that $2N$ 2-spinor fields $\eta_a$ and $\zeta_a$, $a = 1..N$,
 are described by the following Lagrangian
\bel{gamma-two-multy}
L_N = -i \eta_a^T B \gu\mu \partial_\mu\eta_a^* 
   + i \zeta_a^T B\gu\mu\partial_\mu\zeta_a^*
- 2\Re(\zeta_a^T M_{ab}C^\hc \eta_b),
\ee 

with an $N\times N$ complex matrix $M_{ab}$ with non-zero determinant,
$\det M \ne 0$. 
Then to represent this Lagrangian as a sum of $N$ free Dirac fields I need
to diagonalize the matrix $M$. It can be done in the following way.
One can show that the matrix $M$ can be represented as a product 
of a hermitian $M_H$ and a unitary $\tilde{U}$ matrix, i.e.
\be
M = M_H \tilde{U}, \quad M_H^\hc = M_H, \quad \tilde{U}^\hc \tilde{U} = 1.
\ee

As the hermitian matrix $M_H$ may be diagonalized by some unitary matrix $U$,
$M_0 = U M_H U^\hc$, where $M_0$ is a real diagonal matrix. Thus 
\be
M_0 = U M \tilde{U}^{-1} U^\hc  \equiv X^T M Y, 
\ee

where $X$ and $Y$ are also unitary matrices given by
\be
X^T = U, \quad Y = \tilde{U}^{-1} U^\hc, 
\quad Y^\hc Y = X^\hc X = 1.
\ee

Now if I define the new set $\eta'_a$ and $\zeta'_a$ of fields by the relations
\be
\zeta'_a = X_{ab} \zeta_b, \quad \eta'_a =  Y_{ab} \eta_b
,
\ee
then I would have

\be
{\eta'_a}^T B \gu\mu \partial_\mu{\eta'_a}^*
= 
Y_{ab}Y^*_{ac} {\eta_b}^T B \gu\mu \partial_\mu{\eta_c}^*
= 
(YY^\hc)_{bc} {\eta_b}^T B \gu\mu \partial_\mu{\eta_c}^*
=
{\eta_a}^T B \gu\mu \partial_\mu{\eta_a}^*,
\ee

and similarly 
\be
{\zeta'_a}^T B\gu\mu\partial_\mu{\zeta'_a}^* 
= \zeta_a^T B\gu\mu\partial_\mu\zeta_a^*
,
\ee

so the fields $\eta'_a$ and $\zeta'_a$ appear in the same way 
as the set $\eta_a$ and $\zeta_a$ in the
Lagrangian kinetic terms. Also, the mass term has the form
\be
{\zeta'_a}^T M_{ab}C^\hc \eta'_b
= X_{ac} M_{ab} Y_{bd} \zeta_c^T C^\hc \eta_d
= (X^T M Y)_{ab} \zeta_a^T C^\hc \eta_b
= \sum_{a = 1..N} m_a \zeta_a^T C^\hc \eta_a,
\ee

with $m_a = (M_0)_{aa}$. Thus
\bel{gamma-two-multy-diagonalized}
L_N = \sum_{a = 1..N}[
-i {\eta'_a}^T B \gu\mu \partial_\mu{\eta'_a}^* 
   + i {\zeta'_a}^T B\gu\mu\partial_\mu{\zeta'_a}^*
- 2m_a\Re({\zeta'_a}^T C^\hc {\eta'_a})],
\ee

or if I combine the 2-spinor fields $\eta'_a$ and $\zeta'_a$ into 
the single Dirac field $\psi_a$ with the help of~\rf{gamma-4-as-2-and-2},
\be
\psi_a = B^{-1}C^\hc \eta'_a + {\zeta'_a}^*
\ee

I have
\be
\sum_{a = 1..N} \bar\psi_a (i\gu\mu\partial_\mu - m) \psi_a.
\ee

\P
Now I write the results of this section in
the Weyl representation$\gamma_W^\mu$ 
of the Dirac algebra for the Minkowski space.
This representation is defined by 
\bel{gamma-Weyl-representation}
\gamma_W^0 = \pmatrix{0 & 1_{2 \times 2} \cr 1_{2 \times 2} & 0},
\quad
\gamma_W^{i} = \pmatrix{0 & \sigma^i \cr - \sigma^{i} & 0},
\quad i = 1,2,3,
\ee

where the Pauli matrices $\sigma^i$ are given by~\rf{gamma-Pauli-matrices}.

\P
As in the case of the canonical 
representation~\rf{gamma-standard-representation}
$B_W$ can be set to $\gamma_W^0$, thus

\bem
B_W\gamma_W^0 &=& (\gamma_W^0)^2 = 1,
\nel
B_W\gamma_W^i &=& 
\pmatrix{0 & 1_{2 \times 2} \cr 1_{2 \times 2} & 0}
\pmatrix{0 & \sigma^i \cr - \sigma^{i} & 0}
= \pmatrix{-\sigma^i & 0 \cr 0 & \sigma^i}, \quad i = 1..3
.
\ee

The matrix $C_W$ is given by
\be
C_W = \pmatrix{-\varepsilon & 0 \cr 0 & \varepsilon}, 
\quad
\varepsilon = \pmatrix{0 & -1 \cr 1 & 0} = -i\sigma^2.
\ee

\P
In the Weyl representation $\gd5_W$ reads
\bem
\gamma_W^5 & = & i\gu0\gu1\gu2\gu3 = 
i\pmatrix{0 & 1_{2 \times 2} \cr 1_{2 \times 2} & 0}
\pmatrix{0 & \sigma^1 \cr - \sigma^{1} & 0}
\pmatrix{0 & \sigma^2 \cr - \sigma^{2} & 0}
\pmatrix{0 & \sigma^3 \cr - \sigma^{3} & 0}
\nel & = &
i\pmatrix{-\sigma^1 & 0 \cr 0 & \sigma^1}
\pmatrix{-\sigma^2 \sigma^3 & 0 \cr 0 & -\sigma^2 \sigma^3}
=
i\pmatrix{\sigma^1\sigma^2\sigma^3 & 0 \cr 0 & -\sigma^1\sigma^2\sigma^3}
\nel & = &
\pmatrix{-1_{2 \times 2} & 0 \cr 0 & 1_{2 \times 2}}
,
\ee

where I have used
\be
\sigma^1\sigma^2\sigma^3 = 
\pmatrix{ 0 & 1 \cr 1 & 0 }
\pmatrix{ 0 & -i \cr i & 0 }
\pmatrix{ 1 & 0 \cr 0 & -1 }
= 
\pmatrix{ i & 0 \cr 0 & -i }
\pmatrix{ 1 & 0 \cr 0 & -1 }
= i1_{2 \times 2}
.
\ee

It leads to
\bel{gamma-heliscity-in-Weyl}
\PL = \pmatrix{ 1 & 0 \cr 0 & 0 }, 
\quad 
\PR = \PR^* = \pmatrix{ 0 & 0 \cr 0 & 1 }
.
\ee

Thus any 2-spinor $\eta$ defined by~\rf{gamma-2-spinor-is}, 
$\PR^*\eta = \eta$ has the structure
\be
\eta = \pmatrix{ 0 \cr 0 \cr \eta_1 \cr \eta_2 } 
\equiv \pmatrix{ 0 \cr \eta },
\ee
where the spinor components $\eta_1$ and $\eta_2$ are two complex numbers.

If I introduce two-dimensional notations I would have
\be
\zeta^T C^\hc \eta \equiv 
\pmatrix{ 0 & \zeta^T } 
\pmatrix{-\varepsilon & 0 \cr 0 & \varepsilon}
\pmatrix{ 0 \cr \eta }
= \zeta^T \varepsilon \eta
,
\ee

and
\bem
\eta^T B \gu0 \partial_0\eta^*
&= &
\pmatrix{ 0 & \eta^T } 
\pmatrix{ 0 \cr \partial_0\eta^* } = \eta^T \partial_0 \eta^*,
\nel
\eta^T B \gu{i} \partial_{i}\eta^* 
&= &
\pmatrix{ 0 & \eta^T } 
\pmatrix{-\sigma^i & 0 \cr 0 & \sigma^i}
\pmatrix{ 0 \cr \partial_i\eta^* } = \eta^T \sigma^i \partial_i \eta^*,
\ee

with the sum over $i = 1..3$. So if I define $\sigma^0 = 1_{2\times 2}$ then
\bel{charge-two-spinor-3}
\eta^T B \gu\mu \partial_\mu\eta^* \equiv \eta^T \sigma^\mu \partial_\mu \eta^*
.
\ee

Thus in 2-spinor notation, the Dirac Lagrangian is
\be
L = -i \eta^T \sigma^\mu \partial_\mu \eta^*
   + i \zeta^T \sigma^\mu \partial_\mu \zeta^*
- 2m\Re(\zeta^T \varepsilon \eta).
\ee

The case~\rf{gamma-two-multy}
of the several 2-spinors field Lagrangian in Weyl representation reads
\beml{gamma-two-multy-in-Weyl}
L_N & = & -i \eta_a^T  \sigma^\mu \partial_\mu\eta_a^* 
   + i \zeta_a^T \sigma^\mu\partial_\mu\zeta_a^*
- 2\Re(\zeta_a^T M_{ab} \varepsilon \eta_b)
\nel
& = & -i \eta_a^T B \gu\mu \partial_\mu\eta_a^* 
   + i \zeta_a^T B\gu\mu\partial_\mu\zeta_a^*
- 2\Re(\zeta_a^T M_{ab}C^\hc \eta_b)
.
\ee 

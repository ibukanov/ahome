\chapter{$2\times 2$ matrix diagonalization}\label{appMatrix}

Consider a Hermitian $2\times 2$ matrix $H$
\be
H = \left(\ba{cc} x & z \\  z^* & y\ea\right), 
\quad x = x^*, \, y = y^*
\ee

with eigenvalues given by
\bem 
\lambda_1 & = & \frac{x + y}{2} + \sqrt{\frac{(x - y)^2}{4} + \abs{z}^2}, 
\nel
\lambda_2 & = & \frac{x + y}{2} - \sqrt{\frac{(x - y)^2}{4} + \abs{z}^2}, 
\ee 

where I choose 

\bel{lambdaOrder}
\lambda_1 \ge \lambda_2.
\ee

A unitary matrix $U$ that diagonalizes $H$ should fulfill
\bel{diagonalization}
UHU^\hc  = H_0 \quad \mbox{or} \quad UH = H_0U,
\ee
\be
H_0 = \left(\ba{cc} \lambda_1 & 0 \\ 0 &  \lambda_2 \ea\right).
\ee

A possible solution of \rf{diagonalization} can be obtained by writing
\bel{explicit-U-form}
U = \pba{cc} 
   \cos\theta & \sin\theta e^{i\phi} \\
   -\sin\theta e^{-i\phi} & \cos\theta
   \pea,
\ee
and
\bem
\left(\ba{cc} 
x\cos\theta  + z^*\sin\theta e^{i\phi} &
z\cos\theta + y\sin\theta e^{i\phi}\\
-x\sin\theta e^{-i\phi} + z^*\cos\theta &
-z\sin\theta e^{-i\phi} + y\cos\theta
\ea\right)
&&
\nel
\qquad = \quad
\pba{cc} 
   \lambda_1\cos\theta & \lambda_1\sin\theta e^{i\phi} \\
   -\lambda_2\sin\theta e^{-i\phi} & \lambda_2\cos\theta
\pea
&&
\ee

which gives
\bel{matrix-phase-is}
e^{i\phi} = \zOverAbs{z} \quad \mbox{or} \quad \phi = \Arg{z},
\ee 

\be
\abs{z}\sin\theta = (\lambda_1 - x)\cos\theta,
\ee

or
\be
\tan\theta = {\lambda_1 - x \over \abs{z}}
= \frac{y - x}{2\abs{z}} + \sqrt{\frac{(x - y)^2}{4\abs{z}^2} + 1}, 
\ee

\bel{matrix-theta-is}
\tan\theta = \sqrt{1 + \eta^2} - \eta, 
\quad \eta \equiv (x - y)/(2\abs{z}).
\ee

If $z = 0$ and $x \ge y$ the solution is $\theta = 0$ and $U = 1$.
If $z = 0$ and $x < y$ I have $\theta = \pi/2$ and 
\bel{formForU}
U = \left(\ba{cc} 
   0 & e^{i\phi} \\ e^{-i\phi} & 0
\ea\right),
\ee

It just reflects the fact that due to the requirement 
\rf{lambdaOrder} in this situation $U$ should merely interchange $x$ and $y$.

So the domain for $\theta$ can be restricted to the closed interval $[0,\pi/2]$.

\P
The solution~\rf{explicit-U-form} is not unique.
To find all possible $U'$ that satisfy~\rf{diagonalization} I represent 
such $U'$ as a product $\tilde{U}U$ where $U$ is given 
by~\rf{explicit-U-form} and $\tilde{U}$ is some unitary matrix. 
This leads to the equation
\be
H_0 = \tilde{U}U H U^\hc  \tilde{U}^\hc  = \tilde{U} H_0 \tilde{U}^\hc ,  
\ee
\be
H_0\tilde{U} - \tilde{U}H_0 = [H_0, \tilde{U}] =  0.
\ee
But $H_0$ is a diagonal and only diagonal matrices can commute with 
$H_0$ thus
\be
\tilde{U} = \left(\ba{cc} e^{i\psi} & 0 \\ 0 & e^{i\psi'} \ea\right),
\ee

with two arbitrary phase factors $e^{i\psi}$ and $e^{i\psi'}$. So all possible 
solutions of~\rf{diagonalization} have the form
\bel{allDiagonalization}
\tilde{U}U =
   \left(\ba{cc} e^{i\psi} & 0 \\ 0 & e^{i\psi'} \ea\right)
   \left(\ba{cc} 
      \cos\theta & \sin\theta e^{i\phi} \\
      -\sin\theta e^{-i\phi} & \cos\theta
      \ea\right)
.      
\ee

\P
When $H$ is real, $H^* = H$, $z$ also becomes  a real quantity and
\be
z = z^* \Leftrightarrow e^{i\phi} = \zOverAbs{z} \equiv \sign(z) = \pm 1
.
\ee

It is natural to require in this case for $U$ to be also real and 
hence an orthogonal matrix. From \rf{allDiagonalization} I find that 
any orthogonal matrix $O'$ that diagonalizes a real symmetric matrix $H_r$,
\be
H_r = \left(\ba{cc} x & z \\ z & y\ea\right)
= O'\left(\ba{cc} \lambda_1 & 0 \\ 0 & \lambda_2 \ea\right)O'^T
, 
\quad x = x^*, \quad y = y^*, \quad z = z^*,
\ee
\be
\ee

has the form
\bel{orthogonalDiagonalization}
O' =
   \left(\ba{cc} \pm 1 & 0 \\ 0 & \pm 1 \ea\right) O, 
\quad
O = \left(\ba{cc} 
      \cos\theta & \sin\theta \sign(z) \\
      -\sin\theta \sign(z) & \cos\theta
      \ea\right)
.
\ee

\P
Consider now the diagonalization problem for an arbitrary complex 
$2\times 2$ matrix~$C$
\be
C = \pba{cc} a & b \\ c & d\pea, 
\quad C^\hc  = \left(\ba{cc} a^* & c^*  \\ b^* & d^*\ea\right) 
\ee 
by two unitary matrixes $X$, $Y$. By definition they should 
fulfill\footnote{For conventional reasons 
I use $X^*$ and not just $X$ in~\rf{equationForUs}.
}

\bel{equationForUs}
X^*CY^\hc  = C_0 = \left(\ba{cc} \lambda_1 & 0\\ 0 & \lambda_2\ea\right).
\ee

As opposite to the case of the hermitian matrix $H$, in general 
$\lambda_1$, $\lambda_2$ in~\rf{equationForUs} are not eigenvalues 
of $C$ and they can be found by following the procedure 
described in~\cite{howCharginosCome}. I find
\be
C_0C_0^\hc  = C_0^\hc C_0 = X^*CY^\hc YC^\hc X^{*\hc} = X^*CC^\hc X^{*\hc} 
= YC^\hc CY^\hc 
\ee

This means that $X^*$ should diagonalize the hermitian matrix $CC^\hc $ and 
$Y$ should do this for $C^\hc C$ which in general differs from $CC^\hc $ 
but has the same eigenvalues $\abs{\lambda_{1,2}}^2$,
\be
CC^\hc  = \left(\ba{cc} 
   \abs{a}^2 + \abs{b}^2 & ac^* + bd^* \\ 
   ca^* + db^* & \abs{c}^2 + \abs{d}^2
\ea\right),
\quad
C^\hc C = \left(\ba{cc} 
   \abs{a}^2 + \abs{c}^2 & ba^* + dc^* \\ 
   ab^* + cd^* & \abs{b}^2 + \abs{d}^2
\ea\right),
\ee
\beml{lambdaForCompleMatrix}
\abs{\lambda_1}^2 & = & S + \sqrt{T}, 
\qquad
\abs{\lambda_2}^2 \; = \; S - \sqrt{T},
\nel
S &=& \frac{\abs{a}^2 + \abs{b}^2 + \abs{c}^2 + \abs{d}^2}{2} 
,
\nel
T &=& \frac{(\abs{a}^2 + \abs{b}^2 - \abs{c}^2 - \abs{d}^2)^2}{4}
      + \abs{ac^* + bd^*}^2.
\ee 

According to \rf{allDiagonalization} $X^*$ and $Y$ have the following 
representation,
\bel{X-and-Y-representation}
X^* = \tilde{X}X', 
\quad
Y = \tilde{Y}Y'.
\ee

Here $X'$, $Y'$ have the form \rf{explicit-U-form},
\be
X' = \pba{cc} 
   \cos\theta_X & \sin\theta_X e^{i\phi_X} \\
   -\sin\theta_X e^{-i\phi_X} & \cos\theta_X
   \pea,
\quad
Y' = \pba{cc} 
   \cos\theta_Y & \sin\theta_Y e^{i\phi_Y} \\
   -\sin\theta_Y e^{-i\phi_Y} & \cos\theta_Y
   \pea,
\ee

where from~\rf{matrix-phase-is} 
\bel{matrixXandYcoef}
e^{i\phi_X} = \zOverAbs{ac^* + bd^*}, 
\qquad
e^{i\phi_Y} = \zOverAbs{ba^* + dc^*}, 
\quad
\ee

and $\theta_X$ and $\theta_Y$ are defined by~\rf{matrix-theta-is},
\be
\tan\theta_X = \sqrt{1 + \eta_X^2} - \eta_X, 
\quad \eta_X = 
   \frac{\abs{a}^2 + \abs{b}^2 - \abs{c}^2 - \abs{d}^2}
        {2\abs{ac^* + bd^*}},
\ee

\be
\tan\theta_Y = \sqrt{1 + \eta_Y^2} - \eta_Y, 
\quad \eta_Y = 
   \frac{\abs{a}^2 + \abs{c}^2 - \abs{b}^2 - \abs{d}^2}
        {2\abs{ba^* + dc^*}},
\ee

In the general case $\tilde{X}$ and $\tilde{Y}$ cannot be set to $1$ and 
to determine the phase factors present in $\tilde{X}$ and $\tilde{Y}$,
\be
\tilde{X} = \left(\ba{cc} e^{i\psi_X} & 0 \\ 0 & e^{i\psi'_X} \ea\right),
\quad
\tilde{Y} = \left(\ba{cc} e^{i\psi_Y} & 0 \\ 0 & e^{i\psi'_Y} \ea\right),
\ee

I substitute~\rf{X-and-Y-representation} into~\rf{equationForUs} to find 
relations for $\tilde{X}$ and $\tilde{Y}$,

\bel{m-tmp-1}
\tilde{X}{X'}C{Y'}^\hc \tilde{Y}^\hc  = C_0, 
\quad \mbox{or} \quad 
\tilde{X}^\hc C_0\tilde{Y} \; = \; {X'}C{Y'}^\hc .
\ee

By writing the matrices in~\rf{m-tmp-1} explicitly I have
\bem
\lefteqn{
\pba{cc} \lambda_1e^{i(\psi_Y - \psi_X)} & 0 \\ 
    0 & \lambda_2e^{i(\psi'_Y - \psi'_X)} \pea
}
\nel
& = & 
\pba{cc} 
   \cos\theta_X & \sin\theta_X e^{i\phi_X} \\
   -\sin\theta_X e^{-i\phi_X} & \cos\theta_X
\pea
\pba{cc} 
 a & b \\ c & d
\pea
\pba{cc} 
   \cos\theta_Y &  -\sin\theta_Y e^{i\phi_Y} \\
   \sin\theta_Y e^{-i\phi_Y} & \cos\theta_Y
\pea
\nel &=&
\pba{cc} 
   a\cos\theta_X + c\sin\theta_X e^{i\phi_X} 
       & b\cos\theta_X + d\sin\theta_X e^{i\phi_X} \\
   -a\sin\theta_X e^{-i\phi_X} + c\cos\theta_X
       & -b\sin\theta_X e^{-i\phi_X} + d\cos\theta_X
\pea
\nel && \quad {} \times 
\pba{cc} 
   \cos\theta_Y &  -\sin\theta_Y e^{i\phi_Y} \\
   \sin\theta_Y e^{-i\phi_Y} & \cos\theta_Y
\pea
.
\ee

This gives me four equations,
\bem
\lambda_1e^{i(\psi_Y - \psi_X)}
&=&
(a\cos\theta_X + c\sin\theta_X e^{i\phi_X})\cos\theta_Y
\nel &&{} 
+(b\cos\theta_X + d\sin\theta_X e^{i\phi_X})\sin\theta_Y e^{-i\phi_Y},
\nel
0 &=&
-(a\cos\theta_X + c\sin\theta_X e^{i\phi_X})\sin\theta_Y e^{i\phi_Y}   
\nel &&{} 
+(b\cos\theta_X + d\sin\theta_X e^{i\phi_X})\cos\theta_Y
,
\nel
0 & = &
(-a\sin\theta_X e^{-i\phi_X} + c\cos\theta_X)\cos\theta_Y
\nel &&{} 
+
(-b\sin\theta_X e^{-i\phi_X} + d\cos\theta_X)\sin\theta_Y e^{-i\phi_Y}
,
\nel
\lambda_2e^{i(\psi'_Y - \psi'_X)}
& = &
-(-a\sin\theta_X e^{-i\phi_X} + c\cos\theta_X)\sin\theta_Y e^{i\phi_Y}
\nel &&{} 
+
(-b\sin\theta_X e^{-i\phi_X} + d\cos\theta_X)\cos\theta_Y
.
\ee

From the second and third equation it follows that

\bem
(b\cos\theta_X + d\sin\theta_X e^{i\phi_X}) & = &
(a\cos\theta_X + c\sin\theta_X e^{i\phi_X})\tan\theta_Y e^{i\phi_Y},
\nel 
(-a\sin\theta_X e^{-i\phi_X} + c\cos\theta_X) & = &
(b\sin\theta_X e^{-i\phi_X} - d\cos\theta_X)\tan\theta_Y e^{-i\phi_Y},
\ee

Thus the first and fourth equations give

\bem
\lambda_1e^{i(\psi_Y - \psi_X)}
&=&
(a\cos\theta_X + c\sin\theta_X e^{i\phi_X})\cos\theta_Y
\nel &&{} 
+(a\cos\theta_X + c\sin\theta_X e^{i\phi_X})
 \tan\theta_Y e^{i\phi_Y} \sin\theta_Y e^{-i\phi_Y}
\nel & = &
(a\cos\theta_X + c\sin\theta_X e^{i\phi_X})
(\cos\theta_Y + \tan\theta_Y \sin\theta_Y)
\nel
&=&
(a\cos\theta_X + c\sin\theta_X e^{i\phi_X})/\cos\theta_Y,
\nel
\lambda_2e^{i(\psi'_Y - \psi'_X)} & = &
(-b\sin\theta_X e^{-i\phi_X} + d\cos\theta_X)\tan\theta_Y e^{-i\phi_Y}
\sin\theta_Y e^{i\phi_Y}
\nel &&{} 
+
(-b\sin\theta_X e^{-i\phi_X} + d\cos\theta_X)\cos\theta_Y
\nel &=&
(-b\sin\theta_X e^{-i\phi_X} + d\cos\theta_X)
(\cos\theta_Y + \tan\theta_Y\sin\theta_Y)
\nel &=&
(-b\sin\theta_X e^{-i\phi_X} + d\cos\theta_X)/\cos\theta_Y,
\ee

where I used
\be
\cos\theta + \tan\theta\sin\theta
 = {1 \over \cos\theta}(\cos^2\theta + \sin^2\theta) = {1 \over \cos\theta}
 .
\ee

So
\bem
e^{i(\psi_Y - \psi_X)}
&=&
{a\cos\theta_X + c\sin\theta_X e^{i\phi_X} \over \lambda_1 \cos\theta_Y},
\nel
e^{i(\psi'_Y - \psi'_X)} & = &
{-b\sin\theta_X e^{-i\phi_X} + d\cos\theta_X \over \lambda_2\cos\theta_Y}
.
\ee

\P

From the last expressions I find that 
only the absolute values of $\lambda_{1,2}$
are fixed by the matrix $C$ and to determine the phase factors for 
$\lambda_{1,2}$ and to fix the phases $\psi_{X,Y}$, $\psi'_{X,Y}$ 
additional conditions should be given.

\P

For example, under the requirement
for $\lambda_{1,2}$ to be non-positive real numbers,
$\lambda_{1,2} = -\abs{\lambda_{1,2}}$ and for $\tilde{X}$ to be the 
unit matrix, 
$\tilde{X} = 1$ or $\psi_X$ = $\psi'_X$ = $0$,
$X$ and $Y$ have the form
\beml{matrixXandY}
X^* 
& = &
X'
\; = \;
\pba{cc} 
   \cos\theta_X & \sin\theta_X e^{i\phi_X} \\
   -\sin\theta_X e^{-i\phi_X} & \cos\theta_X
\pea
,
\nel
Y 
& = &
\tilde{Y}Y'
\; = \;
\pba{cc} 
   e^{i\psi_Y} & 0 \\ 
   0 & e^{i\psi'_Y} 
\pea 
\pba{cc} 
   \cos\theta_Y & \sin\theta_Y e^{i\phi_Y} \\
   -\sin\theta_Y e^{-i\phi_Y} & \cos\theta_Y
\pea,
\nel & = &
\pba{cc} 
   \cos\theta_Y e^{i\psi_Y} & \sin\theta_Y e^{i\phi_Y} e^{i\psi_Y} \\
   -\sin\theta_Y e^{-i\phi_Y}e^{i\psi'_Y} & \cos\theta_Y e^{i\psi'_Y}
\pea
,
\ee

\be
e^{i\psi_Y}
= 
  \frac{a\cos\theta_X + c\sin\theta_X e^{i\phi_X}}{\lambda_1\cos\theta_Y}
        ,
\quad
e^{i\psi'_Y} = 
{-b\sin\theta_X e^{-i\phi_X} + d\cos\theta_X \over \lambda_2\cos\theta_Y}
      .
\ee


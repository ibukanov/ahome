\chapter{Numerical results}

The chargino production cross-section at tree-level given by~\rf{chargeCross}
depends on the parameters $\mu$, $\beta$, $m_{\tilde\Lambda}$ and the mass $m_{\tilde{\nu}}$
of the lightest scalar neutrino. It also depends on 
the sneutrino decay rate because the possibility to have stable 
sneutrinos is excluded by a combination of the bound 
obtained from precise measurements of the decay width 
of the $Z$ boson at LEP~\cite{LEPConstrainOnStableSneutrino}
and the limits derived from the unsuccessful search for relic 
sneutrinos~\cite{GermaniumConstrainOnStableSneutrino}. In principle
such decay rate can be calculated by means of the radiative corrections
and would eventually depend on almost all MSSM parameters, but I treat it
as an independent parameter. I also will mostly consider the cases of a heavy
scalar neutrino, $m_{\tilde\nu} \approx 500$ GeV. 

\footnote{
The source of C++ code that I have used to calculate the numerical 
results of this chapter is available on WWW as the compressed tar archive at 
``http://www.fi.uib.no/{\char126}boukanov/dipl/prog/thesis-prog-source.tgz''.
}

\P
If CP symmetry is violated in the chargino sector it would mean that 
$\Im(\mu) \ne 0$. This effect would influence on the chargino masses
given by~\rf{charginoMasses} and the production cross section 
through its dependence on the matrices $X$ and $Y$, see~\rf{lagran-X-andY-is}.
According to ~\rf{charginoMasses} and~\rf{lagran-X-andY-is} the contribution
from $\Arg(\mu)$ would be most noticeable in the case when 
$\tan\beta \approx 1$ and $\abs\mu \approx M_{\tilde\Lambda}$.
It is illustrated on the Fig.~5.1 where 
$m_\chi$ and $\sigma^T$ are given as functions
of $\abs\mu$ and $\Arg(\mu)$. 
The graphs also show the rather general property
that the value of $\sigma(\mu)$ lies in the interval 
between $\sigma(-\abs{\mu})$ and $\sigma(\abs{\mu})$. The same holds for 
the dependence of $m_\chi$ on $\mu$. 
Thus although in principle chargino mass measurements may be used to 
test the level of the CP violation if it is present, the accuracy requirements
for these experiments goes far beyond the level that will be available 
in the nearest future. 
I will assume in the following that $\Im(\mu) = 0$. 


\P
To illustrate the contributions from different reaction channels 
to the chargino production cross section given by~\rf{chargeCross}
I present in Fig.~5.2  $\sigma^T_{\gamma\gamma}$, 
$\sigma^T_{ZZ}$, $\sigma^T_{\tilde{\nu}\tilde{\nu}}$ as functions of
the center-of-mass energy. This plot shows a typical behavior 
for the cross section,
i.e. at the threshold the cross section is dominated by boson 
channels of the reaction and at high energies 
$\sigma^T \approx \sigma^T_{\tilde{\nu}\tilde{\nu}}$. It follows 
from the assumption of heavy sneutrino and the asymptotic behavior
of the cross section in the region where $s \gg m_\chi, m_{\tilde{\nu}}$. 
In this case every component but $\sigma^T_{\nu\nu}$ in~\rf{chargeCross}
goes to zero and according to~\rf{charge-total-nu-nu} 
and~\rf{charge-t-plus-minus-is} I have
\bem
\lim_{s \rightarrow \infty } \sigma^T_{\nu\nu}
& = & 
\lim_{s \rightarrow \infty }
{g^4 \cos^4\theta_X\over 64 \pi s^2}
\Biggl\{
{s^2 \over m_{\tilde{\nu}}\Gamma_{\tilde{\nu}}}
\arctg{m_{\tilde{\nu}}\Gamma_{\tilde{\nu}}  \over m_{\tilde{\nu}}^2}
+ 2s\ln\abs{t_+ \over t_- }
+ s
\Biggr\}
\nel
& = & 
{g^4 \cos^4\theta_X\over 64 \pi m_{\tilde{\nu}}\Gamma_{\tilde{\nu}}}
\arctg{\Gamma_{\tilde{\nu}} \over m_{\tilde{\nu}}}
.
\ee

Thus at the high $s$ the cross section will approach its limit given by
\be
\lim_{s \rightarrow \infty } \sigma^T
=
{g^4 \cos^4\theta_X\over 64 \pi m_{\tilde{\nu}}\Gamma_{\tilde{\nu}}}
\arctg{\Gamma_{\tilde{\nu}} \over m_{\tilde{\nu}}}
.
\ee

\P
The plots given in Fig.~5.3--5.5
demonstrates different aspects of the cross section dependence 
on the theory parameters. 

\begin{figure} 
    \begin{center}
        \leavevmode
        \epsfxsize 138 mm 
        \epsfbox{\bodyFigure{mass.mu_re-mu_im.eps}}
    \end{center}
    \begin{center}
        \label{of-mu-re-im}

        \leavevmode
        \epsfxsize 138 mm 
        \epsfbox{\bodyFigure{tcs.mu_re-mu_im.eps}}

        \caption{
            Dependence of the chargino mass and cross section 
            on complex value of $\mu$.
            For both cases (a) and (b) 
            $\tan \beta = 1.0$ and $M_{\tilde\Lambda} = 300$ GeV.
            Part (a) shows the chargino mass, part (b)
            shows the total cross section $\sigma^T$ for
            $m_{\tilde\nu} = 500$ GeV, $\Gamma_{\tilde\nu} = 10$ GeV 
            and $E_{\rm c.m.} = 500$ GeV.
        }

    \end{center}
\end{figure}


\begin{figure} 
    \begin{center}
        \label{num-channels}
        \leavevmode
        \epsfxsize 138 mm 
        \epsfbox{\bodyFigure{channels.eps}}
        \caption{Contribution to the cross-section of 
            $e^+e^-\rightarrow \chi^+\chi^-$ 
            from $\gamma$, $Z^0$ and $\tilde\nu$ channels at 
            $\mu = 200$ GeV, $\tan\beta = 1$, $M_{\tilde\Lambda} = 300$ GeV, 
            $m_{\tilde\nu} = 500$ GeV and $\Gamma_{\tilde\nu} = 10$ GeV.
        }
    \end{center}
\end{figure}


\begin{figure}
    \begin{center}
        \label{fig-picture-A}
        \epsfxsize 138 mm 
        \epsfbox{\bodyFigure{tcs.mu-beta.eps}}
        \caption{
            $\sigma^T$ as a function of $\mu$, $\tan\beta$. 
            The rest set of parameters 
            are fixed by $M_{\tilde\Lambda} = 300$ GeV, 
            $m_{\tilde\nu} = 500$ GeV,
            $\Gamma_{\tilde\nu} = 10$ GeV
            and
            $E_{\rm c.m.} = 500$ GeV.
        }
    \end{center}
\end{figure}


\begin{figure}
    \begin{center}
        \label{fig-picture-B}
        \epsfxsize 138 mm 
        \epsfbox{\bodyFigure{tcs.mu-lambda.eps}}
        \caption{
        $\sigma^T$ as a function of $\mu$, $M_{\tilde\Lambda}$ with 
        $\tan\beta = 1$, 
        $m_{\tilde\nu} = 500$ GeV,            
        $\Gamma_{\tilde\nu} = 10$ GeV
        and
        $E_{\rm c.m.} = 500$ GeV.
        }
    \end{center}
\end{figure}

\begin{figure}
    \begin{center}
        \label{fig-Picture-N}
        \epsfxsize 138 mm 
        \epsfbox{\bodyFigure{tcs.nu-E.eps}}
        \caption{
            $\sigma^T$ as a function of $m_{\tilde\nu}$, $E_{\rm c.m.}$
            with
            $\mu=300$ GeV, 
            $\tan\beta = 1$, 
            $M_{\tilde\Lambda} = 300$ GeV
            and 
            $\Gamma_{\tilde\nu} = 10$ GeV.

        }
    \end{center}
\end{figure}







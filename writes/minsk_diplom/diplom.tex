\documentclass[a4paper,titlepage]{article}
\usepackage[utf8]{inputenc} % Input encoding
\usepackage[T2A]{fontenc} % Output encoding
\usepackage[russian]{babel} % Russian words for Chapter, etc

% Original header: \documentstyle[12pt,russian,titlepage,A4]{article}



\newcommand{\MathBr}[1]{\ifmmode{#1} \else{$#1$} \fi}

\newcommand{\Hhat}{\MathBr{\hat H}}
\newcommand{\Hhatf}[1]{\MathBr{\hat H(#1)}}

\newcommand{\HZhat}{\MathBr{\hat H_0}}
\newcommand{\HZtilde}{\MathBr{\tilde H_0}}

\newcommand{\Ht}{\MathBr{H'}}


\newcommand{\KetVect}[1]{\MathBr{\,|#1\!\!>}}
\newcommand{\BraVect}[1]{\MathBr{<\!\!#1|}}
\newcommand{\DirProd}[2]{\MathBr{<\!\!#1|#2\!\!>}}

\newcommand{\mod}[1]{\MathBr{\mid\!\!#1\!\!\mid}}

\newcommand{\XVect}{\MathBr{\overrightarrow{X}}}
\newcommand{\AForm}{\MathBr{A}}

\newcommand{\rf}[1]{(\ref{#1})}

\newcommand{\der}[1]{\MathBr{\partial_{#1}}}
%\newcommand{\der}[1]{\MathBr{\partial \over \partial #1}}

\begin{document}

\titlepage{
\lineskip=6pt % Это расстояние между строками

\centerline{МИНИСТЕРСТВО ОБРАЗОВАНИЯ И НАУКИ РЕСПУБЛИКИ БЕЛАРУСЬ}

\centerline{БЕЛОРУССКИЙ ОРДЕНА ТРУДОВОГО КРАСНОГО ЗНАМЕНИ}

\centerline{ГОСУДАРСТВЕННЫЙ УНИВЕРСИТЕТ}

\vskip1.5cm

\centerline{Физический факультет}

\centerline{Кафедра теоретической физики}

\vskip1.5cm

\centerline{БУКАНОВ ИГОРЬ ВАЛЕРЬЕВИЧ}

\vskip1cm
\centerline{МЕТОД ВЫЧИСЛЕНИЯ}
\centerline{АБЕЛЕВЫХ ГЕОМЕТРО-ТОПОЛОГИЧЕСКИХ ФАЗ}
\centerline{ДЛЯ МНОГОУРОВНЕВЫХ КВАНТОВЫХ СИСТЕМ}

\vskip1.5cm


\centerline{Дипломная работа}
\centerline{студента 5 курса}

\vskip1.5cm

\vbox{\halign{
# \hfil\hskip10cm & # \hfil\cr
& \qquad\underbar{Руководитель}\cr
& Толкачев Е.\,А. \cr
& главный научный сотрудник, \cr
& доктор физ.--мат. наук \cr
&\cr
\cr\cr
& \qquad \underbar{Рецензент}\cr
& Томильчик Л.\,М. \cr
& член--корр. АН Беларуси, \cr
& профессор, \cr
& доктор физ.--мат. наук \cr
&\cr
}}
\vfill
\centerline{Минск 1995}

}

\newpage

\tableofcontents

%\begin{center}
%{\large\bf  }
%Буканов~И.В.
%\end{center}

\newpage
\section{Введение.}

Как известно, в последние годы уделяется значительное внимание исследованию
геометро - топологических зффектов в квантовых системах. К их числу
относится, в частности, комплекс наблюдаемых и предсказываемых явлений
под общим названием геометрическая фаза или фаза Берри. Данный эффект был
открыт сравнительно недавно в работе~\cite{FirstBerryWork}, где было показано,
что в случае параметрической квантовомеханической системы с циклически
адиабатически меняющимися параметрами у волновой функции возникает
дополнительный фазовый множитель, который может быть обнаружен
экспериментально.

В дальнейшем было установлено, что аналогичное явление имеет место
для целого ряда систем, описываемых классической механикой~\cite{60},
классической электродинамикой~\cite{61}, a также аномальными калибровочными
теориями~\cite{62,63,64}. Универсальная геометрическая природа этой
фазы~\cite{SimonWork} позволяет с единых позиций интерпретировать такие, казалось бы,
разнородные явления, как нейтринные осциляции~\cite{NaumovWork} и некоторые
закономерности в молекулярных спектрах~\cite{AboutPhaseInSpectra}.
Экспериментальное наблюдение фазы Берри в нейтронных и оптических
экспериментах породило определенные надежды на возможность ее практического
использования.

Однако вскоре выяснилось, что несмотря на разнообразные обобщения -- на
случай неадиабатического изменения параметров~\cite{91}, наличие
вырождения~\cite{WilzekZeeWork} и т.д., конкретные вычисления данного эффекта
можно
провести лишь для ряда простых квантовых систем, описываемых квадратичным
гамильтонианом~\cite{About2Hamiltonian} или обладаюших 3-х параметрической
группой динамической симметрии~\cite{FirstBerryWork,AboutExactSolution}.
Попытки в рамках стандартного подхода
получить результат, например, для трехуровневой системы, удались лишь
ценой существенных ограничений на вид соответствующего
гамильтониана ~\cite{CeulemansWork,KorenblitWork}.

Основной задачей данной работы является получение выражений для
фазы Берри в случае произвольной конечноуровневой системы на основе нового
подхода, развитого в работах~\cite{TolkachevTregubovich1,TolkachevTregubovich2}
и пригодного для квантовых систем, описываемых гамильтонианами,
которые реализуют некоторое представление конечномерной алгебры Ли группы
динамической симметрии.

Работа состоит из введения, трех параграфов, заключения и списка цитируемой
литературы.

Первый параграф носит в основном вспомогательный характер и посвящен
рассмотрению некоторых общих вопросов, связанных с возникновением
фазы Берри при циклической эволюции параметров в случае отсутствия вырождения
и ее геометрической интерпретацией. Также будут рассмотрены некоторые подходы
к расчету данного эффекта.

Второй параграф целиком посвящен построению общих формул для вычисления
геометрической фазы в случае конечноуровневых систем. Будет показано, что как
 и в случае двухуровневой системы, в пространстве параметров возникает
индуцированное калибровочное поле, потенциал которого аналогичен потенциалу
поля магнитного монополя.

В третьем параграфе рассмотрены некоторые приложения полученных результатов
в случае трехуровневой системы.

\newpage
\section{Абелева геометрическая фаза.}

Рассмотрим вначале, следуя \cite{AboutPhaseInSpectra}, некоторые общие вопросы,
связанные с геометрической фазой. Пусть имеется некоторая
квантовомеханическая $N$-- уровневая система, описываемая гамильтонианом
$H(\XVect)$, зависяшим от набора параметров
$\XVect = \{x^i|i = \overline{1,d}\}$, где $d$ -- размерность
пространства параметров.

Тогда в случае зависимости \XVect от времени
эволюция состояния $\KetVect{\psi}$ системы будет подчиняться
уравнению Шредингера $(\hbar = 1)$:
\begin{equation}
   i \der{t}\KetVect{\psi(t)} = H(\XVect(t))\KetVect{\psi(t)}.
\label{Shredinger}
\end{equation}

Введем теперь полный набор ортонормированных собственных функций
$\KetVect{n}$:
\begin{equation}
   H(\XVect)\KetVect{n} = E_n(\XVect)\KetVect{n}
   \qquad n = 1..N,
\label{EugenVectors}
\end{equation}
в предположении что спектр $E_n$ невырожден. Тогда \KetVect{n} образуют
базис в гильбертовом пространстве состояний системы и $\KetVect{\psi}$ в
любой момент времени $t$ можно представить в виде:
\begin{equation}
  \KetVect{\psi(t)} = \sum_m c_m(t)\KetVect{m(\XVect(t))}.
\label{PsiRepresent}
\end{equation}

Подставляя \rf{PsiRepresent} в \rf{Shredinger} и учитывая \rf{EugenVectors},
получим уравнение на коэфиценты $c_m$:
\begin{equation}
  \sum_m (i\der{t}c_m - c_m E_m)\KetVect{m}
     + i \sum_m c_m \der{t}\KetVect{m} = 0,
\label{EqnForCN}
\end{equation}
или, используя ортонормированность собственных функций:
\begin{equation}
  i\der{t}c_n + i \sum_m c_m \DirProd{n}{\der{t}m} = c_n E_n.
\label{OneEqnForCN}
\end{equation}

Предположим теперь, что параметры системы \XVect изменяются во времени
достаточно медленно, где под медленностью изменения понимается то, что
спектральная плотность $\XVect(\omega)$ отлична от нуля лишь в области
частот, много меньших собственных частот системы. Тогда в соответствии с
адиабатической теоремой \cite{BornAndFock} можно считать:
\[
  \DirProd{n}{\der{t}m} = 0, \qquad n \ne m.
\]

Откуда:
\[
  \sum_m c_m \DirProd{n}{\der{t}m} = c_n \DirProd{n}{\der{t}n},
\]
и \rf{OneEqnForCN} представляется в виде:
\[
  i\der{t}c_n = c_n(E_n - i\DirProd{n}{\der{t}n}),
\]
что сразу позволяет найти зависимость $c_n$ от времени:
\begin{equation}
  c_n(t) = c_n(t_0)\exp i(\alpha_n(t) + \gamma_n(t)),
\label{CNis}
\end{equation}
где $\alpha_n(t) = -i \int_{t_0}^{t}E_n(t') dt'$
и для фазового множителя $\gamma_n$:
\begin{equation}
 \gamma_n(t) = i \int_{t_0}^{t} \DirProd{n}{\der{t'}n} dt'
  = i \int_{t_0}^{t} \DirProd{n}{\der{x^i}n} {dx^i \over dt'} dt',
\label{FullGeomFase}
\end{equation}
или:
\begin{equation}
 \gamma_n(t) = \gamma_n(С) = i \int_{C} \DirProd{n}{\der{x^i}n} dx^i =
  \int_{C} A(\XVect),
\label{GeomFase}
\end{equation}
\begin{equation}
A(\XVect) = i \DirProd{n}{\der{x^i}n} dx^i.
\label{AFromN}
\end{equation}

Т.\,е. $\gamma_n(t)$ определяется через интеграл по кривой $C$ в пространстве
параметров системы от дифференциальной 1 - формы \AForm, которая имеет смысл
индуцированного калибровочного поля, поскольку при калибровочных, или
фазовых преобразованиях собственных векторов \KetVect{n}:
\begin{equation}
\KetVect{n(\XVect)} \rightarrow \KetVect{\tilde{n}(\XVect)}
 = \exp{i\phi_n(\XVect)} \KetVect{n(\XVect)}
\label{FaseTransform}
\end{equation}
с произвольной вещественной функцией $\phi_n(\XVect)$, \AForm преобразуется
как потенциал в электродинамике:
\begin{equation}
A \rightarrow \tilde{A} = A - \der{x^i}\phi_n dx^i.
\label{ATransform}
\end{equation}
 В самом деле, по определению:
\[
 \tilde{A} = i \DirProd{\tilde{n}}{\der{x^i}\tilde{n}} dx^i
\]
\[
 = ie^{-i\phi_n}\left(\DirProd{n}{\der{x^i}n}e^{i\phi_n}
  + i \DirProd{n}{n}e^{i\phi_n}\der{x^i}\phi_n \right) dx^i
\]
\[
 = A - \der{x^i}\phi_n dx^i.
\]

   Тогда:
\begin{equation}
  \tilde{\gamma_n(t)} = \int_{C} \tilde{A} =
     \gamma_n(t) + \phi_n(\XVect(t_0)) - \phi_n(\XVect(t_0)),
\label{TransGamma}
\end{equation}
 и в случае незамкнутого контура $C$, т.\,е. когда
$\XVect(t_0) \ne \XVect(t)$, выбором $\phi_n$ можно всегда обратить
$\gamma_n(t)$ в ноль.

Если же $C$ является замкнутым контуром в пространстве параметров системы,
то, используя теорему Стокса, $\gamma_n(C)$ можно представить в виде
интеграла по поверхности:

\begin{equation}
  \gamma_n(C) = \int_S F(\XVect),
\label{GammaAsSurfInt}
\end{equation}
  здесь $S$ -- произвольная ориентируемая поверхность,
натянутая на контур $C$, а 2 - форма
\begin{equation}
  F(\XVect) = dA(\XVect) = i \DirProd{\der{x^i}n}{\der{x^j}n} dx^i \wedge dx^j
\label{FFromN}
\end{equation}
 является тензором калибровочного поля.

Из калибровочной инвариантности $F$ относительно преобразований
\rf{FaseTransform}, как впервые показал Берри в своей
работе~\cite{FirstBerryWork}, следует, что $\gamma_n$ -- геометрическая
фаза или фаза Берри -- не зависит от выбора базиса \rf{EugenVectors} и
является наблюдаемой физической величиной, однозначно определяемой
соотношением \rf{GammaAsSurfInt}. При этом значение $\gamma_n$ определяется
только формой контура $C$, в отличии от динамической фазы
$\alpha_n$, которая может быть найдена, если только известна зависимость
$\XVect(t)$.

Т.\,к. $A$ является функцией только $\KetVect{n}$, то для
любых квантовых систем, гамильтонианы которых коммутируют при всех значениях
параметров и вследствии этого обладают одинаковыми системами собственных
векторов~\cite{AboutMatrix}, их геометрические фазы будут совпадать.
В частности, $\gamma_n$ не зависит от следа гамильтониана,
т.\,к. $H$ всегда можно представить в виде:
\begin{equation}
   H = H_0 + {1 \over N}(SpH)I,
\label{ZeroSP}
\end{equation}
 где $SpH_0 = 0$ и $[H_0, H] = [H_0, H_0] + {1 \over N}SpH[H_0, I] = 0$.

Представим теперь $F$ в виде, в котором требование невырожденности
гамильтониана присутствует явным образом. Для этого воспользуемся
полнотой базиса \rf{EugenVectors}:
\[
  F = i \sum_{m} \DirProd{\der{x^i}n}{m}\DirProd{m}{\der{x^j}n}
    dx^i \wedge dx^j
\]
\[
  = i \sum_{m \ne n} \DirProd{\der{x^i}n}{m}\DirProd{m}{\der{x^j}n}
    dx^i \wedge dx^j.
\]

Но из \rf{EugenVectors} получаем:
\[
  \DirProd{m}{\der{x^i}n} =
    {\BraVect{m}\der{x^i}H\KetVect{n} \over E_n - E_m }.
\]

Тогда:
\begin{equation}
  F = i \sum_{m}
  {\BraVect{n}\der{x^i}H\KetVect{m} \BraVect{m}\der{x^j}H\KetVect{n}
     \over (E_n - E_m)^2 } dx^i \wedge dx^j.
\label{FAsSum}
\end{equation}

Откуда следует, что при пересечении уровней $E_n$, $E_m$, когда наступает
вырождение системы, знаменатель в \rf{FAsSum} обращается в ноль и $F$
перестает существовать.

Геометрическая интерпретация фазы Берри была дана в работе \cite{SimonWork}
на основе теории расслоений и вкратце может быть изложена следующим
образом. Т.\,к. область изменения
параметров $x_i$ произвольна, то $x_i$ -- суть локальные координаты на
некотором многообразии $M$. Каждой точке $\XVect \in M$ для фиксированного
уровня $E_n$ соответствует некоторый собственный вектор
$\KetVect{n}$, поэтому $\KetVect{n(\XVect)}$ также образуют некоторое
многообразие $P$. Тогда, как нетрудно видеть,
преобразование~\rf{FaseTransform} реализует
унитарное представление группы $U(1)$, действующей на $P$ и описываемая
конструкция -- это не что иное, как главное \(U(1)\)
расслоение~\cite{PostnikovWork} $\zeta$ c
тотальным пространством $P$, базой $M$ и канонической проекцией
\[
  \pi : P \rightarrow M, \quad \pi(\KetVect{n(\XVect)}) = \XVect.
\]

При этом калибровочный потенциал $A$ является $U(1)$ связностью, а
2-форма $F$ -- кривизной расслоения $\zeta$. Т.\,к. $U(1)$ -- абелева группа,
то \rf{GeomFase} иногда называют абелевой фазой в отличии от случая
вырожденной квантовой системы~\cite{WilzekZeeWork}.

Нетрудно заметить, что рассмотренный выше геометрический подход во многом
аналогичен широко использующемуся в калибровочных теориях
(cм., например \cite{DanielWork}). Отличие состоит в том, что в последних
в качестве базы соответствующего расслоения рассматривается обычно
пространство Минковского $R^{3,1}$, а в данном случае -- произвольное
пространство параметров квантовой системы.

\newpage
\section{Вычисление калибровочного потенциала в общем случае.}

При проведении вычислений фазы Берри использовать
формулы \rf{GeomFase}, \rf{GammaAsSurfInt} довольно трудно, т.\,к.
в конкретных задачах обычно бывают известны только матричные элементы
гамильтониана в каком либо представлении и необходимость
вычислять $\KetVect{\der{x^i}n}$, входящих в \rf{AFromN}, \rf{FFromN}
приводит к очень громоздким вычислениям уже в случае простейших систем.
\rf{FAsSum} несколько сокращает обьем вычислений, однако требует знания
всех собственных векторов, что тоже является неприемлемым требованием в
большинстве случаев.

Поэтому получим в случае конечноуровневой ситемы альтернативные выражения
для $A$, $F$, пригодные для произвольного невырожденного уровня с
энергией $E$. Для этого положим по определению:
\[
   \Ht = H - EI_N,
\]
 т.\,е.
\begin{equation}
  \Ht_{ij} = H_{ij} - E\delta_{ij},
  \qquad \sum_{j = 1}^{N} \Ht_{ij} n_j = 0, \quad i = 1..N,
\label{SelfEqn}
\end{equation}
и пусть $M$ - матрица, составленная из миноров элементов $\Ht_{ij}$.
Тогда в силу невырожденности $E$ $rank \Ht = N - 1$, поэтому
существуют такие индексы $q$ и $p$, что
\begin{equation}
   M_{qp} \ne 0,
\label{MNotZero}
\end{equation}
и с помощью \rf{SelfEqn} можно выразить через $n_p$ остальные компоненты
вектора $n$.

В самом деле, перепишем \rf{SelfEqn} в виде:
\begin{equation}
  \sum_{\nu} \Ht_{\mu\nu} n_\nu = - H_{\mu p} n_p,
\label{WorkEqn}
\end{equation}
где индексы $\mu$, $\nu$ могут принимать следующие значения:
$ \mu = 1..N, \mu \ne q$, $ \nu = 1..N, \nu \ne p$

\rf{WorkEqn} является невырожденной системой линейных уравнений \
относительно $n_\nu$, т.\,к. $det \Ht_{\mu\nu} = M_{qp} \ne 0$,
поэтому решение \rf{WorkEqn} можно представить в следующем виде с помощью
формул Кардано:
\[
   n_\nu = {\Delta_\nu \over M_{qp}},
\]
где $\Delta_\nu$ является определителем матрицы, полученной из $H$
заменой элементов столбца с номером $\nu $ на $-\Ht_{\mu p} n_p$
и вычеркиванием столбца с номером $p$ и строки с номером $q$,
и, как легко показать:
\[
   \Delta_\nu = n_p (-1)^{\sigma_{p\nu}} M_{q\nu}, \qquad
    \sigma_{p\nu} =
     \left\{{\nu + p , \nu < p \atop \nu + p + 1, \nu > p} \right. .
\]

Тогда:
\begin{equation}
  n_\nu = n_p (-1)^{\sigma_{p\nu}} {M_{q\nu} \over M_{qp}}.
\label{GetNFromH}
\end{equation}

Подставим \rf{GetNFromH} в выражение для потенциала
калибровочного поля:
\[
   A = - Im\DirProd{n}{dn} = - \sum_{i=1}^{N} Im(n^\ast_i dn_i) =
     - \sum_{\nu} Im(n^\ast_\nu dn_\nu) - Im(n^\ast_p dn_p)
\]
или:
\[
  A_{qp} = - Im\left(\sum_{\nu} n^\ast_p {M^\ast_{q\nu} \over M^\ast_{qp}}
    d(n_p {M_{q\nu} \over M_{qp}})\right) - Im(n^\ast_p dn_p)
\]
\[
  = - \mod{n_p}^2Im\left(\sum_{\nu} {M^\ast_{q\nu} \over M^\ast_{qp}}
    d({M_{q\nu} \over M_{qp}})\right) -
    \left(1 + \sum_{\nu} {\mod{M_{q\nu}}^2 \over \mod{M_{qp}}^2}\right)
     Im(n^\ast_p dn_p),
\]
  где индексы $q$, $p$ у потенциала $A$ показывают, что данные
выражения справедливы только при выполнении условия \rf{MNotZero}.

Представим теперь $n_p$ в виде: $n_p = r_pe^{i\phi_p}$. Тогда:
\[
   \mod{n_p}^2 = r_p^2, \qquad
   Im(n^\ast_p dn_p) = Im( ir_p^2d\phi_p + r_p dr_p) = r_p^2d\phi_p.
\]

Откуда для $A_{qp}$:
\begin{equation}
  A_{qp} = -r_p^2 \sum_{\nu}
   \left({M^\ast_{q\nu} dM_{q\nu} \over \mod{M_{qp}}^2}
     - {\mod{M_{q\nu}}^2 \over \mod{M_{qp}}^2} {dM_{qp} \over M_{qp}}\right)
   - r_p^2\left({\sum_{\nu} \mod{M_{q\nu}}^2 + \mod{M_{qp}}^2
       \over \mod{M_{qp}}^2}\right) d\phi_p.
\label{TempA}
\end{equation}
  Для нахождения $r_p^2$ воспользуемся условием
нормировки $1 = $\DirProd{n}{n} или:
\[
  1 = \sum_{i = 1}^{N} \mod{n_i}^2 = \sum_{\nu} \mod{n_\nu}^2 + \mod{n_p}^2 =
   \mod{n_p}^2\left(
      1 + \sum_{\nu} {\mod{M_{q\nu}}^2 \over \mod{M_{qp}}^2}\right),
\]
\begin{equation}
   r_p^2 =
     {\mod{M_{qp}}^2 \over \sum_{\nu} \mod{M_{q\nu}}^2 + \mod{M_{qp}}^2}
     = {\mod{M_{qp}}^2 \over L_q}.
\label{TempB}
\end{equation}

Здесь положено:
\[
  L_q = \sum_{\nu} \mod{M_{q\nu}}^2 + \mod{M_{qp}}^2
    = \sum_{i = 1}^N \mod{M_{qi}}^2.
\]
   Подставляя \rf{TempB} в \rf{TempA}, получаем:
\[
  A_{qp} = - {1 \over L_q}\sum_{\nu} Im(M^\ast_{q\nu} dM_{q\nu})
    + {\sum_{\nu} \mod{M_{q\nu}}^2 \over L_q}
      Im\left({dM_{qp} \over M_{qp}}\right) - d\phi_p,
\]
\begin{equation}
  A_{qp} = - {1 \over L_q}\sum_{i = 1}^{N} Im(M^\ast_{qi} dM_{qi})
    + \Pi_{qp},
\label{TempC}
\end{equation}
где по определению:
\[
   \Pi_{qp} = Im({dM_{qp} \over M_{qp}}) - d\phi_p = d(\phi_{qp} - \phi_p),
   \qquad  \phi_{qp} = {M_{qp} \over \mod{M_{qp}}},
\]
т.\,е. $\Pi_{qp}$ является полным дифференциалом, и калибровочным
преобразованием $A$ можно всегда обратить $\Pi_{qp}$ в 0. Тогда в
\rf{TempC} пропадет зависимость $A_{qp}$ от индекса $p$, и полагая
$A_q = A_{qp}$, окончательно имеем:
\begin{equation}
   A_q = - {1 \over L_q}\sum_{i = 1}^{N} Im(M^\ast_{qi} dM_{qi})
      = - {1 \over L_q}\sum_{\mu} Im(M^\ast_{q\mu} dM_{q\mu}),
\label{AExpression}
\end{equation}
где учтено, что $M$ -- эрмитова матрица в силу эрмитовости $\Ht$,
поэтому $Im(M_{qq}) = 0$ и:
\[
   Im(M^\ast_{qq}dM_{qq}) = Im(M_{qq}dM_{qq}) = 0.
\]

Из \rf{AExpression} следует, что калибровочный потенциал $A_q$
не определен при условии:
\begin{equation}
  L_q = 0,
\label{UndefinedA}
\end{equation}
т.\,е. когда все элементы $q$-ой строки матрицы $M$ обращаются в ноль.
При этом очевидно, что при разных $q_1$, $q_2$ в области, где параметры
гамильтониана удовлетворяют условию $L_{q_1} \ne 0$, $L_{q_2} \ne 0$,
$A_{q_1}$, $A_{q_2}$ калибровочно - эквивалентны, т.\,е.
$d(A_{q_1} - A_{q_2}) = 0$ и для вычисления $F$ можно выбрать любое из
выражений \rf{AExpression}

Рассмотрим в качестве примера использования \rf{AExpression} "классический"
случай двухуровневой системы с пространством параметров $R^3$, гамильтониан
которой можно представить в виде:
\begin{equation}
   H = \left(\matrix{z&x+iy\cr
         x-iy&-z}\right)
    = \left(\matrix{r\cos\theta&r\sin\theta e^{i\phi}\cr
      r\sin\theta e^{-i\phi}&-r\cos\theta}\right)
\label{HFor2Level}
\end{equation}
 с собственнымм значениями $\pm r$. Здесь $(x,y,z)$,
$(r,\theta,\phi)$ -- соответственно декартовы и сферические координаты
в $R^3$.

Полагая в \rf{AExpression} $q$ равным $1$, получим:

\[
   M_{\pm} = \left(\matrix{-r\cos\theta \mp r&r\sin\theta e^{-i\phi}\cr
      r\sin\theta e^{i\phi}&r\cos\theta \mp r}\right),
\]
\[
   L_{\pm1} = L_{\pm} = \mod{M_{11}}^2 + \mod{M_{12}}^2
      = 2r^2(1 \pm \cos\theta),
\]
\begin{equation}
   A_{\pm1} = A_{\pm} = -{1 \over L_{\pm}} Im(M^*_{12}dM_{12})
      = {1 \over 2}(1 \mp \cos \theta) d\phi,
\label{AFor2Level}
\end{equation}
\begin{equation}
   F_{\pm} = dA_{\pm} = \pm {1 \over 2}\sin \theta d\theta \wedge d\phi,
\label{FFor2Level}
\end{equation}
в полном соответствии с результатом работы~\cite{FirstBerryWork},
согласно которому \rf{FFor2Level} является полем магнитного монополя,
расположенного в начале координат -- точке вырождения
гамильтониана~\rf{HFor2Level}.

Представим теперь \rf{AExpression} в виде, удобном для
сопоставления с~\rf{AFor2Level}. Для этого положим:
%геометрической интерпретации полученного результата.
\[
   M_{q\mu} = \mod{M_{q\mu}}e^{-i\phi_{q\mu}},
\]
 тогда:
\[
   A_q = - {1 \over L_q}\sum_{\mu}
 Im(\mod{M_{q\mu}} d\mod{M_{q\mu}} - i \mod{M_{q\mu}}^2 d\phi_{q\mu})
 = \sum_{\mu} {\mod{M_{q\mu}}^2 \over L_q} d\phi_{q\mu}.
\]

Т.\,к. $\mod{M_{q\mu}}^2 \le L_q = \sum_{i = 1}^N \mod{M_{qi}}^2$,
то существуют такие $\theta_{q\mu}$, что:
\[
   {1 \over 2}(1 - \cos\theta_{q\mu}) = \sin^2 {\theta_{q\mu} \over 2}
      = {\mod{M_{q\mu}}^2 \over L_q},
\]
и для $A_q$ можно записать:
\begin{equation}
  A_q = {1 \over 2} \sum_{\mu} (1 - \cos\theta_{q\mu}) d\phi_{q\mu}.
\label{AGeomExpression}
\end{equation}

Т.\,к. каждое слагаемое в сумме \rf{AGeomExpression} по своей
структуре аналогично \rf{AFor2Level}, то в случае произвольных
$\theta_{q\mu}$, $\phi_{q\mu}$ можно считать,
что~\rf{AGeomExpression} задает потенциал обобщенного монополя в пространстве
параметров квантовой системы.

\newpage
\section{Трехуровневая система.}

Рассмотрим произвольную 3-ех уровневую квантовую систему с гамильтонианом:
\begin{equation}
   H = \left(\matrix{H_{11}&H_{12}&H_{13}\cr
          H_{21}&H_{22}&H_{23}\cr
          H_{31}&H_{32}&H_{33}}\right),
\label{HForThreeLevel}
\end{equation}
 который, используя эрмитовость $H$, можно представить в виде:
\begin{equation}
   H = \left(\matrix{H_{11}&h_3e^{i\phi_3}&h_2e^{-i\phi_2}\cr
         h_3e^{-i\phi_3}&H_{22}&h_1e^{i\phi_1}\cr
          h_2e^{i\phi_2}&h_1e^{-i\phi_1}&H_{33}}\right),
\label{HShort}
\end{equation}
 где $H_{ii}$, $h_{i}$ вещественны.
В силу \rf{ZeroSP} в \rf{HShort} можно без ограничений общности считать, что
\begin{equation}
 Sp H = H_{11} + H_{22} + H_{33} = 0.
\label{HSpIsZero}
\end{equation}

Тогда уровни энергии системы определяются из уравнения на собственные
значения:
\begin{equation}
   E^3 + Sp{\overline H}E - DetH = 0,
\label{SelfEquation}
\end{equation}
где ${\overline H}$ -- взаимная матрица~\cite{AboutMatrix},
\[
   Sp{\overline H} = - (H_{22})^2 - (H_{33})^2 - H_{22}H_{33}
    - h_1^2 - h_2^2 - h_3^2,
\]
\[
   DetH = (H_{22} + H_{33})(h_1^2 - H_{22}H_{33})
   - H_{22}h_2^2 - H_{33}h_3^2 + 2h_1h_2h_3\cos \Phi.
\]

Выбирая, как и в случае двухуровневой системы, в \rf{AExpression}
$q$ равным единице, получаем для уровня с энергией $E$:
\[
  A_1 =  -{1 \over L_1} Im(M^*_{12}dM_{12} + M^*_{13}dM_{13}),
\]
\[
  L_1 = \mod{M_{11}}^2 + \mod{M_{12}}^2 + \mod{M_{13}}^3,
\]
\[
M_{11} = \Delta_2 \Delta_3 - h_1^2,
\]
\[
M_{12} = - h_2h_1e^{i(\phi_1 + \phi_2)} + \Delta_3 h_3 e^{-\phi_3}
 \, = \, e^{-\phi_3}(\Delta_3 h_3 - h_1 h_2 e^{i\Phi}),
\]
\[
M_{13} = h_1h_3e^{-i(\phi_1 + \phi_3)} - \Delta_2 h_2 e^{\phi_2}
 \, = \, e^{\phi_2}(-\Delta_2 h_2 + h_1 h_3 e^{-i\Phi}),
\]
  где по определению:
\[
\Phi = \phi_1 + \phi_2 + \phi_3,
\]
\[
\Delta_2 = H_{22} - E, \qquad \Delta_3 = H_{33} - E,
\]

 и при произвольной зависимости гамильтониана от времени можно записать:
\[
  A_1 =  {1 \over L_1} (\mod{M_{12}}^2 d\phi_3 - \mod{M_{13}}^2 d\phi_2 + K).
\]

   Здесь:
\[
   \mod{M_{12}}^2 = \Delta_3^2h_3^2 + h_1^2h_2^2 - 2h_1h_2h_3\Delta_3\cos\Phi,
\]
\[
   \mod{M_{13}}^2 = \Delta_2^2h_2^2 + h_1^2h_3^2 - 2h_1h_2h_3\Delta_2\cos\Phi,
\]
  и форма $K$ определяется выражением:
\[
 K = h_1
 \left(h_1(h_3^2 - h_2^2) + h_2h_3\cos\Phi(\Delta_3 - \Delta_2)\right)d\Phi
\]
\[
  + h_2h_3\sin\Phi(\Delta_3 - \Delta_2)dh_1
  + h_1\sin\Phi(\Delta_3 + \Delta_2)(h_3dh_2 - h_2dh_3)
\]
\[
  - h_1h_2h_3\sin\Phi d\left(\Delta_3 -\Delta_2\right).
\]

При расчетах конкретных систем, используя те или иные ограничения на
параметры гамильтониана, можно существенно упростить полученное выражение.
Так, например в случае зависимости от времени только диагональных элементов
$H$ $dh_1=$$dh_2=$$dh_3=$$d\Phi=0$, откуда:
\[
  A_1 = h_1h_2h_3\sin\Phi {d\left(\Delta_2 -\Delta_3\right) \over L_1}
  = h_1h_2h_3\sin\Phi {d\left(H_{22}-H_{33}\right) \over L_1},
\]
и после простых вычислений получаем для $F$:
\begin{equation}
  F =
   6 Im(H_{12}H_{23}H_{31}) { \Delta_2 \Delta_3 - \mod{H_{23}} \over L^2 } E
 (1-{ \partial E \over \partial H_{22} }-{ \partial E \over \partial H_{33} })
  dH_{33} \wedge dH_{22}.
 \label{FOnPlate}
\end{equation}
  Из (\ref{FOnPlate}) следует, что в общем случае произвольной зависимости
$H_{22}$, $H_{33}$ от времени $F \ne 0$, и в данной
системе может наблюдаться геометрическая фаза, в отличии от двухуровневого
гамильтониана, где при изменении только диагональных элементов фаза Берри
тождественно обращается в ноль.

%в отличии от утверждения, содержащегося в работе~\cite{NaumovWork}.

\newpage
\section{Заключение.}
 Полученные в данной работе формулы могут
быть использованы при теоретическом и экспериментальном изучении фазы Берри
в различных квантовомеханических системах. При этом общность примененного
подхода позволяет надееться на возможность обобщения результатов работы на
случай неабелевой геометрической фазы, а также на случай систем с неэрмитовым
гамильтонианом.

\newpage
\begin{thebibliography}{10}

\bibitem{FirstBerryWork}
 Berry~M.~V. // Proc. Roy. Soc. 1987. V. A392. P.~45.
\bibitem{60}
 Hannay J.H. // J. Phys. 1985. V. A18. P. 221
\bibitem{61}
 Berry M.V. // J. Phys. 1985. V. A18. P. 15
\bibitem{62}
 Berry M.V. Fundamental aspects of quantum theory. London, 1986. P. 267.
\bibitem{63}
 Aitchison I.J. // Acta Phys. Pol. 1987. V. B18. No 3. P.207.
\bibitem{64}
 Niemi A.J., Semenoff G.W. // Nucl. Phys. 1986. V. B276. No 1. P. 173.
\bibitem{SimonWork}
 Simon B. // Phys. Rev. Lett. 1983. V. 51. P. 2167.
\bibitem{NaumovWork}
 Наумов~В.~А. // ЖЭТФ. 1992. Т. 101. С.~3.
\bibitem{AboutPhaseInSpectra}
 Виницкий~С.И., Дербов~В.Л., Дубовик~В.М., Марковски~Б.Л., Степановский~Ю.П.
 // УФН. 1990. Т. 160. N6. С.~1.
\bibitem{91}
 Aharonov Y., Anamdan J. // Phys. Rev. Lett. 1987. V. 58. P. 1593
\bibitem{WilzekZeeWork}
  Wilzek F., Zee A. // Phys. Rev. Lett. 1984. V. 52. P.2111
\bibitem{About2Hamiltonian}
 Dodonov~V.V., Man'ko~V.I. in Topological Phases in: Quantum Theory.
 Eds. Markovski~B., Vinitsky~S.I. World Scientific, Singapore, 1989. P.~74.
\bibitem{AboutExactSolution}
 Chaturvedi~S., Sriram~M., Shrinivasan~V. // J.Phys. 1987. V. A20. P.~L1071
\bibitem{CeulemansWork}
 Ceulemans~A., Szopa~M. // J.Phys. 1991. V. A24. P.~4495.
\bibitem{KorenblitWork}
 Korenblit~S.E., Kuznetsov~V.E., Naumov~V.A. in: Proc. Int. Conf.
 Quantum systems : New trends and methods. Minsk, 1994.
 Eds. Barut~A.O. et al. World Scientific, Singapure, 1995. P.~209.
\bibitem{TolkachevTregubovich1}
  Tolkachev~E.A., Tregubovich~A.Ya. in: Proc. Int. Conf.
  Topological phases in quantum theory. Eds. Markovski~B., Vinitsky~S.I.
  World Scientific, Singapure, 1989. P.~119.
\bibitem{TolkachevTregubovich2}
  Tolkachev~E.A., Tregubovich~A.Ya. in: Proc. Int. Conf.
 Quantum systems : New trends and methods. Minsk, 1994.
 Eds. Barut~A.O. et al. World Scientific, Singapure, 1995. P.~218.
\bibitem{BornAndFock}
 Born M. Fock V. // Zs. Phys. 1928. Bd51. S. 165
\bibitem{AboutMatrix}
 Гантмахер Ф. Р. Теория матриц. Москва, 1988.
\bibitem{PostnikovWork}
 Постников М.М. Дифференциальная геометрия. Москва, 1988.
\bibitem{DanielWork}
 Даниэль М., Виалле С.М. // УФН. 1982. Т. 136. Вып. 3. С. 377.
\end{thebibliography}

\end{document}





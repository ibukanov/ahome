\documentclass{article}
\usepackage{amsmath}
\usepackage{theorem}
\usepackage{epic}
%\usepackage{pict2e}

%%No paragraph break on emty lines
\catcode10=10   \catcode13=10


\def\eqn#1{\begin{displaymath}#1\end{displaymath}}
\def\eqnl#1#2{\begin{equation}\label{#1}#2\end{equation}}
\def\eqnm#1{\begin{eqnarray*}#1\end{eqnarray*}}
\def\eqnml#1#2{\begin{eqnarray}\label{#1}#2\end{eqnarray}}

\def\arr#1#2{\begin{array}{#1}#2\end{array}}


\def\rf#1{(\ref{#1})}

\def\p{\par}

\def\EQS{\; = \;}

%\DeclareMathOperator{\ln}{ln}

%I use dirty trick to treat new-lines as ordinary spaces!
%%%\catcode`\^^M=10


\theoremstyle{marginbreak} \theorembodyfont{\itshape}
\newtheorem{Def}{Definition}

\title{Implementation of fem in 2d}
\author {Igor Bukanov}
\date { 08 February 1999 }


\begin{document}

\setlength{\unitlength}{0.01\textwidth}

\begin{abstract}
I present some comments to the fem implementation.
\end{abstract}

\maketitle

\section{General consideration}


The standard equation in an arbitrary coordinate system
\eqnl{eqn-general}{
\frac{1}{\sqrt{g}} \partial_i(k^{ij} \sqrt{g} \partial_j p) = 0,
}
where the summation under repeated indexes from 1 to $N$-the space dimension
is assumed.

Of cause in the Cartesian coordinates the metrix tensor $g$ is the unit matrix
so \rf{eqn-general} simply becomes

\eqnl{eqn-cartesian}{
\partial_i(k^{ij}\partial_j p) = 0,
}

To write down the FEM equivalent of \rf{eqn-general} I multiply it by
an arbitrary function $v$, integrate over the equation domain and
perform the following transformations,

\eqnm{
0 &=& \int_\Omega \sqrt{g} v \times 0 dx^N
  \EQS \int_\Omega v \partial_i(k^{ij} \sqrt{g} \partial_j p) dx^N \\
  &=&  \int_\Omega [\partial_i (v k^{ij} \sqrt{g} \partial_j p)
                   - \partial_i v k^{ij} \sqrt{g} \partial_j p] dx^N
}

Using Green's  theorem I replace integration over full derivation in the
last equation by corresponding surface (or linear one in 2d) integral,

\begin{equation} \label{femfprm-general}
0 = \int_{\partial\Omega} (v k^{ij} \sqrt{g} \partial_j p) n_i dx^{N-1}
  - \int_\Omega \partial_i v k^{ij} \sqrt{g} \partial_j p dx^N,
\end{equation}
where $\vec{n} = n_1 \dots n_N$ is a normal vector to the  surface
$\partial\Omega$.

Now I have to deal with boundary conditions. In general they have the form
(This is the most general form when p and its derivatives are related via
linear expression):

\eqnl{boundary-general}{
\alpha(x) p + \beta(x) k^{ij} \sqrt{g} \partial_j p n_i = \gamma(x).
}

It is important to note that permeability tensor is present in the boundary
condition making somewhat more difficult to interpret the part with
derivatives then in a more standard case of Laplacian equation.

There are two important cases should in \rf{boundary-general}. The first one
is at thouse points on $\partial\Omega$ when $\beta(x) \ne 0$ and the second
one when $\beta(x) = 0$. They split naturally $\partial\Omega$ into two sets
$\partial\Omega_1$ and $\partial\Omega_2$ ,
\eqnl{b-is-zero-split}{
\partial\Omega = \partial\Omega_1 \cup \partial\Omega_2 ,
}
\eqn{
\quad \partial\Omega_1 \cap \partial\Omega_2  = \oslash,
}
\eqn{
\quad \beta(x) \ne 0 \rightarrow x \in  \partial\Omega_1
\quad \beta(x) = 0 \rightarrow \in  \partial\Omega_2
}

Now I will ignore a cases when $x = 0$ just at few points , i.e I assume
$\partial\Omega_1$ and $\partial\Omega_2$ are good subsets. Then I have

\eqnm{
\lefteqn{\int_{\partial\Omega} (v k^{ij} \sqrt{g} \partial_j p) n_i dx^{N-1}}
\\
&&=\int_{\partial\Omega_2} (v k^{ij} \sqrt{g} \partial_j p) n_i dx^{N-1}
+
\int_{\partial\Omega_1} \frac{v}{\beta} (\gamma -  \alpha(x) p) dx^{N-1}.
}

Now I can formulate the integral version of \rf{eqn-general} that satisfy
the boundary conditions \rf{boundary-general}.

\begin{Def}  \label{problem-def}
Find such p that satisfy
\eqnl{formulation-dirty}{
\int_{\partial\Omega_1} \frac{v}{\beta} (\gamma -  \alpha(x) p) dx^{N-1}
 = \int_\Omega \partial_i v k^{ij} \sqrt{g} \partial_j p dx^N,
}\eqn{
 p(x) = \frac{\gamma}{\alpha}, \quad x \in \partial\Omega_2 ,
}
for any sufficiently smooth $v$ that satisfy
$x \in \partial\Omega_2 \rightarrow v(x) = 0$
\end{Def}

\section{Connection with Well}
Here by the termin "Well" I understand a general source/sink that may appear
in~\rf{eqn-general}. In a simple two-dimensional model it may be represented
as a circle of radius $r_w$ with some boundary conditins such as well
pressure $p_w$ or in a more general case a relation between well influx
$q_w$ and
pressure, $F(q_w, p_w) = 0$.
\par
If one use a coarse grid to solve FEM descritisation
of~\rf{formulation-dirty} then I may assume that a typical grid cell size
is significantly bigger then $r_w$ hence I need a method of connecting
well boundary conditions to boundary conditions on the edge of the grid cell
containing well.

\par
In the following I will deal only with well pressure only type of boundary
conditions but the extension to the general case is straightforward.
I also assume that permeability inside the cell with well is constant but
this important restriction is natural because otherwise a numerical grid
becomes too coarse and one can simply decrease the cell size.
Another justification to this assumtion is a simple fact that variations
in reservoir properties occur over lengthes that are significantly greater
then a wellbore size. The figure~\rf{boundaryReconstaructionExample}
shows a typical setup for the problem under the consideration. To simplify
considiration use a coordinate system with origin at the well center.

\par

\begin{figure}
\caption{Example of boundary values reconstruction problem}
\label{boundaryReconstaructionExample}
\begin{picture}(80,30)
\drawline(0,10)(7,17)(17,17)(24,10)(17,3)(7,3)(0,10)
\put(15,12){\circle{3}}
\put(30,16){\vector(-1,0){12}}
\put(31,15){Cell boundary conditions?}
\put(30,12){\vector(-1,0){13.5}}
\put(31,11){given pressure $p_w$}
\put(30,7){\vector(-1,0){15}}
\put(31,6){uniform permeability}
\end{picture}
\end{figure}

The first possible (and I would say very naive) approach is to assume
that the fluid flow inside the cell is strictly radial, i.e. at each
point the fluid current $\vec{u}(\vec{r}) = -k\vec{\nabla} p$
has the form

\eqnl{radial_flow_vector}{ \vec{u}(\vec{r}) = u(r) \frac{\vec{r}}{r} }

\par To fathher explore this assumption I use polar coordinates
where~\rf{eqn-general} has the form
\eqnl{polar-laplas-flow}{
    \frac{1}{r}\frac{\partial}{\partial r} (ru^r)
    + \frac{\partial}{\partial \theta} u^\theta = 0,
}
\eqnl{polar-flow-from-pressure}{
u^\alpha = k^{\alpha\beta}(r,\theta)\partial_{\beta} p,
\quad \alpha = r,\,\theta,
}
where the summation over repeated Greek indixes are assumed.

\par
The assumtion~\rf{radial_flow_vector} now read simply
\eqnl{radial-flow-polar}{ u_\theta = 0 }
that immediately gives from~\rf{polar-laplas-flow}

\eqn{ \frac{1}{r}\frac{\partial}{\partial r} (ru^r) = 0 }
or
\eqnl{polar-radial-flow-is}{u^r = \frac{A(\theta)}{r}}

where $A(\theta)$ is an arbitrary function of $\theta$.
Now~\rf{polar-flow-from-pressure} gives an expression for the pressure,
\eqn{
\partial_{\beta} p
= (k^{-1})_{\beta\alpha}u^\alpha
\equiv (k^{-1})_{\beta r}u^r +  (k^{-1})_{\beta\theta}u^\theta
= (k^{-1})_{\beta r}u^r = \frac{A(\theta)}{r} (k^{-1})_{\beta r}.
}
\par
When $\beta=r$ I have from the last expression
\eqnl{eqn-for-p-r-polar}{
\partial_{r} p = \frac{A(\theta)}{r} (k^{-1})_{rr}.
}

\par
The component $(k^{-1})_{rr}$ of the tensor $k$ has the following expression
in terms of Cartesian components,
\eqnl{k-rr-polar-via-cartesian}{
(k^{-1})_{rr} = (k^{-1})_{xx} \cos^2\theta + (k^{-1})_{yy} \sin^2\theta
+ 2(k^{-1})_{xy} \sin\theta\cos\theta .
}

The last expression shows that if tensor $k$ is constant then
$(k^{-1})_{rr}$ does not depend on $r$ and from~\rf{eqn-for-p-r-polar}
I can write
\eqn{
p = (k^{-1})_{rr}A(\theta) \ln{r} + B(\theta) .
}

The function $B$ is fixed by boundary conditions on the well,
that is
\eqn{ p|_{r_w} = p_w, }
hence for $p$ I finaly have
\eqn{ p = p_w + A(\theta) (k^{-1})_{rr} \ln{\frac{r}{r_w}}}.

\par
Clearly, this simple form for $p$ is a consequence
of the assumption~\rf{radial_flow_vector} and to check how good is it
one has simply calculate $u_\theta$ and compare it with zero.

\par
If this assumption is justified then one can obtain a relation between
$p$ and $u$ at any point by expressing $A$ in terms of $p$,
\eqn{
A(\theta) = \frac{p-p_w}{(k^{-1})_{rr} \ln{\frac{r}{r_w}}},
}
so~\rf{polar-radial-flow-is} gives
\eqnl{u-in-term-p}{
    u^r = \frac{p-p_w}{(k^{-1})_{rr} r\ln{\frac{r}{r_w}}},
    \quad u^\theta=0.
}
\par
The Cartesian components of $u$ are given according to formulae,
\eqn{
u^x = \cos\theta u^r -r \sin\theta u^\theta,
\quad
u^y = \sin\theta u^r +r \cos\theta u^\theta.
}

It provides,
\eqn{
u^x = \frac{p-p_w}{x(k^{-1})_{rr}\ln{\frac{r}{r_w}}},
\quad
u^y = \frac{p-p_w}{y(k^{-1})_{rr}\ln{\frac{r}{r_w}}}
}
or in a more explicit way
\eqn{
\begin{pmatrix}u^x \\ u^y \end{pmatrix}
= 2 \begin{pmatrix} x^{-1} \\ y^{-1} \end{pmatrix}
\frac{p-p_w}{\ln{{r^2}/{r_w^2}}}
\frac{r^2}{(k^{-1})_{xx}x^2 + (k^{-1})_{yy}y^2 + 2(k^{-1})_{xy}xy}
}
where I used~\rf{k-rr-polar-via-cartesian}.

\par
The last expression provides explicit boundary
conditions of the form~\rf{boundary-general}.



\section{Two Dimensional Geometry}

Now some implementation formulae in 2 dimensions and Cartesian coordinates.

\p
If tensor $k$ is constant in $\Omega$ and $v$ and $p$ are liner functions,
$$
v = a_v x + b_v y + c_v, \quad p = a_p x + b_p y + c_p
$$
then
\eqnml{triangle_volume_integral}{
\lefteqn{\int_\Omega  \partial_i v k^{ij} \partial_j p \int_\Omega dx^2
\EQS \partial_i v k^{ij} \partial_j p \int_\Omega  dx^2}
\\ \nonumber
&=& [k^{xx} a_v a_p + k^{yy} b_v b_p + k^{xy} (a_v b_p + b_v a_p)] S(\Omega)
}

For the triangle defined by triple of its vertexes
$[\vec{r_1}, \vec{r_2}, \vec{r_3}]$ its square $S_\Delta$ is given by

\eqn{
S_\Delta = \frac{1}{2}
    \left(\arr{cc}{
        \vec{\Delta r_1} \cdot \vec{\Delta r_1}
            & \vec{\Delta r_1} \cdot \vec{\Delta r_2} \\
        \vec{\Delta r_2} \cdot \vec{\Delta r_1}
            & \vec{\Delta r_2} \cdot \vec{\Delta r_2}
    }\right)^\frac{1}{2},
}
where $\vec{\Delta r_1} \equiv \vec{r_1} - \vec{r_3}$ and
$\vec{\Delta r_2} \equiv \vec{r_2} - \vec{r_3}$.
\p
In 2 dimension the last expression becomes simply
\eqn{
S_\Delta = \frac{1}{2} |\Delta|,
\quad \Delta \equiv\Delta x_1 \Delta y_2 - \Delta x_2 \Delta y_1
}


\p

If linear function $f = ax + by + c$ is defined by its value at triangle
vertexes,
$$
[f(\vec{r_1}), f(\vec{r_2}), f(\vec{r_3})] = [f_1, f_2, f_3],
$$
then coeficients $a$, $b$, $c$
can be found from the following linear system,
\eqn{
ax_1 + by_1 + c = f_1
}\eqn{
ax_2 + by_2 + c = f_2
}\eqn{
ax_3 + by_3 + c = f_3
}

with the solution
\eqnm{
a &=& \frac{1}{\Delta}{\Delta f_1 \Delta y_2 - \Delta f_2 \Delta y_1 }, \\
b &=& \frac{1}{\Delta}{\Delta x_1 \Delta f_2 - \Delta x_2 \Delta f_1 }, \\
c &=& f_ 1 - ax_1 - by_1, \\
\Delta f_1 &=& f_1 - f_3, \\
\Delta f_2 &=& f_2 - f_3,
}

Because $\Delta$ presents in denominators for $a$ and  $b$ I introduce
"scaled" version of these parameters,

\eqnl{scaled_lin_coef}{
\arr{ccc}{
\tilde{a} &=& \Delta f_1 \Delta y_2 - \Delta f_2 \Delta y_1, \\
\tilde{b} &=& \Delta x_1 \Delta f_2 - \Delta x_2 \Delta f_1, \\
} }

and the expression for the integral over triangle area
in~\rf{triangle_volume_integral} becomes

\eqnl{triangle_volume_integral2}{
\int_\Omega  \partial_i v k^{ij} \partial_j p \int_\Omega dx^2
\EQS \frac{1}{2|\Delta|}
[k^{xx} \tilde{a}_v \tilde{a}_p + k^{yy} \tilde{b}_v \tilde{b}_p
+ k^{xy} (\tilde{a}_v \tilde{b}_p + \tilde{b}_v \tilde{a}_p)]
}

If I consider the boundary conditions in the definition \ref{problem-def}
from the implementation point of view, then $\partial\Omega_2$ transforms
itself to the set of {\itshape FEM} grid nodes with the given value of
pressure. To find a numeric interpretaion of the integration over
the $\partial\Omega_1$ domain I take into account that in in my
implementation $\partial\Omega_1$ is a sequence of line segments and for
the segment between $\vec{r_1} = (x_1, y_1)$
and $\vec{r_2} = (x_2, y_2)$ I have
\eqnl{int-line-edge1}{
\int\limits_{\mbox{line {}} \vec{r_1} \dots \vec{r_2}}
\frac{v}{\beta} (\gamma -  \alpha(x) p) ds
=
\int_0^1 \frac{v(\vec{r}(t))}{\beta(\vec{r}(t))}
  [\gamma(\vec{r}(t)) -  \alpha(\vec{r}(t)) p(\vec{r}(t))] L dt,
}
where $L = |\vec{r_2} - \vec{r_1}|$ is the line segment length,
$ds = Ldt$ is the line differntial element,
and the natural line parametrisation $\vec{r}(t)$ is given by
\begin{equation}
\vec{r}(t) = \vec{r_1} + (\vec{r_2} - \vec{r_1}) t, \quad t \in [0; 1]
\end{equation}

Now I take into account that in my simplest approach $v$ and $p$ are linear
functions hence on the line segment
\eqnm{
v(\vec{r}(t))
&=& v(\vec{r_1} + (\vec{r_2} - \vec{r_1}) t)
\\&=& v(\vec{r_1}) + t v(\vec{r_2}) - t v(\vec{r_1})
\\&=& (1 - t)v(\vec{r_1}) + t v(\vec{r_2})
}
and similary
\eqn{
p(\vec{r}(t)) = (1 - t)p(\vec{r_1}) + t p(\vec{r_2})
}

%But by the construction $v(\vec{r_1}) = 1$ and $v(\vec{r_2}) = 0$

Thus for \rf{int-line-edge1} I have
\eqnml{int-line-edge2}{
&
\displaystyle \nonumber
\int_0^1
    \frac{[(1 - t)v(\vec{r_1}) + t v(\vec{r_2}]}{\beta(\vec{r}(t))}
  \lbrace\gamma(\vec{r}(t)) -  \alpha(\vec{r}(t))
  [(1 - t)p(\vec{r_1}) + t p(\vec{r_2}]\rbrace L dt
\\
& \displaystyle
\EQS
v(\vec{r_1}) A + v(\vec{r_2}) B
  + v(\vec{r_1}) p(\vec{r_1}) C + v(\vec{r_2}) p(\vec{r_2}) D
\\ \nonumber
& \displaystyle
  + [v(\vec{r_1}) p(\vec{r_2}) + v(\vec{r_2}) p(\vec{r_1})] E
}
with
\eqnm{
A &=& \int_0^1 \frac{(1 - t)\gamma(\vec{r}(t))}{\beta(\vec{r}(t))} L dt \\
B &=& \int_0^1 \frac{t\gamma(\vec{r}(t))}{\beta(\vec{r}(t))} L dt \\
C &=& \int_0^1 \frac{(1 - t)^2 \alpha(\vec{r}(t))}{\beta(\vec{r}(t))} L dt \\
D &=& \int_0^1 \frac{t^2 \alpha(\vec{r}(t))}{\beta(\vec{r}(t))} L dt \\
E &=& \int_0^1 \frac{t(1 - t) \alpha(\vec{r}(t))}{\beta(\vec{r}(t))} L dt
}

Clearly, the coeficients $A-E$ depend only on the boundary conditions
and can be calculated as long as they are specified. Thus I may use values
of $A-E$ as independent parameters that should be specified for
each line segment.

Structure of \rf{int-line-edge2} shows that $A$ and $B$ may contribute
only to the right-hand side of the {\itshape FEM} equation,
$C$ and $D$ contribute to the diagonal element of the stiffness matrix and
$E$ influences only such element $A_{ij}$ of the stiffness matrix where
$i$ and $j$ are indexes of two connected grid notes on the grid boundary
that belong to $\partial\Omega_1$.

\section{Three Dimensional Geometry}

Here I present some formulae for the 3-d case.
\par

If I have a linear function $f$ define over a pyramid bounded by points
$\vec{r_0}$, $\vec{r_1}$, $\vec{r_2}$, $\vec{r_3}$,

\eqn{ f = ax + by + cz + d, }

and its values at $\vec{r_0}$-$\vec{r_3}$ are $f_0$, $f_1$, $f_2$, $f_3$,
then I can restore $a$-$d$ via the solution of the following system of
linear equations,
\eqnl{linear-coeficients-eqn-three-d}{
    \left\lbrace \arr{rcl}{
    ax_0 + by_0 + cy_0 + d & = & f_0 \\
    ax_1 + by_1 + cy_1 + d & = & f_1 \\
    ax_2 + by_2 + cy_2 + d & = & f_2 \\
    ax_3 + by_3 + cy_3 + d & = & f_3
    } \right.
}

\par

\begin{figure}[h]
\caption{Linear function defined inside tetrahedron formed by
    $\vec{r}_0$..$\vec{r}_1$}
\label{piramidLikeElement}
\begin{picture}(80,45)
\drawline(10,20)(40,40)(50,20)(30,6)(10,20)
\drawline(40,40)(30,6)
\dashline{3}(10,20)(50,20)
\put(7,20){$\vec{r}_1$}
\put(30,4){$\vec{r}_2$}
\put(51,20){$\vec{r}_3$}
\put(40,41){$\vec{r}_0$}
\put(55,35){\vector(-1,-1){18}}
\put(57,35){$f = ax + by + cz + d$}
\end{picture}
\end{figure}

If I need only value of $a$-$c$ then by subtracting the first equation from
the others I have a simple system
\eqnl{linear-coeficients-eqn-three-d-simple}{
\left\lbrace
\arr{rcl}{
a \Delta x_1 + b \Delta y_1 + c \Delta y_1 & = & \Delta f_1 \\
a \Delta x_2 + b \Delta y_2 + c \Delta y_2 & = & \Delta f_2 \\
a \Delta x_3 + b \Delta y_3 + c \Delta y_3 & = & \Delta f_3
}
\right. ,
}

where

\eqn{ \Delta A_i \equiv A_i - A_0, \quad i = 1..3 . }

The solution of \rf{linear-coeficients-eqn-three-d-simple} is given by
\eqnm{
a \Delta_3 &=&\Delta f_1 (\Delta y_2 \Delta z_3 - \Delta y_3 \Delta z_2)
+\Delta f_2 (\Delta y_3 \Delta z_1 - \Delta y_1 \Delta z_3)
\\&&
+\Delta f_3 (\Delta y_1 \Delta z_2 - \Delta y_2 \Delta z_1),
}
\eqnml{linear-coeficients-eqn-three-d-solution}{
b \Delta_3 &=&\Delta f_1 (\Delta x_3 \Delta z_2 - \Delta x_2 \Delta z_3)
+\Delta f_2 (\Delta x_1 \Delta z_3 - \Delta x_3 \Delta z_1)
\\ \nonumber &&
+\Delta f_3 (\Delta x_2 \Delta z_1 - \Delta x_1 \Delta z_2),
}
\eqnm{
c \Delta_3 &=&\Delta f_1 (\Delta x_2 \Delta y_3 - \Delta x_3 \Delta y_2)
+\Delta f_2 (\Delta x_3 \Delta y_1 - \Delta x_1 \Delta y_3)
\\&&
+\Delta f_3 (\Delta x_1 \Delta y_2 - \Delta x_2 \Delta y_1),
}

where by the definition
\eqnl{Delta-three-def}{
\Delta_3 \EQS \left|\arr{ccc}{
\Delta x_1 & \Delta y_1 & \Delta z_1 \\
\Delta x_2 & \Delta y_2 & \Delta z_2 \\
\Delta x_3 & \Delta y_3 & \Delta z_3
}\right|
}
or
\eqnm{
\Delta_3 &=&
\Delta x_1 (\Delta y_2 \Delta z_3 - \Delta y_3 \Delta z_2)
+\Delta x_2 (\Delta y_3 \Delta z_1 - \Delta y_1 \Delta z_3)
\\&&
+\Delta x_3 (\Delta y_1 \Delta z_2 - \Delta y_2 \Delta z_1)
}




During construction of the stiffness matrix I need to
calculate~\rf{linear-coeficients-eqn-three-d} for the following sets
of $f_0$ - $f_3$,
\eqn{
(f_0, f_1, f_2, f_3) \EQS
(1,\,0,\,0,\,0), \; (0,\,1,\,0,\,0), \; (0,\,0,\,1,\,0), \; (0,\,0,\,0,\,1),
}
\eqn{
(\Delta f_1, \Delta f_2, \Delta f_3) \EQS
(-1,\,-1,\,-1), \; (1,\,0,\,0), \; (0,\,1,\,0), \; (0,\,0,\,1).
}

This gives
\begin{description}
\item [Case $(f_0, f_1, f_2, f_3) = (1,0,0,0)$]
\eqnm{
a \Delta_3 &=&
-\Delta y_2 \Delta z_3 + \Delta y_3 \Delta z_2
-\Delta y_3 \Delta z_1 + \Delta y_1 \Delta z_3
\\&& \quad
-\Delta y_1 \Delta z_2 + \Delta y_2 \Delta z_1,
\\
b \Delta_3 &=&
-\Delta x_3 \Delta z_2 + \Delta x_2 \Delta z_3
-\Delta x_1 \Delta z_3 + \Delta x_3 \Delta z_1
\\&& \quad
-\Delta x_2 \Delta z_1 + \Delta x_1 \Delta z_2,
\\
c \Delta_3 &=&
-\Delta x_2 \Delta y_3 + \Delta x_3 \Delta y_2
-\Delta x_3 \Delta y_1 + \Delta x_1 \Delta y_3
\\&& \quad
-\Delta x_1 \Delta y_2 + \Delta x_2 \Delta y_1,
}

or

\eqnm{
a \Delta_3 &=&
y_1 (z_3 - z_2) + y_2 (z_1 - z_3) + y_3 (z_2 - z_1)
\\
b \Delta_3 &=&
z_1 (x_3 - x_2) + z_2 (x_1 - x_3) + z_3 (x_2 - x_1)
\\
c \Delta_3 &=&
x_1 (y_3 - y_2) + x_2 (y_1 - y_3) + x_3 (y_2 - y_1)
}

\item [Case $(f_0, f_1, f_2, f_3) = (0,1,0,0)$]
\eqnm{
a \Delta_3 &=& \Delta y_2 \Delta z_3 - \Delta y_3 \Delta z_2,
\\
b \Delta_3 &=& \Delta x_3 \Delta z_2 - \Delta x_2 \Delta z_3,
\\
c \Delta_3 &=&\Delta x_2 \Delta y_3 - \Delta x_3 \Delta y_2
}

\item [Case $(f_0, f_1, f_2, f_3) = (0,0,1,0)$]
\eqnm{
a \Delta_3 &=&\Delta y_3 \Delta z_1 - \Delta y_1 \Delta z_3,
\\
b \Delta_3 &=& \Delta x_1 \Delta z_3 - \Delta x_3 \Delta z_1,
\\
c \Delta_3 &=& \Delta x_3 \Delta y_1 - \Delta x_1 \Delta y_3,
}

\item [Case $(f_0, f_1, f_2, f_3) = (0,0,0,1)$]
\eqnm{
a \Delta_3 &=& \Delta y_1 \Delta z_2 - \Delta y_2 \Delta z_1,
\\
b \Delta_3 &=& \Delta x_2 \Delta z_1 - \Delta x_1 \Delta z_2,
\\
c \Delta_3 &=& \Delta x_1 \Delta y_2 - \Delta x_2 \Delta y_1
}
\end{description}




\end{document}



\documentclass{article}
\usepackage{amsmath}
\usepackage{a4wide}

\usepackage{igor_macros}

%%No paragraph break on emty lines
\catcode10=10   \catcode13=10

\def\dint#1#2{\int_{#1}^{#2}}


\def\tower3#1#2#3{\overset{#3}{\underset{#2}{#1}}}

\def\braced#1{\left\lbrace{#1}\right\rbrace}


\title {4th order time dependence}
\author {Igor Bukanov}
\date { 2001-11-04 }

\begin{document}

\begin{abstract}
Should be written
\end{abstract}

\maketitle

\section{Should be named 1}

Time-dependent equation in Dirak picture:
\eqn{
\frac{dX}{dt} = \left[\frac{i}{\hbar}H(t), X\right]
}
or 
\eqnl{diff-time-dependence}{
\frac{dX}{dt} = \left[h(t), X\right]
}
with 
\eqn{
h(t) \equiv \frac{i}{\hbar}H(t)
}

\p
Typically $H(t)$ is a small operator (in some particular sence) given by
\eqn{
H(t) = e^{\frac{it}{\hbar}H_0}H_1 e^{-\frac{it}{\hbar}H_0t}
}
where $H_0 + H_1$ form the time independent Hamiltonian with $H_1$ small.


\p
Integral form of \rf{diff-time-dependence} is
\eqnl{int-time-dependence}{
X(T) = X(t_0) + \dint{t_0}{T} dt \left[h(t), X(t)\right]
}

\p The forth order approximation of the solution for \rf{int-time-dependence} is given by
\eqn{
X(T) \approx X_0 + Z_1 + Z_2 + Z_3 + Z_4
}
where 
\eqn{
X_0 \equiv X(t_0) 
}
and $Z_i$ are defined by

\eqnl{Z-definition}{
\begin{split}
Z_1 & = \dint{t_0}{T} dt [h(t), X_0] \\
Z_2 & = \dint{t_0}{T} dt_2 \dint{t_0}{t_2} dt_1 [h(t_2), [h(t_1), X_0]] \\
Z_3 & = \dint{t_0}{T} dt_3 \dint{t_0}{t_3} dt_2 \dint{t_0}{t_2} dt_1 
         [h(t_3), [h(t_2), [h(t_1), X_0]]] \\
Z_4 & = \dint{t_0}{T} dt_4 \dint{t_0}{t_4} dt_3 \dint{t_0}{t_3} dt_2
         \dint{t_0}{t_2} dt_1 [h(t_4), [h(t_3), [h(t_2), [h(t_1), X_0]]]] 
\end{split}
}

\p
The problem with the expression for $Z_i$ in \rf{Z-definition} is that it is not possible to perform integration over $t$ without knowledge of $X_0$. So in the following I will get expression for $Z_i$ that would involve only time integration  of some expressions of $h(t_i)$-commutators with dependence on $x_0$ only via subsequent commutation of the integration results with $X_0$.

\p
New shorthand
\eqn{\begin{split}
x & \equiv X_0 \\
h_i & \equiv h(t_i) 
\end{split}}

And notation for multi intergration
\begin{multline*}
\braced{{i_1}_{a_1}^{b_1}{i_2}_{a_2}^{b_2}\dots{i_N}_{a_N}^{b_N}}
 f(t_{i_1}, t_{i_2}, \dots, t_{i_N})
\\
\equiv \quad 
\int_{t_{a_1}}^{t_{b_1}} dt_{i_1}
\int_{t_{a_2}}^{t_{b_2}} dt_{i_2}
\dots
\int_{t_{a_N}}^{t_{b_N}} dt_{i_N}
f(t_{i_1}, t_{i_2}, \dots, t_{i_N})
\end{multline*}

with additional rule that if some $a_i$ or $b_i$ is $T$, then it stands in the integration limits for $T$ itself and not for $t_T$.
\p
Examples:
\eqn{
\braced{ 2_0^T 1_0^T }
 f(h_1, h_2) \quad
\equiv \quad
\int_{t_0}^T dt_2 \int_{t_0}^T dt_1 f(h(t_1), h(t_2))
}

\eqn{
\braced{ 3_0^T 1_3^T 2_3^1 }
 f(h_3, h_2, h_1) \quad
\equiv \quad
\int_{t_0}^T dt_3 \int_{t_3}^T dt_1 \int_{t_3}^{t_1} dt_2  
f(h(t_3), h(t_2), h(t_1))
}

\begin{multline*}
\braced{4_0^T 3_0^4 2_0^3 1_0^2 }
 f(h_4, h_3, h_2, h_1) \\
\equiv \quad
\int_{t_0}^T dt_4 \int_{t_0}^{t_4} dt_3 \int_{t_0}^{t_3} dt_3
 \int_{t_0}^{t_2} dt_1  f(h(t_4), h(t_3), h(t_2), h(t_1))
\end{multline*}

\p
Now I can state in a compact form the following lemmas that I will need later during $Z_i$ transformatins.

\eqnl{cicle-int-sum-3}{
\left( \braced{3_0^T 2_0^3 1_0^3} + \braced{2_0^T 1_0^2 3_0^2} 
 + \braced{1_0^T 3_0^1 2_0^1}
 \right) F
\, = \, \braced{3_0^T 2_0^T 1_0^T}  F
}

\eqnl{cicle-int-sum-4}{
\left( \braced{4_0^T 3_0^4 2_0^4 1_0^4} + \braced{3_0^T 2_0^3 1_0^3 4_0^3} 
+ \braced{2_0^T 1_0^2 4_0^2 3_0^2} + \braced{1_0^T 4_0^1 3_0^1 2_0^1}
 \right) F
\, = \, \braced{4_0^T 3_0^T 2_0^T 1_0^T}  F
}

where \rf{cicle-int-sum-3} in the left part contains cyclic transposition of $3\,2\,1$ and \rf{cicle-int-sum-4} contains cyclic transposition of
$4\,3\,2\,1$.

\p
These are generalization of 
\eqnl{cicle-int-sum-2}{
\left( \braced{2_0^T 1_0^2} + \braced{1_0^T 2_0^1} 
 \right) F
\, = \, \braced{2_0^T 1_0^T}  F
}

The proof of \rf{cicle-int-sum-3} and \rf{cicle-int-sum-4} uses the following change result for change of integration order:
\eqnl{int-order-change-2}{
\braced{1_B^C 2_A^1}  F
\; = \; 
\left( \braced{2_A^B 1_B^C} + \braced{2_B^C 1_2^C} 
 \right) F
}
or with explicit integral signs:
\eqn{
\int_{t_B}^{t_C} dt_1 \int_{t_A}^{t_1} dt_2 
\; = \; 
\left( \int_{t_A}^{t_B} dt_2 \int_{t_B}^{t_C} dt_1
+  \int_{t_B}^{t_C} dt_2 \int_{t_2}^{t_C} dt_1 \right) F

}

Indeed:
\begin{align*}
\braced{1_B^C 2_A^1}  F 
& = \; \left(\braced{1_B^A 2_A^1} + \braced{1_A^C 2_A^1}\right) F
\; = \; \left(-\braced{1_A^B 2_A^1} + \braced{1_A^C 2_A^1}  \right) F
\\& = \; \left(-\braced{2_A^B 1_2^B} + \braced{2_A^C 1_2^C} \right) F
\\& = \; \left(-\braced{2_A^B 1_2^C} - \braced{2_A^B 1_C^B} 
     + \braced{2_A^C 1_2^C} \right) F
\\& = \; \left(-\braced{2_A^B 1_2^C} + \braced{2_A^B 1_B^C} 
     + \braced{2_A^C 1_2^C}  \right) F
\\& = \; \left(\braced{2_B^A 1_2^C}  + \braced{2_A^B 1_B^C} 
     + \braced{2_A^C 1_2^C} \right) F
\\& = \; \left(\braced{2_A^B 1_B^C}  
     + \braced{2_B^A 1_2^C} + \braced{2_A^C 1_2^C} \right) F
\\& = \; \left(\braced{2_A^B 1_B^C}  + \braced{2_B^C 1_2^C}\right) F,
\end{align*}

where I used:
\eqn{
\int_a^b dt f(t) \; = \; \left(\int_a^c dt + \int_c^b\right) dt f(t),
\qquad \int_a^b dt f(t) \; = \; -\int_b^a dt f(t) 
}
and
\eqn{
\int_a^b dt_1 \int_a^{t_1} dt_2 f(t_1, t_2)
\; = \; \int_a^b dt_2 \int_{t_1}^b dt_1 f(t_1, t_2)

}

Now I have for the left side of \rf{cicle-int-sum-3} by changing integration order in the second and third terms to $3-2-1$:
\begin{align*}
\lefteqn{\braced{3_0^T 2_0^3 1_0^3} + \braced{2_0^T 1_0^2 3_0^2} 
 + \braced{1_0^T 3_0^1 2_0^1}}
\\ & = \; 
\braced{3_0^T 2_0^3 1_0^3} + \braced{2_0^T 3_0^2 1_0^2} 
 + \braced{3_0^T 1_3^T 2_0^1}
\\ & = \; 
\braced{3_0^T 2_0^3 1_0^3} + \braced{3_0^T 2_3^T 1_0^2} 
 + \braced{3_0^T 2_0^3 1_3^T} + \braced{3_0^T 2_3^T 1_2^T}.
\end{align*}

Farther regrouping and fusing according to $\int_a^b = \int_a^c + \int_c^b$ first for $t_1$ and then for $t_2$ prooves \rf{cicle-int-sum-3}:
\begin{align*}
\lefteqn{\braced{3_0^T 2_0^3 1_0^3} + \braced{2_0^T 1_0^2 3_0^2} 
 + \braced{1_0^T 3_0^1 2_0^1}}
\\ & = \; 
\braced{3_0^T 2_0^3 1_0^3}  + \braced{3_0^T 2_0^3 1_3^T} 
+ \braced{3_0^T 2_3^T 1_0^2} + \braced{3_0^T 2_3^T 1_2^T}.
\\ & = \; 
\braced{3_0^T 2_0^3 1_0^T} + \braced{3_0^T 2_3^T 1_0^T} 
\; = \; 
\braced{3_0^T 2_0^T 1_0^T} 

\end{align*}

\p
The procedure works for \rf{cicle-int-sum-4} as well. Changing integration order to $4-3-2-1$ gives:
\begin{align*}
\lefteqn{\braced{4_0^T 3_0^4 2_0^4 1_0^4} + \braced{3_0^T 2_0^3 1_0^3 4_0^3} 
+ \braced{2_0^T 1_0^2 4_0^2 3_0^2} + \braced{1_0^T 4_0^1 3_0^1 2_0^1}}
\\ & = \; 
\braced{4_0^T 3_0^4 2_0^4 1_0^4} 
+ \braced{3_0^T 4_0^3 2_0^3 1_0^3} 
+ \braced{2_0^T 4_0^2 3_0^2 1_0^2}
+ \braced{1_0^T 4_0^1 3_0^1 2_0^1}
\\ & = \; 
\braced{4_0^T 3_0^4 2_0^4 1_0^4} 
+ \braced{4_0^T 3_4^T 2_0^3 1_0^3} 
+ \braced{4_0^T 2_4^T 3_0^2 1_0^2}
+ \braced{4_0^T 1_4^T 3_0^1 2_0^1}
\\ & = \; 
\braced{4_0^T 3_0^4 2_0^4 1_0^4} 
+ \braced{4_0^T 3_4^T 2_0^3 1_0^3} 
+ \braced{4_0^T 3_0^4 2_4^T 1_0^2}+ \braced{4_0^T 3_4^T 2_3^T 1_0^2}
+ \braced{4_0^T 1_4^T 3_0^1 2_0^1}
\\ & = \; 
\braced{4_0^T 3_0^4 2_0^4 1_0^4} 
+ \braced{4_0^T 3_4^T 2_0^3 1_0^3} 
+ \braced{4_0^T 3_0^4 2_4^T 1_0^2} + \braced{4_0^T 3_4^T 2_3^T 1_0^2}
\\ & \qquad
+ \braced{4_0^T 3_0^4 1_4^T 2_0^1} + \braced{4_0^T 3_4^T 1_3^T 2_0^1}
\\ & = \; 
\braced{4_0^T 3_0^4 2_0^4 1_0^4} 
+ \braced{4_0^T 3_4^T 2_0^3 1_0^3} 
+ \braced{4_0^T 3_0^4 2_4^T 1_0^2} + \braced{4_0^T 3_4^T 2_3^T 1_0^2}
\\ & \qquad
+ \braced{4_0^T 3_0^4 2_0^4 1_4^T} + \braced{4_0^T 3_0^4 2_4^T 1_2^T}
+ \braced{4_0^T 3_4^T 2_0^3 1_3^T} + \braced{4_0^T 3_4^T 2_3^T 1_2^T}
\end{align*}

Rearrangements and fusing provide \rf{cicle-int-sum-4}:

\begin{align*}
\lefteqn{\braced{4_0^T 3_0^4 2_0^4 1_0^4} + \braced{3_0^T 2_0^3 1_0^3 4_0^3} 
+ \braced{2_0^T 1_0^2 4_0^2 3_0^2} + \braced{1_0^T 4_0^1 3_0^1 2_0^1}}
\\ & = \; 
\braced{4_0^T 3_0^4 2_0^4 1_0^4} + \braced{4_0^T 3_0^4 2_0^4 1_4^T} 
+ \braced{4_0^T 3_4^T 2_0^3 1_0^3} + \braced{4_0^T 3_4^T 2_0^3 1_3^T}
\\ & \qquad
+ \braced{4_0^T 3_0^4 2_4^T 1_0^2} + \braced{4_0^T 3_0^4 2_4^T 1_2^T}
+ \braced{4_0^T 3_4^T 2_3^T 1_0^2} + \braced{4_0^T 3_4^T 2_3^T 1_2^T}
\\ & = \; 
\braced{4_0^T 3_0^4 2_0^4 1_0^T} + \braced{4_0^T 3_4^T 2_0^3 1_0^T}
+ \braced{4_0^T 3_0^4 2_4^T 1_0^T} + \braced{4_0^T 3_4^T 2_3^T 1_0^T}
\\ & = \; 
\braced{4_0^T 3_0^4 2_0^4 1_0^T}+ \braced{4_0^T 3_0^4 2_4^T 1_0^T}  
+ \braced{4_0^T 3_4^T 2_0^3 1_0^T}+ \braced{4_0^T 3_4^T 2_3^T 1_0^T}
\\ & = \; 
\braced{4_0^T 3_0^4 2_0^T 1_0^T} + \braced{4_0^T 3_4^T 2_0^T 1_0^T}
\; = \; 
\braced{4_0^T 3_0^T 2_0^T 1_0^T} 
\end{align*}

\p
Transformation of $Z_2$ which in new notation is given according to from \rf{Z-definition} by
\eqn{
Z_2  = \braced{2_0^T 1_0^2} [h_2, [h_1, x]]
}

After splitting according to $a=\frac{1}{2}a + \frac{1}{2}a$ and renaming $1\leftrightarrow 2$ I have
\begin{align*}
Z_2  
& = \; \frac{1}{2}\braced{2_0^T 1_0^2} [h_2, [h_1, x]] + \frac{1}{2}\braced{2_0^T 1_0^2} [h_2, [h_1, x]]
\\ & = \;  \frac{1}{2}\braced{2_0^T 1_0^2} [h_2, [h_1, x]] + \frac{1}{2}\braced{1_0^T 2_0^1} [h_1, [h_2, x]]
\end{align*}

Appling
\eqnl{comm-like-diff}{
[a, [b, c]] =  [[a, b], c]] + [b, [a, c]]
}
gives:
\begin{align*}
Z_2 
& = \;  \frac{1}{2}\braced{2_0^T 1_0^2} [h_2, [h_1, x]] + \frac{1}{2}\braced{1_0^T 2_0^1} \left([[h_1, h_2], x]] + [h_2, [h_1, x]]\right)
\\ & = \;  \frac{1}{2}\left(\braced{2_0^T 1_0^2} + \braced{1_0^T 2_0^1}\right)[h_2, [h_1, x]] + \frac{1}{2}\braced{1_0^T 2_0^1} [[h_1, h_2], x]]
\end{align*}

Finally according to \rf{cicle-int-sum-2} I get
\eqn{
Z_2 \; = \;  \frac{1}{2}\braced{2_0^T 1_0^T}[h_2, [h_1, x]] + \frac{1}{2}\braced{1_0^T 2_0^1} [[h_1, h_2], x]]
}

or after reaname $1\leftrightarrow 2$ in the second term
\eqnl{Z-2-transformed}{
Z_2 
\; = \; \frac{1}{2}\braced{2_0^T 1_0^T}[h_2, [h_1, x]] + \frac{1}{2}\braced{2_0^T 1_0^2} [[h_2, h_1], x]]
}

which can be written also as
\eqn{
\begin{split}
Z_2 & = \;  \frac{1}{2}\left[\int_0^T dt\,h(t), \left[\int_0^T dt\,h(t), \; X_0\right]\right]
\\ & + \; \frac{1}{2}\left[\int_0^T dt_2 \int_0^{t_2} dt_1 [h(t_2), h(t_1)],\; X_0\right]
\end{split}
}

that allows to carry out time integration only with knowledge of $h(t)$.

\p Dealing with $Z_3$ and $Z_4$ would be based on similar transformations but of cause the calculation would be rather lengthy.

\p
For $Z_3$ I first use \rf{Z-2-transformed} to write
\eqn{
\begin{split}
Z_3 
& \equiv \;  
\braced{3_0^T 2_0^3 1_0^2} [h(t_3), [h(t_2), [h(t_1), x]]]
\\ & = \;
\frac{1}{2}\braced{3_0^T 2_0^3 1_0^3}[h_3, [h_2, [h_1, x]]] + \frac{1}{2}\braced{3_0^T 2_0^3 1_0^2} [h_3, [[h_2, h_1], x]] 
\end{split}
}
\p
Slitting the first term via $a = \frac{1}{3}a + \frac{1}{3}a + \frac{1}{3}a$ and performing cyclic renames $2\leftrightarrow 3$ in the second part and $1\rightarrow 2 \rightarrow 3 \rightarrow 1$ in the second gives
\eqnl{Z3-calc-1}{
\begin{split}
Z_3 
& = \;
\frac{1}{6}\braced{3_0^T 2_0^3 1_0^3}[h_3, [h_2, [h_1, x]]] 
+ \frac{1}{2}\braced{3_0^T 2_0^3 1_0^2} [h_3, [[h_2, h_1], x]] 
\\ & \quad 
+\frac{1}{6}\braced{3_0^T 2_0^3 1_0^3}[h_3, [h_2, [h_1, x]]] 
\\ & \quad 
+\frac{1}{6}\braced{3_0^T 2_0^3 1_0^3}[h_3, [h_2, [h_1, x]]] 

\\& = \;
\frac{1}{6}\braced{3_0^T 2_0^3 1_0^3}[h_3, [h_2, [h_1, x]]] 
+ \frac{1}{2}\braced{3_0^T 2_0^3 1_0^2} [h_3, [[h_2, h_1], x]] 
\\ & \quad 
+\frac{1}{6}\braced{2_0^T 3_0^2 1_0^2}[h_2, [h_3, [h_1, x]]] 
\\ & \quad 
+\frac{1}{6}\braced{1_0^T 3_0^1 2_0^1}[h_1, [h_3, [h_2, x]]] 
\end{split}
}
 
I use \rf{comm-like-diff} to get for the third and forth term expression that would involve $[h_3, [[h_2, [h_1, x]]]$ plus terms where x nesting is no more then 2. For the commutators in the third and forth terms I have
\eqn{
\begin{split}
[h_2, [h_3, [h_1, x]]]
& = \; [[h_2, h_3], [h_1, x]] + [h_3, [h_2, [h_1, x]]] 
\\
[h_1, [h_3, [h_2, x]]]
& = \; [[h_1, h_3], [h_2, x]] + [h_3, [h_1, [h_2, x]]]
\\ & = \; [[h_1, h_3], [h_2, x]] + [h_3, [[h_1, h_2], x]] + [h_3, [h_2, [h_1, x]]]
\end{split}
}


Thus \rf{Z3-calc-1} becomes
\eqn{
\begin{split}
Z_3 
& = \;
\frac{1}{6}\braced{3_0^T 2_0^3 1_0^3}[h_3, [h_2, [h_1, x]]] 
+ \frac{1}{2}\braced{3_0^T 2_0^3 1_0^2} [h_3, [[h_2, h_1], x]] 
\\ & \quad 
+\frac{1}{6}\braced{2_0^T 3_0^2 1_0^2}
\left([[h_2, h_3], [h_1, x]] + [h_3, [h_2, [h_1, x]]] \right)
\\ & \quad 
+\frac{1}{6}\braced{1_0^T 3_0^1 2_0^1}\left([[h_1, h_3], [h_2, x]] + [h_3, [[h_1, h_2], x]] + [h_3, [h_2, [h_1, x]]]\right)
\end{split}
}

or after regrouping terms and using \rf{cicle-int-sum-3}
\eqn{
\begin{split}
Z_3 
& = \;
\frac{1}{6}\left(\braced{3_0^T 2_0^3 1_0^3} + \braced{2_0^T 1_0^2 3_0^2} + \braced{1_0^T 3_0^1 2_0^1}\right)[h_3, [h_2, [h_1, x]]] 
\\ & \quad 
+ \frac{1}{2}\braced{3_0^T 2_0^3 1_0^2} [h_3, [[h_2, h_1], x]] 
\\ & \quad 
+\frac{1}{6}\braced{2_0^T 3_0^2 1_0^2}[[h_2, h_3], [h_1, x]]
\\ & \quad 
+\frac{1}{6}\braced{1_0^T 3_0^1 2_0^1}\left([[h_1, h_3], [h_2, x]] + [h_3, [[h_1, h_2], x]]\right)

\\& = \;
\frac{1}{6}\braced{3_0^T 2_0^T 1_0^T} [h_3, [h_2, [h_1, x]]] 
+ \frac{1}{2}\braced{3_0^T 2_0^3 1_0^2} [h_3, [[h_2, h_1], x]] 
\\ & \quad 
+\frac{1}{6}\braced{2_0^T 3_0^2 1_0^2}[[h_2, h_3], [h_1, x]]
\\ & \quad 
+\frac{1}{6}\braced{1_0^T 3_0^1 2_0^1}\left([[h_1, h_3], [h_2, x]] + [h_3, [[h_1, h_2], x]]\right)
\end{split}
}

Renaming $2\,3$ back to $3\,2$ in the third term and $1\,3\,2$ back to $3\,2\,1$ in the forth gives
\eqnl{Z3-calc-2}{
\begin{split}
Z_3 
& = \;
\frac{1}{6}\braced{3_0^T 2_0^T 1_0^T} [h_3, [h_2, [h_1, x]]] 
+ \frac{1}{2}\braced{3_0^T 2_0^3 1_0^2} [h_3, [[h_2, h_1], x]] 
\\ & \quad 
+\frac{1}{6}\braced{3_0^T 2_0^3 1_0^3}[[h_3, h_2], [h_1, x]]
\\ & \quad 
+\frac{1}{6}\braced{3_0^T 2_0^3 1_0^3}\left([[h_3, h_2], [h_1, x]] + [h_2, [[h_3, h_1], x]]\right)
\end{split}
}

Transforming the last commutator:
\eqn{
[h_2, [[h_3, h_1], x]] = [[h_2, [h_3, h_1]], x] + [[h_3, h_1], [h_2, x]]
}
and using
\eqn{
\splitDo{
\braced{3_0^T 2_0^3 1_0^3}[[h_3, h_1], [h_2, x]]
 &= \braced{3_0^T 1_0^3 3_0^3}[[h_3, h_1], [h_2, x]]
\\& = \braced{3_0^T 2_0^3 1_0^3}[[h_3, h_2], [h_1, x]]
}
}
where I first change integration order and then renamed $1\leftrightarrow 2$,
provides for~\rf{Z3-calc-2}
\eqn{
\begin{split}
Z_3 
& = \;
\frac{1}{6}\braced{3_0^T 2_0^T 1_0^T} [h_3, [h_2, [h_1, x]]] 
+ \frac{1}{2}\braced{3_0^T 2_0^3 1_0^2} [h_3, [[h_2, h_1], x]] 
\\ & \quad 
+\frac{1}{6}\braced{3_0^T 2_0^3 1_0^3}[[h_3, h_2], [h_1, x]]
\\ & \quad 
+\frac{1}{6}\braced{3_0^T 2_0^3 1_0^3}\left([[h_3, h_2], [h_1, x]] + [[h_2, [h_3, h_1]], x] + [[h_3, h_1], [h_2, x]] \right)

\\& = \;
\frac{1}{6}\braced{3_0^T 2_0^T 1_0^T} [h_3, [h_2, [h_1, x]]] 
+ \frac{1}{2}\braced{3_0^T 2_0^3 1_0^2} [h_3, [[h_2, h_1], x]] 
\\ & \quad 
+\frac{1}{6}\braced{3_0^T 2_0^3 1_0^3}[[h_3, h_2], [h_1, x]]
\\ & \quad 
+\frac{1}{6}\braced{3_0^T 2_0^3 1_0^3}\left([[h_3, h_2], [h_1, x]] + [[h_2, [h_3, h_1]], x] + [[h_3, h_2], [h_1, x]] \right)
\end{split}
}
or 
\eqnl{Z3-calc-3}{
\splitDo{
Z_3 & = \;
\frac{1}{6}\braced{3_0^T 2_0^T 1_0^T} [h_3, [h_2, [h_1, x]]] 
+ \frac{1}{2}\braced{3_0^T 2_0^3 1_0^2} [h_3, [[h_2, h_1], x]] 
\\ & \quad 
+\frac{1}{2}\braced{3_0^T 2_0^3 1_0^3} [[h_3, h_2], [h_1, x]]
+\frac{1}{6}\braced{3_0^T 2_0^3 1_0^3}[[h_2, [h_3, h_1]], x]
}
}

I transform the second term in \rf{Z3-calc-3}:
\eqn{
\splitDo{
\lefteqn{\frac{1}{2}\braced{3_0^T 2_0^3 1_0^2} [h_3, [[h_2, h_1], x]]}
\\ & \; = \;
\frac{1}{2}\braced{3_0^T 2_0^3 1_0^2} \left([[h_3, [h_2, h_1]], x] + [[h_2, h_1], [h_3, x]] \right)
\\ & \; = \;
\frac{1}{2}\braced{3_0^T 2_0^3 1_0^2} [[h_3, [h_2, h_1]], x]
+ \frac{1}{2}\braced{3_0^T 2_0^3 1_0^2} [[h_2, h_1], [h_3, x]]
\\ & \; = \;
\frac{1}{2}\braced{3_0^T 2_0^3 1_0^2} [[h_3, [h_2, h_1]], x] 
+ \frac{1}{2}\braced{1_0^T 3_0^1 2_0^3} [[h_3, h_2], [h_1, x]]
}}
after the rename $3\rightarrow 1\rightarrow 2 \rightarrow 3$ in the second term. Changing integration order in this term back to $3\,2\,1$ gives
\eqn{
\splitDo{
\lefteqn{\frac{1}{2}\braced{3_0^T 2_0^3 1_0^2} [h_3, [[h_2, h_1], x]]}
\\ & \; = \;
\frac{1}{2}\braced{3_0^T 2_0^3 1_0^2} [[h_3, [h_2, h_1]], x]
+ \frac{1}{2}\braced{3_0^T 1_3^T 2_0^3} [[h_3, h_2], [h_1, x]]
\\ & \; = \;
\frac{1}{2}\braced{3_0^T 2_0^3 1_0^2} [[h_3, [h_2, h_1]], x]
+ \frac{1}{2}\braced{3_0^T 2_0^3 1_3^T} [[h_3, h_2], [h_1, x]]
}}

Thus I have for \rf{Z3-calc-3}
\eqn{
\splitDo{
Z_3 
& = \;
\frac{1}{6}\braced{3_0^T 2_0^T 1_0^T} [h_3, [h_2, [h_1, x]]] 
\\ & \quad 
+ \frac{1}{2}\braced{3_0^T 2_0^3 1_0^2} [[h_3, [h_2, h_1]], x] 
+ \frac{1}{2}\braced{3_0^T 2_0^3 1_3^T} [[h_3, h_2], [h_1, x]]
\\ & \quad 
+ \frac{1}{2}\braced{3_0^T 2_0^3 1_0^3} [[h_3, h_2], [h_1, x]]
+ \frac{1}{6}\braced{3_0^T 2_0^3 1_0^3}[[h_2, [h_3, h_1]], x],
}}

\eqn{
\splitDo{
Z_3 
& = \;
\frac{1}{6}\braced{3_0^T 2_0^T 1_0^T} [h_3, [h_2, [h_1, x]]] 
\\ & \quad 
+ \frac{1}{2}\left(\braced{3_0^T 2_0^3 1_3^T} + \braced{3_0^T 2_0^3 1_0^3}\right) [[h_3, h_2], [h_1, x]]
\\ & \quad 
+ \frac{1}{2}\braced{3_0^T 2_0^3 1_0^2} [[h_3, [h_2, h_1]], x]
+ \frac{1}{6}\braced{3_0^T 2_0^3 1_0^3}[[h_2, [h_3, h_1]], x]
}}

or finally
\eqnl{Z-3-transformed}{
\splitDo{
Z_3 
& = \;
\frac{1}{6}\braced{3_0^T 2_0^T 1_0^T} [h_3, [h_2, [h_1, x]]] 
+ \frac{1}{2}\braced{3_0^T 2_0^3 1_0^T} [[h_3, h_2], [h_1, x]]
\\ & \quad 
+ \frac{1}{2}\braced{3_0^T 2_0^3 1_0^2} [[h_3, [h_2, h_1]], x]
+ \frac{1}{6}\braced{3_0^T 2_0^3 1_0^3}[[h_2, [h_3, h_1]], x]
}}

For practical calculations it is better to transform \rf{Z-3-transformed} to a form where it has smaller number of different commutators. I have for the last integration in \rf{Z-3-transformed} via renaming $2\leftrightarrow 3$ and integration order change:
\eqn{
\braced{3_0^T 2_0^3 1_0^3}[[h_2, [h_3, h_1]], x] 
=
\braced{2_0^T 3_0^2 1_0^2}[[h_3, [h_2, h_1]], x] 
=
\braced{3_0^T 2_3^T 1_0^2}[[h_3, [h_2, h_1]], x] 
}

Thus \rf{Z-3-transformed} becomes
\eqn{
\splitDo{
Z_3 
& = \;
\frac{1}{6}\braced{3_0^T 2_0^T 1_0^T} [h_3, [h_2, [h_1, x]]] 
+ \frac{1}{2}\braced{3_0^T 2_0^3 1_0^T} [[h_3, h_2], [h_1, x]]
\\ & \quad 
+ \frac{1}{2}\braced{3_0^T 2_0^3 1_0^2} [[h_3, [h_2, h_1]], x]
+ \frac{1}{6}\braced{3_0^T 2_3^T 1_0^2} [[h_3, [h_2, h_1]], x] 
\\
& = \;
\frac{1}{6}\braced{3_0^T 2_0^T 1_0^T} [h_3, [h_2, [h_1, x]]] 
+ \frac{1}{2}\braced{3_0^T 2_0^3 1_0^T} [[h_3, h_2], [h_1, x]]
\\ & \quad 
+ \left(\frac{1}{2}\braced{3_0^T 2_0^3 1_0^2} 
+ \frac{1}{6}\braced{3_0^T 2_3^T 1_0^2} \right) [[h_3, [h_2, h_1]], x].
}}

Regrouping the term in parenthesis gives:
\eqn{
\splitDo{
\frac{1}{2}\braced{3_0^T 2_0^3 1_0^2} 
+ \frac{1}{6}\braced{3_0^T 2_3^T 1_0^2}
& = \;
\frac{1}{3}\braced{3_0^T 2_0^3 1_0^2} 
+ \frac{1}{6}\left(\braced{3_0^T 2_0^3 1_0^2} 
+ \braced{3_0^T 2_3^T 1_0^2}\right)
\\ & = \;
\frac{1}{3}\braced{3_0^T 2_0^3 1_0^2} 
+ \frac{1}{6}\braced{3_0^T 2_0^T 1_0^2} 
}}

so I get for $Z_3$:
\eqnl{Z-3-transformed-2}{
\splitDo{
Z_3 
& = \;
\frac{1}{6}\braced{3_0^T 2_0^T 1_0^T} [h_3, [h_2, [h_1, x]]] 
+ \frac{1}{2}\braced{3_0^T 2_0^3 1_0^T} [[h_3, h_2], [h_1, x]]
\\ & \quad 
+ \frac{1}{3}\braced{3_0^T 2_0^3 1_0^2} [[h_3, [h_2, h_1]], x]
+ \frac{1}{6}\braced{3_0^T 2_0^T 1_0^2}  [[h_3, [h_2, h_1]], x].
}}

or with the standard notations:
\eqnl{Z-3-transformed-int}{
\splitDo{
Z_3 
& = \;
\frac{1}{6} \left[\int_{t_0}^{T} dt\,h(t), 
    \left[\int_{t_0}^{T} dt\,h(t), \left[\int_{t_0}^{T} dt\,h(t),
      \; x\right]\right]\right] 
\\ &
+ \;
\frac{1}{2}
\left[\int_{t_0}^{T} dt_3 \int_{t_0}^{t_3} dt_2, \left[h(t_3), h(t_2)\right], \left[\int_{t_0}^{T} dt\,h(t), \;x\right]\right]
\\ &  
+ \;
\frac{1}{3}\left[
\int_{t_0}^{T} dt_3 \int_{t_0}^{t_3} dt_2 \int_{t_0}^{t_2} dt_1\,
   [h(t_3), [h(t_2), h(t_1)]], \; x\right]
\\ &  
+ \;
\frac{1}{6}  
\left[\left[\int_{t_0}^{T} dt\,h(t), \; 
\int_{t_0}^{T} dt_2 \int_{t_0}^{t_2} dt_1,
\left[h(t_2), h(t_1)\right]\right], \; x\right].
}}




Lets deal with $Z_4$ in \rf{Z-definition} now:
\eqn{
Z_4 = \braced{4_0^T 3_0^4 2_0^3 1_0^2} [h_4, [h_3, [h_2, [h_1, x]]]] 
}

Applying \rf{Z-3-transformed} gives:
\eqnl{Z-4-calc-1}{
\splitDo{
Z_4
& = \;
\frac{1}{6}\braced{4_0^T 3_0^4 2_0^4 1_0^4} [h_4, [h_3, [h_2, [h_1, x]]]]
+ \frac{1}{2}\braced{4_0^T 3_0^4 2_0^3 1_0^4} [h_4, [[h_3, h_2], [h_1, x]]]
\\ & \quad 
+ \frac{1}{2}\braced{4_0^T 3_0^4 2_0^3 1_0^2} [h_4, [[h_3, [h_2, h_1]], x]]
+ \frac{1}{6}\braced{4_0^T 3_0^4 2_0^3 1_0^3} [h_4, [[h_2, [h_3, h_1]], x]]
}}

I focus on the first term in \rf{Z-4-calc-1} which I will write as $\frac{1}{24}Z_4^1$ with
\eqn{
Z_4^1 = 4\braced{4_0^T 3_0^4 2_0^4 1_0^4} [h_4, [h_3, [h_2, [h_1, x]]]]
}
where the factor $4$ is present to have less fraction in the following. So $Z_3$ becomes
\eqnl{Z-4-calc-2}{
\splitDo{
Z_4
& = \;
\frac{1}{24} Z_4^1
+ \frac{1}{2}\braced{4_0^T 3_0^4 2_0^3 1_0^4} [h_4, [[h_3, h_2], [h_1, x]]]
\\ & \quad 
+ \frac{1}{2}\braced{4_0^T 3_0^4 2_0^3 1_0^2} [h_4, [[h_3, [h_2, h_1]], x]]
+ \frac{1}{6}\braced{4_0^T 3_0^4 2_0^3 1_0^3} [h_4, [[h_2, [h_3, h_1]], x]]
}}

and I split $Z_4^1$ to for term via $4a = a + a + a + a$:
\alignDo{
Z_4^1  &\;  = \; \braced{4_0^T 3_0^4 2_0^4 1_0^4} [h_4, [h_3, [h_2, [h_1, x]]]]
\\&\; + \braced{4_0^T 3_0^4 2_0^4 1_0^4} [h_4, [h_3, [h_2, [h_1, x]]]]
\\&\; + \braced{4_0^T 3_0^4 2_0^4 1_0^4} [h_4, [h_3, [h_2, [h_1, x]]]]
\\&\; + \braced{4_0^T 3_0^4 2_0^4 1_0^4} [h_4, [h_3, [h_2, [h_1, x]]]]
}

Now I do the following renames: $4\leftrightarrow 3$ in the second term, $4\rightarrow 2 \rightarrow 3 \rightarrow 4$ in the third term and $4\rightarrow 1 \rightarrow 2 \rightarrow 3 \rightarrow 4$ in the forth. It gives

\eqnl{Z-4-1-calc-1}{\splitDo{
Z_4^1  &\;  = \; \braced{4_0^T 3_0^4 2_0^4 1_0^4} [h_4, [h_3, [h_2, [h_1, x]]]]
\\&\; + \braced{3_0^T 4_0^3 2_0^3 1_0^3} [h_3, [h_4, [h_2, [h_1, x]]]]
\\&\; + \braced{2_0^T 4_0^2 3_0^2 1_0^2} [h_2, [h_4, [h_3, [h_1, x]]]]
\\&\; + \braced{1_0^T 4_0^1 3_0^1 2_0^1} [h_1, [h_4, [h_3, [h_2, x]]]]
}}

Now I transform commutators in the second, third and forth terms to get $[h_4, [h_3, [h_2, [h_1, x]]]]$ plus terms where $x$ has nesting no more then $3$.
\eqn{
[h_3, [h_4, [h_2, [h_1, x]]]] = 
 [[h_3, h_4], [h_2, [h_1, x]]] + [h_3, [h_3, [h_2, [h_1, x]]]],
} 

\alignDo{
[h_2, [h_4, [h_3, [h_1, x]]]] 
& = 
[[h_2, h_4], [h_3, [h_1, x]]] + [h_4, [h_2, [h_3, [h_1, x]]]]
\\ & = 
[[h_2, h_4], [h_3, [h_1, x]]] 
\\ & \quad
+ [h_4, [[h_2, h_3], [h_1, x]]] + [h_4, [h_3, [h_2, [h_1, x]]]]
} 

\alignDo{
[h_1, [h_4, [h_3, [h_2, x]]]]
& = 
[[h_1, h_4], [h_3, [h_2, x]]] + [h_4, [h_1, [h_3, [h_2, x]]]]

\\& = 
[[h_1, h_4], [h_3, [h_2, x]]] + [h_4, [[h_1, h_3], [h_2, x]]] 
\\& \quad
+ [h_4, [h_3, [h_1, [h_2, x]]]]

\\& = 
[[h_1, h_4], [h_3, [h_2, x]]] + [h_4, [[h_1, h_3], [h_2, x]]] 
\\& \quad
+ [h_4, [h_3, [[h_1, h_2], x]]] + [h_4, [h_3, [h_2, [h_1, x]]]]
} 

Substituting these expressions back to \rf{Z-4-1-calc-1} and putting all $[h_4, [h_3, [h_2, [h_1, x]]]]$ together I have

\eqnl{Z-4-1-calc-2}{\splitDo{
Z_4^1  &\;  
= \; \left(\braced{4_0^T 3_0^4 2_0^4 1_0^4} + \braced{3_0^T 4_0^3 2_0^3 1_0^3} 
+ \braced{2_0^T 4_0^2 3_0^2 1_0^2} + \braced{1_0^T 4_0^1 3_0^1 2_0^1}
\right)
\\ & \qquad\qquad \times
[h_4, [h_3, [h_2, [h_1, x]]]]

\\&\; + \braced{3_0^T 4_0^3 2_0^3 1_0^3} [[h_3, h_4], [h_2, [h_1, x]]]

\\&\; + \braced{2_0^T 4_0^2 3_0^2 1_0^2} 
\left([[h_2, h_4], [h_3, [h_1, x]]] + [h_4, [[h_2, h_3], [h_1, x]]]\right)


\\&\; + \braced{1_0^T 4_0^1 3_0^1 2_0^1} 
\left([[h_1, h_4], [h_3, [h_2, x]]] + [h_4, [[h_1, h_3], [h_2, x]]]\right.
\\&\phantom{\; + \braced{1_0^T 4_0^1 3_0^1 2_0^1} \left(\right.}
+\left.[h_4, [h_3, [[h_1, h_2], x]]]\right)

}}

The result \rf{cicle-int-sum-4} gives:
\eqn{\splitDo{
Z_4^1  &\;  
= \; \braced{4_0^T 3_0^T 2_0^T 1_0^T} [h_4, [h_3, [h_2, [h_1, x]]]]

\\&\; + \braced{3_0^T 4_0^3 2_0^3 1_0^3} [[h_3, h_4], [h_2, [h_1, x]]]

\\&\; + \braced{2_0^T 4_0^2 3_0^2 1_0^2} 
\left([[h_2, h_4], [h_3, [h_1, x]]] + [h_4, [[h_2, h_3], [h_1, x]]]\right)


\\&\; + \braced{1_0^T 4_0^1 3_0^1 2_0^1} 
\left([[h_1, h_4], [h_3, [h_2, x]]] + [h_4, [[h_1, h_3], [h_2, x]]]\right.
\\&\phantom{\; + \braced{1_0^T 4_0^1 3_0^1 2_0^1} \left(\right.}
+\left.[h_4, [h_3, [[h_1, h_2], x]]]\right)

}}

To get the original intergration order I do reverse renames $4\leftrightarrow 3$ in the second term, $4\leftarrow 2 \leftarrow 3 \leftarrow 4$ in the third term and $4\leftarrow 1 \leftarrow 2 \leftarrow 3 \leftarrow 4$ in the forth. It gives
\eqn{\splitDo{
Z_4^1  &\;  
= \; \braced{4_0^T 3_0^T 2_0^T 1_0^T} [h_4, [h_3, [h_2, [h_1, x]]]]

\\&\; + \braced{4_0^T 3_0^4 2_0^4 1_0^4} [[h_4, h_3], [h_2, [h_1, x]]]

\\&\; + \braced{4_0^T 3_0^4 2_0^4 1_0^4} 
\left([[h_4, h_3], [h_2, [h_1, x]]] + [h_3, [[h_4, h_2], [h_1, x]]]\right)


\\&\; + \braced{4_0^T 3_0^4 2_0^4 1_0^4} 
\left([[h_4, h_3], [h_2, [h_1, x]]] + [h_3, [[h_4, h_2], [h_1, x]]]\right.
\\&\phantom{\; + \braced{4_0^T 3_0^4 2_0^4 1_0^4} \left(\right.}
+\left.[h_3, [h_2, [[h_4, h_1], x]]]\right)

}}

or
\eqnl{Z-4-1-calc-4}{\splitDo{
Z_4^1  &\;  
= \; \braced{4_0^T 3_0^T 2_0^T 1_0^T} [h_4, [h_3, [h_2, [h_1, x]]]]

\\&\; + 3 \braced{4_0^T 3_0^4 2_0^4 1_0^4} [[h_4, h_3], [h_2, [h_1, x]]]

\\&\; + 2 \braced{4_0^T 3_0^4 2_0^4 1_0^4} [h_3, [[h_4, h_2], [h_1, x]]]


\\&\; + \braced{4_0^T 3_0^4 2_0^4 1_0^4} [h_3, [h_2, [[h_4, h_1], x]]]

}}

I transform the third and forth terms here to get the same expression as the second term plus terms with $x$ nesting level no more then $2$.
\eqnl{Z-4-1-calc-4-a}{\splitDo{
\lefteqn{\braced{4_0^T 3_0^4 2_0^4 1_0^4} [h_3, [[h_4, h_2], [h_1, x]]]}
\quad
\\ & = 
\braced{4_0^T 3_0^4 2_0^4 1_0^4}[[h_3, [h_4, h_2]], [h_1, x]] 
+ \braced{4_0^T 3_0^4 2_0^4 1_0^4} [[h_4, h_2], [h_3, [h_1, x]]]
\\ & = 
\braced{4_0^T 3_0^4 2_0^4 1_0^4}[[h_3, [h_4, h_2]], [h_1, x]] 
+ \braced{4_0^T 3_0^4 2_0^4 1_0^4} [[h_4, h_3], [h_2, [h_1, x]]]
}}

where to get the last line I used the fact in the $\braced{4_0^T 3_0^4 2_0^4 1_0^4} f(t_3, t_2, t_1)$ the integrations over $t_3$, $t_2$, $t_1$ have the same limits so by changing the integration order and renaming in the resulting expresssion $t_3$, $t_2$, $t_1$ to get the original integration order $\braced{4_0^T 3_0^4 2_0^4 1_0^4}$ I can replace indexes $3$, $2$, $1$ in $f(t_3, t_2, t_1)$ by their arbitrary transposition.

\p
Similary for the forth term in \rf{Z-4-1-calc-4} I have:
\eqn{\splitDo{
\lefteqn{[h_3, [h_2, [[h_4, h_1], x]]]}
\quad
\\& = 
[h_3, [[h_2, [h_4, h_1]], x]] + [h_3, [[h_4, h_1], [h_2, x]]]

\\& = [h_3, [[h_2, [h_4, h_1]], x]] 
+ [[h_3, [h_4, h_1]], [h_2, x]] + [[h_4, h_1], [h_3, [h_2, x]]]

\\& = [[h_3, [h_2, [h_4, h_1]]], x] + [[h_2, [h_4, h_1]], [h_3, x]]
\\& \phantom{=} 
+ [[h_3, [h_4, h_1]], [h_2, x]] + [[h_4, h_1], [h_3, [h_2, x]]],
}}

and by using integration symmetry in $\braced{4_0^T 3_0^4 2_0^4 1_0^4}$ over $3$, $2$, $1$ I have: 

\eqnl{Z-4-1-calc-4-b}{\splitDo{
\lefteqn{\braced{4_0^T 3_0^4 2_0^4 1_0^4} [h_3, [h_2, [[h_4, h_1], x]]]}
\quad
\\& = \braced{4_0^T 3_0^4 2_0^4 1_0^4} [[h_3, [h_2, [h_4, h_1]]], x] 
+ \braced{4_0^T 3_0^4 2_0^4 1_0^4} [[h_2, [h_4, h_1]], [h_3, x]]
\\& \phantom{=} 
+ \braced{4_0^T 3_0^4 2_0^4 1_0^4} [[h_3, [h_4, h_1]], [h_2, x]] 
+ \braced{4_0^T 3_0^4 2_0^4 1_0^4} [[h_4, h_1], [h_3, [h_2, x]]],

\\& = \braced{4_0^T 3_0^4 2_0^4 1_0^4} [[h_3, [h_2, [h_4, h_1]]], x] 
+ \braced{4_0^T 3_0^4 2_0^4 1_0^4} [[h_3, [h_4, h_2]], [h_1, x]]
\\& \phantom{=} 
+ \braced{4_0^T 3_0^4 2_0^4 1_0^4} [[h_3, [h_4, h_2]], [h_1, x]] 
+ \braced{4_0^T 3_0^4 2_0^4 1_0^4} [[h_4, h_3], [h_2, [h_1, x]]].

\\& = \braced{4_0^T 3_0^4 2_0^4 1_0^4} [[h_3, [h_2, [h_4, h_1]]], x] 
+ 2 \braced{4_0^T 3_0^4 2_0^4 1_0^4} [[h_3, [h_4, h_2]], [h_1, x]]
\\& \phantom{=} 
+ \braced{4_0^T 3_0^4 2_0^4 1_0^4} [[h_4, h_3], [h_2, [h_1, x]]].
}}

Substituting \rf{Z-4-1-calc-4-a} and \rf{Z-4-1-calc-4-b} into \rf{Z-4-1-calc-4} gives

\eqn{\splitDo{
Z_4^1  &\;  
= \; \braced{4_0^T 3_0^T 2_0^T 1_0^T} [h_4, [h_3, [h_2, [h_1, x]]]]

\\&\; + 3 \braced{4_0^T 3_0^4 2_0^4 1_0^4} [[h_4, h_3], [h_2, [h_1, x]]]

\\&\; 
+ 2 \braced{4_0^T 3_0^4 2_0^4 1_0^4}[[h_3, [h_4, h_2]], [h_1, x]] 
+ 2 \braced{4_0^T 3_0^4 2_0^4 1_0^4} [[h_4, h_3], [h_2, [h_1, x]]]

\\&\; + \braced{4_0^T 3_0^4 2_0^4 1_0^4} [[h_3, [h_2, [h_4, h_1]]], x] 
+ 2 \braced{4_0^T 3_0^4 2_0^4 1_0^4} [[h_3, [h_4, h_2]], [h_1, x]]
\\& \phantom{\; +} 
+ \braced{4_0^T 3_0^4 2_0^4 1_0^4} [[h_4, h_3], [h_2, [h_1, x]]]
}}

or
\eqn{\splitDo{
Z_4^1  &\;  
= \; \braced{4_0^T 3_0^T 2_0^T 1_0^T} [h_4, [h_3, [h_2, [h_1, x]]]]

+ 6 \braced{4_0^T 3_0^4 2_0^4 1_0^4} [[h_4, h_3], [h_2, [h_1, x]]]

\\&\; 
+ 4 \braced{4_0^T 3_0^4 2_0^4 1_0^4}[[h_3, [h_4, h_2]], [h_1, x]] 

+ \braced{4_0^T 3_0^4 2_0^4 1_0^4} [[h_3, [h_2, [h_4, h_1]]], x] 
}}

\p
Substituting the last result into \rf{Z-4-calc-2} leads to
\eqn{
\splitDo{
Z_4 \;
& = \;
\frac{1}{24}\braced{4_0^T 3_0^T 2_0^T 1_0^T} [h_4, [h_3, [h_2, [h_1, x]]]]
\\&
+ \frac{1}{4} \braced{4_0^T 3_0^4 2_0^4 1_0^4} [[h_4, h_3], [h_2, [h_1, x]]]
\\&
+ \frac{1}{6} \braced{4_0^T 3_0^4 2_0^4 1_0^4} [[h_3, [h_4, h_2]], [h_1, x]] 
+ \frac{1}{24}\braced{4_0^T 3_0^4 2_0^4 1_0^4} [[h_3, [h_2, [h_4, h_1]]], x] 

\\&
+ \frac{1}{2}\braced{4_0^T 3_0^4 2_0^3 1_0^4} [h_4, [[h_3, h_2], [h_1, x]]]
\\ & 
+ \frac{1}{2}\braced{4_0^T 3_0^4 2_0^3 1_0^2} [h_4, [[h_3, [h_2, h_1]], x]]
+ \frac{1}{6}\braced{4_0^T 3_0^4 2_0^3 1_0^3} [h_4, [[h_2, [h_3, h_1]], x]]
}}

or 
\eqnl{Z-4-calc-3}{
\splitDo{
Z_4 \;
& = \;
\frac{1}{24}\braced{4_0^T 3_0^T 2_0^T 1_0^T} [h_4, [h_3, [h_2, [h_1, x]]]]
\\&
+ \frac{1}{4} \braced{4_0^T 3_0^4 2_0^4 1_0^4} [[h_4, h_3], [h_2, [h_1, x]]]
+ \frac{1}{2}\braced{4_0^T 3_0^4 2_0^3 1_0^4} [h_4, [[h_3, h_2], [h_1, x]]]
\\&
+ \frac{1}{6} \braced{4_0^T 3_0^4 2_0^4 1_0^4} [[h_3, [h_4, h_2]], [h_1, x]] 
+ \frac{1}{2}\braced{4_0^T 3_0^4 2_0^3 1_0^2} [h_4, [[h_3, [h_2, h_1]], x]]
\\ & 
+ \frac{1}{6}\braced{4_0^T 3_0^4 2_0^3 1_0^3} [h_4, [[h_2, [h_3, h_1]], x]]
+ \frac{1}{24}\braced{4_0^T 3_0^4 2_0^4 1_0^4} [[h_3, [h_2, [h_4, h_1]]], x] 
}}

Now I consider the second and the third term, the only terms with $x$ nesting level equals 3:
\eqn{
Z_4^2 \equiv
\frac{1}{4} \braced{4_0^T 3_0^4 2_0^4 1_0^4} [[h_4, h_3], [h_2, [h_1, x]]]
+ \frac{1}{2}\braced{4_0^T 3_0^4 2_0^3 1_0^4} [h_4, [[h_3, h_2], [h_1, x]]]
}

Applying \rf{comm-like-diff} to the second term gives
\eqnl{Z-4-2-calc-1}{
\splitDo{
Z_4^2 =&
\frac{1}{4} \braced{4_0^T 3_0^4 2_0^4 1_0^4} [[h_4, h_3], [h_2, [h_1, x]]]
+ \frac{1}{2}\braced{4_0^T 3_0^4 2_0^3 1_0^4} [[h_4, [h_3, h_2]], [h_1, x]]
\\&
+ \frac{1}{2}\braced{4_0^T 3_0^4 2_0^3 1_0^4} [[h_3, h_2], [h_4, [h_1, x]]]
}}

By renaming $3\rightarrow 4\rightarrow 2\rightarrow 3$ and changing integration order I have for the last term in \rf{Z-4-2-calc-1}
\eqn{
\splitDo{
\lefteqn{\braced{4_0^T 3_0^4 2_0^3 1_0^4} [[h_3, h_2], [h_4, [h_1, x]]] 
= \braced{2_0^T 4_0^2 3_0^4 1_0^2} [[h_4, h_3], [h_2, [h_1, x]]]}
&\\&= 
\braced{4_0^T 2_4^T 3_0^4 1_0^2} [[h_4, h_3], [h_2, [h_1, x]]]
= \braced{4_0^T 3_0^4 2_4^T 1_0^2} [[h_4, h_3], [h_2, [h_1, x]]]
}}

Split $a\rightarrow\frac{1}{2}a+\frac{1}{2}a$ and rename $1\leftrightarrow 2$ give:
\eqn{
\splitDo{
\lefteqn{\braced{4_0^T 3_0^4 2_0^3 1_0^4} [[h_3, h_2], [h_4, [h_1, x]]]}
&\\& = 
\frac{1}{2}\braced{4_0^T 3_0^4 2_4^T 1_0^2} [[h_4, h_3], [h_2, [h_1, x]]]
+ \frac{1}{2}\braced{4_0^T 3_0^4 2_4^T 1_0^2} [[h_4, h_3], [h_2, [h_1, x]]]
\\& = 
\frac{1}{2}\braced{4_0^T 3_0^4 2_4^T 1_0^2} [[h_4, h_3], [h_2, [h_1, x]]]
+ \frac{1}{2}\braced{4_0^T 3_0^4 1_4^T 2_0^1} [[h_4, h_3], [h_1, [h_2, x]]]
}}

Now I apply \rf{comm-like-diff} to $[h_1, [h_2, x]]$ and then rename $1\leftrightarrow 2$ back in the second term:
\eqn{
\splitDo{
\lefteqn{\braced{4_0^T 3_0^4 2_0^3 1_0^4} [[h_3, h_2], [h_4, [h_1, x]]]}
&\\& = 
\frac{1}{2}\braced{4_0^T 3_0^4 2_4^T 1_0^2} [[h_4, h_3], [h_2, [h_1, x]]]
+ \frac{1}{2}\braced{4_0^T 3_0^4 2_4^T 1_0^2} [[h_4, h_3], [h_2, [h_1, x]]]
\\& = 
\frac{1}{2}\braced{4_0^T 3_0^4 2_4^T 1_0^2} [[h_4, h_3], [h_2, [h_1, x]]]
+ \frac{1}{2}\braced{4_0^T 3_0^4 1_4^T 2_0^1} [[h_4, h_3], [[h_1, h_2], x]]
\\&
+ \frac{1}{2}\braced{4_0^T 3_0^4 1_4^T 2_0^1} [[h_4, h_3], [h_2, [h_1, x]]]
\\& = 
\frac{1}{2}\braced{4_0^T 3_0^4 2_4^T 1_0^2} [[h_4, h_3], [h_2, [h_1, x]]]
+ \frac{1}{2}\braced{4_0^T 3_0^4 1_4^T 2_0^1} [[h_4, h_3], [h_2, [h_1, x]]]
\\&
+ \frac{1}{2}\braced{4_0^T 3_0^4 2_4^T 1_0^2} [[h_4, h_3], [[h_2, h_1], x]]
}}

Putting the last expression into \rf{Z-4-2-calc-1} I have:
\eqn{
\splitDo{
Z_4^2 &
=\frac{1}{4} \braced{4_0^T 3_0^4 2_0^4 1_0^4} [[h_4, h_3], [h_2, [h_1, x]]]
+ \frac{1}{2}\braced{4_0^T 3_0^4 2_0^3 1_0^4} [[h_4, [h_3, h_2]], [h_1, x]]
\\&
+\frac{1}{4}\braced{4_0^T 3_0^4 2_4^T 1_0^2} [[h_4, h_3], [h_2, [h_1, x]]]
\\&
+ \frac{1}{4}\braced{4_0^T 3_0^4 1_4^T 2_0^1} [[h_4, h_3], [h_2, [h_1, x]]]
+ \frac{1}{4}\braced{4_0^T 3_0^4 2_4^T 1_0^2} [[h_4, h_3], [[h_2, h_1], x]]
}}

or
\eqnl{Z-4-2-calc-2}{
\splitDo{
Z_4^2 &
=\frac{1}{4} \left(
\braced{4_0^T 3_0^4 2_0^4 1_0^4} 
+ \braced{4_0^T 3_0^4 2_4^T 1_0^2}
+ \braced{4_0^T 3_0^4 1_4^T 2_0^1}
\right) [[h_4, h_3], [h_2, [h_1, x]]]
\\&
+ \frac{1}{2}\braced{4_0^T 3_0^4 2_0^3 1_0^4} [[h_4, [h_3, h_2]], [h_1, x]]
+ \frac{1}{4}\braced{4_0^T 3_0^4 2_4^T 1_0^2} [[h_4, h_3], [[h_2, h_1], x]]
}}

Now by using \rf{int-order-change-2} and fusing integration I have for the integrations in the first term in \rf{Z-4-2-calc-2}:
\eqn{
\splitDo{
\lefteqn{
\braced{4_0^T 3_0^4 2_0^4 1_0^4} 
+ \braced{4_0^T 3_0^4 2_4^T 1_0^2}
+ \braced{4_0^T 3_0^4 1_4^T 2_0^1}
}
\\&=
\braced{4_0^T 3_0^4 2_0^4 1_0^4} 
+ \braced{4_0^T 3_0^4 2_4^T 1_0^2}
+ \braced{4_0^T 3_0^4 2_0^4 1_4^T}
+ \braced{4_0^T 3_0^4 2_4^T 1_2^T}
\\&=
\braced{4_0^T 3_0^4 2_0^4 1_0^4} 
+ \braced{4_0^T 3_0^4 2_0^4 1_4^T}
+ \braced{4_0^T 3_0^4 2_4^T 1_0^2}
+ \braced{4_0^T 3_0^4 2_4^T 1_2^T}
\\&=
\braced{4_0^T 3_0^4 2_0^4 1_0^T} 
+ \braced{4_0^T 3_0^4 2_4^T 1_0^T}
= \braced{4_0^T 3_0^4 2_0^T 1_0^T} 
}}

Thus for $Z_4^2$ I have
\eqnl{Z-4-2-calc-3}{
\splitDo{
Z_4^2 &
=\frac{1}{4} 
\braced{4_0^T 3_0^4 2_0^T 1_0^T} [[h_4, h_3], [h_2, [h_1, x]]]
\\&
+ \frac{1}{2}\braced{4_0^T 3_0^4 2_0^3 1_0^4} [[h_4, [h_3, h_2]], [h_1, x]]
+ \frac{1}{4}\braced{4_0^T 3_0^4 2_4^T 1_0^2} [[h_4, h_3], [[h_2, h_1], x]]
}}

Substitution of \rf{Z-4-2-calc-3} into \rf{Z-4-calc-3} to replace the second and the third terms gives:
\eqnl{Z-4-calc-4}{
\splitDo{
Z_4 \;
& = \;
\frac{1}{24}\braced{4_0^T 3_0^T 2_0^T 1_0^T} [h_4, [h_3, [h_2, [h_1, x]]]]
\\&
+ \frac{1}{4} \braced{4_0^T 3_0^4 2_0^T 1_0^T} [[h_4, h_3], [h_2, [h_1, x]]]
\\&
+ \frac{1}{2}\braced{4_0^T 3_0^4 2_0^3 1_0^4} [[h_4, [h_3, h_2]], [h_1, x]]
+ \frac{1}{4}\braced{4_0^T 3_0^4 2_4^T 1_0^2} [[h_4, h_3], [[h_2, h_1], x]]
\\&
+ \frac{1}{6} \braced{4_0^T 3_0^4 2_0^4 1_0^4} [[h_3, [h_4, h_2]], [h_1, x]] 
+ \frac{1}{2}\braced{4_0^T 3_0^4 2_0^3 1_0^2} [h_4, [[h_3, [h_2, h_1]], x]]
\\ & 
+ \frac{1}{6}\braced{4_0^T 3_0^4 2_0^3 1_0^3} [h_4, [[h_2, [h_3, h_1]], x]]
+ \frac{1}{24}\braced{4_0^T 3_0^4 2_0^4 1_0^4} [[h_3, [h_2, [h_4, h_1]]], x]. 
}}

Introducing $Z_4^3$ via
\eqnl{Z-4-3-def}{
\splitDo{
Z_4^3 \;
& = \;
\frac{1}{2}\left(\braced{4_0^T 3_0^4 2_0^3 1_0^4} [[h_4, [h_3, h_2]], [h_1, x]]
+ \braced{4_0^T 3_0^4 2_0^3 1_0^2} [h_4, [[h_3, [h_2, h_1]], x]]\right)
\\&
+ \frac{1}{6}\left( \braced{4_0^T 3_0^4 2_0^4 1_0^4} [[h_3, [h_4, h_2]], [h_1, x]] 
+ \braced{4_0^T 3_0^4 2_0^3 1_0^3} [h_4, [[h_2, [h_3, h_1]], x]]\right)
}}

I have for \rf{Z-4-calc-4}
\eqnl{Z-4-calc-5}{
\splitDo{
Z_4 \;
& = \;
\frac{1}{24}\braced{4_0^T 3_0^T 2_0^T 1_0^T} [h_4, [h_3, [h_2, [h_1, x]]]]
\\&
+ \frac{1}{4} \braced{4_0^T 3_0^4 2_0^T 1_0^T} [[h_4, h_3], [h_2, [h_1, x]]]
\\&
+ \frac{1}{4}\braced{4_0^T 3_0^4 2_4^T 1_0^2} [[h_4, h_3], [[h_2, h_1], x]]
\\&
+ Z_4^3 
+ \frac{1}{24}\braced{4_0^T 3_0^4 2_0^4 1_0^4} [[h_3, [h_2, [h_4, h_1]]], x]. 
}}

Splitting the third term here, renaming $1\leftrightarrow 3$, $2\leftrightarrow 4$ and using \rf{comm-like-diff} I get:
\eqn{
\splitDo{
\lefteqn{\frac{1}{4}\braced{4_0^T 3_0^4 2_4^T 1_0^2} 
[[h_4, h_3], [[h_2, h_1], x]]} \quad&

\\& = 
\frac{1}{8}\braced{4_0^T 3_0^4 2_4^T 1_0^2}
[[h_4, h_3], [[h_2, h_1], x]]
+ \frac{1}{8}\braced{4_0^T 3_0^4 2_4^T 1_0^2}
[[h_4, h_3], [[h_2, h_1], x]]

\\& = 
\frac{1}{8}\braced{4_0^T 3_0^4 2_4^T 1_0^2}
[[h_4, h_3], [[h_2, h_1], x]]
+ \frac{1}{8}\braced{2_0^T 1_0^2 4_2^T 3_0^4}
[[h_2, h_1], [[h_4, h_3], x]]

\\& = 
\frac{1}{8}\braced{4_0^T 3_0^4 2_4^T 1_0^2}
[[h_4, h_3], [[h_2, h_1], x]]
+ \frac{1}{8}\braced{2_0^T 1_0^2 4_2^T 3_0^4}
[[[h_2, h_1], [h_4, h_3]], x]
\\&\quad
+ \frac{1}{8}\braced{2_0^T 1_0^2 4_2^T 3_0^4}
[[h_4, h_3], [[h_2, h_1], x]]

\\& = 
\frac{1}{8}\left(
\braced{4_0^T 3_0^4 2_4^T 1_0^2} + \braced{2_0^T 1_0^2 4_2^T 3_0^4}
\right)
[[h_4, h_3], [[h_2, h_1], x]]
\\&\quad
+ \frac{1}{8}\braced{2_0^T 1_0^2 4_2^T 3_0^4}
[[[h_2, h_1], [h_4, h_3]], x]
}}

By changing the integration order and integral fusing I have for the first term
\eqn{
\splitDo{
\lefteqn{\braced{4_0^T 3_0^4 2_4^T 1_0^2} + \braced{2_0^T 1_0^2 4_2^T 3_0^4}}
\\&
= 
\braced{4_0^T 3_0^4 2_4^T 1_0^2} + \braced{2_0^T 4_2^T 3_0^4 1_0^2}
= 
\braced{4_0^T 3_0^4 2_4^T 1_0^2} + \braced{4_0^T 2_0^4 3_0^4 1_0^2}
\\&
= 
\braced{4_0^T 3_0^4 2_4^T 1_0^2} + \braced{4_0^T 3_0^4 2_0^4 1_0^2}
= 
\braced{4_0^T 3_0^4 2_0^T 1_0^2}
}}

So
\eqn{
\splitDo{
\lefteqn{\frac{1}{4}\braced{4_0^T 3_0^4 2_4^T 1_0^2} 
[[h_4, h_3], [[h_2, h_1], x]]} \quad&

\\& = 
\frac{1}{8}\braced{4_0^T 3_0^4 2_0^T 1_0^2}[[h_4, h_3], [[h_2, h_1], x]]
+ \frac{1}{8}\braced{2_0^T 1_0^2 4_2^T 3_0^4}
[[[h_2, h_1], [h_4, h_3]], x]

\\& = 
\frac{1}{8}\braced{4_0^T 3_0^4 2_0^T 1_0^2}[[h_4, h_3], [[h_2, h_1], x]]
+ \frac{1}{8}\braced{4_0^T 3_0^4 2_4^T 1_0^2}
[[[h_4, h_3], [h_2, h_1]], x]
}}
where in the last term I renamed back $1\leftrightarrow 3$, $2\leftrightarrow 4$.

\p

Substituting this into \rf{Z-4-calc-6} for the third term gives

\eqn{
\splitDo{
Z_4 \;
& = \;
\frac{1}{24}\braced{4_0^T 3_0^T 2_0^T 1_0^T} [h_4, [h_3, [h_2, [h_1, x]]]]
\\&
+ \frac{1}{4} \braced{4_0^T 3_0^4 2_0^T 1_0^T} [[h_4, h_3], [h_2, [h_1, x]]]
\\&
+\frac{1}{8}\braced{4_0^T 3_0^4 2_0^T 1_0^2}[[h_4, h_3], [[h_2, h_1], x]]
+ \frac{1}{8}\braced{4_0^T 3_0^4 2_4^T 1_0^2}
[[[h_4, h_3], [h_2, h_1]], x]
\\&
+ Z_4^3 
+ \frac{1}{24}\braced{4_0^T 3_0^4 2_0^4 1_0^4} [[h_3, [h_2, [h_4, h_1]]], x]. 
}}
or

\eqnl{Z-4-calc-6}{
\splitDo{
Z_4 \;
& = \;
\frac{1}{24}\braced{4_0^T 3_0^T 2_0^T 1_0^T} [h_4, [h_3, [h_2, [h_1, x]]]]
\\&
+ \frac{1}{4} \braced{4_0^T 3_0^4 2_0^T 1_0^T} [[h_4, h_3], [h_2, [h_1, x]]]
+ \frac{1}{8}\braced{4_0^T 3_0^4 2_0^T 1_0^2}[[h_4, h_3], [[h_2, h_1], x]]
\\&
+ Z_4^3 
\\&
+ \frac{1}{8}\braced{4_0^T 3_0^4 2_4^T 1_0^2}
[[[h_4, h_3], [h_2, h_1]], x]
+ \frac{1}{24}\braced{4_0^T 3_0^4 2_0^4 1_0^4} [[h_3, [h_2, [h_4, h_1]]], x]
}}

Now comes the last part which is to deal with $Z_4^3$ defined in \rf{Z-4-3-def}:
\eqn{
\splitDo{
Z_4^3 \;
& = \;
\frac{1}{2}\left(\braced{4_0^T 3_0^4 2_0^3 1_0^4} [[h_4, [h_3, h_2]], [h_1, x]]
+ \braced{4_0^T 3_0^4 2_0^3 1_0^2} [h_4, [[h_3, [h_2, h_1]], x]]\right)
\\&
+ \frac{1}{6}\left( \braced{4_0^T 3_0^4 2_0^4 1_0^4} [[h_3, [h_4, h_2]], [h_1, x]] 
+ \braced{4_0^T 3_0^4 2_0^3 1_0^3} [h_4, [[h_2, [h_3, h_1]], x]]\right)
}}

Application of \rf{comm-like-diff} to the second and forth terms gives
\eqn{
\splitDo{
Z_4^3 \;
& =
\frac{1}{2}\braced{4_0^T 3_0^4 2_0^3 1_0^4} [[h_4, [h_3, h_2]], [h_1, x]]
\\&+
\frac{1}{2}\left(\braced{4_0^T 3_0^4 2_0^3 1_0^2}
[[h_4, [h_3, [h_2, h_1]]], x] + [[h_3, [h_2, h_1]], [h_4, x]]
\right)
\\&
+ \frac{1}{6}\braced{4_0^T 3_0^4 2_0^4 1_0^4} [[h_3, [h_4, h_2]], [h_1, x]] 
\\&
+ \frac{1}{6}\left( \braced{4_0^T 3_0^4 2_0^3 1_0^3} 
[[h_4, [h_2, [h_3, h_1]]], x] + [[h_2, [h_3, h_1]], [h_4, x]]
\right)
}}

or

\eqn{
\splitDo{
Z_4^3 \;
& =
\frac{1}{2}\left(
\braced{4_0^T 3_0^4 2_0^3 1_0^4} [[h_4, [h_3, h_2]], [h_1, x]]
+
\braced{4_0^T 3_0^4 2_0^3 1_0^2} [[h_3, [h_2, h_1]], [h_4, x]]
\right)
\\&
+ \frac{1}{6}\left(
\braced{4_0^T 3_0^4 2_0^4 1_0^4} [[h_3, [h_4, h_2]], [h_1, x]] 
+ 
\braced{4_0^T 3_0^4 2_0^3 1_0^3} [[h_2, [h_3, h_1]], [h_4, x]] 
\right)
\\&
+\frac{1}{2}\braced{4_0^T 3_0^4 2_0^3 1_0^2} [[h_4, [h_3, [h_2, h_1]]], x] 
+ \frac{1}{6}\braced{4_0^T 3_0^4 2_0^3 1_0^3} [[h_4, [h_2, [h_3, h_1]]], x]
}}

Renames $3\rightarrow 4\rightarrow 1\rightarrow 2\rightarrow 3$ in the second and forth terms lead to
\eqn{
\splitDo{
Z_4^3 \;
& =
\frac{1}{2}\left(
\braced{4_0^T 3_0^4 2_0^3 1_0^4} [[h_4, [h_3, h_2]], [h_1, x]]
+
\braced{1_0^T 4_0^1 3_0^4 2_0^3} [[h_4, [h_3, h_2]], [h_1, x]]
\right)
\\&
+ \frac{1}{6}\left(
\braced{4_0^T 3_0^4 2_0^4 1_0^4} [[h_3, [h_4, h_2]], [h_1, x]] 
+ 
\braced{1_0^T 4_0^1 3_0^4 2_0^4} [[h_3, [h_4, h_2]], [h_1, x]] 
\right)
\\&
+\frac{1}{2}\braced{4_0^T 3_0^4 2_0^3 1_0^2} [[h_4, [h_3, [h_2, h_1]]], x] 
+ \frac{1}{6}\braced{4_0^T 3_0^4 2_0^3 1_0^3} [[h_4, [h_2, [h_3, h_1]]], x]
}}

or
\eqnl{Z-4-3-calc-2}{
\splitDo{
Z_4^3 \;
& =
\frac{1}{2}\left(
\braced{4_0^T 3_0^4 2_0^3 1_0^4} + \braced{1_0^T 4_0^1 3_0^4 2_0^3} 
\right) [[h_4, [h_3, h_2]], [h_1, x]]
\\&
+ \frac{1}{6}\left(
\braced{4_0^T 3_0^4 2_0^4 1_0^4} + \braced{1_0^T 4_0^1 3_0^4 2_0^4} 
\right) [[h_3, [h_4, h_2]], [h_1, x]] 
\\&
+\frac{1}{2}\braced{4_0^T 3_0^4 2_0^3 1_0^2} [[h_4, [h_3, [h_2, h_1]]], x] 
+ \frac{1}{6}\braced{4_0^T 3_0^4 2_0^3 1_0^3} [[h_4, [h_2, [h_3, h_1]]], x]
}}

Changing integration ordrer and fusion gives for the first term:
\eqn{
\splitDo{
\lefteqn{\braced{4_0^T 3_0^4 2_0^3 1_0^4} + \braced{1_0^T 4_0^1 3_0^4 2_0^3}}
\\& \quad
=
\braced{4_0^T 3_0^4 2_0^3 1_0^4} + \braced{4_0^T 1_4^T  3_0^4 2_0^3} 
=
\braced{4_0^T 3_0^4 2_0^3 1_0^4} + \braced{4_0^T 3_0^4 2_0^3 1_4^T } 
=
\braced{4_0^T 3_0^4 2_0^3 1_0^T}
}}

and similarly for the second term
\eqn{
\splitDo{
\lefteqn{\braced{4_0^T 3_0^4 2_0^4 1_0^4} + \braced{1_0^T 4_0^1 3_0^4 2_0^4}}
\\& \quad
=
\braced{4_0^T 3_0^4 2_0^4 1_0^4} + \braced{4_0^T 1_4^T 3_0^4 2_0^4}
=
\braced{4_0^T 3_0^4 2_0^4 1_0^4} + \braced{4_0^T 3_0^4 2_0^4 1_4^T}
=
\braced{4_0^T 3_0^4 2_0^4 1_0^T} 
}}

Thus I have for $Z_4^3$ from \rf{Z-4-3-calc-2}:
\eqnl{Z-4-3-calc-3}{
\splitDo{
Z_4^3 \;
& =
\frac{1}{2} \braced{4_0^T 3_0^4 2_0^3 1_0^T} [[h_4, [h_3, h_2]], [h_1, x]]
\\&
+ \frac{1}{6}\braced{4_0^T 3_0^4 2_0^4 1_0^T} [[h_3, [h_4, h_2]], [h_1, x]] 
\\&
+\frac{1}{2}\braced{4_0^T 3_0^4 2_0^3 1_0^2} [[h_4, [h_3, [h_2, h_1]]], x] 
+ \frac{1}{6}\braced{4_0^T 3_0^4 2_0^3 1_0^3} [[h_4, [h_2, [h_3, h_1]]], x]
}}

Substitung this result into \rf{Z-4-calc-6} gives for $Z_4$:
\eqnl{Z-4-calc-7}{
\splitDo{
Z_4 \;
& = \;
\frac{1}{24}\braced{4_0^T 3_0^T 2_0^T 1_0^T} [h_4, [h_3, [h_2, [h_1, x]]]]
\\&
+ \frac{1}{4} \braced{4_0^T 3_0^4 2_0^T 1_0^T} [[h_4, h_3], [h_2, [h_1, x]]]
+ \frac{1}{8}\braced{4_0^T 3_0^4 2_0^T 1_0^2}[[h_4, h_3], [[h_2, h_1], x]]
\\&
+ \frac{1}{2} \braced{4_0^T 3_0^4 2_0^3 1_0^T} [[h_4, [h_3, h_2]], [h_1, x]]
+ \frac{1}{6}\braced{4_0^T 3_0^4 2_0^4 1_0^T} [[h_3, [h_4, h_2]], [h_1, x]] 
\\&
+\frac{1}{2}\braced{4_0^T 3_0^4 2_0^3 1_0^2} [[h_4, [h_3, [h_2, h_1]]], x] 
+ \frac{1}{6}\braced{4_0^T 3_0^4 2_0^3 1_0^3} [[h_4, [h_2, [h_3, h_1]]], x]
\\&
+ \frac{1}{8}\braced{4_0^T 3_0^4 2_4^T 1_0^2}
[[[h_4, h_3], [h_2, h_1]], x]
+ \frac{1}{24}\braced{4_0^T 3_0^4 2_0^4 1_0^4} [[h_3, [h_2, [h_4, h_1]]], x]
}}


\end{document}


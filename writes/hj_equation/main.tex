\documentclass{article}
\usepackage{amsmath}

\usepackage{igor_macros}

%%No paragraph break on emty lines
\catcode10=10   \catcode13=10

%\def\ee{\end{equation}}

\title {Deduction of Hamilton-Jakoby equation}
\author {Igor Bukanov}
\date { 08 February 1999 }

\begin{document}

\begin{abstract}
I present strict deduction of Hamilton-Jakoby eqution and corresponding
approach for solving equation of motion. I decided to write this due to
uneasy feeling that Landau-Lifshic Mechanics left to me. In any case it 
was  useful exercise.
\end{abstract}

\maketitle

\section{Definitions and deduction}
Consider the standard expression for the action of a classical system
given by
\eqgl{
\l{S-def}
S = \int_0^T L[\vec{\dot{q}}(t), \vec{q}(t), t] dt, \\
\l{init-cond}
\vec{q}(0) = \vec{q_0}, \, \vec{q}(T) = \vec{q_1},  \\
\vec{q} \equiv q_1 \dots q_N, \notag
}
where $N$ stands for the number of degree of freedom.

The corresponding Lagrange equations of motion are
\eqnl{L-eqn}{
\frac{\partial L}{\partial q^i}
    = \frac{d}{dt}\frac{\partial L}{\partial \dot{q}^i}
}

A solution of (\ref{L-eqn}) that satisfy the initial conditions 
\rf{init-cond} maybe formally written in the following form,
\eqgl{
\vec{q} = \vec{q}(t, T, \vec{q}_0, \vec{q}_1), \, \notag \\
\vec{q}(0, T, \vec{q}_0, \vec{q}_1) = \vec{q}_0, \, \l{formal-solution} \\
\vec{q}(T, T, \vec{q}_0, \vec{q}_1) = \vec{q}_1 . \notag
}

The form \rf{formal-solution} provides minimal or at least extremal action,
that is a function of initial conditions \rf{init-cond} and time interval $T$,

\eqnl{S-min-def}{
S_{min} \equiv S_{min}(T, \vec{q_0}, \vec{q_1})
    = \int_0^T L[\vec{\dot{q}}(t, T, \vec{q}_0, \vec{q}_1), 
                 \vec{q}(t, T, \vec{q}_0, \vec{q}_1), t] dt
}

With \rf{S-min-def} in hands I can start deduction of a equation for $S_{min}$
itself. In the following I assume $S \equiv S_{min}$.

\par
I calculate time derivative for it as a derivative of parametric
integral ,
\eqnl{der-S-a}{
\frac{\partial S}{\partial T} 
    = \left.L[\vec{\dot{q}}, \vec{q}, t)\right|_{t=T}
    + \int_0^T \frac{\partial}{\partial T} L[\vec{\dot{q}}, \vec{q}, t] dt .
}

To father transform \rf{der-S-a} I consider a general expression of the form
\eqn{
F(a) =
\int_0^T dt 
    \frac{\partial}{\partial a} L[\vec{\dot{q}}(t, a), \vec{q}(t, a), t]
}
with some arbitrary parameter $a$. I have
\eqn{
F(a) = 
\int_0^T dt \left\lbrace
\frac{\partial L}{\partial\vec{\dot{q}}}\frac{\partial\vec{\dot{q}}}{\partial a}
+ \frac{\partial L}{\partial\vec{q}}\frac{\partial\vec{q}}{\partial a}
\right\rbrace .
}

But 
\eqn{
\frac{\partial\vec{\dot{q}}}{\partial a} 
    \equiv \frac{\partial}{\partial a}\frac{\partial}{\partial t}\vec{q}(t, a) 
    = \frac{\partial}{\partial t}\frac{\partial}{\partial a}\vec{q}(t, a) 
}
under the assumtion that $\vec{q}(t, a)$ is smoth enough to permit interchange
of the differention order. This gives
\eqn{
\begin{split}
F(a) &= 
\int_0^T dt \left\lbrace
\frac{\partial L}{\partial\vec{\dot{q}}}\frac{\partial}{\partial t}
\frac{\partial\vec{q}}{\partial a}
+ \frac{\partial L}{\partial\vec{q}}\frac{\partial\vec{q}}{\partial a}
\right\rbrace . \\
&= 
\int_0^T dt \left\lbrace
\frac{\partial}{\partial t}\left[
\frac{\partial L}{\partial\vec{\dot{q}}}
\frac{\partial\vec{q}}{\partial a}\right]
+ \frac{\partial L}{\partial\vec{q}}\frac{\partial\vec{q}}{\partial a}
- \frac{\partial}{\partial t}\frac{\partial L}{\partial\vec{\dot{q}}}
    \frac{\partial\vec{q}}{\partial a}
\right\rbrace . \\
&= 
\left.
\frac{\partial L}{\partial\vec{\dot{q}}}
\frac{\partial\vec{q}}{\partial a}\right|_{t=0}^T
+
\int_0^T dt \frac{\partial\vec{q}}{\partial a} \left\lbrace
\frac{\partial L}{\partial\vec{q}}
- \frac{\partial}{\partial t}\frac{\partial L}{\partial\vec{\dot{q}}}
\right\rbrace . \\
\end{split}
}

Assuming now that $\vec{q}$ satisfies the Lagrange equations \rf{L-eqn},
the second term in the last expression gives 0 and I have
\eqgl{
\l{param-derivative}
\int_0^T \frac{\partial}{\partial a} L[\vec{\dot{q}}(t, a), \vec{q}(t, a), t] dt
= 
\left.
\frac{\partial L}{\partial\vec{\dot{q}}}
\frac{\partial\vec{q}}{\partial a}\right|_{t=0}^T ,
\\
\vec{\dot{q}}(t, a) \equiv 
\left.\frac{\partial\vec{q}}{\partial t}\right|_a \notag
}

When $a \equiv T$ the last equation gives for \rf{der-S-a}
\eqn{
\int_0^T dt 
    \frac{\partial}{\partial T} L[\vec{\dot{q}}(t, T), \vec{q}(t, T), t]
= 
\left.
\frac{\partial L}{\partial\vec{\dot{q}}}
\frac{\partial\vec{q}}{\partial T}\right|_{t=0}^T ,
}
and finally I have for \rf{der-S-a}
\eqnl{der-S-b}{
\frac{\partial S(T, \vec{q}_0, \vec{q}_1)}{\partial T} 
    = \left. L[\vec{\dot{q}}, \vec{q}, t]\right|_{t=T}
    + \left.\frac{\partial L}{\partial\vec{\dot{q}}}
      \frac{\partial\vec{q}}{\partial T}\right|_{t=0}^T .
}

Now I take into account that
\eqnl{L-at-second-edge}{
\begin{split}
\left. L[\vec{\dot{q}}, \vec{q}, t]\right|_{t=T}
&\equiv \left. L[\vec{\dot{q}}(t, T, \vec{q}_0, \vec{q}_1), 
 \vec{q}(t, T, \vec{q}_0, \vec{q}_1), t]
\right|_{t=T}
\\
&= L[\vec{\dot{q}}(T, T, \vec{q}_0, \vec{q}_1), 
    \vec{q}(T, T, \vec{q}_0, \vec{q}_1), T]
\\ 
&= L[\vec{\dot{q}}(T, T, \vec{q}_0, \vec{q}_1), \vec{q}_1, T]
\end{split}
}

from the boundary condition \rf{formal-solution}, 
$\vec{q}(T, T, \vec{q}_0, \vec{q}_1) = \vec{q}_1$.
\par

From the same condition it is also follows 
\eqn{
\frac{\partial}{\partial T}
\vec{q}(t=T, T, \vec{q}_0, \vec{q}_1)
=
\frac{\partial \vec{q}_1}{\partial T}
=
\left.\frac{\partial}{\partial t}
\vec{q}(t, T, \vec{q}_0, \vec{q}_1)\right|_{t=T}
+
\left.\frac{\partial}{\partial T}
\vec{q}(t, T, \vec{q}_0, \vec{q}_1)\right|_{t=T} ,
}
but $\frac{\partial \vec{q}_1}{\partial T} \equiv 0$ because 
the boundary conditions themself are not depend on $T$, so

\eqn{
0 =
\left.\frac{\partial}{\partial t}
\vec{q}(t, T, \vec{q}_0, \vec{q}_1)\right|_{t=T}
+
\left.\frac{\partial}{\partial T}
\vec{q}(t, T, \vec{q}_0, \vec{q}_1)\right|_{t=T} .
}
In the same way 
\eqn{
0 \equiv \frac{\partial \vec{q}_0}{\partial T} 
= \frac{\partial}{\partial T}
\vec{q}(t=0, T, \vec{q}_0, \vec{q}_1)
=
\left.\frac{\partial}{\partial T}
\vec{q}(t, T, \vec{q}_0, \vec{q}_1)\right|_{t=0} .
}
This gives for \rf{der-S-b} 
\eqnl{der-S-c}{
\begin{split}
\frac{\partial S(T, \vec{q}_0, \vec{q}_1)}{\partial T} 
&= 
L[\vec{\dot{q}}(T, T, \vec{q}_0, \vec{q}_1), \vec{q}_1, T]
- \left.\frac{\partial L}{\partial\vec{\dot{q}}}
  \frac{\partial}{\partial t}
  \vec{q}(t, T, \vec{q}_0, \vec{q}_1)\right|_{t=T}
\\ &\equiv
L[\vec{\dot{q}}(T, T, \vec{q}_0, \vec{q}_1), \vec{q}_1, T]
- \vec{\dot{q}}(T, T, \vec{q}_0, \vec{q}_1)
\frac{\partial }{\partial\vec{\dot{q}}} 
L[\vec{\dot{q}}, \vec{q}_1, T]
  
\end{split}
}

Now I consider ${\partial S}/{\partial \vec{q}_1}$.
\eqn{
\frac{\partial S(T, \vec{q}_0, \vec{q}_1)}{\partial \vec{q}_1}
 = \int_0^T \frac{\partial}{\partial \vec{q}_1}
     L[\vec{\dot{q}}(t, T, \vec{q}_0, \vec{q}_1), 
                 \vec{q}(t, T, \vec{q}_0, \vec{q}_1), t] dt
 
}

From \rf{param-derivative} with $a \equiv \vec{q}_1$ I get
\eqn{
\frac{\partial S(T, \vec{q}_0, \vec{q}_1)}{\partial \vec{q}_1}
=
\left.\frac{\partial L}{\partial\dot{q^i}}
\frac{\partial q^i}{\partial \vec{q}_1}\right|_{t=0}^T 
\equiv
\left.\frac{\partial L}{\partial\dot{q^i}}
\frac{\partial}{\partial \vec{q}_1}q^i(t, T, \vec{q}_0, \vec{q}_1)
\right|_{t=0}^T .
}

Using the boundary conditions in \rf{formal-solution} one more time I have
\eqn{
\left.Y(t)\frac{\partial}{\partial q^j_1}q^i(t, T, \vec{q}_0, \vec{q}_1)
\right|_{t=0}^T 
= 
Y(T)\frac{\partial q^i_1}{\partial q^j_1} 
    - Y(0)\frac{\partial q^i_0}{\partial q^j_1}
= 
Y(T)\delta^i_j - 0 = Y(T)\delta^i_j .
}

Thus 
\eqnl{der-S-d}{
\frac{\partial S(T, \vec{q}_0, \vec{q}_1)}{\partial \vec{q}_1}
= 
\left.\frac{\partial L}{\partial\vec{\dot{q}}}\right|_{t=T} ,
}

Equations \rf{der-S-c} and \rf{der-S-d} together form a closed system
\eqnl{S-der-system}{
\left\lbrace
\begin{split}
\frac{\partial S(T, \vec{q}_0, \vec{q}_1)}{\partial T} 
&= 
L[\vec{\dot{q}}(T, T, \vec{q}_0, \vec{q}_1), \vec{q}_1, T]
- \dot{q^i}(T, T, \vec{q}_0, \vec{q}_1)
    \frac{\partial }{\partial\dot{q^i}} 
    L[\vec{\dot{q}}, \vec{q}_1, T]

\\
\frac{\partial S(T, \vec{q}_0, \vec{q}_1)}{\partial \vec{q}_1}
&= \frac{\partial}{\partial\vec{\dot{q}}}
    L[\vec{\dot{q}}(T, T, \vec{q}_0, \vec{q}_1), \vec{q}_0, \vec{q}_1)]
\end{split}
\right.
}

or simply
\eqnl{S-der-system-b}{
\left\lbrace
\begin{split}
\frac{\partial S(T, \vec{q}_1)}{\partial T} 
&= 
L[\vec{\dot{q}}, \vec{q}_1, T]
- \dot{q}^i \partial_{\dot{q}^i}L [\vec{\dot{q}}, \vec{q}_1, T]
\\
\frac{\partial S(T, \vec{q}_1)}{\partial q^i_1}
&= 
\partial_{\dot{q}^i}L[\vec{\dot{q}}, \vec{q}_1, T]
\end{split}
\right. ,
}
where I skiped a dependence of $S$ on $\vec{q}_0$, which reflects the fact
that a general solution of \rf{S-der-system-b} will include some constants
$C_\alpha$, that assume dependence on $\vec{q}_0$

From a pure algebraic point of view I can use the second equation 
in \rf{S-der-system-b} to expres $\vec{\dot{q}}$ as a function of
$\vec{q}_1$, $T$ and $\frac{\partial S}{\partial \vec{q}_1}$:
\eqn{
\vec{\dot{q}} = 
    \vec{\dot{q}}(\vec{q}_1, T, \frac{\partial S}{\partial \vec{q}_1})
}

[ Note: I skipped completely the issue about the existence of such solution.]

This gives from \rf{S-der-system-b} the single partial differencal equation 
for $S$ as a function of $\vec{q}_1$ and $T$,
\eqnl{S-eqn}{
\begin{split}
\frac{\partial S(T, \vec{q}_1)}{\partial T}
= &   
L[\vec{\dot{q}}(\vec{q}_1, T, 
    \frac{\partial S(T, \vec{q}_1)}{\partial \vec{q}_1}), 
    \vec{q}_1, T]
\\
& - \dot{q}^i(\vec{q}_1, T, 
        \frac{\partial S(T, \vec{q}_1)}{\partial \vec{q}_1})
 \frac{\partial S(T, \vec{q}_1)}{\partial q^i_1}
\end{split}
}

The equation \rf{S-eqn} is called Hamilton-Jakoby equation. 
It is the first order partial non-linear differential equation. 
In short I can write it as 
\eqnl{JH-eqn}{
\frac{\partial S(T, \vec{q})}{\partial T}
=    
L[\vec{\dot{q}}, \vec{q}, T]
 - \frac{\partial S(T, \vec{q})}{\partial q^i} \dot{q}^i
}

where $\vec{\dot{q}} = \vec{\dot{q}}(\partial_{\vec{q}}S, T, q)$ 
can be considered as a symbol representing a function
 obtained from the solution of the algebraic system
\eqnl{JH-qdot}{
\frac{\partial S(T, \vec{q})}{\partial q^i}
= 
\partial_{\dot{q}^i}L[\vec{\dot{q}}, \vec{q}, T]
}

\par 

Now comes technically more difficult in my opinion part to 
deduce the solution of the equation of motion \rf{L-eqn} from 
the solution of the JH-equation \rf{JH-eqn}.

\par

As for a first-oder partial differncial equation for $S$ in $N+1$ space, 
the general solution of \rf{JH-eqn} can be written as function
of $T$, $\vec{q}$ and $N + 1$ constants $\vec{\alpha}$ and $\alpha_0$:
\eqn{
    S = \tilde{S}(T, \vec{q}, \vec{\alpha}) + \alpha_0,
}
where $\alpha_0$ is written explicitly as additive constant. 
It is possible because $S$ is present in \rf{JH-eqn} only via
its patial derivatives $\partial_T S$ and $\partial_{\vec{q}} S$.

\par
In the following the value of $\alpha_0$ is not important and for 
this reason I simply assume that
\eqnl{S-is-T-q-alpha}{
    S = S(T, \vec{q}, \vec{\alpha}).
}

I define 
\eqnl{beta-definition}{
    \vec{\beta}(T, \vec{q}, \vec{\alpha}) = \frac{\partial S}{\partial \alpha}.
}

This definition also can be seen as introduction of new coordinates 
$T$, $\vec{\beta}$, $\vec{\alpha}$ instead of the set 
$T$, $\vec{q}$, $\vec{\alpha}$. I will assume in the following that 
\rf{beta-definition} can be resolved with respect to $\vec{q}$. 
[Warning: under what condition this can be true?] I.e. I can assume
that 
\eqn{
    \vec{q} = \vec{q}(T, \vec{\beta}, \vec{\alpha})
        \equiv \vec{q}(T, \frac{\partial S}{\partial \alpha}, \vec{\alpha}).
}

With this definitions/assumesiones in hands I differenciate \rf{JH-eqn}
with respect to $\alpha$,
\eqn{
\begin{split}
\frac{\partial^2 S}{\partial \vec{\alpha} \partial T}
 & = \frac{\partial^2 S}{\partial T \partial \vec{\alpha}}
\\ & 
\equiv \frac{\partial \vec{\beta}}{\partial T}
= \frac{\partial}{\partial \vec{\alpha}}
    L[\vec{\dot{q}}(\partial_{\vec{q}}S(T, \vec{q}, \vec{\alpha}), T, q), 
        \vec{q}, T]
-
\frac{\partial}{\partial \vec{\alpha}}\left\lbrace 
 \frac{\partial S(T, \vec{q}, \vec{\alpha})}{\partial q^i} \dot{q}^i
\right\rbrace
\\ & 
= \frac{\partial L}{\partial \dot{q}^i}
\frac{\partial \dot{q^i}}{\partial \vec{\alpha}}
-\frac{\partial^2 S}{\partial \vec{\alpha} \partial q^i}\dot{q}^i
-\frac{\partial S}{\partial q^i} 
    \frac{\partial \dot{q}^i}{\partial \vec{\alpha}}

= 
\frac{\partial S}{\partial q^i} 
    \frac{\partial \dot{q}^i}{\partial \vec{\alpha}}
-\frac{\partial^2 S}{\partial \vec{\alpha} \partial q^i}\dot{q}^i
-\frac{\partial S}{\partial q^i} 
    \frac{\partial \dot{q}^i}{\partial \vec{\alpha}}
\end{split},
}
or 
\eqnl{beta-over-T}{
\left.\frac{\partial \vec{\beta}}{\partial T}\right|_{\alpha\vec{q}}
\; = \; -\frac{\partial^2 S}{\partial \vec{\alpha} \partial q^i}\dot{q}^i
\; = \; 
-\left.\frac{\partial\vec\beta}{\partial q^i}
  \right|_{\alpha T}\dot{q}^i 
}

where I used \rf{JH-qdot} to replace 
$\partial_{\dot{q}^i}L[\vec{\dot{q}}, \vec{q}, T]$ by 
$\partial_{q^i}S$.

\par

I must note here that $\dot{\vec{q}}$ in \rf{beta-over-T} is just a symbol
that denote a function defined as the solution of 
\rf{JH-qdot}, i.e. it just represents some function of $T$, $\beta$,
$\alpha$:
\eqn{
\vec{\dot{q}}
 = 
\vec{\dot{q}}(\partial_{\vec{q}}S(\vec{q}, T, \vec{\alpha}), T, q)
=
\vec{\dot{q}}(\partial_{\vec{q}}S(\vec{q}(T, \vec{\beta}, \vec{\alpha}), 
  T, \vec{\alpha}), T, q(T, \vec{\beta}, \vec{\alpha}))
}

\par
Now I return to the interpretation of \rf{beta-definition} as a non-explicit
definition of function $\vec{q}(\vec{\beta}, T, \vec{\alpha})$. I will show 
that with fixed $\vec{\beta}$, $\vec{\alpha}$ this function $\vec{q}$ as
function of $T$ satisfies the equation of motion \rf{L-eqn}. Indeed, I have
for 
\eqn{
\left.\frac{\partial{q}}{\partial{T}}\right|_{\beta\alpha} 
\; \equiv \;
\vec{q}_T
}
from $\vec{\beta} = \vec{\beta}(\vec{q}, T, \vec{\alpha})$:

\eqnl{q-dot-q-T-deduction}{
\begin{split}
0 & = 
\left.\frac{\partial\vec{\beta}}{\partial T}
(\vec{q}, T, \vec{\alpha})\right|_{\alpha\beta}
\; = \;
\left.\frac{\partial\vec{\beta}}{\partial q^i}\right|_{\alpha T}
\left.\frac{\partial q^i}{\partial T}\right|_{\alpha\beta} 
+ 
\left.\frac{\partial\vec{\beta}}{\partial T}
\right|_{\alpha\vec{q}}
\\ & =
\left.\frac{\partial\vec{\beta}}{\partial q^i}\right|_{\alpha T}
q^i_T
+ 
\left.\frac{\partial\vec{\beta}}{\partial T}
\right|_{\alpha\vec{q}}
\; = \;
\left.\frac{\partial\vec{\beta}}{\partial q^i}\right|_{\alpha T}
q^i_T
-\left.\frac{\partial\vec\beta}{\partial q^i}
  \right|_{\alpha T}q^i_T\dot{q}^i 
\\ & =
\left.\frac{\partial\vec{\beta}}{\partial q^i}\right|_{\alpha T}
\left(q^i_T - \dot{q}^i\right)
\end{split}
}

\par
Assuming that the matrix
$\frac{\partial\vec{\beta}}{\partial q^i} 
\equiv \frac{\partial\beta^j}{\partial q^i}$ has non-zero determinant
[ I need to refer to a proper mathematics for exact reasons ]
I multiply \rf{q-dot-q-T-deduction} by 
$\left(\frac{\partial\beta^j}{\partial q^i}\right)^{-1}$ which gives
\eqnl{q-dot-is-q-T}{
\dot{q}^i = q^i_T 
\equiv \left.\frac{\partial{q^i}}{\partial{T}}\right|_{\alpha\beta} .
} 

That is, an abstract symbol $\dot{\vec{q}}$ is equal to the partial derivative
$\vec{q}_T$. This in turns means from \rf{JH-qdot} that $\vec{q}_T$ satisfies
\eqnl{JH-qdot-alpha}{
\frac{\partial S(T, \vec{q}, \vec{\alpha})}{\partial q^i}
= 
\partial_{q_T^i}L[\vec{q}_T, \vec{q}, T].
}

I differenciate the last equation over T to get
\eqn{
\begin{split}
\frac{d}{d T}\partial_{q_T^i}L[\vec{q}_T, \vec{q}, T]
& \equiv  
\left.\frac{\partial}{\partial T}\right|_{\alpha \vec{\beta}}
\partial_{q_T^i}L[\vec{q}_T, \vec{q}, T]
\; = \;
 \left.\frac{\partial}{\partial T}\right|_{\alpha\vec{\beta}} 
  \left.\frac{\partial S(T, \vec{q}, \vec{\alpha})}{\partial q^i}
    \right|_{\alpha T}
\\
& = \left.\frac{\partial}{\partial T}\right|_{\alpha\vec{q}} 
  \left.\frac{\partial S(T, \vec{q}, \vec{\alpha})}{\partial q^i}
    \right|_{\alpha T}
+ 	
\left.\frac{\partial}{\partial q^j}\right|_{\alpha T} 
  \left.\frac{\partial S(T, \vec{q}, \vec{\alpha})}{\partial q^i}
    \right|_{\alpha T}
  \left.\frac{\partial q^j}{\partial T}\right|_{\alpha \vec{\beta}} 		
\\
& =  \left.\frac{\partial}{\partial q^i}\right|_{\alpha T}
  \left\{L[\vec{\dot{q}}, \vec{q}, T]
 - \frac{\partial S(T, \vec{q}, \vec{\alpha})}{\partial q^j} \dot{q}^j
    \right\} 
\\
& \phantom{ = }
+
\left.\frac{\partial}{\partial q^j}\right|_{\alpha T} 
  \left.\frac{\partial S(T, \vec{q}, \vec{\alpha})}{\partial q^i}
    \right|_{\alpha T}
  q^j_T
\end{split}
}

due to \rf{JH-eqn}, thus

\eqn{
\begin{split}
\frac{d}{d T}\partial_{q_T^i}L[\vec{q}_T, \vec{q}, T]
& =  \left.\frac{\partial}{\partial q^i}\right|_{\alpha T}
  \left\{L[\vec{q_T}, \vec{q}, T]
 - \frac{\partial S(T, \vec{q}, \vec{\alpha})}{\partial q^j} q_T^j
    \right\} 
\\
& \phantom{ = }

+
\left.\frac{\partial}{\partial q^j}\right|_{\alpha T} 
  \left.\frac{\partial S(T, \vec{q}, \vec{\alpha})}{\partial q^i}
    \right|_{\alpha T}
  q^j_T
\\
& =  \left.\frac{\partial}{\partial q^i}\right|_{\alpha T}
  L[\vec{q_T}, \vec{q}, T]
 - \frac{\partial^2 S(T, \vec{q}, \vec{\alpha})}{\partial q^i \partial q^j} 
    q_T^j
 -  \frac{\partial S(T, \vec{q}, \vec{\alpha})}{\partial q^j}   
    \frac{\partial q_T^j} {\partial q^i }
\\
& \phantom{ = }
+
\frac{\partial^2 S(T, \vec{q}, \vec{\alpha})}{\partial q^j \partial q^i}
  q^j_T

\\
& =  \left.\frac{\partial}{\partial q^i}\right|_{\alpha T}
  L[\vec{q_T}, \vec{q}, T]
 -  \frac{\partial S(T, \vec{q}, \vec{\alpha})}{\partial q^j}   
    \frac{\partial q_T^j} {\partial q^i }

\\
& =  
\left.\frac{\partial}{\partial q^i}\right|_{\alpha \vec{q}_T}
  L[\vec{q_T}, \vec{q}, T]

+
\left.\frac{\partial}{\partial q_T^j}\right|_{\alpha \vec{q}}
  L[\vec{q_T}, \vec{q}, T] 
\left.\frac{\partial q_T^j}{\partial q^i}\right|_{\alpha T}
\\
& \phantom{ = }

 -  \frac{\partial S(T, \vec{q}, \vec{\alpha})}{\partial q^j}   
    \frac{\partial q_T^j} {\partial q^i }.

\end{split}
}

Using \rf{JH-qdot-alpha} one more time I finally have
\eqn{
\begin{split}
\frac{d}{d T}\partial_{q_T^i}L[\vec{q}_T, \vec{q}, T]
& =  
\left.\frac{\partial}{\partial q^i}\right|_{\alpha \vec{q}_T}
  L[\vec{q_T}, \vec{q}, T]

+  \frac{\partial S(T, \vec{q}, \vec{\alpha})}{\partial q^j}   
    \frac{\partial q_T^j} {\partial q^i }.
\\
& \phantom{ = }
 -  \frac{\partial S(T, \vec{q}, \vec{\alpha})}{\partial q^j}   
    \frac{\partial q_T^j} {\partial q^i }.

\end{split}
}

or simply
\eqn{
\frac{d}{d T}\partial_{q_T^i}L[\vec{q}_T, \vec{q}, T]
\; = \; 
\frac{\partial}{\partial q^i} L[\vec{q_T}, \vec{q}, T].
}

Thus $\vec{q}(T, \vec{\beta}, \vec{\alpha})$ defined in inexplicit way
via \rf{beta-definition} is a solution of equations of motion \rf{L-eqn}
that depends on $2N$ constants $\alpha_i$, $\beta_i$, $i=1..N$

\par

\section{Bottom line}

In short, I can describe the Hamilton-Jakoby method as following procedure.

\begin{enumerate}
\item
Given the Lagrangian $L=L(\dot{\vec{q}}, \vec{q}, T)$ express $\dot{\vec{q}}$
as a function of $\partial_{\dot{\vec{q}}}L$, $q$ and~$T$:
\eqn{
\dot{\vec{q}} = \dot{\vec{q}}
    (\frac{\partial L}{\partial \dot{\vec{q}}}, \vec{q}, T)
}

\item
Substitude $\partial_{\dot{\vec{q}}}L$ by $\partial_{\vec{q}}S(T, \vec{q})$
in the expression for $\dot{\vec{q}}$ and find a general solution 
of the Hamilton-Jakoby equation \rf{JH-eqn} for $S(T, \vec{q})$:

\eqn{
\partial_T S(T, \vec{q}) =    
L[\vec{\dot{q}}(\partial_{\vec{q}}S, \vec{q}, T), \vec{q}, T]
 - \partial_{q^i} S(T, \vec{q}) \dot{q}^i(\partial_{\vec{q}}S, \vec{q}, T).
}

\item
Present the general solution in the following form:
\eqn{
S = S(T, \vec{q}, \vec{alpha}) + \alpha_0,
}
where $\vec{alpha})$ and $\alpha_0$ are $N+1$ arbitrary constants.

\item

From $beta^i \equiv \partial_{\alpha_i} S(T, \vec{q}, \vec{\alpha})$ express
$\vec{q}$ as a function of $\vec{\alpha}$, $\vec{\beta}$ and~$T$:
\eqn{
\vec{q} = \vec{q}(\vec{\alpha}, \vec{\beta}, T).
}

\item

Use definitions $\vec{q_0} = \vec{q}(\vec{\alpha}, \vec{\beta}, T_0)$, 
$\vec{v_0} = \partial_T \vec{q}(\vec{\alpha}, \vec{\beta}, T_0)$ to express
$q$ via initial conditions:
\eqn{
\vec{q} = \vec{q}(\vec{q_0}, \vec{v_0}, T).
}

\end{enumerate}

And that is it.

\end{document}

